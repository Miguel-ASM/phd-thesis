%!TEX root = ../Thesis.tex
%Conclusions Chapter

\chapter*{Conclusions}
\label{Conclusions}
\lhead{\emph{Conclusions}}
\addcontentsline{toc}{chapter}{Conclusions}
\null
% \vfill
\textit{``Don't adventures ever have an end? I suppose not. Someone else always has to carry on the story.''}
\begin{flushright}
  {\bf J. R. R. Tolkien}\\
  The Fellowship of the Ring
\end{flushright}
%\vfill
\null

In this Thesis I have presented the most relevant results of the research I have conducted during my PhD, which embraces the study of asymmetric scattering and heat transport. The general goal of my research was to design devices that alow asymmetric transport and that are reallistic enough, so they can be implemented experimentally. In this chapter I will summarize the main results of this Thesis and present the conclusions.

% Since this Thesis is divided in a first part, which adresses the design of asymmetric devices based on non-Hermitian and non-local potentials, and a second part, which focus on the design of thermal rectifiers, I will present the conclusions to each part in different sections.
% Finally, I shall present my general conclusions for the entirety of my work.


\section*{Conclusions to part I}

\begin{itemize}
  \item {\bf Asymmetric scattering by non-Hermitian potentials}
  \begin{itemize}
    \item Six types of devices with asymmetric scattering are possible when imposing 0 or 1 for the values of the scattering probabilities.

    \item Hermitian Hamiltonians do not allow for any asymmetry in transmission and
    reflection probabilities, therefore in order to design asymmetric devices non-Hermitian
    Hamiltonians are needed. Besides, non-local potentials are needed for asymmetric scattering.

    \item There are 8 symmetries that generate all the possible transformations of the potential matrix elements, which consist in complex conjugation, coordinate inversion, the identity and transposition. The eight symmetries arise from the commutation or pseudohermiticity of the potential with an element of the Klein’s 4-group $\mathbb{K}_4 = \{1,\Pi,\Theta,\Pi\theta\}$. The symmetries impose selection rules for the scattering amplitudes that conditions the design of some of the devices.

    \item The conventional definition of a symmetry in terms of commutation with a unitary/antiunitary operator $A$ is extended with the concept of $A$-pseudohermiticity for non-Hermitian Hamiltonians. Both commutation and $A$-pseudohermiticity must be considered on the same footing.

    \item Some example potentials are given for the different asymmetric devices, in particular a local PT-potential that works as a transparent 1-way reflector in a broad domain of incident momenta.

  \end{itemize}

  \item {\bf $S$-matrix pole symmetries for non-Hermitian scattering Hamiltonians}
  \begin{itemize}
    \item The symmetries of a non-Hermitian Hamiltonian, understood as commutation or $A$-pseudohermiticity, can be rewritten as the invariance of $H$ with respect to the action of a unitary or antiunitary superoperator, $H = \mathcal{L}(H)$. Following this approach with the 8 symmetries described in chapter \ref{Chapter1}, a group structure is unveiled: the 8 symmetries form the elementary abelian group E8.

    \item In refs. \cite{Mostafazadeh2002,Mostafazadeh2002a,Mostafazadeh2002b} it was shown that
    $A$-pseudohermiticity (with $A$ linear and Hermitian) or commutavity with an antilinear Hermitian operator were necessary and sufficient conditions for a discrete Hamiltonian to have conjugate pairs of discrete eigenenergies. I show that this result can be extended to scattering Hamiltonians. Scattering Hamiltonians that satisfy the same conditions, have the poles of their $S$-matrix forming conjugate pairs in the complex energy plane.

    \item I provided examples of the distribution of poles using separable potentials. The two examples correspond to the non-trivial symmetries: time-reversal and parity-pseudohermicity.

  \end{itemize}

  \item {\bf Quantum-optical implementation of non-Hermitian potentials for asymmetric scattering}
  \begin{itemize}
    \item I propose a quantum-optical implementation of non-local and non-Hermitian potentials
    with asymmetric scattering amplitudes. Since they are non-local and also non-PT symmetrical they allow asymmetric transmission.

    \item The non-Hermitian potentials are effective interactions for the ground state of a two-level atom impinging on a laser field. They are found using Feshbach projection technique.

    \item I present examples of a $\mathcal{T/A}$ device (One-way T-filter), a $\mathcal{R/A}$ device (One-way R-filter) and a partial $\mathcal{TR/A}$ device (One-way mirror).

  \end{itemize}

\end{itemize}

\section*{Conclusions to part II}

\begin{itemize}
  \item {\bf Local rectification of heat flux}
  \begin{itemize}
    \item I have presented a design for a thermal rectifier based on a localized impurity in a chain of atoms. The on-site potential and interatomic interactions are modeled with harmonic and Morse potentials, respectively.

    \item As oposed to other models, the chain is homogeneous and the only structural asymmetry is the impurity.

    \item The numerical results show normal heat conduction without the impurity and rectification when it is present.

    \item Rectification also occurs when the Morse interaction is substituted by a harmonic one, although it is somewhat weaker.

  \end{itemize}

  \item {\bf Asymmetric heat transport in ion crystals}
  \begin{itemize}
    \item I introduced a model of a chain of ions trapped in individual microtraps and in contact at both ends with thermal baths mediated by optical molasses.

    \item Numerical results show that there is rectification when the microtrap frequencies are graded along the chain.

    \item In this model I explore some of the mechanisms that have been proposed in the literature to improve rectification, namely long range interactions and graded structures.

    \item This model could be implemented in a trapped-ion platform, which is interesting because is one of the most controllable quantum technologies platform. Besides, this work connects two different scientific communities: ion trappers and\\
    researchers in thermal rectification.
  \end{itemize}

  \item {\bf Heat rectification with a minimal model of two harmonic oscillators}
  \begin{itemize}
    \item  I study thermal rectification in an analytically treatable model
    composed of two coupled harmonic oscillators connected to Langevin Baths.

    \item The results demonstrate that thermal rectification is also possible in
    harmonic systems if there are temperature-dependent features. In this case, the
    temperature dependence is in the coupling of the oscillators to the baths. This
    temperature dependence arises naturally in Doppler-cooled trapped-ion setups.

    \item The phonon band match-mismatch description that was proposed for non-harmonic
    systems also applies to this harmonic model.
  \end{itemize}

\end{itemize}

% \section*{General conclusions}
