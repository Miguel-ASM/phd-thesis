%!TEX root = ../Thesis.tex
%Chapter 1

\chapter{Symmetries and Invariants for Non-Hermitian Hamiltonians}
\label{ChapterSymAndInvariants}
\lhead{Chapter Sym, Invariants. \emph{Symmetries and Invariants for Non-Hermitian Hamiltonians}} % Write in your own chapter title to set the page header
%
We discuss Hamiltonian symmetries and invariants for quantum systems driven by
non-Hermitian Hamiltonians. For time-independent Hermitian Hamiltonians, a unitary or antiunitary transformation
$AHA^\dagger$ that leaves the Hamiltonian $H$
unchanged represents a symmetry of the Hamiltonian, which implies the commutativity $[H,A]=0$
and, if $A$ is linear and  time-independent,  a conservation law,
namely the invariance of expectation values of $A$.
% and provides a way to create degenerate eigenfunctions
%to span a degenerate subspace.
For non-Hermitian Hamiltonians, $H^\dagger$ comes into play as a distinct operator that complements $H$
in generalized unitarity relations.
The above description of symmetries has to be extended to include also $A$-pseudohermiticity
relations of the form
$AH=H^\dagger A$. A superoperator formulation of Hamiltonian symmetries is provided and exemplified
for Hamiltonians
of a particle moving in one-dimension considering  the set of $A$ operators
that form Klein's 4-group: parity, time-reversal, parity\&time-reversal, and unity.
The link between symmetry and conservation laws is
discussed and shown to be richer and subtler for non-Hermitian than for Hermitian Hamiltonians.
%
\newpage
%

\section{Introduction}
%
%
%
%
%
%
The intimate link between invariance and symmetry is well studied and understood
for
%behind the paradoxical fact that dynamics is often
%understood and characterized by means of elementary invariant objects. Invariance and symmetry are
%indeed well understood for
Hermitian Hamiltonians but non-Hermitian Hamiltonians pose some interesting
conceptual and formal challenges.
% that we want to address here.
Non-Hermitian Hamiltonians arise naturally in quantum systems as effective interactions for a subsystem.
These Hamiltonians may be proposed phenomenologically or may be
found exactly or approximately by applying Feshbach's projection technique to describe the dynamics in
the subsystem \cite{Feshbach1958,Ruschhaupt2004a}.
It is thus  important to understand
%the corresponding dynamics, forwards and backwards in time,  and
how common concepts for Hermitian Hamiltonians such as ``symmetry'',
``invariants'', or ``conservation laws'' generalize.
A lightning review in this section of concepts and formal relations for a time-independent Hermitian Hamiltonian $H$ will
be helpful as the starting point to
address generalizations for a non-Hermitian $H$. Unless stated otherwise, $H$ is time-independent in the following.
In quantum mechanics $A$ (unitary or antiunitary) represents a symmetry of the Hamiltonian if, together with its adjoint $A^\dagger$,
satisfies
%
\begin{equation}
A^\dagger HA=H,
\label{symme}
\eeq
%
so that
%
\beq
[H,A]=0,
\label{conmu}
\eeq
%
and thus $A$,  which we assume to be time-independent unless stated otherwise,
represents also a conserved quantity when A is unitary (and therefore linear),
%
\beq
\la \psi(t),A\psi(t)\ra=\la \psi(0),A \psi(0)\ra,
\label{conser}
\eeq
%
(We use the ordinary quantum inner product notation.)
where $|\psi(t)\ra=U(t)|\psi(0)\ra$ is the time-dependent wave function
satisfying the  Schr\"odinger equation
%
\begin{equation}
    i\hbar\partial_t\ket{\psi(t)} = H \ket{\psi(t)},
    \label{eq:ScrdngrEqn}
\end{equation}
%
and $U(t)=e^{-iHt/\hbar}$ is the unitary evolution operator from $0$ to $t$,
$U(t)U^\dagger(t)=U^\dagger(t)U(t)=1$.

Backwards evolution in time from $t$ to 0 is
represented by $U(-t)=U(t)^\dagger$ so that the initial state is recovered by
a forward and backward sequence,
$U(t)^{\dagger}U(t)|\psi(0)\ra=|\psi(0)\ra$.

Equation (\ref{conser}) may formally be found for an antiunitary $A$ if $AH = -HA$. However,
for antilinear operators expectation  values are ambiguous since multiplication of the state by a
unit modulus phase factor $e^{i\phi}$ changes the expectation value by $e^{-2i\phi}$.
This ambiguity
does not mean at all that antilinear symmetries do not have physical consequences.
They affect, for example, selection rules for possible transitions.
%The possible symmetry relations originated by anticonmutation of the Hamiltonian with a unitary/antiunitary operator
%will not be discussed in this article.


More generally, time-independent linear operators $A$ satisfying (\ref{conmu}), fullfill (\ref{conser}) without the
need to be unitary,
and represent also invariant quantities.
A further property from (\ref{conmu}) is that if $|\phi_E\ra$ is an eigenstate of $H$ with (real) eigenvalue $E$, then
$A|\phi_E\ra$ is also an eigenstate of $H$ with the same eigenvalue.
% for $A$ linear, or its complex conjugate for $A$
%antilinear.
%
%
%
\section{Dual character of $H$ and $H^\dagger$}
%
%
%
For $H\neq H^\dagger$,
we find the generalized unitarity relations
$U(t)\widehat{U}^\dagger(t)=\widehat{U}^\dagger U(t)=1$, where $\widehat{U}(t)=e^{-iH^\dagger t/\hbar}$.
Backwards evolution with $H^\dagger$
compensates the changes induced forwards by $H$. Similar generalized unitarity relations exist for the scattering
$S$ matrix (for evolution with $H$) and the corresponding $\widehat{S}$ (for evolution with $H^\dagger$), with
important physical consequences discussed e.g. in \cite{Muga2004,Ruschhaupt2017}.
%

%
%{\it{Formal generalizations.}}
%
%
%
Now consider the following two formal generalizations of the element $\la \psi(t),A\psi(t)\ra$ in Equation (\ref{conser}),
%
\beqa
&&\la e^{-iH^\dagger t/\hbar} \psi(0), A e^{-iHt/\hbar} \psi(0)\ra=\la \widehat{\psi}(t),A\psi(t)\ra,
\label{sec}
\\
%
&&\la e^{-iH t/\hbar}  \psi(0), A e^{-iHt/\hbar} \psi(0)\ra=\la \psi(t), A \psi(t)\ra,
\label{pri}
\eeqa
%
and the  generalizations of (\ref{conmu})
%
\beqa
AH&=&HA,
\label{usu}
\\
AH&=&H^\dagger A.
\label{unu}
\eeqa
%
We name (\ref{unu}) $A$-pseudohermiticity of $H$ \cite{Mostafazadeh2010}.
(This is here a formal definition that does not presupose any  further property
on $A$.)
Up  to normalization, which will be discussed in the following section, Equation (\ref{pri}) corresponds to the
usual rule to
define expectation values, whereas (\ref{sec}), where $|\widehat{\psi}(t)\ra\equiv e^{-iH^\dagger t/\hbar} |\psi(0)\ra$,
is unusual, and its physical meaning is not obvious. Note, however, that for linear $A$,
$AH=HA$ implies the conservation of the unusual quantity (\ref{sec}), whereas
$A$-pseudohermiticity $AH=H^\dagger A$ implies the conservation of the usual quantity (\ref{pri}) \cite{Mostafazadeh2002,Vitanov2016},
as discussed mostly for local PT-symmetrical potentials with $A$ being the parity operator \cite{Bagchi2001,Zezyulin2013,Konotop2016}.
At this point we might be tempted to discard (\ref{usu}) as less useful or significant physically.
This is however premature for several reasons. One is the following (others will be seen in Sections 4 to 6):
Unlike Hermitian Hamiltonians, non-Hermitian ones may have generally different
right and left eigenvectors. We assume the existence of the resolution
%
\beq
H=\sum_j |\phi_j\ra E_j \la\widehat{\phi}_j|,
\label{res}
\eeq
%
where the $E_j$ may be complex and where
%
\beqa
H|\phi_j\ra=E_j|\phi_j\ra, \;\;\;
H^\dagger|\widehat{\phi}_j\ra=E_j^* |\widehat{\phi}_j\ra.
\eeqa
%
We have used a simplifying notation for  a discrete spectrum, but a continuum
part could be treated similarly with integrals rather than sums and continuum-normalized states.
Note that left eigenstates of $H$ are right eigenstates of $H^\dagger$
with a complex conjugate eigenvalue. If $|\phi_j\ra$ is  a right eigenstate of $H$
with eigenvalue $E_j$, Equation (\ref{usu}) implies that
$A|\phi_j\ra$ is also a right  eigenstate of $H$, with the
same eigenvalue if $A$ is linear, and with the complex conjugate eigenvalue $E_j^*$ if $A$ is antilinear.
Instead, Equation (\ref{unu}) implies that $A|\phi_j\ra$ is a right eigenstate of $H^\dagger$
with eigenvalue $E_j$ for $A$ linear or $E_j^*$ for $A$ antilinear, or a left eigenstate of $H$ with eigenvalue
$E_j^*$ for $A$ linear, or $E_j$ for $A$ antilinear.
%For real-energy scattering
%eigenfunctions in the continuum, the ones we are interested in here, $E^*=E$.
%When eq. (\ref{pseudo}) holds we say that $H$ is $A$-pseudohermitian \cite{mostareview}.
As right and left eigenvectors must be treated on equal footing, since both are needed for the
resolution (\ref{res}), this argument points at a similar importance of the relations (\ref{usu}) and (\ref{unu}).
%
%
%
%
\section{Time evolution for normalized states.}
%
%
For a quantum system following the  Schr\"odinger equation (\ref{eq:ScrdngrEqn})
%Consider a quantum system whose time evolution follows the
%
%\begin{equation}
%    i\hbar\partial_t\ket{\psi(t)} = H \ket{\psi(t)},
%    \label{eq:ScrdngrEqn}
%\end{equation}
%
with $H$ non-Hermitian, in general the evolution will not  be unitary and
the norm $N_{\psi}(t) \equiv \braket{\psi(t)}{\psi(t)}$ is not conserved.
%
%\begin{equation}
%    N_{\psi}(t) \equiv \braket{\psi(t)}{\psi(t)}.
%   \label{eq:Norm}
%\end{equation}
%
We shall asume the initial condition $N_{\psi}(0)=1$. Using Equation \eqref{eq:ScrdngrEqn},
the rate of change of the norm is
%
\begin{equation}
    \partial_t \braket{\psi(t)}{\psi(t)} = \frac{1}{i\hbar}\bra{\psi(t)}H-H^\dagger\ket{\psi(t)}.
    \label{eq:NormChange}
\end{equation}
%
%
%
\subsection{Expectation Values} %We set this as subsection format, please confirm.{\it{Expectation Values.}}
%
%
%
%
We now restrict the discussion to linear (not necessarily unitary) observables $A$. Since the state of the system is not normalized to 1
for $t>0$, the expectation value formula has to take into account the norm explicitly,
%
\begin{equation}
  \expval{A}(t) = \frac{\ev{A}{\psi(t)}}{\braket{\psi(t)}}.
  \label{eq:ExpectValue}
\end{equation}
%
Since here $A$ is linear we may use
the standard Dirac ``braket'' notation for matrix elements with vertical bars.
Using Equations (\ref{eq:ScrdngrEqn}) and (\ref{eq:NormChange})
the rate of change of the expectation value of $A$ is
%
\begin{equation}
  \partial_t \ev{A}(t) = \frac{1}{i\hbar}\frac{\braket{\psi(t)}\ev{AH-H^\dagger A}{\psi(t)}-\ev{H-H^\dagger}{\psi(t)}\ev{A}{\psi(t)}}{\braket{\psi(t)}^2}.
  \label{eq:ExpectValueChange}
\end{equation}
%
%
%
%{\it{Expectation values.}}
%
%
%
For Hermitian Hamiltonians the commutation of $A$ and $H$ leaves  the expectation values of $A$ invariant.
For non-Hermitian Hamiltonians the symmetry Equation (\ref{unu})
applied to Equation \eqref{eq:ExpectValueChange} gives
%
\begin{equation}
\partial_t \ev{A}(t) = \frac{-1}{i\hbar}\frac{\ev{H-H^\dagger}{\psi(t)}\ev{A}{\psi(t)}}{\braket{\psi(t)}^2}.
\label{eq:ExpectValueChangeSymmetry}
\end{equation}
%
If we use Equations \eqref{eq:NormChange} and \eqref{eq:ExpectValue} in Equation
\eqref{eq:ExpectValueChangeSymmetry},
%
\beqa
\frac{\ev{A}}{\braket{\psi(t)}} \partial_t\braket{\psi(t)} &=& -  \partial_t \ev{A} ,
%  \\\braket{\psi(t)}\partial_t \ev{A} + \ev{A}\partial_t\braket{\psi(t)} &=0,
%  \\\partial_t\left( \ev{A}\braket{\psi(t)} \right) &= 0,
\\
\ev{A}\braket{\psi(t)} &=& \text{Constant}.
\eeqa
%
Applying the initial condition $\braket{\psi(0)}=1$,
%
\begin{equation}
  \ev{A}(t) = \frac{\ev{A}(0)}{\braket{\psi(t)}},
  \label{eq:ExpectValueScaling}
\end{equation}
%
so the expectation value of an $A$ that obeys
$AH = H^\dagger A$,  is simply rescaled by the norm of the wave function as it increases or decreases.
%
%

\subsection{Lower Bound on the Norm of the Wave Function}%{\it{Lower bound on the norm of the wave function.}}
%
%
The symmetry condition $AH = H^\dagger A$ may set lower bounds to the norm along the dynamical process.
Consider a linear observable $A$ with real eigenvalues $\{a_i\}$ bounded by ${max\{\abs{a_i}\}}$.
%(the spectrum could also be continuous).
Then, the expectation values satisfy $\abs{\ev{A}}\leq max\{\abs{a_i}\}$. If we use the result in \eqref{eq:ExpectValueScaling} we get
%
\begin{equation}
  \braket{\psi(t)} \geq \frac{\abs{\ev{A}(0)}}{max\{\abs{a_i}\}}.
  \label{eq:LowerBound}
\end{equation}
%
Equation \eqref{eq:LowerBound} bounds the norm of the state due to symmetry conditions.
A remarkable case is parity pseudohermiticity, $\Pi H = H^\dagger \Pi$, where the (unitary)
parity operator acts on the position eigenstates as $\Pi\ket{x} = \ket{-x}$ and has eigenvalues $\{-1,1\}$. Under this symmetry, Equation \eqref{eq:LowerBound} gives
%
\begin{equation}
  \braket{\psi(t)} \geq \abs{\ev{\Pi}(0)},
  \label{eq:LowerBoundParity}
\end{equation}
%
where $\ev{\Pi}(0)$ is the expectation value of the state at $t=0$.
% (at $t=0$ the state is suposed to be normalized to 1) \cite{Muga2004}
%
%
%
\section{Generic symmetries}
%
%
We postulate that both (\ref{usu}) and (\ref{unu}), for $A$ unitary or antiunitary,  are symmetries of the Hamiltonian.
A superoperator framework helps to understand why (\ref{unu}) also represents a symmetry.
Let us define the superoperators ${\cal L}_A(\cdot) \equiv A^\dagger(\cdot) A$, ${\cal L}_\dagger (\cdot)\equiv (\cdot)^\dagger$ and ${\cal L}_{A,\dagger}(\cdot)\equiv {\cal L}_A\left({\cal L}_\dagger (\cdot)\right) = {\cal L}_\dagger\left({\cal L}_A (\cdot)\right) $. For  linear operators $B$ and a complex number $a$ they satisfy
%
%
\beqa
{\cal L}_A (a B)&=& a A^\dagger  B A,\;\;\,  A\, {\rm unitary},
\\
{\cal L}_A (a B)&=& a^* A^\dagger  B A,\;  A\, {\rm antiunitary},
\\
{\cal L}_{\dagger} (a B)&=& a^*B^\dagger,
\\
{\cal L}_{A,\dagger} (a B)&=& {\cal L}_{\dagger} {\cal L}_A  (aB)=    {\cal L}_A  {\cal L}_{\dagger} (aB) = a^* A^\dagger B^\dagger A,\;
A\, {\rm{unitary}},
\\
{\cal L}_{A,\dagger} (a B)&=& {\cal L}_{\dagger} {\cal L}_A  (aB)=    {\cal L}_A  {\cal L}_{\dagger} (aB) = a A^\dagger B^\dagger A,\;\;\,
A\, {\rm{antiunitary}}.
\eeqa
%
As the product of two antilinear operators is a linear operator,
the resulting operators (on the right hand sides) are linear in all cases,
independently of the linearity or antilinearity of $A$. This should not be confused with the linearity or antilinearity
of the superoperators ${\cal L}$ that may be checked by the invariance (for a linear superoperator)
or complex conjugation (for an antilinear
superoperator) of the constant $a$.
%The first superoperator is linear and the second and third are antilinear.
Using the scalar product for linear operators $F$ and $G$,
%
\beq
\la\la F,G\ra\ra=Tr (F^\dagger G),
\eeq
%
we find the adjoints,
%
%
\beqa
{\cal L}_A^\dagger(\cdot)&=& {\cal L}_{A^\dagger}(\cdot)\equiv A(\cdot)A^\dagger,
\\
{\cal L}_{\dagger}^\dagger(\cdot) &=&{\cal L}_{\dagger}(\cdot),
\\
{\cal L}_{A,\dagger}^\dagger (\cdot) &=& {\cal L}_{A^\dagger, \dagger}(\cdot)\,,
\eeqa
%
where $\la\la F,{\cal L}^\dagger G\ra\ra=\la\la G,{\cal L} F\ra\ra^*$ for ${\cal L}$ linear
and $\la\la F,{\cal L}^\dagger G\ra\ra=\la\la G,{\cal L} F\ra\ra$ for ${\cal L}$ antilinear.

All the above transformations are unitary or antiunitary (in a superoperator sense),
${\cal L}^\dagger= {\cal L}^{-1}$,
and they keep ``transition probabilities'' among
two states, most generally represented by density operators
$\rho_1$ and $\rho_2$, invariant, namely
%
\beq
\la\la \rho_1,\rho_2\ra\ra=\la\la {\cal L}\rho_1,  {\cal L} \rho_2\ra\ra.
\label{Wigner}
\eeq
%
Due to the Hermicity of the density operators, $\la\la \rho_1,\rho_2\ra\ra$ is a real number
(both for unitary or antinuitary ${\cal{L}}$). This result is reminiscent of Wigner's theorem, originally formulated
for pure states \cite{Wigner1959},
but considering a more general set of states and transformations.
%
%to density operators: transition-probability preserving transformations are represented by unitary or antiunitary superoperators.
%(To prove (\ref{Wigner}) for a linear ${\cal L}$ note that

We conclude that all the above ${\cal L}$ superoperators may represent symmetry transformations, and in particular
Hamiltonian symmetries if they leave the Hamiltonian invariant, namely,  if
${\cal L} H= H$. The following section provides specific examples for the set of symmetry transformations
that may leave Hamiltonians for a particle in one dimension invariant,
making use of transposition, complex conjugation, and inversion of
coordinates or momenta.

As for the connection between symmetries and conservation laws, the results of the previous sections apply.
It is possible
to find quantities that on calculation remain invariant, but they are not necessarily physically significant.

%
%
%
%
\section{Example of physical relevance of the relations $AH=HA$ or $AH=H^\dagger A$ as symmetries}
%
%
%
%
In this section we exemplify the above general formulation of Hamiltonian symmetries
for Hamiltonians
of the form $H_0+V$ corresponding to a  spinless particle of mass $m$ moving in one dimension,
where $H_0=P^2/(2m)$
is the kinetic energy, $P$ is the  momentum operator, and $V$ is a generic potential that may be
non-Hermitian and non-local (non-local means that matrix elements in coordinate representation,
$\la x|V|y\ra$, may be nonzero  for
$x\ne y$. Non-locality is as common as non-Hermiticity, in the sense that Feschbach's projection framework
typically  provides non-local effective Hamiltonians for the subsystems).
We assume that $H$ is diagonalizable, possibly with discrete and continuum parts.
By inspection of  Table \ref{tableSymmetries}, one finds a set of possible Hamiltonian symmetries described
by the eight relations of the second column. They imply  the invariance of the Hamiltonian with respect to the
transformations represented by the superoperators in the third column.
In coordinate or momentum representation, see the last two columns,
each symmetry amounts to the invariance of the potential matrix elements with respect to some combination
of transposition, complex conjugation and inversion of coordinates or momenta.
($H_0$ is invariant with respect to the eight transformations.)

Table \ref{tableSymmetries} demonstrates that the eight transformations are complete when
making only use of transposition, complex conjugation, inversion of the coordinates (or momenta),
and their combinations.
The eight superoperators form the elementary abelian group of order eight \cite{Rose2009},
with a minimal set of three generators
${\cal L}_\dagger, {\cal L}_\Pi, {\cal L}_\Theta$, from which all elements may be formed by multiplication,
i.e., successive
application. ($\Theta$ is the antilinear (antiunitary) time-reversal operator acting as
$\Theta a |x\ra=a^*|x\ra$ in coordinate representation, and as $\Theta a |p\ra=a^*|-p\ra$ in momentum
representation.)
The eight superoperators may also be found from the generating set $\{{\cal L}_A\}, {\cal L}_\dagger$,
where  $A$ is one of the elements of Klein's 4-group
$\{1, \Theta,\Pi,\Pi\Theta\}$.  These four operators commute. Moreover they are Hermitian and  equal to their own inverses.
The superoperators in the third column may be classified as antiunitary (symmetries II, IV, V, and VII)
and unitary (symmetries I, III, VI, and VIII).


In \cite{Ruschhaupt2017} these symmetries are exploited to find relations among matrix elements of the scattering operators and selections rules that allow or disallow certain asymmetries
in the reflection or transmission amplitudes for right and left incidence, a relevant information to implement
microscopic asymmetrical devices such as
diodes or rectifiers in quantum circuits \cite{Ruschhaupt2004}.

To end this section we note the use of Equation (\ref{unu}) with differential operators different from
the Klein's four-group set to generate non-PT local potentials with real spectra
\cite{Nixon2016}.





\begin{table}[h]
%\label{tableSymmetries}
\caption{Symmetries of the potential dependent on the commutativity or pseudo-hermiticity of $H=H_0+V$ with the elements of
Klein's 4-group  $\{1,\Pi,\Theta,\Pi\Theta\}$ (second column). Each symmetry has a roman number code in the first column.
Each symmetry may also be regarded as the invariance of the potential with respect to the transformations represented by superoperators
${\cal L}$ in the third column. The kinetic part $H_0$ is invariant in all cases.
In coordinate (fourth column) or momentum representation (last column), the eight transformations
correspond to all possible combinations of transposition, complex conjugation, and inversion. \label{tableSymmetries}}
\centering
\begin{tabular}{ccccc}
\toprule
\textbf{Code} & \textbf{Symmetry}&  \textbf{Superoperator} &\boldmath{$\la x|V|y\ra=$} &    \boldmath{$\la p|V|p'\ra=$}
\\
\midrule
I & $1H=H1$ &  ${ {\cal L}_1}$  &  $\la x|V|y\ra$ &  $\la p|V|p'\ra$
\\
II & $1H=H^\dagger 1$ &  ${\cal L}_\dagger$ & $\la y|V|x\ra^*$ &   $\la p'|V|p\ra^*$
\\
III & $\Pi H=H\Pi$ & ${\cal L}_\Pi$  & $\la -x|V|-y\ra$ &   $\la -p|V|-p'\ra$
\\
IV & $\Pi H=H^\dagger \Pi$ & ${\cal L}_{\Pi\dagger}$  & $\la -y|V|-x\ra^*$ &   $\la -p'|V|-p\ra^*$
\\
V & $\Theta H=H\Theta$ &  ${\cal L}_\Theta$ & $\la x|V|y\ra^*$ &   $\la -p|V|-p'\ra^*$
\\
VI & $\Theta H=H^\dagger\Theta$ &  ${\cal L}_{\Theta\dagger}$ & $\la y|V|x\ra$  & $\la -p'|V|-p\ra$
\\
VII & $\Theta\Pi H=H\Theta \Pi$ &  ${\cal L}_{\Pi\Theta}$ & $\la -x|V|-y\ra^*$ &  $\la p|V|p'\ra^*$
\\
VIII& $\Theta\Pi H=H^\dagger \Pi\Theta$ & ${\cal L}_{\Pi\Theta\dagger}$ & $\la -y|V|-x\ra$ &  $\la p'|V|p\ra$
\\
\bottomrule
\end{tabular}
%\end{tabular}
%}
\end{table}

\section{Discussion}
%
%
The relations between invariance and symmetry are often emphasized,
but for non-Hermitian Hamiltonians, which occur naturally as effective interactions, they become more complex
and subtler than for Hermitian Hamiltonians.
We have discussed these relations for time-independent Hamiltonians.

Time-dependent non-Hermitian
Hamiltonians require a specific analysis and will be treated in more detail
elsewhere. (In particular
for a time-dependent $H$, exceptional points, not addressed here,  may be crossed.)
We briefly advance here some important differences with the time-independent Hamiltonians.
1969, Lewis and Riesenfeld \cite{Lewis1969} showed that the motion of a system subjected to time-varying
forces admits a simple decomposition into elementary, independent motions characterized by constant
values of some quantities (eigenvalues of the invariant).
In other words, the dynamics is best
understood, and is most economically described, in terms of invariants even for time-dependent Hamiltonians.
In fact the powerful link between
forces and invariants can be used in reverse order to inverse engineer from the invariant associated with
some desired dynamics the necessary driving forces.

Invariants for Hermitian, time-dependent Hamiltonians obey the
invariance condition
%
\beq
\frac{\partial I(t)}{\partial t}-\frac{1}{i\hbar}[H(t),I(t)]=0,
\label{gene}
\eeq
%
so that $\frac{d}{dt}\la {\psi}(t)|I(t)|\psi(t)\ra=0$ for states $\psi(t)$ that evolve  with $H(t)$
(we assume that the invariant is linear).
In general the  operator $I(t)$ may depend on time and the invariant quantity is the expectation
value $\la \psi(t)|I(t)|\psi(t)\ra$.
In this context a  Hamiltonian symmetry, defined
by the commutativity of $A$ with $H$ as in (\ref{usu}) does not lead necessarily to a  conservation
law, unless $A$ is time independent.
%

Invariant operators  are useful to express the dynamics of the state $\psi(t)$ in terms of
superpositions of their  eigenvectors with constant coefficients \cite{Lewis1969}; also to do inverse
engineering, as in shortcuts to adiabaticity,
so as to find $H(t)$ from the desired dynamics \cite{Ibanez2011,Torrontegui2013}.

$I(t)$ may be formally defined by (\ref{gene}) for non-Hermitian Hamiltonians too, and its
roles to provide a basis for useful state decompositions and inverse engineering are still
applicable \cite{Ibanez2011}.
Note however that in this context $I(t)$ is not invariant in an ordinary sense, but rather
%
\beq
\frac{d}{dt}\la \widehat{\psi}(t)|I(t)|\psi(t)\ra=0.
\eeq
%
The alternative option, yet to be explored for inverse engineering the Hamiltonian,
is to consider (linear) operators $I'(t)$ such that
%
\beq
\frac{\partial I'(t)}{\partial t}-\frac{1}{i\hbar}[H(t)^\dagger I'(t)-I'(t)H(t)]=0,
\label{gene2}
\eeq
%
and thus  $\frac{d}{dt}\la \psi(t)|I'(t)|\psi(t)\ra=0$.

As an outlook for further work, it would be interesting to extend the present formalism
to field theories \cite{Alexandre2017},
and to other generalized symmetries
where  intertwining operators $A$ relate $H$ to  operators different from $H^\dagger$
\cite{Correa2015,Guilarte2017,Deak2012}, for example $-H$ \cite{Chen2017}.
Klein's group may also be augmented by considering further symmetries due to internal states \cite{Kartashov(2014)}.
Applications in optical devices   \cite{Longhi2017a} and quantum circuits \cite{Ruschhaupt2017} may be expected.
