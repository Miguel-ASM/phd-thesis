%!TEX root = ../Thesis.tex
%Chapter 6

\chapter{Rectification in a minimal model}
\label{Chapter6}
\lhead{Chapter 6. \emph{Rectification in a minimal model}} % Write in your own chapter title to set the page header
%
We study heat rectification in a minimalistic model composed of two unequal atoms subjected to linear forces and in contact with effective Langevin baths
induced by Doppler lasers. Analytic expressions of the heat currents in the steady state are spelled out. Asymmetric heat transport is found in this linear system if both the bath temperatures and the temperature dependent bath-system couplings are exchanged. The model can be realized with two ions  in  either common or individual traps. This physical setting allows for a natural temperature
dependence of the coupling to the baths.
We also explore the parameter space of the model to optimize asymmetric heat current and find
conditions for maximal rectification. High rectification corresponds to a good match of the power spectra of the ions for forward temperature bias and
mismatch  for reverse bias, which may be understood by the behavior of dissipative normal modes.
%
\newpage
%
\section{Introduction \label{sec:Introduction}}
%
Heat rectification is the physical phenomenon, analogous to electrical current rectification in diodes, in which heat current through a device or medium (the thermal diode or rectifier) is not symmetric with respect to the exchange of the bath temperatures at the boundaries. It was  first observed in 1936 by Starr in a junction between copper and
cuprous oxide \cite{Starr1936}. The theoretical work started much later  using as rectifiers simple anharmonic chain models
with different segments \cite{Terraneo2002,Li2004}.
These papers sparked much research that
continues to this day. The field remains very active driven by potential applications in fundamental science and technology for thermal management and signal processing, and also because
none of the proposals so far appears to be efficient enough for
practical purposes, i.e., highly conducting in one direction and insulating in the other one. The studies have also branched into different subfields
and systems (e.g. quantum or  classical \cite{Pereira2019}, and for macroscopic, mesoscopic,  or microscopic devices),
that need  specific treatments. A full account of the
developments and results is out of the scope of this introduction, but  several good reviews  are available for a broad perspective  \cite{Roberts2011,Li2012,Pereira2019,Ma2019}. We merely mention in passing,  important  progress on
nanostructures \cite{Li2012,Ma2019},
macroscopic devices \cite{Roberts2011},  or time dependent drivings \cite{Li2012,Riera-Campeny2019}.
Instead we shall focus  on some aspects more closely connected to the present work that help to  motivate it and put it in context.
The rectification in the first  models was explained by the different temperature dependences of the phonon bands (power spectra) of the  segments  \cite{Terraneo2002,Li2004}. A match or mismatch of the spectra of neighboring parts implies corresponding good or bad conduction so the
sign of the temperature bias may affect the conduction and lead to rectification when the spectra of the parts are affected differently
by the bias reversal. Interaction potential anharmonicities (i.e. non-linear forces) imply dependences of the spectra on temperature and thus have been regarded recurrently as an essential element for rectification   \cite{Li2012,Li2008,Hu2006,Zeng2008,Segal2005,Segal2005b,Katz2016,Benenti2016}.
However Pereira pointed out \cite{Pereira2017} that anharmonicity is not a necessary condition for rectification.
Rectification also occurs in simple (minimalistic) harmonic models
that incorporate some structural asymmetry and temperature-dependence of the model parameters. This dependence may indeed result from
an underlying, more intricate  anharmonic  system by linearization of the stochastic dynamics \cite{Pereira2017,Pereira2019},
or it may have a different origin \cite{Simon2019}.
In general minimalistic models, harmonic or otherwise,  provide insight and serve to guide further work towards effective rectifiers.

To look for higher rectification factors, the use of graded  materials \cite{Yang2007}
and long range interactions (LRI) was  put forward \cite{Pereira2013,Chen2015}.
It was noted recently that LRI naturally occurs as a result of the Coulomb interaction in
a  chain of cold ions in Paul traps \cite{Simon2019}, and that this system may serve to bridge the gap between
simple models and experimental realizations. In Ref. \cite{Simon2019} the gradation was incorporated
by ramping the frequencies of individual traps. Moreover the linear regime (when the potentials are well approximated harmonically)
is realistic for trapped ions, and  shows rectification because of the temperature dependence of  the coupling to the effective bath
induced by Doppler cooling lasers.  Trapped ions constitute a well-developed and tested  architecture for fundamental research,
quantum information processing and
quantum technologies such as detectors or metrology. This architecture is  in principle scalable in driven ion circuits, see e.g.  \cite{Bruzewicz2019}.
Controllable heat rectification in this context
would be a useful asset for energy management.

%Several of the trends and ideas mentioned so far motivate this article. In particular
In this article we put forward, along this line of trapped ion physics, an even simpler, minimalistic model
for rectification
implemented by two neighboring atoms of different mass interacting harmonically
and in contact with thermal baths with temperature dependent couplings (LRI does not play any role in the two-ion configuration but it
would affect the dynamics of longer chains).
The model admits a  natural realization in terms of two trapped ions subjected to  Doppler cooling lasers, which provide the
necessary temperature dependence of the coupling parameters. Apart from the possibility of a physical realization, another interesting
feature is the analytical treatment, which facilitates greatly the exploration in parameter  space
to  identify regimes of maximal rectification.
The explicit solution of the stationary regime also provides tools for a better understanding of the physics and enhanced control.
For example the match or mismatch of the spectra of the two masses for forward and reverse bias configurations, which will be made evident
for the parameters with maximal
rectification, may be analyzed in terms of dissipative normal modes characterized by complex eigenvalues.


The model may be compared and related to other simple models. The localization of the
structural asymmetry in a  small spatial region, by a ``defect'', ``impurity'', or asymmetrical molecule has been proposed e.g. in
several anharmonic models \cite{Segal2005b,Pons2017,Alexander2020}.
Segal and Nitzan proposed models with some similarities to ours \cite{Segal2005,Segal2005b}, specifically an anharmonic chain
with different couplings to both baths. They also worked out  quantum models \cite{Segal2005,Segal2005b} in terms of an N-level
system asymmetrically coupled to the baths. Both types of models have ``harmonic limits'', which in the chain is reached by making the potentials
harmonic, and in the quantum one by taking $N$ to infinity assuming equispaced levels.
The asymmetrical couplings however, were not interchanged when reversing the temperature bias
(in these models that interchange would have suppressed the asymmetry because the forward bias configuration becomes a mirror image of the reverse bias one), so that
the harmonic limit did not give any rectification.










%In 2002 a paper by Terraneo \textit{et al.} \cite{Terraneo2002} demonstrated heat rectification numerically for a chain of nonlinear oscillators in contact with two thermal baths at different temperatures. Since then, there has been a growing interest in heat rectification  \cite{Roberts2011,Ye2017,Wang2008,Wang2007,Casati2006,Joulain2016,Chang2006,Kobayashi2009,Leitner2013,Elzouka2017,Pons2017,Alexander2020}, and
%Much effort has been devoted to  understand the underlying physical mechanism responsible for  rectification \cite{Pereira2019}.
%
%However, a work by Pereira \textit{et al.} \cite{Pereira2017} showed that rectification can also be found in effective harmonic systems if two requirements are met: some kind of structural asymmetry, and features that depend on the temperature so they change as the baths are inverted. Indeed,  in this article we demonstrate rectification in a minimalistic model of two harmonic oscillators where the coupling to the baths depends on the temperature.
%
%This will be justified with a particular physical set up with trapped ions and lasers.

%

The article is organized as follows: In Section \ref{sec:Physical_Model}
we describe the physical model and its dynamical equations. In Section \ref{sec:covMatrix} we introduce the  covariance matrix and  derive the equation  that it satisfies in the steady state. In Section \ref{sec:solutions} we solve the covariance matrix equation and find analytical expressions for the steady-state temperatures of the masses and heat currents. In Section \ref{sec:TrappedIonSetUp} we relate the parameters of the model to those of Doppler cooled trapped ions. In Section \ref{sec:lookingForR} we look for configurations with high rectification. We also study the power spectra of the oscillators, which confirm the match or mismatch pattern for rectification. Finally, in Section \ref{sec:Conclusions} we summarize the results and present the conclusions.

\begin{figure}
  \center
  \includegraphics[width=1.1\linewidth]{Figures/model_diagram.pdf}
  \caption{Diagram of the model described in Section \ref{sec:Physical_Model}. Two masses are coupled to each other through a spring constant $k$. Each mass is harmonically trapped and connected to a bath characterized by its temperature $T_i$ and its friction coefficient $\gamma_i$. }
  \label{fig:model_diagram}
\end{figure}
%
%
%
%
\section{Physical Model \label{sec:Physical_Model}}
%
%
%
%
The physical model consists of two masses $m_1$ and $m_2$ coupled to each other by a harmonic interaction with spring constant $k$ and natural length $x_e$. The masses $m_1$ and $m_2$ are confined by  harmonic potentials centered at $x_L$, $x_R$ with spring constants $k_L$, $k_R$  respectively (see Fig. \ref{fig:model_diagram}). The Hamiltonian describing this model is
%
\begin{equation}
  H = \frac{p_1^2}{2m_1} + \frac{p_2^2}{2m_2} + V(x_1,x_2),
  \label{eq:HamiltonianOriginalCordinates}
\end{equation}
%
with $V(x_1,x_2)=\frac{k}{2}\left( x_1 - x_2 - x_e \right)^2 + \frac{k_L}{2}\left( x_1 - x_L \right)^2 + \frac{k_R}{2}\left( x_2 - x_R \right)^2$,  where $\{x_i,p_i\}_{i=1,2}$ are the position and momentum of each mass. Switching from the original coordinates $x_i$ to displacements with respect to the equilibrium positions of the system $q_i = x_i - x_i^{eq}$, where $x_i^{eq}$ are the solutions to $\partial_{x_i}V(x_1,x_2)=0$, the Hamiltonian can be written as
%
\begin{align}
  H &= \frac{p_1^2}{2m_1} + \frac{p_2^2}{2m_2} + \frac{k+k_L}{2}q_1^2\nonumber\\ &+ \frac{k+k_R}{2}q_2^2 - k q_1 q_2 + V(x_1^{eq},x_2^{eq}).
  \label{eq:Hamiltonian}
\end{align}
%
Dropping the constant term, this has the form of  the Hamiltonian of a system around a stable equilibrium point,
%
\begin{equation}
  H = \frac{1}{2} \overrightarrow{p}^\mathsf{T}\mathbb{M}^{-1}\overrightarrow{p} + \frac{1}{2} \overrightarrow{q}^\mathsf{T}\mathbb{K}\overrightarrow{q},
\label{generic}
\end{equation}
%
where $\overrightarrow{q} = \left(q_1,q_2\right)^\mathsf{T}$, $\overrightarrow{p} = \left(p_1,p_2\right)^\mathsf{T}$, $\mathbb{M} = diag(m_1,m_2)$ is the mass matrix of the system and $\mathbb{K}$ is the Hessian matrix of the potential at the equilibrium point, i.e., $\mathbb{K}_{ij} = \partial^2_{x_i,x_j}V(\overrightarrow{x})\Big|_{\overrightarrow{x} = \overrightarrow{x}^{eq}}$. In this model  $\mathbb{K}_{11} = k + k_L$, $\mathbb{K}_{22} = k + k_R$ and $\mathbb{K}_{12} = \mathbb{K}_{21} = -k$.
We shall see later that
the generic form (\ref{generic}) can be adapted to different physical settings, in particular to
two ions in individual traps, or to two ions in a common trap.

The  masses are in contact with Langevin baths, which will be denoted as $L$ (for left) and $R$ (for right), at temperatures $T_{L}$ and $T_R$ for  the mass $m_1$ and $m_2$ respectively (see Fig. \ref{fig:model_diagram}). The equations of motion of the system, taking into account the Hamiltonian and the Langevin baths are
%
\begin{align}
  \dot{q}_1 &= \frac{p_1}{m_1},\;\;\;\;
  \dot{q}_2 = \frac{p_2}{m_2},\nonumber
  \\
  \dot{p}_1 &= -(k+k_L)q_1 + k q_2 -\frac{\gamma_L}{m_1} p_1 + \xi_L(t),\nonumber
  \\
  \dot{p}_2 &= -(k+k_R)q_2 + k q_1 -\frac{\gamma_R}{m_2} p_2 + \xi_R(t),
\end{align}
%
where $\gamma_L$, $\gamma_R$ are the friction coefficients of the baths and $\xi_L(t)$, $\xi_R(t)$ are Gaussian white-noise-like forces. The Gaussian forces have zero mean over noise realizations ($\expval{ \xi_L(t) } = \expval{ \xi_R(t) } = 0 $) and satisfy the correlations $\expval{ \xi_L(t)\xi_R(t') } = 0$, $\expval{ \xi_L(t)\xi_L(t') } = 2D_L\delta(t-t')$, $\expval{ \xi_R(t)\xi_R(t') } = 2D_R\delta(t-t')$. $D_L$ and $D_R$ are the diffusion coefficients, which satisfy the fluctuation-dissipation theorem, $D_L = \gamma_L k_B T_L$, $D_R =\gamma_R k_B T_R$, where  $k_B$ is the Boltzmann constant.

It is useful to define the phase-space vector $\overrightarrow{r}(t) = \left( \overrightarrow{q}, \mathbb{M}^{-1}\overrightarrow{p} \right)^\mathsf{T}$ (note that $\overrightarrow{v} = \mathbb{M}^{-1}\overrightarrow{p}$ is just the velocity vector).  The equations of motion are
%
\begin{equation}
  \dot{\overrightarrow{r}}(t) = \mathbb{A} \, \overrightarrow{r}(t) + \mathbb{L}\overrightarrow{\xi}(t),
  \label{eq:vectorEqOfMotion}
\end{equation}
%
with
%
\begin{align}
  \mathbb{A} &=
  \left(
  \begin{array}{cc}
    \mathbb{0}_{2 \times 2} & \mathbb{1}_{2 \times 2}
    \\
    -\mathbb{M}^{-1}\mathbb{K} & -\mathbb{M}^{-1}\bbGamma
  \end{array}
  \right),
  \nonumber
  \\
  \mathbb{L} &=
  \left(
  \begin{array}{c}
    \mathbb{0}_{2\times 2} \\ \mathbb{M}^{-1}
  \end{array}
  \right),
  \label{eq:Dynamical_Matrix}
\end{align}
%
and $\overrightarrow{\xi}(t) = \left( \xi_L(t),\xi_R(t) \right)^\mathsf{T}$, $\bbGamma = diag(\gamma_L,\gamma_R)$. $\mathbb{0}_{n\times n}$ and $\mathbb{1}_{n\times n}$ are the $n$-dimensional squared 0 matrix and identity matrix respectively. With the vector notation the correlation of the white-noise forces can be written as
%
\begin{equation}
  \expval{\overrightarrow{\xi}(t)\overrightarrow{\xi}(t')^\mathsf{T}} = 2 \mathbb{D}\delta(t-t'),
\end{equation}
%
where $\mathbb{D} = diag(D_L,D_R)$.
%
%
%
%
%
%
\section{Covariance matrix in the steady state\label{sec:covMatrix}}
%
%
%
%
%
%
We define the covariance matrix of the system as
%
\begin{equation}
\mathbb{C}(t) = \expval{\overrightarrow{r}(t)\overrightarrow{r}(t)^\mathsf{T}}.
\end{equation}
%
This matrix is important because the heat transport properties can be extracted from it. In particular, the kinetic temperatures of the masses, $T_1(t)$ and  $T_2(t)$, are
%
\begin{align}
  T_1(t) &= \frac{\expval{ p_1^2(t)}}{m_1 k_B} = \frac{m_1 C_{3,3}(t)}{k_B},
  \nonumber\\
   T_2(t) &= \frac{\expval{ p_2^2(t)}}{m_2 k_B} = \frac{m_2 C_{4,4}(t)}{k_B}.
  \label{eq:Temperature_definition}
\end{align}
%
One approach to find the covariance matrix is to solve Eq. \eqref{eq:vectorEqOfMotion}. However, this requires solving the equations explicitly or simulate them numerically many times to find the covariance matrix for the ensemble of simulated stochastic trajectories. Instead, we proceed by looking for an ordinary differential equation that gives the evolution of the covariance matrix as described in \cite{Sarkka2019,Rieder1967,Casher1971}. Differentiating $\mathbb{C}(t)$ with respect to time and using Eq. \eqref{eq:vectorEqOfMotion} we get
%
\begin{align}
  \frac{d}{dt}\mathbb{C}(t) &=
  \mathbb{A}\mathbb{C}(t) +
  \mathbb{C}(t) \mathbb{A}^\mathsf{T}
  \nonumber\\
  &+
  \mathbb{L}\expval{ \overrightarrow{\xi}(t)\overrightarrow{r}(t)^\mathsf{T}}
  %\nonumber\\
 % &+
+
  \expval{ \overrightarrow{r}(t)\overrightarrow{\xi}(t)^\mathsf{T}}\mathbb{L}^\mathsf{T}.
  \label{eq:evolutionOfCovariances}
\end{align}
%
The solution of Eq. \eqref{eq:evolutionOfCovariances} allows us to find the local temperatures of the masses as a function of the bath temperatures (Eq. \eqref{eq:Temperature_definition}) at all times. In particular, we are interested in the covariance matrix in the steady state, i.e., for $t\to \infty$. %According to the Novikov Theorem \cite{Novikov1965} we can write down the covariance matrix in the steady state without having to integrate the differential equation. We now show how to get the steady-state covariance matrix.

In the steady state, the covariance matrix is constant ($\frac{d}{dt}\mathbb{C}(t)=0$), therefore it satisfies
%
\begin{align}
  &\mathbb{A}\mathbb{C}^{s.s.} +
  \mathbb{C}^{s.s.} \mathbb{A}^\mathsf{T}=
  \nonumber\\
  &- \mathbb{L}\expval{ \overrightarrow{\xi}\overrightarrow{r}^\mathsf{T}}^{s.s.}
  - \expval{ \overrightarrow{r}\overrightarrow{\xi}^\mathsf{T}}^{s.s.}\mathbb{L}^\mathsf{T},
  \label{eq:SteadyStateEquationToyModel_raw}
\end{align}
%
with $\small\{\cdot\small\}^{s.s.}\equiv \lim\limits_{t \to \infty} \small\{\cdot\small\}(t)$. Equation \eqref{eq:SteadyStateEquationToyModel_raw} is an algebraic equation whose solution is the steady-state covariance matrix $\mathbb{C}^{s.s.}$. However, the two terms $\expval{ \overrightarrow{\xi}\overrightarrow{r}^\mathsf{T}}^{s.s.}$ and  $\expval{\overrightarrow{r}\overrightarrow{\xi}^\mathsf{T}}^{s.s.}$ need to be calculated before working out the solution.
%One approach to calculate $\expval{\overrightarrow{\xi}\overrightarrow{r}^\mathsf{T}}^{s.s.}$ would be to solve Eq. , but this is exactly what we are trying to avoid. It is here when
%The Novikov theorem comes useful, since it lets us compute $\expval{ \overrightarrow{\xi}\overrightarrow{r}^\mathsf{T}}^{s.s.}$ without having to integrate the equations of motion.
Using Novikov's theorem and the $\delta$-correlation of the noises, we find the $ij$-th component of $\expval{ \overrightarrow{\xi}(t)\overrightarrow{r}(t)^\mathsf{T}}$ without solving Eq. \eqref{eq:vectorEqOfMotion},
%
\begin{align}
  \expval{ \xi_i(t) r_j(t) } &= \sum_{k=1}^2 \int_0^t d\tau\,\expval{ \xi_i(t) \xi_k(\tau)}
  \,
  \expval{ \frac{\delta r_j(t)}{\delta \xi_k(\tau)} }\nonumber
  \\
  &= \sum_{k=1}^2 \mathbb{D}_{ik}
  \,
  \lim_{\tau \to t^{-}}
  \,
  \expval{ \frac{\delta r_j(t)}{\delta \xi_k(\tau)} },
\end{align}
%
where $\lim_{\tau \to t^{-}}$ is the limit when $\tau$ goes to $t$ from below. Evaluation of the functional derivative ${\delta r_j(t)}/{\delta \xi_k(\tau)}$ for the $\tau \to t^{-}$ limit gives
%
\begin{equation}
  \expval{ \overrightarrow{\xi}(t)\overrightarrow{r}(t)^\mathsf{T}} = \mathbb{D}\mathbb{L}^\mathsf{T}.
\end{equation}
%
Now, the algebraic equation that gives the steady-state covariance matrix becomes
%
\begin{equation}
  \mathbb{A}\mathbb{C}^{s.s.} +
  \mathbb{C}^{s.s.}\mathbb{A}^\mathsf{T}
  =
  -\mathbb{B},
  \label{eq:SteadyStateEquationToyModel}
\end{equation}
%
with $\mathbb{B} = 2 \mathbb{L}\mathbb{D}\mathbb{L}^\mathsf{T}$. By definition, the covariance matrix is  symmetric, but there are also  additional restrictions imposed by the equations of motion and the steady-state condition, which reduce the dimensionality of the problem of solving Eq. \eqref{eq:SteadyStateEquationToyModel} \cite{Simon2019}. Since ${d \expval{ q_i q_j }}/{dt} = 0$ in the steady state, we have
%
\begin{align}
  \expval{ p_1 q_1}^{s.s.} &= \expval{ p_2 q_2}^{s.s.} = 0,\nonumber\\
  \frac{\expval{ p_1 q_2}^{s.s.}}{m_1}&=-\frac{\expval{ q_1 p_2}^{s.s.}}{m_2}.
  \label{eq:ExtraConditionSteadyState}
\end{align}
%
Taking \eqref{eq:ExtraConditionSteadyState} into account, the steady-state covariance matrix takes the form
%
\begin{equation}
  \begin{split}
    \mathbb{C}^{s.s.} =
    \left(
    \begin{array}{cccc}
      \expval{ q_1^2}^{s.s.}  & \expval{ q_1 q_2}^{s.s.}  & 0 & \frac{\expval{ p_2 q_1}^{s.s.} }{m_2} \\
      \expval{ q_1 q_2}^{s.s.}  & \expval{ q_2^2}^{s.s.}  & -\frac{\expval{ p_2 q_1}^{s.s.} }{m_2} & 0 \\
      0 & -\frac{\expval{ p_2 q_1}^{s.s.} }{m_2} & \frac{\expval{ p_1^2}^{s.s.} }{m_1^2} & \frac{\expval{ p_1 p_2}^{s.s.} }{m_1 m_2} \\
      \frac{\expval{ p_2 q_1}^{s.s.} }{m_2} & 0 & \frac{\expval{ p_1 p_2}^{s.s.} }{m_1 m_2} & \frac{\expval{ p_2^2}^{s.s.} }{m_2^2} \\
      \end{array}
      \right)
    \end{split}
    \label{eq:steadyStateCovarianceMatrix}\,.
\end{equation}
%
The explicit set of equations for the components of $\mathbb{C}^{s.s}$ can be found in Appendix \ref{Appendix:SteadyStateEquations}.
%
%
%
%
%
\section{Solutions\label{sec:solutions}}
%
%
%
%

%
In this section we use the solution to Eq. \eqref{eq:SteadyStateEquationToyModel} to write down the temperatures and currents in the steady state. We use {\it Mathematica} to find analytic expressions for the temperatures,
%
\begin{align}
  T_1 &= \frac{T_L \mathcal{P}_{1,L}(k) + T_R \mathcal{P}_{1,R}(k)}{\mathcal{D}(k)},\nonumber
  %
  \\
  %
  T_2 &= \frac{T_L \mathcal{P}_{2,L}(k) + T_R \mathcal{P}_{2,R}(k)}{\mathcal{D}(k)},
  %
  \label{eq:ModelBTemperatures}
\end{align}
%
where $\mathcal{D}(k) =  \sum\limits_{n=0}^2 \mathcal{D}_n k^n$ and $\mathcal{P}_{i,(L/R)}(k) = \sum\limits_{n=0}^2 a_{i,n,(L/R)} k^n$ are polynomials in the coupling constant $k$ with coefficients
%

%\begin{widetext}
  \begin{align}
    \mathcal{D}_0 &= a_{1,0,L} = a_{2,0,R}
    \nonumber\\
    & = \gamma _L \gamma _R\! \left[h^{(1)}\! \left(\gamma_L k_R +\gamma_R k_L \right)+\left(m_1 k_R-m_2 k_L\right)^2\right]\!,
    \nonumber\\
    %
    \mathcal{D}_1 &= a_{1,1,L} = a_{2,1,R}
    \nonumber\\
    &= \gamma _L \gamma _R\! \left[h^{(0)} h^{(1)}\!+2 \left(m_1-m_2\right) \left(m_1 k_R-m_2 k_L\right)\right]\!,
    %
    \nonumber\\
    %
    \mathcal{D}_2 &= h^{(0)} h^{(2)},\nonumber
    %
    \\
    %
    a_{1,2,L} &= \gamma _L \left(m_2 h^{(1)} + \gamma_R (m_1 - m_2)^2 \right),\nonumber
    %
    \\
    %
    a_{1,2,R} &= h^{(1)} m_1 \gamma_R,\nonumber
    %
    \\
    %
    a_{2,2,L} &= h^{(1)} m_2 \gamma_L,\nonumber
    %
    \\
    %
    a_{2,2,R} &= \gamma _R \left( m_1 h^{(1)} + \gamma_L (m_1-m_2)^2 \right),\nonumber
    %
    \\
    %
    a_{1,0,R} &= a_{1,1,R} = a_{2,0,L} = a_{2,1,L} = 0,
    %
    \label{eq:SolutionPolynomialCoefficients}
  \end{align}
%\end{widetext}
%
where
%
\begin{equation}
h^{(n)}\equiv \gamma_R m_1^n + \gamma_L m_2^n.
\end{equation}
%
The currents from the baths to the masses \cite{Simon2019} are
%
\begin{equation}
%  \begin{split}
    J_L = k_B \frac{\gamma_L}{m_1} \left( T_L - T_1 \right),\;\;\;
    J_R = k_B \frac{\gamma_R}{m_2} \left( T_R - T_2 \right),
    \label{eq:currents_definition}
%  \end{split}
\end{equation}
\\
%
with $T_i$ given by Eq. \eqref{eq:ModelBTemperatures}. Since, in the steady state, $J_L = -J_R$ we will use the shorthand notation $J \equiv J_L$. Substituting Eq. \eqref{eq:ModelBTemperatures} into Eq.  \eqref{eq:currents_definition} we get for the heat current
%
% \begin{equation}
%   J = k_B \frac{k^2\gamma_L \gamma_R h^{(1)}}{\mathcal{D}(k)}(T_L - T_R).
%   \label{eq:CurrentsInModelB}
% \end{equation}
%
\begin{equation}
  J = \kappa\;(T_L - T_R),
  \label{eq:CurrentsInModelB}
\end{equation}
%
where $\kappa = k_B {k^2\gamma_L \gamma_R h^{(1)}}/{\mathcal{D}(k)}$ acts as an effective thermal conductance.
%, which depends on the parameters of the system, i.e., the masses and spring constants, and also on the friction coefficients of the baths.
%From Eq. \eqref{eq:CurrentsInModelB} it could be thought that inverting the temperatures of the baths would only lead to an exchange of heat currents. However, since the thermal conductance $\kappa$ depends on the friction coefficients, the exchange of the baths implies a change in its value. Moreover, it is possible to have temperature-dependent friction coefficients, as it happens in the physical set-up of laser-cooled trapped ions described in Section \ref{sec:TrappedIonSetUp}.
%
%
%
%
\section{Relation of the Model to a trapped ion setup \label{sec:TrappedIonSetUp}}
%
%
%
%
%As we mentioned, the parameters  can be related to the elements of the Hessian matrix of a system in a stable equilibrium position.
In this section we discuss the realization of the model with  a pair of trapped ions. We consider two different setups: two ions in a collective trap, and two ions in individual traps. Later in Section \ref{sec:lookingForR} we shall focus on two ions in individual traps to illustrate the analysis of rectification.

In both setups we assume strong confinement in the radial direction, making the effective dynamics one-dimensional. We will also assume that the confinement in the axial direction is purely electrostatic, which makes the effective spring constant independent of the mass of the ions \cite{Leibfried2003}. Additionally, we will relate the temperatures and friction coefficients of the Langevin baths to those corresponding to Doppler cooling.
%
%
\subsection{Collective trap}
%
%
Consider two ions of unit charge with masses $m_1$ and $m_2$ trapped in a collective trap. Assuming strong radial confinement and purely electrostatic axial confinement, both ions feel the same harmonic oscillator potential with trapping constant $k_{trap}$ \cite{Leibfried2003}. The potential describing the system is
%
\begin{equation}
  V_{collective} = \frac{1}{2}k_{trap} \left( x_1^2 + x_2^2\right) + \frac{\mathcal{C}}{x_2-x_1},
\end{equation}
%
with $\mathcal{C}={Q^2}/({4\pi\varepsilon_0})$. The equilibrium positions for this potential are
%
\begin{equation}
  x_2^{eq} = -x_1^{eq} =
  \label{eq:equilibriumPositionsCollectiveTrap}\left(\frac{1}{2}\right)^{2/3} \left(\frac{Q^2}{4\pi\varepsilon_0 k_{trap}}\right)^{1/3}.
\end{equation}
%
Assuming small oscillations of the ions around the equilibrium positions, the Hessian matrix of the system is
%
\begin{align}
  \mathbb{K}_{1,2} &= -\frac{Q^2}{2\pi\varepsilon_0}\frac{1}{(x_2^{eq}-x_1^{eq})^3} = -k_{trap},\nonumber
  \\
  \mathbb{K}_{1,1} &= k_{trap} + \frac{Q^2}{2\pi\varepsilon_0}\frac{1}{(x_2^{eq}-x_1^{eq})^3} = 2 k_{trap},\nonumber
  \\
  \mathbb{K}_{2,2} &= k_{trap} + \frac{Q^2}{2\pi\varepsilon_0}\frac{1}{(x_2^{eq}-x_1^{eq})^3} = 2 k_{trap}.
  \label{eq:HessianOffDiagonalCollective}
\end{align}
%
Using Eq. \eqref{eq:HessianOffDiagonalCollective} we can relate the parameters of this physical setup to those of the model described in Section \ref{sec:Physical_Model},
%
\begin{equation}
  k_L = k_R = k = k_{trap}.
\end{equation}
%
%
%
\subsection{Individual on-site traps}
%
%
%
We can make the same assumptions for the axial confinement as in the previous subsection but now each of the ions is in an individual trap with spring constants $k_{trap,L}$ and $k_{trap,R}$ respectively. The potential of the system is
%
\begin{align}
    V_{individual} &= \frac{1}{2}k_{trap,L}\left(x_1 -x_L\right)^2 +\frac{1}{2}k_{trap, R}\left(x_2 -x_R\right)^2 \nonumber \\&+ \frac{\mathcal{C}}{x_2-x_1},
\end{align}
%
where $x_L$ and $x_R$ are the center positions of the on-site traps. The elements of the Hessian matrix in the equilibrium position are
%
\begin{align}
  \mathbb{K}_{1,2} &= -\frac{Q^2}{2\pi\varepsilon_0}\frac{1}{(x_2^{eq}-x_1^{eq})^3},\nonumber
  \\
  \mathbb{K}_{1,1} &= k_{trap,L} + \frac{Q^2}{2\pi\varepsilon_0}\frac{1}{(x_2^{eq}-x_1^{eq})^3},\nonumber
  \\
  \mathbb{K}_{2,2} &= k_{trap,R} + \frac{Q^2}{2\pi\varepsilon_0}\frac{1}{(x_2^{eq}-x_1^{eq})^3}.
  \label{eq:HessianOffDiagonalOnSite}
\end{align}
%
Comparing the parameters in Eq. \eqref{eq:HessianOffDiagonalOnSite} with those in the model described in Section \ref{sec:Physical_Model} we identify
\begin{align}
  k_L &= k_{trap,L},\nonumber\\
  k_R &= k_{trap,R},\nonumber\\
  k &= \frac{Q^2}{2\pi\varepsilon_0}\frac{1}{(x_2^{eq}-x_1^{eq})^3}\,.
\end{align}
%
In this case, the analytic expressions for the equilibrium positions are more complicated. We get for the distance between the equilibrium positions of the ions
%
\begin{align}
  &(x_2 - x_1)^{(eq)} = \frac{1}{3} \Delta x_{LR}\nonumber\\
  &- \frac{1}{6}\Big[ \frac{2^{2/3}\zeta}{k_{trap,L} k_{trap,R} (k_{trap,L} + k_{trap,R})}\nonumber\\
  &+ \frac{2^{4/3} k_{trap,L} k_{trap,R} (k_{trap,L} + k_{trap,R}) (x_R-x_L)^2}{\zeta} \Big]\,,
\end{align}
%
where $\Delta x_{LR} = (x_R-x_L)$ and $\zeta = \left( Y - \eta \right)^{(1/3)}$, with
%
\begin{eqnarray}
&&Y = 3 \sqrt{3} \bigg\{\mathcal{C} k_{trap,L}^4 k_{trap,R}^4 \left(k_{trap,L}+k_{trap,R}\right)^{7}
\nonumber\\
&&\times\left[4 k_{trap,L} k_{trap,R} \Delta x_{LR}^3+27 \mathcal{C} \left(k_{trap,L}+k_{trap,R}\!\right)\!\right]\!\!\bigg\}^{\!1/2}\!,
\nonumber
%
\\
  %%
&&\eta =  k_{trap,L}^2 k_{trap,R}^2 \left(k_{trap,L}+k_{trap,R}\right)^{3}
\nonumber\\
&&\times\left[2 k_{trap,L} k_{trap,R} \Delta x_{LR}^3+27 \mathcal{C} \left(k_{trap,L}+k_{trap,R}\right)\right]\!.
\end{eqnarray}
%
In this setup, the coupling between the ions $k$ can be controlled by changing the distance between the on-site traps.
%
%
%
\subsection{Optical molasses and Langevin baths}
%
%
%
Trapped ions may be cooled down by counterpropagating lasers which are red-detuned with respect to an internal atomic transition of the ions. This technique is known as Doppler cooling or optical molasses \cite{Chu1985,Cohen1992,Metcalf1999,Metcalf2003}. The off-resonant absorption of laser photons by the ions exerts a damping-like force that slows them down. The spontaneous emission of the ions produces heating due to the random recoil generated by the emitted photons. The friction and recoil force balance, so eventually the ion thermalizes to a finite temperature.
Thus the effect of the lasers on the ion is equivalent to a Langevin bath with temperature $T_{molass}$ and friction coefficient $\gamma_{molass}$. The temperature and friction coefficients are controlled with the laser intensity $I$ and frequency detuning $\delta$ with respect to the selected internal atomic transition by the expressions \cite{Cohen1992,Metcalf2003,Ruiz2014},
%
\begin{align}
  \gamma_{molass}(I,\delta) &= -4 \hbar \left(\frac{\delta + \omega_0}{c}\right)^2 \left(\frac{I}{I_0}\right)\frac{2\delta/\Gamma}{\left[1 + (2\delta/\Gamma)^2\right]^2},\nonumber\\
  %
  T_{molass}(\delta) &= -\frac{\hbar \Gamma}{4 k_B} \frac{1+(2\delta/\Gamma)^2}{(2\delta/\Gamma)},
  \label{eq:DopplerCoolingToyModel}
\end{align}
%
where $\omega_0$ is the frequency of the internal atomic transition, $\Gamma$ is the natural width (decay rate) of the excited state, and $I_0$ is the saturation intensity. For fixed $\Gamma$ and $I$, $\gamma_{molass}$ depends on $\delta$, and thus, indirectly, on the temperature $T_{molass}$.
%
%
%
\section{Looking for rectification\label{sec:lookingForR}}
%
%
%
%First let us define what we exactly mean by \textit{rectification}.
There is rectification if the flux $J$  for the forward temperature bias is different from the flux $\tilde{J}$ for reverse bias
with the baths exchanged.  To measure rectification, we will use the rectification coefficient $0\le R\le 1$ defined as
%
\begin{equation}
  R = \frac{\abs{|J|-|\tilde{J}|}}{\max(|J|,|\tilde{J}|)}.
  \label{eq:Rectification}
\end{equation}
%
The important point here is to define what is  meant by \textit{exchanging the baths}. We consider that a bath is characterized, not only by its temperature $T$ but also by its coupling  to the system by means of the friction coefficient $\gamma$, so, exchanging the baths is achieved by exchanging both the temperatures and the friction coefficients, as summarized in Table \ref{tab:reversed_bath}. For generic models this
choice is a matter of definition, but for trapped ions it is a natural way to proceed.

When implementing temperatures and friction coefficients by lasers according to
%, this exchange operation is performed by changing the values of the intensities and detunings acting on each ion (
Eq. \eqref{eq:DopplerCoolingToyModel}, the exchange operation is straightforward when the two ions are either of the same species or isotopes of each other, since the only required action to exchange temperatures is to exchange the detunings without modifying the intensities. The detuning exchange in turn automatically exchanges the friction coefficients. However, for two different species, which involve two different atomic transitions, the laser wavelengths and the decay rates $\Gamma$ depend on the species. Then, exchanging the temperatures by modifying the detunings, keeping the laser intensities constant, does not necessarily imply an exchange of the friction coefficients. Nevertheless it is possible to adjust the laser intensities so that the friction coefficients get exchanged and that is the assumption hereafter. In terms of the analysis of rectification in Ref. \cite{Pereira2017}, we are adding a temperature dependent feature to the system, namely,  the friction coefficients depend on the bath temperature
and are exchanged as the baths are reversed.

\begin{figure}
  \center
  \includegraphics[width=\linewidth]{Figures/RwMPlota.pdf}
  \caption{Rectification, $R$, in the $k_L k_R$ plane for $k = 1.17$ fN/m, $\gamma_L = 6.75\times 10^{-22}$ kg/s, and $\gamma_R = 4.64\gamma_L$, $m_1 = 24.305$ a.u., $m_2 = 40.078$ a.u. The dashed  line represents Eq.  (\ref{eq:MaxRLines}).}
  \label{fig:Fig_rectification_K_plane}
\end{figure}


%that is, the ratio between the difference of heat currents and the largest one. As defined, $R=0$ for no asymmetry of the heat currents and $R=1$ when they are maximally asymmetric.

\begin{table}[]
\center
\caption{Definition of forward and reversed (exchanged) bath configurations.}
\begin{tabular}{lcc}
\hline
                 & forward                & reversed                                                       \\ \hline
Bath Friction    & $\gamma_L$, $\gamma_R$ & $\tilde{\gamma}_L =\gamma_R $,  $\tilde{\gamma}_R =\gamma_L $   \\
Bath Temperature & $T_L$, $T_R$           & $\tilde{T}_L =T_R $,  $\tilde{T}_R =T_L $                     \\
\hline
\end{tabular}
\label{tab:reversed_bath}
\end{table}
%
%
\subsection{Parametric exploration}
%
%
%
We have explored thoroughly the space formed by the parameters of the model $m_1,m_2,k,k_L,k_R,\gamma_L,\gamma_R$, to find
and maximize asymmetric heat transport. We have fixed the values of some of the parameters to realistic ones while varying the rest. Unless stated otherwise the masses are
$m_1 = 24.305$ a.u. and $m_2 = 40.078$ a.u., which correspond to Mg and Ca, whose ions are broadly used in trapped-ion physics. According to Eq. \eqref{eq:CurrentsInModelB} and the corresponding expression for $\tilde{J}$ with the substitutions in Table \ref{tab:reversed_bath},
rectification does not formally depend on the bath temperatures in this model for given friction coefficients.
Of course the friction coefficients depend on the temperature indirectly, but also on laser intensities, see Eq. \eqref{eq:DopplerCoolingToyModel}, so in the parametric space $m_1,m_2,k,k_L,k_R,\gamma_L,\gamma_R$ there is no need to specify the bath temperatures to analyze the rectification in the following. The bath temperatures will be needed though
to calculate the power spectra, and play an implicit role in the central assumption that their exchange implies an exchange of
friction coefficients.

Figure \ref{fig:Fig_rectification_K_plane} depicts the values of the rectification after sweeping the $k_L k_R$ plane for fixed values of $k$, $\gamma_L$, and $\gamma_R$. There is a ridge in the $k_L,k_R$ plane for which the rectification is maximal. $\partial_{k_L}R = 0$ may be
solved explicitly but the solution is too long to be displayed here. In a weak dissipation regime
(${\gamma_L}/{m_1}<<\sqrt{{k}/{m_1}}$, ${\gamma_R}/{m_2}<<\sqrt{{k}/{m_2}}$), a Taylor series around $(\gamma_L,\gamma_R) = (0,0)$ gives in zeroth order
%.i.e., ${\gamma_L}/{m_1}<<\sqrt{{k}/{m_1}}$, ${\gamma_R}/{m_2}<<\sqrt{{k}/{m_2}}$, the ridge is given approximately by
a straight line for the ridge,
%
\begin{equation}
  \frac{k+k_R}{m_2} = \frac{k+k_L}{m_1}.
  \label{eq:MaxRLines}
\end{equation}
%
Eq. \eqref{eq:MaxRLines} implies the resonance condition $\omega_L = \omega_R$
for the effective oscillation frequencies $\omega_L = \sqrt{{(k+k_L)}/{m_1}}$ and $\omega_R = \sqrt{{(k+k_R)}/{m_2}}$,
see Eq. (\ref{eq:Hamiltonian}). The lowest order correction to  Eq. \eqref{eq:MaxRLines} implies a small shift of the line,
keeping the same slope,
%
\begin{equation}
  \frac{k+k_R}{m_2} = \frac{k+k_L}{m_1} + \frac{(m_2\gamma_L+m_1\gamma_R)(m_1\gamma_L+m_2\gamma_R)}{2m_1m_2(m_2^2-m_1^2)}.
  \label{eq:MaxRLines_correction}
\end{equation}
%
In a trapped-ion context the condition \eqref{eq:MaxRLines} may be imposed by adjusting the distance of the traps for fixed $k_L$ and $k_R$. Besides the line of maximum rectification, Fig. \ref{fig:Fig_rectification_K_plane} also shows two lines where rectification is zero.
At these lines forward and backward fluxes cross.
%, The lines correspond to the boundaries in which the heat conductance for the reversed configuration surpasses the forward configuration. Inside of these boundaries heat propagates easily for a forward bias ($T_L>T_R$), whereas the opposite happens outside. In Fig. \ref{fig:Fig_rectification_K_plane}, the lines of zero rectification look parallel to the one of maximum rectification. This is however only a limiting behavior for weak dissipation.
Solving $R=0$ with a Taylor series around $(\gamma_L,\gamma_R) = (0,0)$ gives, up to second order in friction coefficients,  the two approximate solutions
%
\begin{align}
k_R &= k_L\left[\frac{m_2}{m_1}\pm\frac{1}{2k}\sqrt{\frac{m_2\gamma_L\gamma_R^3}{m_1^3}}\right]
\nonumber\\
&+k\left[\frac{m_2}{m_1}\left(1\pm \frac{ 2 m_1 m_2 \gamma_R + (m_1^2 + m_2^2)\gamma_L }{2\sqrt{m_1 m_2^3 \gamma_L \gamma_R}} \right)-1\right]
\nonumber\\
&\pm\frac{1}{2}\sqrt{\frac{m_2\gamma_L\gamma_R^3}{m_1^3}} + \gamma_R\frac{(m_1^2+m_2^2)\gamma_L + m_1m_2\gamma_R}{2m_1^2(m_2-m_1)}.
\label{eq:zeroRlines}
\end{align}
%
The term $\pm\frac{1}{2k}\sqrt{{m_2\gamma_L\gamma_R^3}/{m_1^3}}$ in Eq. \eqref{eq:zeroRlines} makes the slopes of the
two zero-rectification lines different from each other and also from the maximum-rectification line. This difference is however
hardly noticeable for weak dissipation as in  Fig. \ref{fig:Fig_rectification_K_plane}.

Interestingly, along the maximum line  \eqref{eq:MaxRLines} the rectification no longer depends on the spring constants of the model,
see  Eqs. \eqref{eq:CurrentsInModelB}  and \eqref{eq:Rectification},
%
\begin{equation}
    R=
    \begin{cases}
      1-\frac{a+g}{1+ag} &\text{ if }a>1,g>1\text{ or }a<1,g<1\\
      1-\frac{1+ag}{a+g} &\text{ if }a>1,g<1\text{ or }a<1,g>1,
    \end{cases}
  \label{eq:maxRExpression}
\end{equation}
%
it only depends on the mass and friction coefficient ratios $a$ and $g$
%
\begin{align}
  a = m_2/m_1,\;\;\;\;
  g = \gamma_R/\gamma_L.
\end{align}
%
%The maximal rectification found does not scale with the magnitude of the masses or the friction coefficients, just with their ratios.
Besides a high value of $R$, it is desirable to have a significant current $J_{max}$
%when a forward temperature bias is applied to the rectifier
%, as in some cases an increase of $R$ is accompanied by an overall decrease of the heat currents
\cite{Simon2019}. Using again  Eq. \eqref{eq:MaxRLines} in the expression for the currents \eqref{eq:CurrentsInModelB}, the maximum current $J_{\max} = \max(\big|{J}\big|,\big|\tilde{J}\big|)$ is
%
\begin{align}
    &J_{\max}=\begin{cases}
   \frac{k_B g\gamma_L k^2 \abs{T_L-T_R}}{(a+g)(g\gamma_L^2(k_L+k)+k^2m_1)} &
   \text{ if }\begin{cases}a>1,g>1\\
   \text{ or }a<1,g<1\end{cases}
    \\
    \frac{k_B g\gamma_L k^2 \abs{T_L-T_R}}{(1+ag)(g\gamma_L^2(k_L+k)+k^2m_1)}&\text{ if }
    \begin{cases}a>1,g<1
    \\\text{ or }a<1,g>1\end{cases}
    \end{cases}
    \label{eq:maxJExpression}
\end{align}
%
Now we analyze how the parameters $a$ and $g$ affect the maximum current $J_{max}$ in \eqref{eq:maxJExpression}. To do this, we can divide the $ag$ plane in four quadrants by the axes $a = 1$ and $g = 1$ (in those axes $R = 0$). In Eq. \eqref{eq:maxJExpression} the parameter $a$ appears only in the denominator, thus for a higher $a$, a smaller current is found. The quadrants with $a < 1$ will be better for achieving large currents. $g$ appears both in the numerator and denominator so there is no obvious advantageous quadrant for this parameter.

Equation \eqref{eq:maxRExpression} is symmetric upon the transformations $a \leftrightarrow 1/a$ and $g \leftrightarrow 1/g$. Using a logarithmic scale for $a$ and $g$, the resulting $R$ map is symmetric with respect to the $a=1$ and $g=1$ axes. We can thus limit ourselves to analyze the quadrant $a > 1$, $g > 1$.
%, as the results in other quadrants will be equivalent upon transformations $a \leftrightarrow 1/a$ and $g \leftrightarrow 1/g$.


\begin{figure}
  \center
  \includegraphics[width=\linewidth]{Figures/Rade.pdf}
  \caption{Rectification factor, $R$, given by Eq. \eqref{eq:maxRExpression}.}
  \label{fig:R_g_a_plane}
\end{figure}

Figure \ref{fig:R_g_a_plane} shows the rectification given by Eq. \eqref{eq:maxRExpression} in terms of $a$ and $g$. Along any diagonal line (parallel to the solid cyan or the dashed green lines), the maximum value is at the center, that is, when $a = g$. For constant $a$, a larger $g$ always increases $R$, but making the  ratio between friction coefficients $g$ arbitrarily large is not  realistic in a trapped-ion setup.
%Since $g$ is defined as the ratio between the friction coefficients, increasing it means making either $\gamma_L$ go to 0 or $\gamma_R$ to infinity.
Making $\gamma_L$ go to 0 decouples one of the ions from the bath, so the heat current tends to vanish in any direction. Also, increasing $\gamma_R$ arbitrarily is impossible since it is a function of the laser detuning (Eq. \eqref{eq:DopplerCoolingToyModel}) which is physically bounded
by the existence of other levels. Although Eq. \eqref{eq:DopplerCoolingToyModel} suggests that boosting the laser intensity can also increase the friction coefficient, this is not an option since Eq. \eqref{eq:DopplerCoolingToyModel} is just an approximation for low laser intensities. When going to higher intensities, the emission/absorption of photons by the ion is saturated and the friction coefficient reaches a finite value proportional to the width $\Gamma$ of the excited state \cite{Metcalf2003}. As a compromise between feasibility and high $R$, let as assume that the ratio between the friction coefficients $g$ to be equal to the mass ratio $a$. As shown  in Fig. \ref{fig:R_g_a_plane}, along the solid-cyan and dashed-green diagonal lines the maximum $R$ is achieved for $a = g$. The effect of varying  the common value $c$ of $a$ and $g$, $c=a = g$, may be seen in
Fig. \ref{fig:Fig_PerfectRectification}, which  shows the rectification in Eq. \eqref{eq:maxRExpression}. $R$ tends to one for large $c$.
%
%
\subsection{Spectral match/mismatch approach to rectification}
%
%
%
\begin{figure}
  \center
  \includegraphics[width=\linewidth]{Figures/CC.pdf}
  \caption{Rectification for different values of $c=m_2/m_1=\gamma_R/\gamma_L$ when the maximum condition in the $k_L k_R$ plane is satisfied (Eq. \eqref{eq:MaxRLines}).}
  \label{fig:Fig_PerfectRectification}
\end{figure}

%The match/mismatch between the power spectra of the particles controls the heat currents in the system . A good match between the power spectra of the two ions in a large range of frequencies yields a higher heat current through the system while the mismatch  reduces the heat current.
%Therefore, we can understand rectification through the match/mismatch of the phonon bands of the ions \cite{Terraneo2002}.
If there is a good match between the phonon spectra of the ions (i.e., their peaks overlap in a broad range of frequencies) for a certain baths configuration, and mismatch when the baths exchange, the system will present heat rectification \cite{Terraneo2002,Li2004}.
We have studied the spectra of the ions in our model for several sets of parameters exhibiting no rectification or strong rectification. The spectra are calculated  through the spectral density matrix. For a real-valued stochastic process $\overrightarrow{x}(t)$, its spectral density matrix is defined as \cite{Sarkka2019}
%
\begin{equation}
  \mathbb{S}_{\overrightarrow{x}}(\omega) \equiv \expval{ \overrightarrow{X}(\omega) \overrightarrow{X}^\mathsf{T}(-\omega) },
  \label{eq:SpectralDensityDefinition}
\end{equation}
%
with $\overrightarrow{X}(\omega)$ being the Fourier transform of $\overrightarrow{x}(t)$ (we use the convention of factors of $1$ and ${1}/{(2\pi)}$ for the transform and the inverse transform). A justification of the use of the spectral density matrix to understand heat transport arises from the Wiener-Khinchin theorem \cite{Sarkka2019}, which says that the correlation matrix of a stationary stochastic process in the steady state is the inverse Fourier transform of its spectral density matrix $\expval{\overrightarrow{r}(t)\overrightarrow{r}^\mathsf{T}(t+\tau)} = \mathcal{F}^{-1}[\mathbb{S}_{\overrightarrow{r}}(\omega)](\tau)$. Thus  the covariance matrix in the steady state is
%
\begin{equation}
  \mathbb{C}^{s.s.} = \frac{1}{2\pi} \int_{-\infty}^{\infty}d\omega\;\mathbb{S}_{\overrightarrow{r}}(\omega).
  \label{eq:Wiener-Khinchin}
\end{equation}
%
Eq. \eqref{eq:Wiener-Khinchin} directly connects the spectral density matrix to the steady-state temperature
since  $T_1^{s.s.} = {m_1 C_{3,3}^{s.s.}}/{k_B}$ and $T_2^{s.s.} = {m_2 C_{4,4}^{s.s.}}/{k_B}$, and, therefore, to the heat currents,
see  Eqs.(\ref{eq:Temperature_definition}) and (\ref{eq:currents_definition}).


\begin{figure*}[t]
  \center
  \includegraphics[width=.8\linewidth]{Figures/SpectrumComparative.pdf}
  \caption{Spectral densities of the velocities of the ions ($r_3$ and $r_4$) corresponding to $T_L=\tilde{T}_R=2$ mK, $T_R=\tilde{T}_L=1$ mK, and  two values of $c$ in Fig. \ref{fig:Fig_PerfectRectification}: (a), (b) for $c=1$ and (c), (d) for $c=10$. Solid, black lines are for the left ion spectral density ${\cal{S}}_1(\omega)$ and dashed, blue lines for the right ion spectral density
 ${\cal{S}}_2(\omega)$. Dot-dashed, vertical lines mark the frequencies of the normal modes of the system. The spectra are multiplied by their corresponding masses so that  the areas are proportional to the steady-state temperatures, see  Eq. \eqref{eq:Wiener-Khinchin}. (a) and (b) correspond to $R = 0$:  the overlap between the phonon bands is the same in forward and reversed configurations. (c) and (d) correspond to $R\approx 0.8$:  in the forward configuration (c)  the phonons match better than in the reversed configuration (d).}
  \label{fig:Figure_Spectra}
\end{figure*}

The Fourier transform of the vector process $\overrightarrow{r}(t)$ describing the evolution of our system, see Eq. (\ref{eq:vectorEqOfMotion}),
is $\overrightarrow{R}(\omega) = \left( i \omega - \mathbb{A} \right)^{-1}\mathbb{L}\overrightarrow{\Xi}(\omega)$ with $\overrightarrow{\Xi}(\omega)$ being the Fourier transform of the white noise $\overrightarrow{\xi}(t)$. Note that $\overrightarrow{\Xi}(\omega)$ is not square-integrable, however its spectral density is $\mathbb{S}_{\overrightarrow{\xi}}(\omega) = 2 \mathbb{D}$ \cite{Sarkka2019}, which is flat as expected for a white noise. Therefore, the spectral density matrix of the system is
%
\begin{equation}
  \mathbb{S}_{\overrightarrow{r}} (\omega)= 2 \left(  \mathbb{A} - i\omega\right)^{-1}\mathbb{L}\mathbb{D}\mathbb{L}^\mathsf{T}\left(  \mathbb{A} + i\omega\right)^{-\mathsf{T}}.
  \label{eq:SpectralDensityToyModelB}
\end{equation}
%
%As we can see in Eq. \eqref{eq:SpectralDensityToyModelB},
The imaginary part of the eigenvalues of the dynamical matrix $\mathbb{A}$ correspond to the peaks in the spectrum whereas the real part dictates their width. Eq. \eqref{eq:SpectralDensityToyModelB} gives after direct computation
%
\begin{equation}
  \mathbb{S}_{\overrightarrow{r}}(\omega) = 2 k_B \frac{\gamma_L T_L\mathbb{S}_L(i\omega)+\gamma_L T_R\mathbb{S}_R(i\omega)}{(m_1 m_2)^2 P_\mathbb{A}(i\omega)P_\mathbb{A}(-i\omega)},
\end{equation}
%
where $P_\mathbb{A}(\lambda)$ is the characteristic polynomial of the dynamical matrix $\mathbb{A}$ and $\mathbb{S}_L(\omega)$, $\mathbb{S}_R(\omega)$ are the matrix polynomials in the angular frequency $\omega$ whose coefficients are defined in Appendix \ref{Appendix:SpectralDensity}. We give
%Equation \eqref{eq:SpectralDensitiesVelocities} gives the full expressions of
the spectral densities for the velocities, ${\cal{S}}_1\equiv\mathbb{S}_{3,3}(\omega) = \expval{R_3(\omega)R_3(-\omega)}$ for the left ion, and ${\cal S}_2\equiv\mathbb{S}_{4,4}(\omega) = \expval{R_4(\omega)R_4(-\omega)}$ for the right ion, since they are the elements related to the calculation of the heat current using Eq. \eqref{eq:Wiener-Khinchin},
%
  \begin{align}
    {\cal S}_1(\omega) &= 2 k_B \frac{\gamma_R k^2 T_R \omega ^2+\gamma_L T_L \left[\omega ^4 \left(\gamma_R^2-2 k m_2-2 k_R m_2\right)+\omega ^2 (k+k_R)^2+m_2^2 \omega ^6\right]}{(m_1 m_2)^2 P_\mathbb{A}(i\omega)P_\mathbb{A}(-i\omega)},\nonumber\\
    %
%    \nonumber\\
    %
%    \mathbb{S}_{4,4}(\omega)
{\cal S}_2(\omega)
&= 2 k_B \frac{\gamma_L k^2 T_L \omega ^2+\gamma_R T_R \left[\omega ^4 \left(\gamma_L^2-2 k m_1-2 k_L m_1\right)+\omega ^2 (k+k_L)^2+m_1^2 \omega ^6\right]}{(m_1 m_2)^2 P_\mathbb{A}(i\omega)P_\mathbb{A}(-i\omega)}.
    \label{eq:SpectralDensitiesVelocities}
  \end{align}
%
%The spectral densities of the masses depend explicitly on the temperatures on the baths, as well as implicitly through the dependence of the friction coefficients if the laser cooling baths are used.
Figure \ref{fig:Figure_Spectra} depicts a series of plots of the spectra given by Eq. \eqref{eq:SpectralDensitiesVelocities}, corresponding to two points in Fig. \ref{fig:Fig_PerfectRectification}. (The calculation for the reverse bias is done with the substitutions in Table \ref{tab:reversed_bath}.)
For $c=1$ (Fig. \ref{fig:Figure_Spectra}(a) and (b)) there is no rectification, since the spectra match in the forward (a) and reversed (b) configurations. However, for $c=10$ (Fig. \ref{fig:Figure_Spectra}(c) and (d), $R\approx 0.8$) the picture is very different: there is a good match between the spectra in the forward configuration but not for the reversed configuration. It is interesting to analyze how the system changes from $c=1$ to $c=10$
% ( Fig. \ref{fig:Figure_Spectra} (a),(b) to Fig. \ref{fig:Figure_Spectra} (c),(d)
using the dissipative normal modes of the system, which may be found by diagonalizing the dynamical matrix $\mathbb{A}$, Eq. \eqref{eq:Dynamical_Matrix}. The frequencies of the peaks in Fig. \ref{fig:Figure_Spectra} are given by the imaginary part of the eigenvalues of $\mathbb{A}$. Likewise, the width depends on  the real part of the eigenvalues. For the forward configuration, the normal frequencies (position of the peaks) come closer to each other as $c$ is increased, while the widths remain practically constant. To understand why the real part remains practically constant, we recall that we have chosen to work with spring constants that satisfy Eq. \eqref{eq:MaxRLines} and making the mass and friction coefficient ratios equal to $c$, \textit{i.e.}, $ c\equiv m_2/m_1 = \gamma_R/\gamma_L$. The (dissipative) terms in $\mathbb{A}$ responsible for the real parts in the eigenvalues are,
for the forward configuration,  $\gamma_L/m_1$ and $\gamma_R/m_2 = (c \gamma_L)/(c m_1) = \gamma_L/m_1$, which are constant for every value of $c$. On the contrary, in the reverse bias configuration  the dissipative terms in the dynamical matrix  are $\tilde{\gamma}_L/m_1 = \gamma_R/m_1 = c\gamma_L/m_1$ and $\tilde{\gamma}_R/m_2 = \gamma_L/ (c m_1)$, with opposite behavior with respect to $c$. The real parts of the eigenvalues
also behave quite differently for reverse bias, one of them gets closer to the imaginary axis for $c=10$,
see Fig. \ref{fig:Figure_Spectra} (d), where this mode  concerns mostly the right ion,
the only one excited at the peak frequency, while the other eigenvalue  moves far from the imaginary axis so a peak is not noticeable
at the imaginary value (left dotted-dashed line) any more.
%The different behavior of the spectra in the reversed bias with respect to the forward bias explains why heat currents are lower for the reversed bias.
%In fact, in the limit for high values of $c$, which leads to perfect rectification, two purely dissipative eigenvalues appear (the imaginary part is 0) which could indicate low heat current.


%Figure \ref{fig:Figure_Spectra} only shows the elements (3,3) and (4,4) in the diagonal of $\mathbb{S}$ but the remaining elements, including off-diagonal ones, exhibit a similar behavior.
%
\section{Conclusions \label{sec:Conclusions}}
%
We have studied heat rectification in a model composed of two coupled harmonic oscillators connected to Langevin baths, which could be realized with trapped ions and optical molasses. This simple model allows analytical treatment but still has enough complexity to examine different ingredients that can produce rectification. %We have also derived analytical expressions for the heat currents and local temperatures.
Our results demonstrate in a simple but realistic model that harmonic systems can rectificate heat current if they have features which depend on the temperature  \cite{Pereira2017}. We implement this notion of temperature-dependent features by defining the baths exchange operation as an exchange of both temperatures and coupling parameters of the baths to the system. The temperature dependence of the bath-system coupling  occurs naturally in laser-cooled trapped ion setups.

We have also studied the phonon spectra of the system, aided by a normal mode analysis,
comparing the match/mismatch of the phonon bands, to reach the conclusion that the band match/mismatch description for heat rectification is also valid for systems which are purely harmonic, as long as there are temperature-dependent features.
We hope this article sheds more light into the topic of heat rectification and that encourages more research regarding its physical implementation on chains of trapped ions.
