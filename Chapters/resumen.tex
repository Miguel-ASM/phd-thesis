\chapter{Resumen} % Write in your own chapter title
\label{Resumen}
\lhead{\emph{Resumen}} % Write in your own chapter title to set the page header

Los dispositivos que controlan el flujo de energía o materia desempeñan un papel destacado en la tecnología. Un dispositivo clave es el rectificador, que permite que las corrientes sólo vayan en una dirección. El más notable de estos dispositivos es el diodo eléctrico, que es una parte vital de los ordenadores, los dispositivos digitales y los sistemas de conversión de corriente AC/DC. Sin el diodo no existiría la mayor parte de la tecnología que tenemos hoy en día.

El diodo eléctrico es un componente eléctrico que permite que la corriente eléctrica fluya de forma asimétrica con respecto al signo de la diferencia de potencial que se le aplica. Normalmente, un diodo está compuesto por la unión de un semiconductor $p$ con un semiconductor $n$. Cuando se aplica un potencial de polarización directa $\Delta V$ a la unión $p$-$n$ (conectando el polo positivo de una batería al semiconductor $p$), la corriente eléctrica fluirá a través del diodo. Sin embargo, la unión $p$-$n$ actúa como un aislante eléctrico si se aplica un potencial de polaridad inversa $-\Delta V$.

El impacto tecnológico del diodo ha motivado el desarrollo de dispositivos análogos en otros escenarios físicos, como la óptica. Un equivalente óptico al diodo es el aislador óptico, que se utiliza para permitir la propagación unidireccional de la luz. Este dispositivo se basa en la rotación no recíproca de la dirección de polarización de la luz polarizada en materiales que se encuentran en un campo magnético, conocida como Rotación de Faraday. El aislador óptico es un componente crítico en los dispositivos ópticos para proteger las fuentes de luz delicadas de la retropropagación de la luz.

Llegados a este punto podemos ver que un ingrediente común entre los dispositivos que muestran un comportamiento similar al de un diodo es algún tipo de asimetría estructural interna. En el diodo eléctrico esta asimetría proviene de la distribución asimétrica de los portadores de carga: electrones en el lado $n$ y huecos en el lado $p$. En el aislador óptico, la orientación del campo magnético rompe la simetría del sistema.

El objetivo de esta Tesis es explorar la física y los posibles diseños de dispositivos que implementen un mecanismo rectificador para un transporte asimétrico de materia o energía. Esta Tesis se divide en dos partes: En la primera parte, estudio el scattering asimétrico de partículas por potenciales cuánticos unidimensionales y en la segunda parte, estudiaré la rectificación térmica en cadenas de osciladores. A continuación, presento una introducción a estas dos partes.

\section*{Parte I}


El interés actual por desarrollar nuevas tecnologías cuánticas está impulsando la investigación aplicada y fundamental sobre los fenómenos y sistemas cuánticos con posibles aplicaciones en circuitos lógicos, metrología, comunicaciones o sensores. Se necesitan dispositivos básicos robustos que realicen operaciones elementales para llevar a cabo tareas complejas cuando se combinan en un circuito. Con el desarrollo de nuevas tecnologías cuánticas en mente, el objetivo de esta parte de la Tesis es diseñar potenciales para el scattering unidimensional de una partícula cuántica sin espín que conduzcan a coeficientes de transmisión y reflexión que difieran para paquetes de ondas procedentes de la izquierda o de la derecha.


Para obtener scattering asimétrico, es necesario utilizar potenciales no hermíticos y no locales. Aunque los potenciales no locales y no hermíticos pueden parecer poco comunes y extraordinarios en la física cuántica para algunos, aparecen de forma natural cuando se aplican técnicas de partición para describir interacciones efectivas en un subespacio de un sistema mayor con un hamiltoniano hermítico. Los hamiltonianos no hermíticos que representan interacciones efectivas tienen una larga historia en física nuclear, atómica y molecular, y se han vuelto comunes en la óptica, donde las ecuaciones de onda en guías de onda podrían simular la ecuación de Schr\"{o}dinger. Los hamiltonianos no hermíticos también pueden establecerse fenomenológicamente, por ejemplo, para describir la disipación. Recientemente ha habido mucho interés en los hamiltonianos no hermíticos, en particular, en aquellos que tienen simetría PT por sus propiedades espectrales y sus útiles aplicaciones, sobre todo en óptica. Sin embargo, es importante destacar que existen simetrías diferentes a la PT y que son necesarias para producir ciertas formas de scattering asimétrico.

El contenido de esta parte de la Tesis está organizado como sigue. En el capítulo I, utilizaré potenciales no hermíticos y no locales para diseñar potenciales con coeficientes de scattering asimétricos. Las simetrías para los hamiltonianos no hermíticos se generalizarán utilizando el concepto de pseudohermiticidad y se utilizarán para derivar reglas de selección útiles para los coeficientes de transmisión y reflexión. En el capítulo II, derivaré un conjunto de propiedades de los valores propios de los potenciales de scattering que extienden los resultados anteriores para los hamiltonianos discretos no hermíticos utilizando las simetrías generalizadas. En el capítulo III, presentaré una posible realización física de los hamiltonianos de scattering asimétrica en un contexto de óptica cuántica.


\section*{Parte II}


La radiación, el calor y la electricidad son algunos de los principales mecanismos físicos de transporte de energía. En particular, los dos últimos mecanismos desempeñan un papel importante en la tecnología. El procesamiento moderno de la información se basa en dispositivos electrónicos como el diodo y el transistor. Sin embargo, no existe una tecnología análoga para controlar las corrientes de calor transportadas por fonones. Una explicación podría ser que los fonones son más difíciles de controlar que los electrones, ya que (al contrario que ellos) no tienen masa ni carga eléctrica. Sin embargo, sería interesante explorar el diseño de dispositivos fonónicos debido a la riqueza de diferentes mecanismos físicos que intervienen en el transporte de calor. El rectificador térmico, o diodo térmico, sería un componente elemental para el desarrollo de dispositivos fonónicos. En esta parte de la Tesis estudio la rectificación térmica en cadenas de osciladores, teniendo la posibilidad de diseñar un diodo térmico como motivación principal.

La rectificación térmica es el fenómeno físico, análogo a la rectificación de la corriente eléctrica en los diodos, en el que la corriente de calor a través de un dispositivo o medio (el diodo térmico o rectificador) no es simétrica con respecto al intercambio de las temperaturas de los baños térmicos con los que está en contacto. Fue observado por primera vez en 1936 por Starr en una unión entre cobre y óxido cuproso. Los trabajos teóricos se iniciaron mucho más tarde utilizando como rectificadores modelos simples de cadenas sgmentadas de osciladores anarmónicos. Estos trabajos desencadenaron mucha investigación que continúa hasta hoy. La investigación sobre la rectificación térmica ha ganado mucha atención en los últimos años como ingrediente clave para construir dispositivos que controlen los flujos de calor de forma similar a las corrientes eléctricas. Existen propuestas para diseñar circuitos lógicos térmicos en los que la información, almacenada en memorias térmicas, se procesaría en puertas lógicas térmicas. Estas puertas lógicas térmicas, al igual que sus homólogas electrónicas, requerirían diodos térmicos y transistores térmicos para funcionar.
Los dispositivos rectificadores de calor también serían muy útiles en los circuitos nanoelectrónicos, ya que permitirían a los componentes delicados disipar el calor mientras están protegidos de las fuentes de calor externas.

La mayoría de los trabajos sobre diodos térmicos han sido teóricos, con sólo unos pocos experimentos. Un intento relevante de construir un rectificador térmico se basó en una estructura graduada hecha de nanotubos de carbono y nitruro de boro que transporta el calor entre un par de circuitos de calefacción/sensores. Uno de los extremos del nanotubo está cubierto con una deposición de otro material, lo que hace que el calor fluya mejor desde el extremo cubierto al descubierto. Sin embargo, la rectificación obtenida fue pequeña, con factores de rectificación en torno al $7\%$.

Gran parte del esfuerzo teórico en la investigación de la rectificación térmica se ha dirigido a mejorar los factores de rectificación y las características de los rectificadores. La primera aproximación al diseño de diodos térmicos consistió en utilizar cadenas de osciladores segmentados en dos o más regiones con propiedades diferentes. Sin embargo, pronto se observó que el rendimiento de los rectificadores segmentados era muy sensible al tamaño del dispositivo: la rectificación disminuye al aumentar la longitud del rectificador. Para superar esta limitación se propusieron dos ideas. La primera consiste en utilizar cadenas escalonadas en lugar de segmentadas, es decir, cadenas en las que alguna propiedad física varía de forma continua a lo largo de la cadena, como por ejemplo la masa de las partículas que la componen. La segunda consiste en utilizar cadenas con interacciones de largo alcance, de forma que todos los elementos de la cadena interactúan con todos los demás. El fundamento de estas propuestas era que en un sistema escalonado se crean nuevos canales rectificadores asimétricos, mientras que las interacciones de largo alcance crean
también nuevos canales de transporte, evitando el habitual decaimiento del flujo de calor con el tamaño. Además de un mayor poder de rectificación, se espera que las cadenas escalonadas tengan una mejor conductividad térmica que las segmentadas. Este es un punto importante para las aplicaciones tecnológicas, ya que los dispositivos con altos factores de rectificación no son útiles si las corrientes que fluyen a través de ellos son muy pequeñas.

Otro foco principal de la investigación teórica en rectificación térmica es la búsqueda de los factores fundamentales que contribuyen a la aparición de la rectificación. Históricamente, los elementos cruciales para que haya rectificación han sido la presencia de alguna asimetría estructural en el sistema y de fuerzas no lineales (anarmónicas), que conducen a una dependencia con la temperatura de las bandas fonónicas. El solapamiento de las bandas fonónicas de las distintas partes de la cadena implica una buena o mala conducción térmica, por lo que el signo de la diferencia de temperatura aplicada puede afectar a la conducción y dar lugar a rectificación cuando. Sin embargo, investigaciones más recientes han señalado que la anarmonicidad no es una condición necesaria para un solapamiento asimétrico y, por tanto, para la rectificación. La rectificación también se produce en modelos armónicos simples (minimalistas) que incorporan alguna asimetría estructural y en la que los valores de algunos de sus parametros físicos dependen de la temperatura.

El contenido de esta parte de la Tesis está organizado como sigue. En el capítulo IV, presento un modelo de rectificador térmico que se basa en una impureza localizada en medio de una cadena de átomos. En el capítulo V, se presenta una propuesta de rectificador térmico en una cadena de iones atrapados con una distribución de frecuencia escalonada. Finalmente, en el capítulo VI, se estudia el transporte de calor en un modelo de dos osciladores conectados para explorar el origen y la optimización de la rectificación térmica.
