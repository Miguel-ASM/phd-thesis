%!TEX root = ../Thesis.tex

% \chapter*{Introduction} % Write in your own chapter title
\label{Introduction}
\lhead{\emph{Introduction}} % Write in your own chapter title to set the page header

Devices that control the flow of energy or matter play a prominent role in technology. A key device is the rectifier, which allows currents only one way. A rectifier behaves like a corridor with a trap door that can be opened from left to right but is closed otherwise. The most notable of such devices is the electric diode, which is a vital part of computers, digital devices, and AC/DC current conversion systems. Without the diode most of the technology that we have today would not exist.

The electric diode is an electrical component that allows electrical current to flow asymmetrically with respect to the sign of the potential difference that is applied to it. Typically, a diode is composed by the union of a $p$-semiconductor with an $n$-semiconductor. When a forward-bias potential $\Delta V$ is applied to the $p$-$n$ junction (by connecting the positive pole of a battery to the $p$-semiconductor), electrical current will flow through the diode. However, the $p$-$n$ junction acts as an electrical insulator if a reversed-bias potential $-\Delta V$ is applied.

Motivated by the technological impact of the diode, analogous devices have been developed in other physical scenarios, like optics. An optical equivalent to the diode is the optical isolator, which is used to allow one-way light propagation \cite{Saleh1991}. This device is based on the non-reciprocal rotation of the polarization direction of polarized light in materials that are in a magnetic field, known as Faraday Rotation (see ref. \cite{Yariv1984}). The optical isolator is a critical component in optical devices to protect delicate light sources from back-propagating light.

At this point we can see that a common ingredient between devices which show a diode-like behaviour is some kind of internal structural asymmetry. In the electric diode this asymmetry comes from the asymmetric distribution of charge carriers: electrons in the $n$-side, and holes in the $p$-side. In the optical isolator the orientation of the magnetic field breaks the symmetry of the system.

This Thesis is devoted to explore the physics and possible designs of devices that implement a \textit{diodic} or rectifying mechanism for an asymmetric transport of matter or energy. The Thesis
is divided into two parts: In part \ref{partI}, I look for asymmetric particle scattering of 1-dimensional quantum potentials and in part \ref{partII}, I will study thermal rectification in chains of oscillators. There follows an introduction to these two parts.


\section*{Introduction to part I: non-Hermitian systems and asymmetric scattering}

The current interest to develop new quantum technologies is boosting applied
and fundamental research on quantum phenomena and systems with potential
applications in logic circuits, metrology, communications or sensors. Robust basic devices performing elementary operations are needed to perform complex tasks when combined in a circuit. With the development of new quantum technologies in mind, the objective of this part is to design 1-dimensional scattering potentials for a quantum, spinless particle of mass $m$ that lead to transmission and reflection coefficients (squared modulus of the amplitudes) which differ for wave packets coming from the left or the right.

To find an asymmetric scattering behavior, I will use non-Hermitian and non-local potentials \cite{Muga2004,Mostafazadeh2018}. Although non-local and non-Hermitian potentials might seem uncommom and extraordinary in quantum physics to some, they appear naturally when applying partitioning techniques to describe the effective interactions in a subspace of a larger system with a Hermitian Hamiltonian by projection \cite{Feshbach1958,Ruschhaupt2004,Muga2004}. Non-Hermitian Hamiltonians representing effective interactions have a long history in nuclear, atomic, and molecular physics, and have become common in optics, where wave equations in waveguides could simulate the Schr\"odinger equation \cite{Ruschhaupt2005,Longhi2017a,Konotop2016}. Non-Hermitian Hamiltonians can also be set phenomenologically, e.g. to describe dissipation \cite{Ruschhaupt2005}. Recently there has been a lot of interest in non-Hermitian Hamiltonians \cite{Nixon2016,Nixon2016a,Chen2017,Ruschhaupt2017,Simon2018,Simon2019a,Alana2020,Bernard2002,Kawabata2019}, in particular, the ones having parity-time (PT) symmetry \cite{Bender1998,Znojil2015} because of their spectral properties and useful applications, mostly in optics  \cite{Longhi2017a,Konotop2016,Longhi2014}. However, I shall emphasize that symmetries different from PT exist and are necessary to produce certain forms of asymmetric scattering.

The contents of this part of the Thesis will be organized as follows. In chapter \ref{Chapter1}, I will use non-Hermitian and non-local potentials to design potentials with asymmetric scattering coefficients for left/right incidence. Symmetries for non-Hermitian Hamiltonians will be generalized using the concept of pseudohermiticity \cite{Mostafazadeh2002} and used to derive useful selection rules for the transmission and reflection coefficients. In chapter \ref{Chapter2}, I will derive a set of properties of the eigenvalues of scattering potentials that extend previous results for discrete non-Hermitian Hamiltonians by using the generalized symmetries. In chapter \ref{Chapter3}, I will present a possible physical realization for asymmetric scattering Hamiltonians in a quantum optics setup.


\section*{Introduction to part II: Heat rectification in mesoscopic systems}

Radiation, heat and electricity are prominent mechanisms of energy transport. In particular, the two last mechanisms play a dominant role in technology. Modern information processing rests on electronic devices like the diode and the transistor. However, there is not an analogous technology to control heat currents driven by phonons. An explanation could be that phonons are more difficult to control than electrons since (contrary to them) they do not have mass or electrical charge \cite{Li2012}. However, it would be interesting to explore the design of \textit{phononic} devices due to the richness of different physical mechanisms that mediate heat transport. The thermal rectifier, or thermal diode, would be a primary building block to develop \textit{phononics} \cite{Li2012}. In this part I study thermal rectification in chains of oscillators with the design of a thermal diode in mind.

Thermal rectification is the physical phenomenon, analogous to electrical current rectification in diodes, in which heat current through a device or medium (the thermal diode or rectifier) is not symmetric with respect to the exchange of the bath temperatures at the boundaries. It was  first observed in 1936 by Starr in a junction between copper and cuprous oxide \cite{Starr1936}. The theoretical work started much later using as rectifiers simple anharmonic chain models
with different segments \cite{Terraneo2002,Li2004}. These papers sparked much research that continues to this day. Research on thermal rectification has gained a lot of attention in recent years as a key ingredient to build prospective devices to control heat flows similarly to electrical currents \cite{Roberts2011,Li2012}. There are  proposals to engineer thermal logic circuits \cite{Ye2017} in which information, stored in thermal memories \cite{Wang2008}, would be processed in thermal gates \cite{Wang2007}. Such thermal gates, as their electronic counterparts,  would require thermal diodes and thermal transistors to operate \cite{Li2006,Joulain2016}.
Heat rectifying devices would also be quite useful in nanoelectronic circuits, letting delicate components dissipate heat while being protected from external heat sources \cite{Roberts2011}.

Most work on thermal diodes has been theoretical with only a few experiments (see refs. \cite{Chang2006,Kobayashi2009,Leitner2013,Elzouka2017}).
A relevant attempt to build a thermal rectifier was based on a graded structure made of carbon and boron nitride nanotubes that transports heat between a pair of heating/sensing circuits \cite{Chang2006}. One of the ends of the nanotube is covered with a deposition of another material, which makes the heat flow better from the covered end to the uncovered end. However, rectifications were small, with rectification factors around $7\%$.

Much of the theoretical effort in thermal rectification research has been aimed at improving the rectification factors and the features of the rectifiers. The first approach to designing thermal diodes consisted in using chains of oscillators segmented into two or more regions with different properties \cite{Terraneo2002,Li2004,Li2008,Hu2006}, which is reminiscent of the idea of the $p$-$n$ junction in electric diodes. However, it was soon noticed that the performance of segmented rectifiers was very sensitive to the size of the device, \textit{i.e.}, rectification decreases when increasing the length of the rectifier \cite{Hu2006}. To overcome this limitation two ideas were proposed. The first one consisted in using graded rather than segmented chains, \textit{i.e.}, chains where some physical property varies continuously along the site position such as the mass of particles in the chain \cite{Wang2012,Chen2015,Romero-Bastida2017,Yang2007,Romero-Bastida2013,Dettori2016,Pereira2010,Pereira2011,Avila2013}. The second one consisted in using chains with long-range interactions (LRI), such that all the elements in the chain interact with all the rest \cite{Chen2015,Bagchi2017,Pereira2013}. The rationale behind these proposals was that in a graded system, new asymmetric rectifying channels are created, while the long-range interactions create
also new transport channels, avoiding the usual decay of heat flow with size \cite{Chen2015}. Besides a stronger rectification power, LRI graded chains are expected to have better heat conductivity than segmented ones. This is an important point for technological applications, because devices with high rectification factors are not useful if the currents that flow through them are very small.

Another main focus of the theoretical research in thermal rectification is the search for the fundamental factors that contribute to the emergence of rectification. Historically, the crucial elements for having rectification have been the presence of some structural asymmetry in the system and of non-linear (anharmonic) forces \cite{Zeng2008,Katz2016,Li2008,Hu2006,Benenti2016,Li2012,Segal2005,Segal2005b}, which lead to a temperature dependence of the phonon bands or power spectral densities. A match or mismatch of the phonon bands of neighboring parts of the chain implies corresponding good or bad conduction so the
sign of the temperature bias may affect the conduction and lead to rectification when the spectra of the parts are affected differently by the bias reversal. However, more recent research pointed out that anharmonicity is not a necessary condition for an asymmetric match/mismatch and therefore for rectification \cite{Pereira2017}. Rectification also occurs in simple (minimalistic) harmonic models that incorporate some structural asymmetry and temperature-dependence of the model parameters \cite{Pereira2017}. This dependence may indeed result from an underlying, more intricate  anharmonic system by linearization of the stochastic dynamics \cite{Pereira2017,Pereira2019}, or it may have a different origin \cite{Simon2019}.


The contents of this part of the Thesis will be organized as follows. In chapter \ref{Chapter4}, I present a model of a thermal rectifier that relies on a localized impurity in the middle of a chain of atoms. In chapter \ref{Chapter5}, a proposal for a thermal rectifer in a chain of trapped ions with a graded frequency distribution is presented. Finally, in chapter \ref{Chapter6}, I study heat transport in a solvable model of two connected oscillators to explore the origin an optimization of thermal rectification.
