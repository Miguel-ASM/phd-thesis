%!TEX root = ../Thesis.tex

\chapter*{Introduction} % Write in your own chapter title
\label{Introduction}
\lhead{\emph{Introduction}} % Write in your own chapter title to set the page header

It is indisputable that the invention of the electric diode [] has played a major role in the technological advancement during the second half of the 20$^\text{th}$ century. In particular, they are one of the main responsibles, together to transistors, of the revolution of modern digital electronics. Among other applications of the diode, AC to DC current conversion has also made possible the advancement of electronics to the point that we know today.

A diode, also known as rectifier, is an electrical component that allows electric current to flow in one direction but acts as an insulator in the opposite one []. Typically, the fundamental part in a diode is a p-n junction [], which is a physical union of two different types of semiconductor materials. The first one of these semiconductors is a p-type semiconductor, i.e., a semiconductor that has been dopped with donor impurities and hence, electrical current is mainly transported by electrons. The other semiconductor is a n-type semiconductor i.e., a semiconductor that has been dopped with acceptor impurities and hence, electrical current is mainly transported by electron holes (electron vacancy with positive charge). In the absence of a external applied voltage, the junction reaches an equilibrium state in which the region near the two semiconductors interface is emptied of free charge carriers and an electric potential difference forms between the p and n semiconductors (with the n semiconductor being at higher potential). If a battery is connected with its positive pole connected to the p-side of the junction and its negative pole connected to the n-side of the junction, the potential difference between the semiconductors will decrease. If the voltage of the battery is increased further, the potential difference in the junction will eventually change its sign, allowing for a current to pass through the junction. If however, the junction is connected to the battery with its negative pole attached to the p-side and its positive pole attached to the n-side, the potential barrier in the junction will be increased for the free charge carriers. As a result, the p-n junction allows current to flow through the junction only when a forward bias voltage, i.e. the p-side connected to the positive pole, is applied to it and acts as an insulator otherwise.

The diode would fall in a more general category of devices that are called non-reciprocal or asymmetric devices. There are many devices like this in very different contexts. We have one way mirrors, unidirectional valves, atomic diodes, magnetooptic devices....


About nonreciprocal devices \cite{Deak2012}.

Atom diode \cite{Raizen2005,Ruschhaupt2004}

In this thesis, we explore the possibility of designing nano devices that implement a \textit{diodic} in other context different from semiconductor physics. We want to design devices which allow asymmetric transport of matter and energy. This thesis is divided in 2 parts, where each of the parts deals with a different objective.

In the first part we study the design of quantum potentials for a particle in a 1 dimensional motion that allow asymmetric particle flow: one way absorbers, one way mirrors,... . The minimal condition to have this kind of behaviour is having non-hermitian Hamiltonians. We also analyze what kind of hamiltonians, and which hamiltonian symmetries allow this kind of behavior.

In the second part of this thesis we study heat rectification in mesoscopical systems. The objective in this part is designing an analog device to the diode that allows heat to propagate in just one direction. Besides designing such devices, we investigate on which are the requirements for having heat rectification with a solvable model.




\section{Proposal of content}

\begin{itemize}
  \item PART I: Non-Hermitian Asymmetric Transmission Devices

    \begin{itemize}
      \item Chapter 1. \textbf{Asymmetric scattering by non-Hermitian potentials}.
      This chapter is based in the following article:

      \begin{itemize}
        \item Asymmetric scattering by non-Hermitian potentials
      \end{itemize}

      \item Chapter 2. \textbf{Symmetries of non-hermitian potentials}.
      This chapter is based in the following articles:

      \begin{itemize}
        \item Symmetries and invariants for non-Hermitian Hamiltonians
        \item $S$-matrix pole symmetries for non-Hermitian scattering Hamiltonians
        \item Symmetries of ($N \times N$) non-Hermitian Hamiltonian matrices
      \end{itemize}

      \item Chapter 3. \textbf{Physical Implementation of non-hermitian and non-local hamiltonians}.
      This chapter is based in the following article:

      \begin{itemize}
        \item Quantum-optical implementation of non-Hermitian potentials for asymmetric scattering.
      \end{itemize}

    \end{itemize}

  \item PART II: Heat Rectification in mesoscopical devices/systems

    \begin{itemize}
      \item Chapter 4. \textbf{Heat Rectification with local impurities}.
      This chapter is based in the following article:

      \begin{itemize}
        \item Local rectification of heat flux.
      \end{itemize}

      \item Chapter 5. \textbf{Heat Rectification in graded chains of trapped ions}.
      This chapter is based in the following article:

      \begin{itemize}
        \item Asymmetric heat transport in ion crystals
      \end{itemize}

      \item Chapter 6. \textbf{Rectification in a minimal model}.
      This chapter is based in the following article:

      \begin{itemize}
        \item Heat rectification with a minimal model of two harmonic oscillators
      \end{itemize}

    \end{itemize}

  \item Conclusions

  \item Appendices

  \item Biliography

\end{itemize}
