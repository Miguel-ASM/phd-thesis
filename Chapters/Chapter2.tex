%!TEX root = ../Thesis.tex
%Chapter 2

\chapter{$S$-matrix pole symmetries for non-Hermitian scattering Hamiltonians}
\label{Chapter2}
\lhead{Chapter 2. \emph{$S$-matrix pole symmetries for non-Hermitian scattering Hamiltonians}}

The complex eigenvalues of some non-Hermitian Hamiltonians, e.g. parity-time symmetric Hamiltonians, come in complex-conjugate pairs. We show that for non-Hermitian scattering Hamiltonians (of a structureless particle in one dimension) possesing one of four certain symmetries, the poles of the $S$-matrix eigenvalues in the complex  momentum plane are symmetric about the imaginary axis, i.e. they  are complex-conjugate pairs in complex-energy plane. This applies even to states which are not bounded eigenstates of the system, i.e. antibound or virtual states, resonances, and antiresonances. Example potentials with such symmetries are constructed and their pole structures and scattering properties are calculated.
%
\newpage
%
\section{Introduction}

Much work on non-hermitian (NH) physics has focused on PT-symmetric Hamiltonians, as they may have a purely real spectrum \cite{Bender1998}. More recently, other NH and non-PT Hamiltonians, have been shown to hold real eigenvalues \cite{Nixon2016,Chen2017,Yang2017}. Work on scattering by PT-symmetric potentials was at first rather scarce \cite{Muga2004,Ruschhaupt2005,Cannata2007,Znojil2015}. However, scattering has been later investigated intensely in connection with spectral singularities and reflection asymmetries for left or right incidence (i.e. unidirectional invisibility) \cite{Mostafazadeh2009,Longhi2014,Mostafazadeh2013}, in most cases restricting the analysis to local potentials. As was shown in chapter \ref{Chapter1}, different devices with asymmetrical scattering responses (i.e., with different transmission and/or reflection for right and left incidence in a 1D setting) are possible if one makes use of non-local potentials. Chapter \ref{Chapter1} provides the selection rules for the transmission and reflection coefficient asymmetries based on eight basic Hamiltonian symmetries. Four of theses symmetries are a standard conmutation between a symmetry operator $A$ and the Hamiltonian, eq. \eqref{eq:chapter1_symmetry}, and the other four are $A$-pseudohermicity, \eqref{eq:chapter1_pseudoSymmetry}. $A$ is an element of the Klein 4-group $\mathbf{K}_4 = \left\{1,\Pi,\Theta,\Theta\Pi\right\}$.

A set of works by A. Mostafazadeh \cite{Mostafazadeh2002,Mostafazadeh2002a,Mostafazadeh2002b} gives the sufficient and necesary conditions for hamiltonians with a discrete spectrum to have real or complex conjugate pairs of eigenvalues. Hamiltonians satisying eq. \eqref{eq:chapter1_pseudoSymmetry} for a hermitian linear operator $A$, \textit{i.e.} $A$-pseudohermicity have a spectrum in which the eigenvalues come in complex conjugate pairs, some of them can even be real-valued. Moreover, the results of \cite{Mostafazadeh2002b} show that if a Hamiltonian conmutes with a hermitian antilinear operator, then there will exist a hermitian linear operator $A$ for which the Hamiltonian is $A$-pseudohermitian and, therefore, it will have complex conjugate pairs of eigenvalues. However, an aspect uncovered in refs. \cite{Mostafazadeh2002,Mostafazadeh2002a,Mostafazadeh2002b} is that when the Hamiltonian is $A$-pseudohermitian, the poles of the scattering matrix can have the same properties as the discrete spectrum, they can also come in complex conjugate pairs.

This chapter aims at extending the results in chapter \ref{Chapter1} and refs. \cite{Mostafazadeh2002,Mostafazadeh2002a,Mostafazadeh2002b} in several directions:

i) I will provide an alternative characterization of the 8 symmetries formed by the elements of the Klein 4-group and the relations \eqref{eq:chapter1_symmetry}, \eqref{eq:chapter1_pseudoSymmetry} in terms of the invariance of $H$ with respect to the action of superoperators.

ii) Moreover,
four of these eight symmetries imply the same
type of pole structure of $S$-matrix eigenvalues in the complex momentum plane that was found for PT symmetry \cite{Muga2004},
namely, zero-pole correspondence at complex-conjugate points, and poles on the imaginary axis or forming symmetrical pairs with respect to the imaginary
axis. This configuration with poles located on the imaginary  axis or as symmetrical pairs has some important consequences. In particular, it provides stability of the real energy eigenvalues with respect to parameter variations of the potential. While a simple pole on the imaginary axis can move along that axis when a parameter is changed, it cannot move off this axis (since this would violate the pole-pair symmetry) or bifurcate. The formation of pole pairs occurs near special  parameter values for which two poles on the imaginary axis collide. When the poles are mapped to the energy complex plane $E = p^2/2m$, they have the same symmetry structure of complex conjugate pairs as the discrete eigenvalues of $A$-pseudohermitian Hamiltonian, which expands the results of refs. \cite{Mostafazadeh2002,Mostafazadeh2002a,Mostafazadeh2002b}.

The remainder of the chapter is organized as follows. In section \ref{sec:SymTheory} we characterize the symmetry operations defined in chapter \ref{Chapter1} as the invariance of the Hamiltonian with respect to the action of eight linear or antilinear superoperators. In section \ref{sec:SPoles} I discuss the physical consequences of the symmetries in the pole structure of the scattering matrix eigenvalues. Four symmetries are shown to lead to complex poles corresponding to real energies or conjugate (energy) pairs.  In section \ref{sep_pot_sec} I exemplify the general results with separable potentials exhibiting parity-pseudohermiticity and time-reversal symmetry. These are the two non-trivial symmetries of the four (in the sense that the other two, hermiticity and PT-symmetry, are already well discussed). In section \ref{sec:RealEigenConclusions} we discuss and summarize our results.
