%!TEX root = ../Thesis.tex
%IntroductionPartII

\chapter*{Introduction to Part II}
\addcontentsline{toc}{chapter}{Introduction to Part II}
\label{IntroductionPartII}
\lhead{Introduction to Part II} % Write in your own chapter title to set the page header

Radiation, heat and electricity are prominent mechanisms of energy transport. In particular, the two last mechanisms have a dominant role in technology. Modern information processing rests on electronic devices like the diode and the transistor. However, there is not an existing analogous technology to the diodes and transistors based on control of heat currents driven by phonons, \textit{i.e.} the quasiparticles describing vibrational modes. An explanation to this phenomenom could be that phonons are more difficult to control than electrons since, as oposed to electrons, they don't have mass or electrical charge.

Energy transport due to phonons has a rich diversity of physical mechanisms worth exploring to design phononic devices. Together with advancements in nanotechnology, this could boost the growth of phononics. The thermal rectifier, or thermal diode would be a primary building block to develop phononic devices and that is why I decided to study it.

Thermal rectification is the physical phenomenon, analogous to electrical current rectification in diodes, in which heat current through a device or medium (the thermal diode or rectifier) is not symmetric with respect to the exchange of the bath temperatures at the boundaries. It was  first observed in 1936 by Starr in a junction between copper and cuprous oxide \cite{Starr1936}. The theoretical work started much later  using as rectifiers simple anharmonic chain models
with different segments \cite{Terraneo2002,Li2004}. These papers sparked much research that continues to this day. Research on thermal rectification has gained a lot of attention in recent years as a key ingredient to build prospective devices to control heat flows similarly to electrical currents \cite{Roberts2011,Li2012}. There are  proposals to engineer thermal logic circuits \cite{Ye2017} in which information, stored in thermal memories \cite{Wang2008}, would be processed in thermal gates \cite{Wang2007}. Such thermal gates, as their electronic counterparts,  will require thermal diodes and thermal transistors  to operate \cite{Li2006,Joulain2016}.
Heat rectifying devices would also be quite useful in nano electronic circuits, letting delicate components dissipate heat while being protected from external heat sources \cite{Roberts2011}.

Most work on thermal diodes has been theoretical with only a few experiments
like \cite{Chang2006,Kobayashi2009,Leitner2013,Elzouka2017}.
A relevant attempt to build a thermal rectifier was based on a graded structure made of carbon and boron nitride nanotubes that transports heat between a pair of heating/sensing circuits \cite{Chang2006}. One of the ends of the nanotube is loaded with a deposition of another material, which makes the heat flow better from the loaded end to the unloaded end. However, rectifications were small, with rectification factors (relative
heat-flow differentials) around $7\%$.

Much of the theoretical effort in thermal rectification research has been aimed at improving the rectification factors and the features of the rectifiers. The first approach to designing thermal diodes consisted in using chains of oscillators segmented into two or more regions with different properties \cite{Terraneo2002,Li2004,Li2008,Hu2006}, which is reminiscent of the idea of the $p-n$ junction in electric diodes. It was soon realized, however, that the performance of segmented rectifiers was very sensitive to the size of the device, i.e., rectification decreases with increasing the length of the rectifier \cite{Hu2006}. To overcome this limitation two ideas were proposed. The first one consists in using graded rather than segmented chains, i.e., chains where some physical property varies continuously along the site position such as the mass of particles in the chain \cite{Wang2012,Chen2015,Romero-Bastida2017,Yang2007,Romero-Bastida2013,Dettori2016,Pereira2010,Pereira2011,Avila2013}. The second one uses chains with long-range interactions (LRI), such that all the sites in the lattice interact with each other \cite{Chen2015,Bagchi2017,Pereira2013}. The rationale behind was that in a graded system new asymmetric, rectifying channels are created, while the long-range interactions create
also new transport channels, avoiding the usual decay of heat flow with size \cite{Chen2015}. Besides a stronger rectification power, LRI graded chains are expected to have better heat conductivity than segmented ones. This is an important point for technological applications, because devices with high rectification factors are not useful if the currents that flow through them are very small.

Another main focus of the theoretical research in thermal rectification is the search for the fundamental factors that contribute to the emergence of rectification. Historically the crucial elements for having rectification were identified as the presence of some structural asymmetry in the system and of nonlinear (anharmonic) forces \cite{Zeng2008,Katz2016,Li2008,Hu2006,Benenti2016,Li2012,Segal2005,Segal2005b}, which lead to a temperature dependence of the phonon bands or power spectral densities. A match or mismatch of the phonon bands of neighboring parts of the chain implies corresponding good or bad conduction so the
sign of the temperature bias may affect the conduction and lead to rectification when the spectra of the parts are affected differently by the bias reversal. However more recent research in the topic pointed out  that anharmonicity is not a necessary condition for an asymmetric match/mismatch and therefore for rectification \cite{Pereira2017}. Rectification also occurs in simple (minimalistic) harmonic models that incorporate some structural asymmetry and temperature-dependence of the model parameters \cite{Pereira2017}. This dependence may indeed result from an underlying, more intricate  anharmonic system by linearization of the stochastic dynamics \cite{Pereira2017,Pereira2019}, or it may have a different origin \cite{Simon2019}.

In the chapters of this section I will present the contributions I made together with my group and other collaborators to the field of thermal rectification in chains of oscillators. In Chapter \ref{Chapter4} I present a model of a thermal rectifier that relies on a localized impurity in the middle of the chain. With this approach we make a proposal that diverges from the prevalent approach of using segmented chains. In Chapter \ref{Chapter5} we make a proposal for a thermal rectifer in a chain of trapped ions with a graded frequency distribution. This prototype seizes the combined power of long range interactions, which are naturally present in trapped ion chains due to the Coulomb force, and graded structures. This proposal brings the added value of being possible to be experimentally realized, since trapped ions are a solid technology in quantum technologies. Finally, in Chapter \ref{Chapter6} I study heat transport in a solvable model of two connected oscillators to explore the origin of thermal rectification. It is believed that non-linear interactions are needed in order to have rectification, since a system with non-linear interactions will have a temperature-dependent spectra. I show, however that it is possible to have temperature dependent features in a linear system that lead to rectification, in agreement with other works like \cite{Pereira2017}.
