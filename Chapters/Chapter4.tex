%!TEX root = ../Thesis.tex
%Chapter 4

\chapter{Local Rectification of Heat Flux}
\label{Chapter4}
\lhead{Chapter 4. \emph{Local Rectification of Heat Flux}} % Write in your own chapter title to set the page header
%
In this chapter, a model for an atom-chain thermal rectifier is presented. The atoms in the chain are trapped in on-site harmonic potentials, and interact with their nearest neighbours by Morse potentials (or also by harmonic potentials in a simplified version). The chain is homogeneous except for a local modification of the interactions and trapping potential at one site, the ``impurity''. The rectification mechanism is due here to the localized impurity, the only asymmetrical element of the structure, apart from the externally imposed temperature bias, and does not rely on putting in contact different materials or other known mechanisms such as grading or long-range interactions.  The effect survives if all interaction forces are linear except the ones for the impurity.

The rest of the chapter is organized as follows. In section \ref{sec:homogeneous_chain}, I shall describe the homogeneous 1D chain, without the impurity.  For this system, I numerically solve the dynamical equations, to show that the usual heat conduction applies. In section \ref{sec:Impurity_rectifier}, I modify the potentials for one of the atoms and demonstrate the rectification effect. I also observe rectification when all the interaction Morse potentials are substituted by harmonic oscillators. Finally, in section \ref{sec:chapter4_Discussion}, I summarize and discuss the results of this chapter.

\section{Homogeneous one-dimensional chain\label{sec:homogeneous_chain}}

I start with a homogeneous 1D chain with $N$ atoms coupled at both extremes to heat baths, at different temperatures $T_h$ and $T_c$ for ``hot" and ``cold" respectively. The baths are modeled with a Nos\' e-Hoover method as described in \cite{Martyna1992}. Atoms $1$ and $N$ represent the first and the $N$-th atom in the chain, from left to right, that will be in contact with the baths. All the atoms are subjected to on-site potentials and to nearest-neighbor interactions, and their equilibrium positions $y_{i0}$ are assumed to be equally spaced by a distance $a$.
$x_i= y_i-y_{i0}$,
$i=1,...,N$, represent the displacements from the equilibrium positions of the corresponding atoms
with positions $y_i$.

%%%%%%%%%%%%%%%%%%%%%%%
\begin{figure}
\centering
\includegraphics[width=0.65\linewidth]{Figures/FIG1.pdf}
\caption{(a) On-site potentials: harmonic potential centered at the equilibrium position of each atom (dashed blue line) as a function of the displacement from this position $x_i=y_i-y_{i0}$ in $a-$units, and the on-site potential for the impurity, $i=N/2+1$
($N$ even, red solid line). (b) Interaction potentials as a function of the distance between nearest neighbors: Morse potential
(blue dashed line) valid for all atoms except for $i=N/2+1$, $N$ even, where the modified potential (red solid line) is used.
The harmonic approximation of the Morse potential is also depicted (eq. (\ref{Vhar}), black dots, only used for fig. \ref{fig:chapter4_figure5}, below).
Parameters: $D=0.5$, $g=1$, $\gamma = 45$, $d=100$ and $b=105$, used throughout the chapter.
}
\label{fig:chapter4_figure1}
\end{figure}
%%%%%%%%%%%%%%%%%%%%%%%%%

The classical Hamiltonian of the atom chain can be written in a general form as
%
\begin{equation}
\label{GH}
%GH=general Hamiltonian
H=\sum_{i=1}^{N} H_i,
\end{equation}
%
with
%
\begin{eqnarray}
\label{GH2}
%GH=general Hamiltonian
H_1&=&{{p^2_1} \over {2m}} +U_1(x_1)+V_L,
\nonumber\\
H_i&=&{{p^2_i} \over {2m}} +U_i(x_i)+V_i(x_{i-1},x_i)  \quad i=2,...,N-1,
 \nonumber\\
H_N&=&{{p^2_N} \over {2m}} +U_N(x_N)+V_N(x_{N-1},x_N) + V_R,
\end{eqnarray}
%
where the $p_i$ are the momenta, $U_i(x_i)$ is the on-site potential for the $i$th atom, and $V_i(x_{i-1},x_i)$ represents the atom-atom interaction potential. $V_R$ and $V_L$ are the interactions coupling the boundary atoms to the Nos\'e-Hoover thermostats, see \cite{Martyna1992}.

%%%%%%%%%%%%%%%%%%%%%%%%%%%%
\begin{figure}
\centering
\includegraphics[width=0.65\linewidth]{Figures/FIG2.pdf}
\caption{Symmetric temperature profiles for a homogeneous chain, without impurity.  For $T_{h}=T_{L}$, $T_c=T_R$ (red solid dots) the (absolute value of) the heat flux is $J_{L\rightarrow R}$, equal to $J_{R\rightarrow L}$ for the reverse configuration of the bath temperatures, $T_{h}=T_{R}$, $T_c=T_L$
(black empty squares). Parameters as in fig. \ref{fig:chapter4_figure1}.}
\label{fig:chapter4_figure2}
\end{figure}
%%%%%%%%%%%%%%%%%%%%%%%%%%%%%%

There are a large number of 1D models that obey this general Hamiltonian. Different choices of the trapping and interaction potentials would give different conductivity behaviors. I choose a simple form of the Hamiltonian in which each atom is subjected to a harmonic on-site potential and a Morse interaction potential between nearest neighbors (see fig. \ref{fig:chapter4_figure1}, dashed lines),
%
\begin{eqnarray}
\label{HO}
%HO=Harmonic oscillator
U_i(x_i)&=&{1 \over 2} m \omega^2 x^2_i,
%\end{equation}
%\begin{equation}
\\
\label{IH}
%IP=Interaction potential
V_i(x_{i-1},x_i)&=&D\left \{e^{-\alpha [x_i-x_{i-1}]}-1\right \}^2,
\end{eqnarray}
%
where $\omega$ is the trapping angular frequency, and $D$ and $\alpha$ are time-independent parameters of the Morse potential.
A ``minimalist version'' of the model where $V$ becomes the harmonic limit of eq. (\ref{IH}), dotted line in fig. 1,
 will also be considered in the final discussion,
%
\begin{equation}
\label{Vhar}
{V}_i(x_{i-1},x_i)=k(x_i-x_{i-1})^2/2,\;k=2D\alpha^2.
\end{equation}
%
For convenience, dimensionless units are used and the mass of all particles is set to unity.

I start by studying the homogeneous configuration with no impurity and potentials (\ref{HO}) and (\ref{IH}), solving numerically the dynamical equations for  the Hamiltonian (\ref{GH}) with a Runge-Kutta-Fehlberg algorithm. I have chosen a low number of atoms, $N=20$,  with thermal baths at $T_h=0.20$ and $T_c=0.15$ at both ends of the chain with 16 thermostats each. The real temperature is related to the dimensionless one through $T_{real}=T m a^2 \omega^2/k_B$ so, for typical values  $m\approx10^{-26}$ kg, $\omega \approx 10^{13}$ s$^{-1}$, $a\approx 10^{-10}$ m, and using $k_B =1.38 \times 10^{-23}$ JK$^{-1}$,
the dimensionless temperatures $0.15,\, 0.20$, translate into $100,\, 150$ K. It is advisable to use temperatures around these values in order to ensure that the displacements of the particles are realistic \cite{Casati1984}.

%%%%%%%%%%%%%%%%%%%%%%%%%%%%%%%%%%
\begin{figure}
\centering
\includegraphics[width=0.65\linewidth]{Figures/FIG3.pdf}
\caption{Temperature profile along the homogeneous chain for different number of atoms: 100 (dotted black line), 125 (dashed blue line) and 150 (solid red line). The atom sites have been rescaled with the total number of atoms.
%, showing the convergence of the spatial profile of the local temperature $T_i$.
The time averages have been carried over a time interval of $\approx 2 \times 10^6$ after a transient of $\approx 1\times 10^5$. In the inset (a), the product $JN$ vs. $N$ demonstrates that for long chains $JN$ is independent of $N$. In (b) the linear dependence of $J$ with $\Delta T$ for a fixed number of atoms, $N=100$, is shown. Parameters as in fig. \ref{fig:chapter4_figure1}.}
\label{fig:chapter4_figure3}
\end{figure}
%%%%%%%%%%%%%%%%%%%%%%%%%%%%%%%%%%%%%%%%

First I demonstrate the conductivity behavior of the model.
%that our system satisfies Fourier's heat law for the heat flux, $J=\kappa \nabla T$.
%, so it shows normal thermal conductivity. ESTE CONCEPTO TIENE QUE VER CON EL TAMA�O?
To this end, I calculate the local heat flux $J_i$ and temperature $T_i$, performing the numerical integration
%of eq. (\ref{GH2})
for long enough times to reach the stationary state.
The local temperature is found as the time average $T_i= \langle p_i^2 / m \rangle$, whereas
%After a transient, the local temperature is given by the time average $T_i=\langle p_i^2\rangle$.
$J_i$,  from the continuity equation
%, $\dot H(x,t)+divJ(x,t)=0,$
\cite{Hu1998}, is given by
%
%Fourier law: temperature gradient vanishes with N
\begin{equation}
\label{heatflux}
J_i=-\dot x_i {{\partial V(x_{i-1},x_{i})} \over {\partial x_i}}.
\end{equation}
%
From now on I only consider the time average $\langle J_i (t)\rangle$, which converges to a constant value for all sites once the system is in the stationary nonequilibrium state. I depict the temperature profiles, for $N=20$, first with $T_L=T_h$ and $T_R=T_c$
($L$ and $R$ stand for left and right) and after switching the positions of the thermal baths in fig. \ref{fig:chapter4_figure2}. The profiles are symmetric, as expected, and the heat flux does not have a preferred direction  \cite{Hu1998,Terraneo2002}. Denoting the absolute values of the fluxes from the left (when $T_L=T_h$) as
$J_{L\rightarrow R}$, and from the right (when $T_R=T_h$) as
$J_{R\rightarrow L}$, I find that $J_{L\rightarrow R}=J_{R\rightarrow L}=J=1.6\times 10^{-2}$, in the dimensionless units, consistent with the values found in other models \cite{Terraneo2002,Hu1998}.

%%%%%%%%%%%%%%%%%%%%%%%%%%%%%%%%%%%
\begin{figure}
\centering
\includegraphics[width=0.65\linewidth]{Figures/FIG4b.pdf}
\caption{Temperature profile for the chain of $N=20$ atoms, with an impurity in the $N/2+1$ position, with $T_L=T_h$ and $T_R=T_c$ (circles) and with the thermostat baths switched (squares).
Parameters as in fig. \ref{fig:chapter4_figure1}.
(a) $T_c=0.15$, $T_h=0.2$. $J_{L\rightarrow R}=0.00769$ vs $J_{R\rightarrow L}=0.00581$, with gives a rectification $R=31 \% $; (b) $T_c=0.025$, $T_h=0.325$. $J_{L\rightarrow R}=0.0499$ vs  $J_{R\rightarrow L}=0.0140$, with $R=256 \%$.}
\label{fig:chapter4_figure4}
\end{figure}
%%%%%%%%%%%%%%%%%%%%%%%%%%%%%%%%%%%%%%%%%

The profile of the temperature is linear with boundary non-linearities at the edges, close to the thermal baths,  due to the boundary conditions \cite{Lepri1997}. In fig. \ref{fig:chapter4_figure3}, I depict $T_i$ vs $i/N$ for $N=100, 125$ and $150$ with the same boundary conditions. For these
larger atom numbers  I have connected the first 3 and the last 3 atoms to the Nos\'e-Hoover baths.
%The temperature gradient scales as $N^{-1}$, which is also true for many other different models \cite{Hu1998}.
In the inset (a) of fig. \ref{fig:chapter4_figure3}  the product $JN$ vs. $N$ is plotted, showing that for a low $N$ limit there is a well defined conductivity per unit length whereas for longer chains, $JN$ tends to be constant  which indicates a normal thermal conductivity independent of the length. Fixing the number of atoms to 100, as in the inset (b) of fig. \ref{fig:chapter4_figure3},  I observe a linear dependence between the flux and $\Delta T$.
%Fourier law, $J=\kappa \nabla T$, is fulfilled.

\section{Impurity-based thermal rectifier \label{sec:Impurity_rectifier}}

To rectify the heat flux I modify the potentials for site $j=N/2+1$ with $N$ even, as
%
%\begin{equation}
%\label{IMP1}
%IMP1=impurity in absolute position
%U_j(y_j,t)=d e^{-b [y_j(t)-y_{d}]} +ge^{-\gamma [y_j(t)-y_{j-1}(t)-\epsilon]}
%\end{equation}
%with $y_{d}=y_{d,0}-a/3$.  Written in terms of the displacements, $x_j$,
%
\begin{eqnarray}
\label{IMP}
%IMP=impurity
U_j(x_j,t)&=&d e^{-b [x_j(t)+a/3]},
\\
V_j(x_{j-1},x_j,t)&=&ge^{-\gamma [x_j(t)-x_{j-1}(t)+a/2]}.
\end{eqnarray}
%
All the parameters involved, $d, b$, and $g,\gamma$ are time-independent. In fig. \ref{fig:chapter4_figure1} the modifications introduced with respect to the ordinary sites are shown (solid lines).  The different on-site and interaction terms introduce soft-wall potentials
(instead of hard-walls to aid in integrating the dynamical equations) that make it difficult for the impurity to transmit its excitation to the left whereas left-to-right transmission is still possible.
This effect is facilitated by the relative size of the coefficients, $a/3<a/2$, that determine the position of the walls.
% that I fixed after some experimentation.
These positions imply that an impurity excited by a hot right bath cannot affect its left cold neighbour near its equilibrium position at the $j-1$ site.
However, if the left $j-1$ atom is excited from a hot bath on the left,
it can get close enough to the impurity to kick it and transfer kinetic energy.

\begin{figure}
\centering
\includegraphics[width=0.65\linewidth]{Figures/FIG5new.pdf}
\caption{Rectification factor $R$ as a function of the temperature difference between ends of the chain of atoms, $\Delta T$.
%The rectification factor shows a very strong dependency on $\Delta T$.
I have changed both $T_h$ and $T_c$ according to $T_c=0.15-(\Delta T-0.05)/2$ and $T_h=0.2+(\Delta T-0.05)/2$, with $N=20$,  keeping the rest of parameters as in fig. \ref{fig:chapter4_figure1}.
Interatomic potentials: Morse potential, eq. (\ref{IH}) (black line with circles, see the temperature profiles of extreme points in fig. \ref{fig:chapter4_figure4}); harmonic potential, eq. (\ref{Vhar}) (red line with squares).}
\label{fig:chapter4_figure5}
\end{figure}

After extensive numerical simulations, I have chosen the values of these parameters as in fig. \ref{fig:chapter4_figure1}, such that the conductivity in the forward direction, $J_{L\rightarrow R}$, and the rectification factor, defined as $R=(J_{L\rightarrow R}-J_{R\rightarrow L}) / J_{R\rightarrow L}\times 100$,
are both large for $T_h=0.2$, $T_c=0.15$. A large $R$ without a large $J_{L\rightarrow R}$ could in fact be useless \cite{Roberts2011}.
%($R=0$ would represent a perfectly symmetric heat conduction.).
Note that the parameters are not necessarily the optimal combination, which in any case would depend on the exact definition of ``optimal'' (technically on how $J_{L\rightarrow R}/J$ and $R$ are weighted and combined in a cost function and on the limits imposed on the
parameter values). This definition is an interesting question but it goes beyond the scope of this chapter, which is to demonstrate and discuss the effect of the localized impurity.

I have used again $N=20$ atoms connected to baths of 16 thermostats each, with the same temperatures as for the homogeneous chain, and numerically solved the dynamical equations
to calculate the local temperature and the heat flux for both configurations of the baths. The interatomic potential for the regular atoms is the Morse potential (\ref{IH}).
In fig. \ref{fig:chapter4_figure4}(a), the temperature profiles show a clear asymmetry between ${L\rightarrow R}$ and ${R\rightarrow L}$. Specifically, I find $J_{L\rightarrow R}=7.6 \times 10^{-3}$ and $J_{R\rightarrow L}=5.8 \times 10^{-3}$ which gives
$R=31\%$. The effect decays with longer chains,  with, for example, $R=19\%$ for $N=100$, and R=17.8\% for $N=150$.

\begin{figure}
\centering
\includegraphics[width=0.65\linewidth]{Figures/FIG6.pdf}
\caption{Temperature profile for a harmonic interacting chain of $N=20$ atoms, with an impurity in the $N/2+1$ position, with $T_L=T_h$ and $T_R=T_c$ (circles) and with the thermostat baths switched (squares), for (a) $\Delta T = 0.05$ and (b)  $\Delta T = 0.3$. The corresponding rectification factors are (a) $R=18\%$ and (b) $R=85\%$. Parameters regarding the impurity are the same as in fig. \ref{fig:chapter4_figure1}.
}
\label{fig:chapter4_figure6}
\end{figure}

These temperature profiles depend on the difference between the bath temperatures, see e.g. fig. \ref{fig:chapter4_figure4}(b). Increasing the temperature gap, but  keeping $T_h$ low enough so that the displacement of the atoms from their equilibrium positions is realistic, I find higher values of $R$. Figure \ref {fig:chapter4_figure5} shows the strong dependence of $R$ with $\Delta T$ (black circles). I have changed both $T_h$ and $T_c$ so that the mean temperature $(T_c+T_h)/2$ remains constant.

\section{Discussion\label{sec:chapter4_Discussion}}

I have presented  a scheme for thermal rectification using a one-dimensional chain of atoms which is homogeneous except
for the special interactions of one of them, the impurity, and the couplings with the baths at the boundaries. These proof-of-principle results for an impurity-based rectification mechanism may encourage further exploration of the impurity-based rectification, in particular of the effect of different forms for the impurity on-site potential and its interactions with neighboring atoms.
In contrast to the majority of chain models, the structural asymmetry in the present model is only in the impurity. The idea of a localized effect was already implicit in early works on a two-segment Frenkel-Kontorova
model \cite{Li2004,Hu2006}, where rectification depended crucially on the interaction constant coupling between the two segments.
However, the coupling interaction was symmetrical and the asymmetry was provided by the different nature
(parameters) of the segments put in contact.
Also different from common chain models are the potentials chosen here. Instead of using the Morse potential as an on-site model, see e.g.  \cite{Terraneo2002},
I have considered a natural setting where this potential characterizes the interatomic interactions,
and the on-site potential is symmetrical with respect to the equilibrium position, and actually harmonic.
The numerical results indicate that this model is consistent with normal conduction,
and also helps to isolate and identify the local-impurity mechanism for rectification.
In this regard it is useful to consider a further simplification, in the spirit of the minimalists models
proposed by Pereira \cite{Pereira2017}, so as to distill further the essence of the local rectification mechanism.
If the Morse interatomic interaction is substituted by the corresponding harmonic interaction, see the black dotted line in fig. \ref{fig:chapter4_figure1}(b), the rectification effect remains, albeit slightly reduced, see fig. \ref{fig:chapter4_figure5}. The chain is then perfectly linear with the only non-linear exception  localized
at the impurity.
The temperature dependent feature mentioned in \cite{Pereira2017} as the second necessary condition for rectification besides asymmetry, is here localized in the impurity too, and consists of a different
capability to transfer kinetic energy depending on the temperatures on both sides of the impurity.
Figure \ref{fig:chapter4_figure6} shows temperature profiles for the purely harmonic chain to be compared with the Morse-interaction
chain in fig. \ref{fig:chapter4_figure4}. Flatter profiles are found on both sides of the impurity, as corresponds to the abnormal transport expected for harmonic chains \cite{Lepri2003}. It would be interesting to combine the impurity effect with other rectification mechanisms (such as grading, long-range interactions, or use of different segments), or with more impurities in series to enhance further the rectification effect.

Even though the motivation was to mimic the effect of a localized atom diode that lets atoms pass only one way,
unlike the atom diode \cite{Ruschhaupt2004}, all interactions in the present model
are elastic. The model may be extended by adding an irreversible,  dissipative element so as to induce not only rectification but a truly Maxwell demon for heat transfer \cite{Skordos1992,Ruschhaupt2006}.
On the experimental side, one dimensional chains of neutral atoms in optical lattices can be implemented with cold atoms \cite{Bloch2005}.
An impurity with different internal structure could be subjected to a different on-site potential imprinted by a holographic mask \cite{Bakr2009}, and asymmetrical interatomic interactions could be implemented by trapping a controllable polar molecule or mediated by atoms in parallel lattices \cite{Gollub2014}.
