%% ----------------------------------------------------------------
%% Thesis.tex -- MAIN FILE (the one that you compile with LaTeX)
%% ----------------------------------------------------------------

% ------------------ Set up the document -----------------------------------------------------------
%\documentclass[a4paper, 11pt, twoside, openright]{Thesis} %en a4
\documentclass[a4paper, 12pt, twoside, openright]{Thesis}  %para reducir a b5

% ------------------ Include any extra LaTeX packages required -------------------------------
\usepackage[square, numbers, comma, sort&compress]{natbib}  % Use the "Natbib" style for the references in the Bibliography
\usepackage{verbatim}         % Needed for the "comment" environment to make LaTeX comments
\usepackage{vector}           % Allows "\bvec{}" and "\buvec{}" for "blackboard" style bold vectors in maths
\usepackage[table]{xcolor}

\usepackage{parskip}          %para que haga el indent al principio de cada parrafo
\setlength{\parindent}{15pt}
\usepackage{indentfirst}      %para que ponga el indent en el primer parrafo tambien, sino no lo hace

\usepackage{lscape}           %Poner una página apaisada (util para tablas muy anchas)

\usepackage{ragged2e}         %Forzar el full justify en las caption
\usepackage{caption}
\captionsetup[figure]{
  justification   = justified,
  singlelinecheck = off
}
\captionsetup[table]{
  justification   = justified,
  singlelinecheck = off
}

\hypersetup{urlcolor=blue, colorlinks=true}  % Colours hyperlinks in blue, but this can be distracting if there are many links.


% ------------------ Abreviaciones de comandos y nuevos comandos -----------------------------------
\newcommand {\absq}[1] {\left| #1 \right|^2}

\def\la{\langle}
\def\ra{\rangle}
\def\blankpage{
\newpage
\begin{equation}
  \nonumber
\end{equation}
\newpage}
\def\ssst{\scriptscriptstyle}   %%para hacer más pequeñas algunas ecuaciones

%% Para habilitar la gamma mayuscula con blackboard style
\newcommand{\bbGamma}{{\mathpalette\makebbGamma\relax}}
\newcommand{\makebbGamma}[2]{%
  \raisebox{\depth}{\scalebox{1}[-1]{$\mathsurround=0pt#1\mathbb{L}$}}%
}

% Para doublestruck 0
\usepackage[bb=boondox]{mathalfa}

% Para tachar
\usepackage{soul}

% Remove hyphen word split
% \hyphenpenalty=10000

% No number in part pages
\makeatletter
\renewcommand\part{%
  \if@openright
    \cleardoublepage
  \else
    \clearpage
  \fi
  \thispagestyle{empty}%   % Original »plain« replaced by »emptyx
  \if@twocolumn
    \onecolumn
    \@tempswatrue
  \else
    \@tempswafalse
  \fi
  \null\vfil
  \secdef\@part\@spart}
\makeatother


%% ----------------- START OF THE DOCUMENT ---------------------------------------------------------
\begin{document}

% ------------------ Set up the Title Page ---------------------------------------------------------
\begin{titlepage}
\thispagestyle{empty} %\vspace*{0.1cm}
% \hspace{-0.3cm}\includegraphics[scale=0.25]{ehu}\hspace{0.3cm}
{\centering
\includegraphics[scale=0.25]{ehu}
}
\bigskip
{\centering \large
\par \vspace{1.5cm}

\hrule\vspace*{0.3cm}

{\LARGE \bf {Asymmetric Quantum Devices\\and Heat Transport}}

\vspace{0.3cm}\hrule \vspace{2cm}
{\LARGE \bf{Miguel \'{A}ngel Sim\'{o}n Mart\'{i}nez}}\\
\vspace{1.25cm}
{\it{Supervisors:}} \\
\vspace{0.1cm}
{\large \bf {Prof. Juan Gonzalo Muga Francisco}}\\
{\large \bf {Prof. Maria Luisa Pons Barba}}\\
% \vspace{2.2cm}
\vfill
% \begin{figure}[h]
% {\centering {
%
% }\par}
% \end{figure}
% \vspace{1.0cm}
Departamento de Qu\'{\i}mica-F\'{\i}sica\\
Facultad de Ciencia y Tecnolog\'ia\\
Universidad del Pa\'is Vasco/Euskal Herriko Unibertsitatea\\ (UPV/EHU)\\
\vspace{1.0cm}
Leioa, Abril 2021\\
} \pagebreak
\end{titlepage}
       %we use the title page document


\setstretch{1.5}        % It is better to have smaller font and larger line spacing than the other way round


% --------- Define the page headers using the FancyHdr package and set up for one-sided printing

\fancyhead{}        % Clears all page headers and footers
\rhead{\thepage}    % Sets the right side header to show the page number
\lhead{\thepage}    % Sets the left side header to show the page number
\pagestyle{fancy}   % Finally, use the "fancy" page style to implement the FancyHdr headers
\clearpage          % Declaration ended, now start a new page

% Meto una pagina en blanco
\pagestyle{empty}   % No headers or footers for the following pages
\null\vfill

\vfill\vfill\vfill\vfill\vfill\vfill\null
\clearpage          % Empty page ended, start a new page



% ------------------ The "Funny Quote Page" --------------------------------------------------------
\pagestyle{empty}  % No headers or footers for the following pages

\null\vfill
% Now comes the "Funny Quote", written in italics
\textit{``Look at me. Look at me. I am the \textst{captain} doctor now.''}
\begin{flushright}
  Captain Phillips
\end{flushright}
\vfill\vfill\vfill\vfill\vfill\vfill\null
\clearpage  % Funny Quote page ended, start a new page
% ----------------------------------------------------------------

% Meto una pagina en blanco
\pagestyle{empty}  % No headers or footers for the following pages

\null\vfill

\vfill\vfill\vfill\vfill\vfill\vfill\null
\clearpage  % Empty page ended, start a new page

%% ----------------------------------------------------------------
\lhead{\emph{Contents}}  % Set the left side page header to "Contents"
\tableofcontents  % Write out the Table of Contents
\addtocontents{toc}{\protect\thispagestyle{empty}}
\pagenumbering{gobble}



\pagestyle{fancy}  %The page style headers have been "empty" all this time, now use the "fancy" headers as defined before to bring them back

\frontmatter	           % Begin Roman style (i, ii, iii, iv...) page numbering

% ------------------ The Acknowledgements, for thanking everyone -----------------------------------
% \addtocontents{toc}{\vspace{1em}}  % Add a gap in the Contents, for aesthetics
\input{./Chapters/agradecimientos}

% ------------------ The Abstract (spanish) --------------------------------------------------------
% \addtocontents{toc}{\vspace{1em}}  % Add a gap in the Contents, for aesthetics
\chapter{Resumen} % Write in your own chapter title
\label{Resumen}
\lhead{\emph{Resumen}} % Write in your own chapter title to set the page header

Los dispositivos que controlan el flujo de energía o materia desempeñan un papel destacado en la tecnología. Un dispositivo clave es el rectificador, que permite que las corrientes sólo vayan en una dirección. El más notable de estos dispositivos es el diodo eléctrico, que es una parte vital de los ordenadores, los dispositivos digitales y los sistemas de conversión de corriente AC/DC. Sin el diodo no existiría la mayor parte de la tecnología que tenemos hoy en día.

El diodo eléctrico es un componente eléctrico que permite que la corriente eléctrica fluya de forma asimétrica con respecto al signo de la diferencia de potencial que se le aplica. Normalmente, un diodo está compuesto por la unión de un semiconductor $p$ con un semiconductor $n$. Cuando se aplica un potencial de polarización directa $\Delta V$ a la unión $p$-$n$ (conectando el polo positivo de una batería al semiconductor $p$), la corriente eléctrica fluirá a través del diodo. Sin embargo, la unión $p$-$n$ actúa como un aislante eléctrico si se aplica un potencial de polaridad inversa $-\Delta V$.

El impacto tecnológico del diodo ha motivado el desarrollo de dispositivos análogos en otros escenarios físicos, como la óptica. Un equivalente óptico al diodo es el aislador óptico, que se utiliza para permitir la propagación unidireccional de la luz. Este dispositivo se basa en la rotación no recíproca de la dirección de polarización de la luz polarizada en materiales que se encuentran en un campo magnético, conocida como Rotación de Faraday. El aislador óptico es un componente crítico en los dispositivos ópticos para proteger las fuentes de luz delicadas de la retropropagación de la luz.

Llegados a este punto podemos ver que un ingrediente común entre los dispositivos que muestran un comportamiento similar al de un diodo es algún tipo de asimetría estructural interna. En el diodo eléctrico esta asimetría proviene de la distribución asimétrica de los portadores de carga: electrones en el lado $n$ y huecos en el lado $p$. En el aislador óptico, la orientación del campo magnético rompe la simetría del sistema.

El objetivo de esta Tesis es explorar la física y los posibles diseños de dispositivos que implementen un mecanismo rectificador para un transporte asimétrico de materia o energía. Esta Tesis se divide en dos partes: En la primera parte, estudio el scattering asimétrico de partículas por potenciales cuánticos unidimensionales y en la segunda parte, estudiaré la rectificación térmica en cadenas de osciladores. A continuación, presento una introducción a estas dos partes.

\section*{Parte I}


El interés actual por desarrollar nuevas tecnologías cuánticas está impulsando la investigación aplicada y fundamental sobre los fenómenos y sistemas cuánticos con posibles aplicaciones en circuitos lógicos, metrología, comunicaciones o sensores. Se necesitan dispositivos básicos robustos que realicen operaciones elementales para llevar a cabo tareas complejas cuando se combinan en un circuito. Con el desarrollo de nuevas tecnologías cuánticas en mente, el objetivo de esta parte de la Tesis es diseñar potenciales para el scattering unidimensional de una partícula cuántica sin espín que conduzcan a coeficientes de transmisión y reflexión que difieran para paquetes de ondas procedentes de la izquierda o de la derecha.


Para obtener scattering asimétrico, es necesario utilizar potenciales no hermíticos y no locales. Aunque los potenciales no locales y no hermíticos pueden parecer poco comunes y extraordinarios en la física cuántica para algunos, aparecen de forma natural cuando se aplican técnicas de partición para describir interacciones efectivas en un subespacio de un sistema mayor con un hamiltoniano hermítico. Los hamiltonianos no hermíticos que representan interacciones efectivas tienen una larga historia en física nuclear, atómica y molecular, y se han vuelto comunes en la óptica, donde las ecuaciones de onda en guías de onda podrían simular la ecuación de Schr\"{o}dinger. Los hamiltonianos no hermíticos también pueden establecerse fenomenológicamente, por ejemplo, para describir la disipación. Recientemente ha habido mucho interés en los hamiltonianos no hermíticos, en particular, en aquellos que tienen simetría PT por sus propiedades espectrales y sus útiles aplicaciones, sobre todo en óptica. Sin embargo, es importante destacar que existen simetrías diferentes a la PT y que son necesarias para producir ciertas formas de scattering asimétrico.

El contenido de esta parte de la Tesis está organizado como sigue. En el capítulo I, utilizaré potenciales no hermíticos y no locales para diseñar potenciales con coeficientes de scattering asimétricos. Las simetrías para los hamiltonianos no hermíticos se generalizarán utilizando el concepto de pseudohermiticidad y se utilizarán para derivar reglas de selección útiles para los coeficientes de transmisión y reflexión. En el capítulo II, derivaré un conjunto de propiedades de los valores propios de los potenciales de scattering que extienden los resultados anteriores para los hamiltonianos discretos no hermíticos utilizando las simetrías generalizadas. En el capítulo III, presentaré una posible realización física de los hamiltonianos de scattering asimétrica en un contexto de óptica cuántica.


\section*{Parte II}


La radiación, el calor y la electricidad son algunos de los principales mecanismos físicos de transporte de energía. En particular, los dos últimos mecanismos desempeñan un papel importante en la tecnología. El procesamiento moderno de la información se basa en dispositivos electrónicos como el diodo y el transistor. Sin embargo, no existe una tecnología análoga para controlar las corrientes de calor transportadas por fonones. Una explicación podría ser que los fonones son más difíciles de controlar que los electrones, ya que (al contrario que ellos) no tienen masa ni carga eléctrica. Sin embargo, sería interesante explorar el diseño de dispositivos fonónicos debido a la riqueza de diferentes mecanismos físicos que intervienen en el transporte de calor. El rectificador térmico, o diodo térmico, sería un componente elemental para el desarrollo de dispositivos fonónicos. En esta parte de la Tesis estudio la rectificación térmica en cadenas de osciladores, teniendo la posibilidad de diseñar un diodo térmico como motivación principal.

La rectificación térmica es el fenómeno físico, análogo a la rectificación de la corriente eléctrica en los diodos, en el que la corriente de calor a través de un dispositivo o medio (el diodo térmico o rectificador) no es simétrica con respecto al intercambio de las temperaturas de los baños térmicos con los que está en contacto. Fue observado por primera vez en 1936 por Starr en una unión entre cobre y óxido cuproso. Los trabajos teóricos se iniciaron mucho más tarde utilizando como rectificadores modelos simples de cadenas sgmentadas de osciladores anarmónicos. Estos trabajos desencadenaron mucha investigación que continúa hasta hoy. La investigación sobre la rectificación térmica ha ganado mucha atención en los últimos años como ingrediente clave para construir dispositivos que controlen los flujos de calor de forma similar a las corrientes eléctricas. Existen propuestas para diseñar circuitos lógicos térmicos en los que la información, almacenada en memorias térmicas, se procesaría en puertas lógicas térmicas. Estas puertas lógicas térmicas, al igual que sus homólogas electrónicas, requerirían diodos térmicos y transistores térmicos para funcionar.
Los dispositivos rectificadores de calor también serían muy útiles en los circuitos nanoelectrónicos, ya que permitirían a los componentes delicados disipar el calor mientras están protegidos de las fuentes de calor externas.

La mayoría de los trabajos sobre diodos térmicos han sido teóricos, con sólo unos pocos experimentos. Un intento relevante de construir un rectificador térmico se basó en una estructura graduada hecha de nanotubos de carbono y nitruro de boro que transporta el calor entre un par de circuitos de calefacción/sensores. Uno de los extremos del nanotubo está cubierto con una deposición de otro material, lo que hace que el calor fluya mejor desde el extremo cubierto al descubierto. Sin embargo, la rectificación obtenida fue pequeña, con factores de rectificación en torno al $7\%$.

Gran parte del esfuerzo teórico en la investigación de la rectificación térmica se ha dirigido a mejorar los factores de rectificación y las características de los rectificadores. La primera aproximación al diseño de diodos térmicos consistió en utilizar cadenas de osciladores segmentados en dos o más regiones con propiedades diferentes. Sin embargo, pronto se observó que el rendimiento de los rectificadores segmentados era muy sensible al tamaño del dispositivo: la rectificación disminuye al aumentar la longitud del rectificador. Para superar esta limitación se propusieron dos ideas. La primera consiste en utilizar cadenas escalonadas en lugar de segmentadas, es decir, cadenas en las que alguna propiedad física varía de forma continua a lo largo de la cadena, como por ejemplo la masa de las partículas que la componen. La segunda consiste en utilizar cadenas con interacciones de largo alcance, de forma que todos los elementos de la cadena interactúan con todos los demás. El fundamento de estas propuestas era que en un sistema escalonado se crean nuevos canales rectificadores asimétricos, mientras que las interacciones de largo alcance crean
también nuevos canales de transporte, evitando el habitual decaimiento del flujo de calor con el tamaño. Además de un mayor poder de rectificación, se espera que las cadenas escalonadas tengan una mejor conductividad térmica que las segmentadas. Este es un punto importante para las aplicaciones tecnológicas, ya que los dispositivos con altos factores de rectificación no son útiles si las corrientes que fluyen a través de ellos son muy pequeñas.

Otro foco principal de la investigación teórica en rectificación térmica es la búsqueda de los factores fundamentales que contribuyen a la aparición de la rectificación. Históricamente, los elementos cruciales para que haya rectificación han sido la presencia de alguna asimetría estructural en el sistema y de fuerzas no lineales (anarmónicas), que conducen a una dependencia con la temperatura de las bandas fonónicas. El solapamiento de las bandas fonónicas de las distintas partes de la cadena implica una buena o mala conducción térmica, por lo que el signo de la diferencia de temperatura aplicada puede afectar a la conducción y dar lugar a rectificación cuando. Sin embargo, investigaciones más recientes han señalado que la anarmonicidad no es una condición necesaria para un solapamiento asimétrico y, por tanto, para la rectificación. La rectificación también se produce en modelos armónicos simples (minimalistas) que incorporan alguna asimetría estructural y en la que los valores de algunos de sus parametros físicos dependen de la temperatura.

El contenido de esta parte de la Tesis está organizado como sigue. En el capítulo IV, presento un modelo de rectificador térmico que se basa en una impureza localizada en medio de una cadena de átomos. En el capítulo V, se presenta una propuesta de rectificador térmico en una cadena de iones atrapados con una distribución de frecuencia escalonada. Finalmente, en el capítulo VI, se estudia el transporte de calor en un modelo de dos osciladores conectados para explorar el origen y la optimización de la rectificación térmica.


% ------------------ The List of Publications ------------------------------------------------------
% \addtocontents{toc}{\vspace{1em}}  % Add a gap in the Contents, for aesthetics
\chapter{List of publications} % Write in your own chapter title
\label{Publications}
\lhead{\emph{List of publications}} % Write in your own chapter title to set the page header
\hspace{-0.5 cm}{{\large\bf \rm I)}} {\bf The results of this Thesis are based on the following articles}
\section*{Published Articles}

\begin{enumerate}

  \item M. Pons, Y. Y. Cui, A. Ruschhaupt, {\bf M. A. Sim\'{o}n} and J. G. Muga\\
  {\it Local rectification of heat flux}\\
  \href{https://doi.org/10.1209/0295-5075/119/64001}{EPL {\bf 119}, 64001 (2017).}

  \item A. Ruschhaupt, T. Dowdall, {\bf M. A. Sim\'{o}n} and J. G. Muga\\
  {\it Asymmetric scattering by non-Hermitian potentials}\\
  \href{https://doi.org/10.1209/0295-5075/120/20001}{EPL {\bf 120}, 20001 (2017).}

  \item {\bf M. A. Sim\'{o}n}, A. Buend\'{i}a and J. G. Muga\\
  {\it Symmetries and Invariants for Non-Hermitian Hamiltonians}\\
  \href{https://doi.org/10.3390/math6070111}{Mathematics {\bf 6}, 111 (2018).}

  \item {\bf M. A. Sim\'{o}n}, A. Buend\'{i}a, A. Kiely, A. Mostafazadeh and J. G. Muga\\
  {\it $S$-matrix pole symmetries for non-Hermitian scattering Hamiltonians}\\
  \href{https://doi.org/10.1103/PhysRevA.99.052110}{Phys. Rev. A {\bf 99}, 052110 (2019).}

  \item {\bf M. A. Sim\'{o}n}, S. Mart\'{i}nez-Garaot, M. Pons and J. G. Muga\\
  {\it Asymmetric heat transport in ion crystals}\\
  \href{https://doi.org/10.1103/PhysRevE.100.032109}{Phys. Rev. E {\bf 100}, 032109 (2019).}

  \item A. Alaña, S. Mart\'{i}nez-Garaot, {\bf M. A. Sim\'{o}n} and J. G. Muga\\
  {\it Symmetries of (${N \times N}$ ) non-Hermitian Hamiltonian matrices}\\
  \href{https://doi.org/10.1088/1751-8121/ab7781}{J. Phys. A: Math. Theor. {\bf 53}, 135304 (2020).}

\end{enumerate}

\section*{Preprints}

\begin{enumerate}

  \item A. Ruschhaupt, A. Kiely, {\bf M. A. Sim\'{o}n} and J. G. Muga\\
  {\it Quantum-optical implementation of non-Hermitian potentials for asymmetric scattering}\\
  \href{https://arxiv.org/abs/2008.01702}{arXiv:2008.01702 [quant-ph] (2020)}\\
  (\href{https://journals.aps.org/pra/accepted/2f07eYb5Ibe1ea6603ef59232d2f3864c12b1a62d}{Accepted for publication in Phys. Rev. A})

  \item {\bf M. A. Sim\'{o}n}, A. Alaña, M. Pons, A. Ruiz-Garc\'{i}a and J. G. Muga\\
  {\it Heat rectification with a minimal model of two harmonic oscillators}\\
  \href{https://arxiv.org/abs/2010.10432}{arXiv:2010.10432 [cond-mat.stat-mech] (2020)}

\end{enumerate}

\vspace{1.25 cm}

\hspace{-0.65  cm}{{\large\bf \rm II)}} {\bf Other articles produced during the Thesis period}
\section*{Published  Articles not included in this Thesis}

\begin{enumerate}

  \item M. Palmero, {\bf M. A. Sim\'{o}n} and D. Poletti\\
  {\it Towards Generation of Cat States in Trapped Ions Set-Ups via FAQUAD Protocols and Dynamical Decoupling}\\
  \href{https://doi.org/10.3390/e21121207}{Entropy {\bf 21}, 1207 (2019)}

  \item {\bf M. A. Sim\'{o}n}, M. Palmero, S. Mart\'{i}nez-Garaot and J. G. Muga\\
  {\it Trapped-ion Fock-state preparation by potential deformation}\\
  \href{https://doi.org/10.1103/PhysRevResearch.2.023372}{Phys. Rev. Research {\bf 2}, 023372 (2020)}

\end{enumerate}


% --------------------------------------------------------------------------------------------------
% ------------------ Include the chapters of the thesis, as separate files -------------------------
% --------------------------------------------------------------------------------------------------

\addtocontents{toc}{\vspace{2.0em}}

\mainmatter	       % Begin normal, numeric (1,2,3...) page numbering
\pagestyle{fancy}  % Return the page headers back to the "fancy" style
% ------------ INTRODUCTION ------------------------------------------------------------------------
%!TEX root = ../Thesis.tex

% \chapter*{Introduction} % Write in your own chapter title
\label{Introduction}
\lhead{\emph{Introduction}} % Write in your own chapter title to set the page header

Devices that control the flow of energy or matter play a prominent role in technology. A key device is the rectifier, which allows currents only one way. A rectifier behaves like a corridor with a trap door that can be opened from left to right but is closed otherwise. The most notable of such devices is the electric diode, which is a vital part of computers, digital devices, and AC/DC current conversion systems. Without the diode most of the technology that we have today would not exist.

The electric diode is an electrical component that allows electrical current to flow asymmetrically with respect to the sign of the potential difference that is applied to it. Typically, a diode is composed by the union of a $p$-semiconductor with an $n$-semiconductor. When a forward-bias potential $\Delta V$ is applied to the $p$-$n$ junction (by connecting the positive pole of a battery to the $p$-semiconductor), electrical current will flow through the diode. However, the $p$-$n$ junction acts as an electrical insulator if a reversed-bias potential $-\Delta V$ is applied.

Motivated by the technological impact of the diode, analogous devices have been developed in other physical scenarios, like optics. An optical equivalent to the diode is the optical isolator, which is used to allow one-way light propagation \cite{Saleh1991}. This device is based on the non-reciprocal rotation of the polarization direction of polarized light in materials that are in a magnetic field, known as Faraday Rotation (see ref. \cite{Yariv1984}). The optical isolator is a critical component in optical devices to protect delicate light sources from back-propagating light.

At this point we can see that a common ingredient between devices which show a diode-like behaviour is some kind of internal structural asymmetry. In the electric diode this asymmetry comes from the asymmetric distribution of charge carriers: electrons in the $n$-side, and holes in the $p$-side. In the optical isolator the orientation of the magnetic field breaks the symmetry of the system.

This Thesis is devoted to explore the physics and possible designs of devices that implement a \textit{diodic} or rectifying mechanism for an asymmetric transport of matter or energy. The Thesis
is divided into two parts: In part \ref{partI}, I look for asymmetric particle scattering of 1-dimensional quantum potentials and in part \ref{partII}, I will study thermal rectification in chains of oscillators. There follows an introduction to these two parts.


\section*{Introduction to part I: non-Hermitian systems and asymmetric scattering}

The current interest to develop new quantum technologies is boosting applied
and fundamental research on quantum phenomena and systems with potential
applications in logic circuits, metrology, communications or sensors. Robust basic devices performing elementary operations are needed to perform complex tasks when combined in a circuit. With the development of new quantum technologies in mind, the objective of this part is to design 1-dimensional scattering potentials for a quantum, spinless particle of mass $m$ that lead to transmission and reflection coefficients (squared modulus of the amplitudes) which differ for wave packets coming from the left or the right.

To find an asymmetric scattering behavior, I will use non-Hermitian and non-local potentials \cite{Muga2004,Mostafazadeh2018}. Although non-local and non-Hermitian potentials might seem uncommom and extraordinary in quantum physics to some, they appear naturally when applying partitioning techniques to describe the effective interactions in a subspace of a larger system with a Hermitian Hamiltonian by projection \cite{Feshbach1958,Ruschhaupt2004,Muga2004}. Non-Hermitian Hamiltonians representing effective interactions have a long history in nuclear, atomic, and molecular physics, and have become common in optics, where wave equations in waveguides could simulate the Schr\"odinger equation \cite{Ruschhaupt2005,Longhi2017a,Konotop2016}. Non-Hermitian Hamiltonians can also be set phenomenologically, e.g. to describe dissipation \cite{Ruschhaupt2005}. Recently there has been a lot of interest in non-Hermitian Hamiltonians \cite{Nixon2016,Nixon2016a,Chen2017,Ruschhaupt2017,Simon2018,Simon2019a,Alana2020,Bernard2002,Kawabata2019}, in particular, the ones having parity-time (PT) symmetry \cite{Bender1998,Znojil2015} because of their spectral properties and useful applications, mostly in optics  \cite{Longhi2017a,Konotop2016,Longhi2014}. However, I shall emphasize that symmetries different from PT exist and are necessary to produce certain forms of asymmetric scattering.

The contents of this part of the Thesis will be organized as follows. In chapter \ref{Chapter1}, I will use non-Hermitian and non-local potentials to design potentials with asymmetric scattering coefficients for left/right incidence. Symmetries for non-Hermitian Hamiltonians will be generalized using the concept of pseudohermiticity \cite{Mostafazadeh2002} and used to derive useful selection rules for the transmission and reflection coefficients. In chapter \ref{Chapter2}, I will derive a set of properties of the eigenvalues of scattering potentials that extend previous results for discrete non-Hermitian Hamiltonians by using the generalized symmetries. In chapter \ref{Chapter3}, I will present a possible physical realization for asymmetric scattering Hamiltonians in a quantum optics setup.


\section*{Introduction to part II: Heat rectification in mesoscopic systems}

Radiation, heat and electricity are prominent mechanisms of energy transport. In particular, the two last mechanisms play a dominant role in technology. Modern information processing rests on electronic devices like the diode and the transistor. However, there is not an analogous technology to control heat currents driven by phonons. An explanation could be that phonons are more difficult to control than electrons since (contrary to them) they do not have mass or electrical charge \cite{Li2012}. However, it would be interesting to explore the design of \textit{phononic} devices due to the richness of different physical mechanisms that mediate heat transport. The thermal rectifier, or thermal diode, would be a primary building block to develop \textit{phononics} \cite{Li2012}. In this part I study thermal rectification in chains of oscillators with the design of a thermal diode in mind.

Thermal rectification is the physical phenomenon, analogous to electrical current rectification in diodes, in which heat current through a device or medium (the thermal diode or rectifier) is not symmetric with respect to the exchange of the bath temperatures at the boundaries. It was  first observed in 1936 by Starr in a junction between copper and cuprous oxide \cite{Starr1936}. The theoretical work started much later using as rectifiers simple anharmonic chain models
with different segments \cite{Terraneo2002,Li2004}. These papers sparked much research that continues to this day. Research on thermal rectification has gained a lot of attention in recent years as a key ingredient to build prospective devices to control heat flows similarly to electrical currents \cite{Roberts2011,Li2012}. There are  proposals to engineer thermal logic circuits \cite{Ye2017} in which information, stored in thermal memories \cite{Wang2008}, would be processed in thermal gates \cite{Wang2007}. Such thermal gates, as their electronic counterparts,  would require thermal diodes and thermal transistors to operate \cite{Li2006,Joulain2016}.
Heat rectifying devices would also be quite useful in nanoelectronic circuits, letting delicate components dissipate heat while being protected from external heat sources \cite{Roberts2011}.

Most work on thermal diodes has been theoretical with only a few experiments (see refs. \cite{Chang2006,Kobayashi2009,Leitner2013,Elzouka2017}).
A relevant attempt to build a thermal rectifier was based on a graded structure made of carbon and boron nitride nanotubes that transports heat between a pair of heating/sensing circuits \cite{Chang2006}. One of the ends of the nanotube is covered with a deposition of another material, which makes the heat flow better from the covered end to the uncovered end. However, rectifications were small, with rectification factors around $7\%$.

Much of the theoretical effort in thermal rectification research has been aimed at improving the rectification factors and the features of the rectifiers. The first approach to designing thermal diodes consisted in using chains of oscillators segmented into two or more regions with different properties \cite{Terraneo2002,Li2004,Li2008,Hu2006}, which is reminiscent of the idea of the $p$-$n$ junction in electric diodes. However, it was soon noticed that the performance of segmented rectifiers was very sensitive to the size of the device, \textit{i.e.}, rectification decreases when increasing the length of the rectifier \cite{Hu2006}. To overcome this limitation two ideas were proposed. The first one consisted in using graded rather than segmented chains, \textit{i.e.}, chains where some physical property varies continuously along the site position such as the mass of particles in the chain \cite{Wang2012,Chen2015,Romero-Bastida2017,Yang2007,Romero-Bastida2013,Dettori2016,Pereira2010,Pereira2011,Avila2013}. The second one consisted in using chains with long-range interactions (LRI), such that all the elements in the chain interact with all the rest \cite{Chen2015,Bagchi2017,Pereira2013}. The rationale behind these proposals was that in a graded system, new asymmetric rectifying channels are created, while the long-range interactions create
also new transport channels, avoiding the usual decay of heat flow with size \cite{Chen2015}. Besides a stronger rectification power, LRI graded chains are expected to have better heat conductivity than segmented ones. This is an important point for technological applications, because devices with high rectification factors are not useful if the currents that flow through them are very small.

Another main focus of the theoretical research in thermal rectification is the search for the fundamental factors that contribute to the emergence of rectification. Historically, the crucial elements for having rectification have been the presence of some structural asymmetry in the system and of non-linear (anharmonic) forces \cite{Zeng2008,Katz2016,Li2008,Hu2006,Benenti2016,Li2012,Segal2005,Segal2005b}, which lead to a temperature dependence of the phonon bands or power spectral densities. A match or mismatch of the phonon bands of neighboring parts of the chain implies corresponding good or bad conduction so the
sign of the temperature bias may affect the conduction and lead to rectification when the spectra of the parts are affected differently by the bias reversal. However, more recent research pointed out that anharmonicity is not a necessary condition for an asymmetric match/mismatch and therefore for rectification \cite{Pereira2017}. Rectification also occurs in simple (minimalistic) harmonic models that incorporate some structural asymmetry and temperature-dependence of the model parameters \cite{Pereira2017}. This dependence may indeed result from an underlying, more intricate  anharmonic system by linearization of the stochastic dynamics \cite{Pereira2017,Pereira2019}, or it may have a different origin \cite{Simon2019}.


The contents of this part of the Thesis will be organized as follows. In chapter \ref{Chapter4}, I present a model of a thermal rectifier that relies on a localized impurity in the middle of a chain of atoms. In chapter \ref{Chapter5}, a proposal for a thermal rectifer in a chain of trapped ions with a graded frequency distribution is presented. Finally, in chapter \ref{Chapter6}, I study heat transport in a solvable model of two connected oscillators to explore the origin an optimization of thermal rectification.
 %Introduction

% ------------ PART I: NH-Devices ------------------------------------------------------------------
\part{Non-Hermitian systems and asymmetric scattering\label{partI}}
%!TEX root = ../Thesis.tex
%Chapter 1

\chapter{Asymmetric scattering by non hermitian potentials}
\label{Chapter1}
\lhead{Chapter 1. \emph{Asymmetric scattering by non hermitian potentials}}
%
\null\vfill
\textit{``You must unlearn what you have learned.''}
\begin{flushright}
  {\bf Master Yoda}\\
  The Empire Strikes Back
\end{flushright}
\vfill\null

The scattering of quantum particles by non hermitian (generally non-local)  potentials in one dimension may result in asymmetric transmission and/or reflection for left and right incidence.
Extending the concept of symmetry for non hermitian potentials, eight generalized symmetries based on the discrete  Klein's four-group
(formed by parity, time reversal, their product, and identity) are found. Together with generalized unitarity relations they determine selection rules for the possible and/or forbidden scattering asymmetries. Six basic device types are identified when the scattering coefficients adopt zero/one values, and transmission and/or reflection are asymmetric. They can pictorically be described as a
one-way mirror, a one-way barrier (a Maxwell  pressure demon), one-way (transmission or reflection) filters, a mirror with unidirectional transmission, and a transparent, one-way reflector. In this chapter I present potentials that have been designed to implement some of these devices and also demonstrate that the  behavior of the scattering
coefficients can be extended to a broad range of incident momenta.
%
\newpage
%

\section{Introduction\label{sec:chapter1_Introduction}}

In this chapter I study the properties of potentials with asymmetric transmission or reflection for a quantum, spinless particle of mass $m$ satisfying a one-dimensional (1D) Schr\"{o}dinger equation. I propose six types of asymmetric devices according to the asymmetries of the transmission/reflection coefficients, see fig. \ref{cases}. Non hermitian and non local potentials will be necessary to construct this kind of devices, therefore an important part of this chapter will consist in studying their properties. In particular their behaviour w.r. to symmetries such as parity, time-reversal and PT. Symmetries can be used, analogously to their standard application in atomic physics to determine selection rules for allowed/forbidden transitions, to predict whether a certain potential may or may not lead to asymmetric scattering. The concept of symmetry, however, must be generalized when dealing with non hermitian potentials.

The theory in this chapter is worked out for particles and the Schr\"odinger equation but it is clearly of relevance for optical devices
due to the much exploited analogies and connections between Maxwell's equations and the Schr\"odinger equation,
which were used, e.g., to propose  the realization of PT-symmetric potentials in optics \cite{Ruschhaupt2005}.

I will first start by introducing the formalism of quantum scattering in one dimension by non-hermitian potentials and the concept of left/right eigenvectors and biorthogonal partners, which will be fundamental to understand the rest of the chapter. It will be followed by the definition of generalized symmetries for non hermitian Hamitonians, that will be studied in detail for the 4-group composed by parity, time reversal, their product, and identity. I then derive the selection rules for the scattering coefficients corresponding to the different symmetries. Finally I will describe the types of asymmetric devices that can be built and how the different symmetries affect them.


\section{non hermitian scattering in 1 dimension \label{sec:chapter1_ScattFormalism}}

In this section I will put together the minimum set of ideas and tools needed to describe scattering in 1 dimension. The ideas in the section can be found with more detail in \cite{Muga2004}, which generalizes the results in the celebrated book by Taylor \cite{Taylor1972} to non hermitian hamiltonians. In scattering theory one deals with the usual Hamiltonian for a particle of mass $m$ subjected to the action of a potential $V$
%
\begin{equation}
  H =  H_0 + V,
\end{equation}
%
where $H_0=\frac{P^2}{2m}$ is the kinetic energy operator, with $P$ the momentum operator. The potential $V$ is not assumed to be either hermitian or local. Before continuing, I shall clarify these two statements. A linear operator $\mathcal{O}$ is hermitian, or also self-adjoint, if it is equal to its adjoint operator $\mathcal{O}^\dagger$. The definition of the adjoint of a linear operator is given by the equation
%
\begin{equation}
  \braket{\phi}{\mathcal{O}\psi} = \braket{\mathcal{O}^\dagger\phi}{\psi}\quad\forall\ket{\psi},\,\ket{\phi}\in\mathcal{H},
  \label{eq:chapter1_adjointLinearOperators}
\end{equation}
%
Where $\mathcal{H}$ is the Hilbert space of the particle. A potential $V$ with a diagonal representation in the position basis $\{\ket{x}\}$ of the particle's Hilbert space is said to be local
%
\begin{equation}
  \mel{x}{V_{local}}{x'} = \delta(x-x')V(x'),
  \label{eq:chapter1_localPotential}
\end{equation}
%
whereas a non local potential has off-diagonal elements
%
\begin{equation}
  \mel{x}{V_{non local}}{x'} = V(x,x').
  \label{eq:chapter1_nonLocalPotential}
\end{equation}
%
The potentials that we study are then non-diagonal and satisfy $V \neq V^\dagger$, therefore, the Hamiltonians will also be non-hermitian, $H \neq H^\dagger$.

Scattering theory tries to answer the following question: given an input (incident) wave packet $\ket{\psi_{in}}$, in which output (outcoming) wave packet $\ket{\psi_{out}}$ transforms into after interacting with the potential? Another way of formulating the same question is: given an incoming state with a momentum $p$, what are the probability amplitudes of the state being reflected and transmited elastically (with the same energy)? To answer these questions the scattering theory brings in the scattering states, eigenstates of the Hamiltonian which belong to the continuum part of the spectrum, \textit{i.e.}, eigenstates of the Hamiltonian which are not bounded to a finite portion of the space and behave asymptotically as plane waves far from the range of action of the potential. The scattering states represent incoming waves with momentum $p$ and energy $E_p = \frac{p^2}{2m}>0$ that are partially reflected and transmitted by the potential. For a plane wave with momentum $p>0$ incoming from the left which is transmited to the right and reflected back, the asymptotic scattering state is $\ket{p^+}$
%
\begin{equation}
  \braket{x}{p^+} =
  \begin{cases}
    \braket{x}{p} + R^l(p)\braket{x}{-p} \quad &\text{if}\; x \to -\infty
    \\
    T^l(p) \braket{x}{p} \quad &\text{if}\; x \to \infty
  \end{cases},
  \label{eq:chapter1_leftState}
\end{equation}
%
where $T^l(p)$ and $R^l(p)$ are the transmission and reflection coefficients for left incidence. $\braket{x}{p} = \frac{1}{ \sqrt{2\pi\hbar} } e^{-i p x /\hbar} $ is the unnormalized momentum eigenstate. Similarly, for a plane wave incoming from the right with momentum $-p \;(p>0)$ which is transmited to the left and reflected back, the scattering state is $\ket{-p^+}$
%
\begin{equation}
  \braket{x}{-p^+} =
  \begin{cases}
    T^r(p) \braket{x}{-p} \quad &\text{if}\; x \to -\infty
    \\
    \braket{x}{-p} + R^r(p)\braket{x}{p} \quad &\text{if}\; x \to \infty
  \end{cases},
  \label{eq:chapter1_rightState}
\end{equation}
%
where $T^r(p)$ and $R^r(p)$ are the transmission and reflection coefficients for right incidence. %Additionally to the scattering states in eqs. \eqref{eq:chapter1_leftState} and \eqref{eq:chapter1_rightState} it is possible to take a negative value of the momentum $p<0$ that will give formally correct scattering eigenstates $\ket{\pm p^-}$ of energy $E_p$ that are, however more difficult to interpret. Roughly, they can be interpreted as a time-reversed scattering process since they consist in a superpositions of 2 plane waves, one coming from the left and the other one from the right which after interacting with the potential recombine into an outgoing plane wave.
We shall see that in general $T^l\neq T^r$, $R^l\neq R^r$ for Hamiltonians that are nonhermitian and non-local. Taking the adjoint of the Hamiltonian $H^\dagger = \frac{P^2}{2m} + V^\dagger$, we can also obtain the scattering states and scattering coefficients (reflection and transmission) for left and right incidence for the adjoint: $\widehat{T}^l(p)$, $\widehat{R}^l(p)$, $\widehat{T}^r(p)$ and $\widehat{R}^r(p)$. In the rest of this thesis I will use the convention that hatted variables $\widehat{...}$ will refer to the adjoint hamiltonian $H^\dagger$. In the following, all the results I write down for $H$ can be obtained also for $H^\dagger$ by making the change $H \to H^\dagger$.

Scattering theory provides the method to obtain the scattering amplitudes through the transition operator $T_{op}(z)$, which is defined as
%
\begin{equation}
T_{op}(z) = V + VG(z)V,
\label{eq:chapter1_transitionOperator_definition}
\end{equation}
%
with $G(z) = (z-H)^{-1}$ being the Green's operator. In \cite{Muga2004} the transition operator is used to obtain the scattering eigenstates as
%
\begin{equation}
  \ket{\pm p^+} =  \ket{\pm p} + \lim_{ \varepsilon\to 0^+ } G^0(E_p + i\varepsilon)T_{op}(E_p + i\varepsilon)\ket{\pm p}\quad (p>0),
  \label{eq:chapter1_scatteringStateFormal}
\end{equation}
%
where $G^0(z) = (z-H_0)^{-1}$ is the Green's operator of the free propagation Hamiltonian $H_0$. Now, to obtain the scattering amplitudes of reflection and transmission, one has to take the limits of $\braket{x}{\pm p^+}$, where $\ket{\pm p^+}$ is given by eq. \eqref{eq:chapter1_scatteringStateFormal}, when $|x|$ goes to infinite and compare with eqs. \eqref{eq:chapter1_leftState}, \eqref{eq:chapter1_rightState} to get
%
\begin{align}
R^l&=-i\frac{2\pi m}{p}\la -p|T_{op}(+)|p\ra\nonumber,
\\
T^l&=1-i\frac{2\pi m}{p} \la p|T_{op}(+)|p\ra\nonumber,
\\
R^r&=-i\frac{2\pi m}{p}\la p|T_{op}(+)|-p\ra\nonumber,
\\
T^r&=1-i\frac{2\pi m}{p}\la -p|T_{op}(+)|-p\ra,
\label{eq:chapter1_amplitudesFromTOperator}
\end{align}
%
where $T_{op}(+)$ is a shorthand notation for $\lim_{ \varepsilon\to 0^+ } T_{op}(E_p + i\varepsilon)$. The same procedure can be followed using the transition operator $\widehat{T}_{op}$ for $H^\dagger$ to obtain $\widehat{T}^l(p)$, $\widehat{R}^l(p)$, $\widehat{T}^r(p)$ and $\widehat{R}^r(p)$. For what comes next, it is useful to introduce the on-shell scattering matrix, defined as
%
\begin{equation}
  \mathsf{S}(p) =
  \left(
  \begin{array}{cc}
    T^l(p)&R^r(p)
    \\
    R^l(p)&T^r(p)
  \end{array}
  \right),
  \label{eq:chapter1_onShellMatrix}
\end{equation}
%
which gives the reflected and transmited components for an incident plane wave. Waves propagating from left to right are represented as $\left(1,0\right)^\mathsf{T}$ and waves propagating from right to left as $\left(0,1\right)^\mathsf{T}$. Therefore, a left incident wave will be scattered to $\left(T^l(p),R^l(p)\right)^\mathsf{T}$ and a right incident wave to $\left(R^r(p),T^r(p)\right)^\mathsf{T}$.

As I mentioned before, one of the goals of scattering theory is connecting an input state $\ket{\psi_{in}}$ with an output state $\ket{\psi_{out}}$. In scattering theory the way of doing this is through the collision or scattering operator $S$
%
\begin{equation}
  \ket{\psi_{out}} = S \ket{\psi_{in}}
  \label{eq:chapter1_actuationOfSOperator}.
\end{equation}
%
$S$ is obtained as the product of the M\"{o}ller operators $S = \Omega_{-}^\dagger\Omega_{+}$ which are defined as
%
\begin{align}
    \Omega_+ &= \lim_{t \to -\infty}e^{i H t / \hbar}e^{-i H_0 t/ \hbar},\nonumber\\
    \Omega_- &= \lim_{t \to +\infty}e^{i H^\dagger t/ \hbar}e^{-i H_0 t/ \hbar}.
    \label{eq:chapter1_MollerDefinition}
\end{align}
%
The M\"{o}ller $\widehat{\Omega}_\pm$ and scattering $\widehat{S}$ operators for the adjoint Hamiltonian can be found by substituting $H$ for $H^\dagger$ in eq. \eqref{eq:chapter1_MollerDefinition}. The M\"{o}ller operator with the +(-) symbol connects the input (output) state with the scattering states of the Hamiltonian. The M\"{o}ller operators satisfy the isometry relation
%
\begin{equation}
  \Omega_\pm^\dagger \Omega_\pm = 1,
  \label{eq:chapter1_MollerIsometry}
\end{equation}
%
and the following intertwining equations with the complete and free Hamiltonians $H$, $H_0$
%
\begin{align}
  H\Omega_+ &= \Omega_+ H_0,\nonumber
  \\
  H^\dagger\Omega_- &= \Omega_- H_0.
  \label{eq:chapter1_MollerIntertwining}
\end{align}
%
The hatted quantities for the adjoint Hamiltonian have expressions similar to those in eqs. \eqref{eq:chapter1_MollerIsometry} and \eqref{eq:chapter1_MollerIntertwining} with the substitution $H\leftrightarrow H^\dagger$. Because of eq. \eqref{eq:chapter1_MollerIntertwining} and because the momentum eigenstates $\ket{p}$ are eigenstates of $H_0$ with energy $E_p$ we have that the scattering states in eqs. \eqref{eq:chapter1_leftState}, \eqref{eq:chapter1_rightState} are also given by
%
\begin{equation}
  \ket{\pm p^+} = \Omega_+ \ket{\pm p}\quad (p>0).
  \label{eq:chapter1_scatteringStateFromMoller}
\end{equation}
%
%
It is inmediate to see, that since the Hamiltonian may not be hermitian, the scattering operator will not be unitary in general. Because of this non-unitarity, input states may completely be absorved by the potential if the norm goes to zero. However, the scattering operator $S$ and the scattering operator $\widehat{S}$ of the adjoint Hamiltonian $H^\dagger$ satisfy the generalized unitarity relation
%
\begin{equation}
  \widehat{S}^\dagger S = S\widehat{S}^\dagger= 1.
  \label{eq:chapter1_SMatrixUnitarityGeneralized}
\end{equation}
%
The generalized unitary relation \eqref{eq:chapter1_SMatrixUnitarityGeneralized} implies a set of relations for the scattering amplitudes of non hermitian hamiltonians. To find these relations one needs to find the matrix elements of $S$, $\widehat{S}$ in the momentum basis. According to \cite{Muga2004} the matrix elements of the scattering operator are related to the transition operator by
%
\begin{equation}
    \bra{p} S \ket{p'} = \delta (p-p') -2 i \pi \delta (E_p-E_{p'}) \bra{p}T_{op}(+)\ket{p'},
    \label{eq:SmatrixElements}
\end{equation}
%
If we factor the dirac delta in momentum using that $\delta(p-p') = \frac{|p|}{m}\delta(E_{p}-E_{p'})$ and consider positive momentums $p$ and $p'$ we get
%
\begin{equation}
  \left(
  \begin{array}{cc}
    \mel{p}{S}{p'} & \mel{p}{S}{-p'}\\
    \mel{-p}{S}{p'} & \mel{-p}{S}{-p'}
  \end{array}
  \right)
  = \frac{|p|}{m}\delta(E_{p}-E_{p'})\mathsf{S}(p).
  \label{eq:chapter1_onShellMatrixRelationToS}
\end{equation}
%
Because of \eqref{eq:chapter1_onShellMatrixRelationToS}, the on-shell scattering matrices $\mathsf{S}$ and $\mathsf{\widehat{S}}$ (of the adjoint Hamiltonian) inherit the generalized unitarity relation of the scattering operators \eqref{eq:chapter1_SMatrixUnitarityGeneralized}, yielding the following useful relations for the scattering amplitudes
%
\begin{equation}
  \mathsf{\widehat{S}}^\dagger\mathsf{S} = \mathsf{1}
  \Longrightarrow
  \begin{cases}
    %
    \widehat T^l(p) T^{l*}(p) + \widehat R^l(p) R^{l*}(p) &= 1,
    %
    \\
    %
    \widehat T^r(p) T^{r*}(p) + \widehat R^r(p) R^{r*}(p) &= 1,
    %
    \\
    %
    \widehat T^{l*}(p) R^r(p) + T^r(p) \widehat R^{l*}(p) &= 0,
    %
    \\
    %
    T^l(p) \widehat R^{r*}(p) + \widehat T^{r*}(p) R^l(p) &= 0,
    %
  \end{cases}
  \label{eq:chapter1_unitarityCoefficients}
\end{equation}
%
The relations in \eqref{eq:chapter1_unitarityCoefficients} will be extremely relevant for the rest of the thesis, as they set extra conditions for the scattering amplitudes of asymmetry devices.

\subsection{Why is non-hermiticity needed for asymmetric scattering?}

Asymmetric scattering is achieved when $\left|T^l\right|\neq\left|T^r\right|$ and $\left|R^l\right|\neq\left|R^r\right|$. We shall see that if the Hamiltonian is hermitian $H = H^\dagger$ the later requirement can not be fulfilled. When the Hamiltonian is hermitian, the hatted ($\widehat{\cdot}$) quantities in eq. \eqref{eq:chapter1_unitarityCoefficients} are equal to the unhatted ones and therefore eq. \eqref{eq:chapter1_unitarityCoefficients} becomes
%
\begin{align}
  \left|T^l(p)\right|^2 +  \left|R^l(p)\right|^2  &= 1,\nonumber
  %
  \\
  %
  \left|T^r(p)\right|^2 +  \left|R^r(p)\right|^2  &= 1,\nonumber
  %
  \\
  %
   T^{l*}(p) R^r(p) + T^r(p)  R^{l*}(p) &= 0.
  %
  \label{eq:chapter1_unitarityCoefficients_HermitianCase}
\end{align}
%
Taking absolute values in the last equation in \eqref{eq:chapter1_unitarityCoefficients_HermitianCase} and solving for $T^{r}(p)$ gives
%
\begin{equation}
  \left|T^r(p)\right|  = \left| T^{l}(p) \frac{R^r(p)}{R^{l}(p)} \right|.
\end{equation}
%
Now, because of the first equation in \eqref{eq:chapter1_unitarityCoefficients_HermitianCase} we get $\left|R^r(p)\right| = \left|R^l(p)\right|$. Finally, substracting the two first equations in \eqref{eq:chapter1_unitarityCoefficients_HermitianCase} one arrives at $\left|T^r(p)\right| = \left|T^l(p)\right|$. Therefore, it is impossible to build asymmetric devices with hermitian Hamiltonians.

\section{Right and left eigenvectors of non-hermitian Hamiltonians\label{sec:chapter1_LeftAndRightEigenstates}}

Eigenstates in the discrete part of the spectrum (bound-in-space eigenstates) of a hermitian Hamiltonian corresponding to different eigenvalues are orthogonal, \textit{i.e.} $\braket{E_i}{E_j} =0$ if $E_i\neq E_j$. Assuming there are no degenerate eigenvalues (for simplicity) one can choose the normalization $\braket{E_i}{E_j} =\delta_{ij}$, which makes the  eigenstates an orthonormal set. Similarly, the scattering states $\ket{p^+}$ satisfy (in the hermitian case) $\braket{p^+}{p'^+} =\delta(p-p')$. The discrete and scattering eigenstates are orthonormal bases in their corresponding subspaces and can be used to form a basis of the complete hilbert space in the following way
%
\begin{equation}
  1 = \sum_i \ketbra{E_i}{E_i} + \int dp\; \ketbra{p^+}{p^+}
  \label{eq:chapter1_HermitianHamiltonianBasis}
\end{equation}
%
However, if the Hamiltonian is not hermitian, the completeness formula \eqref{eq:chapter1_HermitianHamiltonianBasis} will not be correct anymore. It is possible, however, to generalize \eqref{eq:chapter1_HermitianHamiltonianBasis} to the hermitian case with the concept of left eigenvectors and bi-orthogonal partners. We say that the vectors $\ket{\lambda}$ and $\ket{\widehat{\lambda}}$ are a right and a left pair of eigenvectors if there exists a $\lambda$ eigenvalue such that
%
\begin{align}
  H \ket{\lambda} &= \lambda \ket{\lambda},\nonumber\\
  \bra{\widehat{\lambda}} H  &= \bra{\widehat{\lambda}}\lambda.
  \label{eq:chapter1_leftAndRightEigenvector}
\end{align}
%
If one takes two different eigenvalues $\lambda_1$ and $\lambda_2$, it is easy to show that $\braket{\widehat{\lambda_1}}{\lambda_2}=\braket{\widehat{\lambda_2}}{\lambda_1}=0$, and for this reason, the pairs $\ket{\lambda},\,\ket{\widehat{\lambda}}$ are called biorthogonal partners. This result is applied to scattering hamiltonians in the following way. If we assume, for simplicity, that there is no degeneracy in the discrete spectrum of $H$, we can choose the normalization $\braket{\widehat{E_i}}{E_j} = \delta_{ij}$ for the point spectrum eigenvectors. The right scattering states are given by eq. \eqref{eq:chapter1_scatteringStateFromMoller}. Using the intertwining relations \eqref{eq:chapter1_MollerIntertwining} for the hatted Moller operators (after the $H\leftrightarrow H^\dagger$ substitution) one can prove that the states $\ket{\widehat{p}^+} = \widehat{\Omega}_+\ket{p}$ are left eigenvectors of $H$ with eigenvalue $E_p$. The isometry relation \eqref{eq:chapter1_MollerIsometry} implies $\braket{\widehat{p}^+}{q^+} = \delta(q-p)$, therefore the pairs $\ket{\widehat{p}^+}$, $\ket{p^+}$ are biorthogonal partners. Finally, the completeness formula of the Hilbert space \eqref{eq:chapter1_HermitianHamiltonianBasis} is generalized to the non hermitian case as
%
\begin{equation}
  1 = \sum_i \ketbra{E_i}{\widehat{E}_i} + \int dp\; \ketbra{p^+}{\widehat{p}^+}
  \label{eq:chapter1_NonHermitianHamiltonianBasis}
\end{equation}
%

\section{Generalized symmetries}

For hermitian  Hamiltonians, symmetries are represented by the commutation of some symmetry operator $A$ with the Hamiltonian. According to Wigner theorem \cite{Wigner1959}, a symmetry operator $A$ has to be either unitary or antiunitary. Unitary and antiunitary operators are defined by the relation $A^\dagger A = A A^\dagger = 1$, however, the meaning of the hermitian adjoint $^\dagger$ is different in each case. On one hand, unitary operators are linear, and therefore the action of the adjoint is defined by eq. \eqref{eq:chapter1_adjointLinearOperators}. On the other hand, antiunitary operators are antilinear, in which case the action of the adjoint is defined by
%
\begin{equation}
  \braket{\phi}{\mathcal{O}\psi} = \braket{\mathcal{O}^\dagger\phi}{\psi}^*\quad\forall\ket{\psi},\,\ket{\phi}\in\mathcal{H}.
  \label{eq:chapter1_adjointAntiLinearOperators}
\end{equation}
%
In scattering theory, symmetry plays an important role  as it implies relations among
the S-matrix elements beyond those implied by its unitarity, see e.g. \cite{Taylor1972} and, for scattering in one dimension, Sec. 2.6 in \cite{Muga2004}. Symmetries are also useful for  non hermitian Hamiltonians, but the mathematical and conceptual
framework must be generalized. We consider that a unitary or antiunitary operator $A$ represents a symmetry of $H$ if it satisfies at least one of these relations,
%
\begin{eqnarray}
  AH&=&HA,
  \label{eq:chapter1_symmetry}
  \\
  AH&=&H^\dagger A,
  \label{eq:chapter1_pseudoSymmetry}
\end{eqnarray}
%
For a right eigenstate of $H$, $|\psi\rangle$,
with eigenvalue $E$, eq. (\ref{eq:chapter1_symmetry}) implies that
$A|\psi\rangle$ is also a right  eigenstate of $H$, with the
same eigenvalue if $A$ is unitary, and with the complex conjugate eigenvalue $E^*$ if $A$ is antiunitary.
Equation (\ref{eq:chapter1_pseudoSymmetry}) implies that $A|\psi\rangle$ is a right eigenstate of $H^\dagger$
with eigenvalue $E$ for $A$ unitary or $E^*$ for $A$ antiunitary, or a left eigenstate of $H$ with eigenvalue $E^*$ for $A$ unitary, or $E$
for $A$ antiunitary. For real-energy scattering
eigenfunctions in the continuum, the ones we are interested in here, $E^*=E$.
When eq. (\ref{eq:chapter1_pseudoSymmetry}) holds we say that $H$ is $A$-pseudohermitian \cite{Mostafazadeh2010}.
Parity-pseudohermiticity has played an important role as being equivalent to space-time reflection (PT) symmetry for {\it local} potentials
\cite{Mostafazadeh2010,Znojil2015}. A large set of these equivalences
will be discussed below.
A relation of the form (\ref{eq:chapter1_pseudoSymmetry}) has been also used with differential operators  to get real spectra beyond
PT-symmetry for local potentials  \cite{Nixon2016,Nixon2016a}.

Here we consider
%that $H$ may be non-local, and
$A$ to be a member of the
Klein 4-group $K_4=\{1,\Pi, \Theta, \Pi\Theta\}$ formed by unity, the parity operator $\Pi$, the antiunitary time-reversal operator $\Theta$, and their product
$\Pi\Theta$. The Klein 4-group is a discrete, abelian group and every element $A$ satisfies $A^2 = 1$. We also assume that the  Hamiltonian is  of the form $H=H_0+V$, with $H_0$, the kinetic energy operator of the particle,
being hermitian and
satisfying $[H_0,A]=0$ for all members of the group, whereas the potential $V$ may be non-local in position representation.
The  motivation to use Klein's group is that the eight relations implied by eqs. (\ref{eq:chapter1_symmetry}) and (\ref{eq:chapter1_pseudoSymmetry}) generate all
possible symmetry transformations of a non-local potential due to the identity, complex conjugation, transposition, and sign inversion,
both in coordinate or momentum representation, see 3rd and 4th columns of table \ref{tab:chapter1_SymmetriesTable}, where each symmetry has been labeled by a roman number.

% ----------------------------------------------------------------------------------------------
\begin{landscape}
  \begin{table}
    \caption{Symmetries of the potential classified in terms of the commutativity or pseudo-hermiticity of $H$ with the elements of
    Klein's 4-group  $\{1,\Pi,\theta,\Pi\theta\}$ (second column). The first column sets a simplifying roman-number code for each symmetry.
    The relations among potential matrix elements are given in coordinate and momentum representations in the third and fourth columns.
    The fifth column gives the relations they imply in the matrix elements of $S$ and/or $\widehat{S}$ matrices ($S$ is for scattering by $H$
    and $\widehat{S}$ for scattering by $H^\dagger$). From them  the next four columns set the relations implied on scattering amplitudes.
    Together with generalized unitarity relations (\eqref{eq:chapter1_unitarityCoefficients}) they also imply relations for the moduli (tenth column), and phases (not shown). The last two columns indicate the possibility to achieve perfect asymmetric transmission or reflection:  ``${P}$" means possible (but not necessary),
    ``No'' means impossible.  In some cases ``$P$" is accompanied by a condition that must be satisfied.\vspace*{.2cm}
    \label{tab:chapter1_SymmetriesTable}}
    \centering
    \scalebox{.8}{
    \begin{tabular}{cccccccccccc}
      \hline\hline\\
      (1)&(2)&(3)&(4)&(5)&(6)&(7)&(8)&(9)&(10)&(11)&(12)\\
      Code & Symmetry&  $\langle x|V|y\rangle=$ & $\langle p|V|p'\rangle=$ & $\langle p|S|p'\rangle=$ & $T^l\!=$ & $T^r\!=$ & $R^l\!=$& $R^r\!=$& from eq. (\eqref{eq:chapter1_unitarityCoefficients})&$|T^l|\!=\!1$&$|R^l|\!=\!1$
      \\
      &&&&&&&&&&$|T^r|\!=\!0$&$|R^r|\!=\!0$
      \\
      \hline
      I & $1H=H1$ &   $\langle x|V|y\rangle$ & $\langle p|V|p'\rangle$ & $\langle p|S|p'\rangle$ & $T^l$ & $T^r$ & $R^l$ & $R^r$ & & $P$ & $P$
      \\
      II & $1H=H^\dagger 1$ &  $\langle y|V|x\rangle^*$ & $\langle p'|V|p\rangle^*$ &$\langle p|\widehat{S}|p'\rangle$ & $\widehat{T}^l$& $\widehat{T}^r$ & $\widehat{R}^l$ & $\widehat{R}^r$
      & $|T^l|\!=\!|T^r|$, $|R^l|\!=\!|R^r|$&No&No
      \\
      III & $\Pi H=H\Pi$ &  $\langle -x|V|-y\rangle$ & $\langle -p|V|-p'\rangle$ &$\langle -p|S|-p'\rangle$ & $T^r$ & $T^l$ & $R^r$ & $R^l$& $|T^l|\!=\!|T^r|$,$ |R^l|\!=\!|R^r|$ &No&No
      \\
      IV & $\Pi H=H^\dagger \Pi$ &  $\langle -y|V|-x\rangle^*$ & $\langle -p'|V|-p\rangle^*$ & $\langle -p|\widehat{S}|-p'\rangle$ & $\widehat{T}^r$ & $\widehat{T}^l$ & $\widehat{R}^r$ & $\widehat{R}^l$&&$P$, $R^rR^{l*}=1$&$P$, $T^r{T^l}^*=1$
      \\
      V & $\Theta H=H\Theta$ &  $\langle x|V|y\rangle^*$& $\langle -p|V|-p'\rangle^*$ & $\langle -p'|\widehat{S}|-p\rangle$ & $\widehat{T}^r$ & $\widehat{T}^l$ & $\widehat{R}^l$& $\widehat{R}^r$
      &$|R^l|=|R^r|$&$P$, $|R^{r,l}|=1$&No
      \\
      VI & $\Theta H=H^\dagger\Theta$ &  $\langle y|V|x\rangle$& $\langle -p'|V|-p\rangle$ & $\langle -p'|S|-p\rangle$ & $T^r$& $T^l$ & $R^l$& $R^r$&$|T^l| = |T^r|$&No&$P$
      \\
      VII & $\Theta\Pi H=H\Theta \Pi$ &  $\langle -x|V|-y\rangle^*$ & $\langle p|V|p'\rangle^*$ & $\langle p'|\widehat{S}|p\rangle$ &$\widehat{T}^l$& $\widehat{T}^r$ & $\widehat{R}^r$& $\widehat{R}^l$&$|T^l|=|T^r|$&No&$P$, $|T^{r,l}|=1$
      \\
      VIII& $\Theta\Pi H=H^\dagger \Theta \Pi$ &  $\langle -y|V|-x\rangle$ & $\langle p'|V|p\rangle$ & $\langle p'|S|p\rangle$ & $T^l$ & $T^r$ & $R^r$ & $R^l$&$|R^l|=|R^r|$&$P$&No
      \\
      \hline\hline
    \end{tabular}}
  \end{table}
\end{landscape}
% ----------------------------------------------------------------------------------------------


\begin{table}
  \caption{Transformation rules of the M\"oller and scattering operators under symmetries/pseudo-symmetries with linear or antilinear operators.}
  \label{tab:chapter1_MollerOperatorSymmetries}
  \centering
  \begin{tabular}{lcc}
  \hline\hline\\
  \textbf{Type of symmetry} & \textbf{$A$ linear} & \textbf{$A$ antilinear}

  \\
  \hline
  $A H = H A$
  &
  $
  \begin{array}{ccc}
    &&
    \\
    A \Omega_{\pm}&=&\Omega_{\pm} A
    \\
    A S&=&S A
    \\
    &&
  \end{array}
  $
  &
  $
  \begin{array}{ccc}
    &&
    \\
    A \Omega_{\pm}&=&\widehat{\Omega}_{\mp} A
    \\
    A S&=&\widehat{S}^\dagger A
    \\
    &&
  \end{array}
  $
  \\
  $A H = H^\dagger A$
  &
  $
  \begin{array}{ccc}
    &&
    \\
    A \Omega_{\pm}&=&\widehat{\Omega}_{\pm} A
    \\
    A S&=&\widehat{S} A
    \\
    &&
  \end{array}
  $
  &
  $
  \begin{array}{ccc}
    &&
    \\
    A \Omega_{\pm}&=&\Omega_{\mp} A
    \\
    A S&=&S^\dagger A
    \\
    &&
  \end{array}
  $
  \\
  \hline\hline
  \end{tabular}
\end{table}

With the definitions of symmetry in eqs. \eqref{eq:chapter1_symmetry}, \eqref{eq:chapter1_pseudoSymmetry} and the tools from scattering formalism in \ref{sec:chapter1_ScattFormalism} we can now derive the effects of the symmetries in the scattering operator $S$, which will pose new restrictions in the scattering amplitudes. We start by deriving the transformation rules of the M\"{o}ller operators \eqref{eq:chapter1_MollerDefinition} under a symmetry operator $A$ of the Klein 4-group. The M\"{o}ller operators are transformed differently depending on whether $A$ is unitary or antiunitary and whether the Hamiltonian obeys usual symmetry \eqref{eq:chapter1_symmetry} or pseudohermiticity \eqref{eq:chapter1_pseudoSymmetry}. For example, for $A$ unitary and \eqref{eq:chapter1_symmetry} we have
%
\begin{equation}
  \begin{split}
    A \Omega_+ &= A \lim_{t \to -\infty} e^{i H t/\hbar}e^{-i H_0 t/\hbar}\\
    &= \lim_{t \to -\infty}e^{i H t/\hbar}Ae^{-i H_0 t/\hbar}\\
    &= \lim_{t \to -\infty} e^{i H t/\hbar}e^{-i H_0 t/\hbar}A\\
    &= \Omega_+ A,
  \end{split}
  \label{eq:chapter1_MollerPositiveTransformAUnitary_Symmetry}
\end{equation}
%
and
%
\begin{equation}
  \begin{split}
    A \Omega_- &= A \lim_{t \to +\infty} e^{i H^\dagger t/\hbar}e^{-i H_0 t/\hbar}\\
    &= \lim_{t \to +\infty}e^{i H^\dagger t/\hbar}Ae^{-i H_0 t/\hbar}\\
    &= \lim_{t \to +\infty} e^{i H^\dagger t/\hbar}e^{-i H_0 t/\hbar}A\\
    &= \Omega_- A.
  \end{split}
  \label{eq:chapter1_MollerNegativeTransformAUnitary_Symmetry}
\end{equation}
%
Using eqs. \eqref{eq:chapter1_MollerPositiveTransformAUnitary_Symmetry} and \eqref{eq:chapter1_MollerNegativeTransformAUnitary_Symmetry} we finally get
%
\begin{equation}
  \begin{split}
    A S &= A \Omega_{-}^\dagger\Omega_{+} \\
    &= \Omega_{-}^\dagger A \Omega_{+} \\
    &= \Omega_{-}^\dagger\Omega_{+} A \\
    &= S A,\\
    &\Downarrow\\
    S &= A^{\dagger}SA.
  \end{split}
  \label{eq:chapter1_Moller-TransformAUnitary_Symmetry}
\end{equation}
%
The rest of combinations of the type of $A$ (unitary/antiunitary) with the type of symmetry (\eqref{eq:chapter1_symmetry} or \eqref{eq:chapter1_pseudoSymmetry}) is summarized in table \ref{tab:chapter1_MollerOperatorSymmetries}. Since the symmetry operators conmute with the free hamiltonian $H_0$, the action of any of them on a an momentum eigenstate will give as a result a state with the same energy. In fact, as can be seen in table \ref{tab:chapter1_KleinGroupOnPosAndMomentBases} the result of the action of any element $A$ of the 4-group on a state $\ket{p}$ is either $\ket{p}$ or $\ket{-p}$. For this reason, we can work with the on-shell representation of the scattering operator to obtain extra relations between the scattering amplitudes. As an example I will demonstrate what happens to the scattering amplitudes when the symmetry III is satisified, \textit{i.e.}, $\Pi H = H \Pi$. Considering a momentum $p>0$ and using $S = \Pi^{\dagger}S \Pi$ (see table \ref{tab:chapter1_MollerOperatorSymmetries}) I find
%
\begin{equation}
  \left(
  \begin{array}{cc}
    \mel{p}{S}{p} & \mel{p}{S}{-p}\\
    \mel{-p}{S}{p} & \mel{-p}{S}{-p}
  \end{array}
  \right)
  =
  \left(
  \begin{array}{cc}
    \mel{-p}{S}{-p} & \mel{-p}{S}{p}\\
    \mel{p}{S}{-p} & \mel{p}{S}{p}
  \end{array}
  \right).
\end{equation}
%
Now, using the on-shell representation of the scattering operator \eqref{eq:chapter1_onShellMatrixRelationToS} and the definition of $\mathsf{S}(p)$ \eqref{eq:chapter1_onShellMatrix} I arrive at
%
\begin{align}
  T^l(p) &= T^r(p),\nonumber\\
  R^l(p) &= R^r(p).
  \label{eq:chapter1_effectOfPArityOnamplitudes}
\end{align}
%
Instead of deriving the equivalent relation to \eqref{eq:chapter1_effectOfPArityOnamplitudes} for all the symmetries explicitely, I summarize the results for the matrix elements of the $S$ operator in the 5th column and for the scattering amplitudes in the columns 6-9 of table \ref{tab:chapter1_SymmetriesTable}. Explicit examples on how to find the relations in the 5th and 6th columns of table \ref{tab:chapter1_SymmetriesTable} for other symmetries can be found in \cite{Muga2004}.

If we now take into account the generalized unitary relations $\widehat{S}^\dagger S=S\widehat{S}^\dagger=1$, in terms of amplitudes \eqref{eq:chapter1_unitarityCoefficients}, the columns 6-9 of table \ref{tab:chapter1_SymmetriesTable} imply further consequences on the amplitudes' moduli (tenth column of table \ref{tab:chapter1_SymmetriesTable}) and phases (not shown). The final two columns use the previous results to determine if perfect asymmetry is possible for transmission or reflection.
This makes evident that hermiticity (II) and parity (III) preclude, independently, any asymmetry in the scattering coefficients;
PT-symmetry (VII) or  $\Theta$-pseudohermiticity
(VI) forbid transmission asymmetry (all local potentials  satisfy automatically
symmetry VI),  whereas time-reversal symmetry (i.e., a real potential in coordinate space)
(V) or  PT-pseudohermiticity (VIII) forbid reflection asymmetry.


\begin{table}
  \caption{Action of the Klein 4-group on the position and momentum bases. Each cell corresponds to the action of the elements of the Klein 4-group (in the first column) on the position and momentum eigenstates (in the first row). A scalar $\alpha \in \mathbb{C}$ is included in the table to make the unitary/antiunitary nature of the elements of the group explicit.}
  \label{tab:chapter1_KleinGroupOnPosAndMomentBases}
  \center
  \begin{tabular}{lcc}
    \hline\hline
    & $\alpha \ket{x}$ & $\alpha \ket{p}$\\
    \hline
    $1$ & $\alpha \ket{x}$ & $\alpha \ket{p}$\\ %Identity
    $\Pi$  & $\alpha \ket{-x}$ & $\alpha \ket{-p}$\\ %Parity
    $\Theta$ & $\alpha^* \ket{x}$ & $\alpha^* \ket{-p}$\\ %Time Reversal
    $\Pi\Theta$ & $\alpha^* \ket{-x}$ & $\alpha^* \ket{p}$\\ %PT
    \hline\hline
  \end{tabular}
\end{table}




\subsection{Equivalences between symmetries}

The occurrence of one particular symmetry in the potential (conventionally  ``first symmetry'')
does not exclude a second symmetry to be satisfied as well.
When a double symmetry holds, excluding the identity,  the ``first'' symmetry  implies the equivalence of the second symmetry with a third symmetry.
We have already mentioned that $\Pi$-pseudohermiticity (IV) is equivalent to $PT$-symmetry (VII) for local potentials.
Being local is just one particular way to satisfy symmetry VI, namely $\Theta$-pseudohermiticity. The reader may verify with the aid of
the third column for $\langle x|V|y\rangle$  in table \ref{tab:chapter1_SymmetriesTable}, that indeed, if symmetry VI is satisfied (first symmetry), symmetry IV has the same effect as symmetry VII.
They become equivalent. Other well known example is  that for a local potential (symmetry VI is satisfied), a real potential in coordinate space  is necessarily hermitian,
so symmetries V and II become equivalent.
These examples are just particular cases of the full set of equivalences given in table \ref{tab:chapter1_equivalencesBetweenSymmetries}.

\begin{table}
  \caption{Equivalences among symmetries for the potential elements.
  Given the symmetry of the upper row, the table provides the equivalent symmetries.
  For example, if II is satisfied, then III=IV holds. In words, if the potential is hermitian,  parity symmetry amounts to
  parity pseudohermiticity. In terms of the matrix elements of the potential, if  $\langle x|V|y\rangle=\langle y|V|x\rangle^*$ and also
  $\langle x|V|y\rangle=\langle -x|V|-y\rangle$, $\forall (x,y)$, then $\langle x|V|y\rangle=\langle -y|V|-x\rangle^*$ holds as well. One may proceed similarly for all other relations.
  The commutation with the identity (I) is excluded as this symmetry is satisfied by all potentials.\vspace*{.2cm}
  \label{tab:chapter1_equivalencesBetweenSymmetries}}
  \centering
  \scalebox{1}{
  \begin{tabular}{ccccccc}
    \hline\hline
    II & III& IV& V& VI & VII &VIII
    \\
    \hline
    III=IV & II=IV & II=III & II=VI &II=V&II=VIII&II=VII
    \\
    V=VI&V=VII&V=VIII&III=VII&III=VIII&III=V&III=VI
    \\
    VII=VIII&VI=VIII&VI=VII&IV=VIII&IV=VII&IV=VI&IV=V
    \\
    \hline\hline
  \end{tabular}}

\end{table}



\section{Asymmetric devices\label{sec:chapter1_AsymmetricDevices}}

Now, I will describe the set of devices that we want to design and how the symmetries of the Klein 4-group make this possible or impossible in some cases. Table \ref{tab:chapter1_DevicesDescription} shows a descriptive list of the 6 kinds of asymmetric devices that we consider. The first column gives the name that we have chosen for each of the devices. Second and third columns show the expected behaviour for left or right incidence, respectively. The fourth column shows the descriptive code for each of the devices. The descriptive codes have always the following structure, \textit{LI/RI}, where LI and RI are codes that describe the behaviour for left and right incidence respectively. LI and RI can be composed by the following symbols $\mathcal{T}$, $\mathcal{R}$, $\mathcal{A}$. $\mathcal{T}$ stands for devices with full transmision ($T=1$), $\mathcal{R}$ for full reflection ($R = 1$). $\mathcal{A}$ is the code for full absortion ($R=T=0$). For example, a device with reflection asymmetry $R^l=1$, $R^r=0$ and with $T^r=T^l=1$ would in our case be a particular ``transparent, one-way reflector'', as full transmission occurs from both sides and its descriptive code would be $\mathcal{T}\mathcal{R}/\mathcal{T}$. This effect has however become popularized as ``unidirectional invisibility'' \cite{Lin2011,Yin2013}. The device denominations in fig. \ref{cases} or table \ref{tab:chapter1_DevicesDescription} are intended as short and meaningful, and do not necessarily coincide with some extended terminology, in part because the range of possibilities is broader here than those customarily considered, and because we use a 1 or 0 condition for the moduli.

Combining the information of the last two-columns in table \ref{tab:chapter1_SymmetriesTable} with the additional condition that all scattering coefficients
be 0 or 1 we elaborate the last two columns of table \ref{tab:chapter1_DevicesDescription}, which provides the symmetries
that do not allow the implementation of the devices in fig. \ref{cases}. The complementary table \ref{tab:chapter1_AllowedDevices} gives instead the symmetries that allow, but do not necessarily imply, a given device type.




\begin{landscape}
  \begin{table}
    \caption{Device types for  transmission and/or reflection asymmetry, restricted to 1 or 0 moduli for the scattering amplitudes.¡''
    % (i.e., zero scattering amplitudes for transmission and reflection from the right).
    The fifth column indicates the symmetries in table \ref{tab:chapter1_SymmetriesTable} that forbid the device. Figures S2, S3, S5 and S6 can be found in the supplemental material to this paper.\vspace*{.2cm}}
    \label{tab:chapter1_DevicesDescription}
    \centering
    \scalebox{1}{
    \begin{tabular}{ccccccc}
      \hline\hline
      Device type & Left incidence& Right incidence&Code& Forbidden by & Example
      \\
      \hline
      One-way mirror&transmits and reflects&absorbs&$\cal{TR/A}$&II, III, IV, V, VI,  VII, VIII& fig. S1
      \\
      One-way barrier&transmits&reflects&$\cal{T/R}$&II, III, IV, V, VI, VII, VIII&fig. S2
      \\
      One-way T-filter&transmits&absorbs&$\cal{T/A}$&II, III, IV, V, VI, VII&fig. \ref{fig_device_T_A}, S3
      \\
      Mirror\&1-way transmitter&transmits and reflects&reflects&$\cal{TR/R}$&II, III, VI, VII&fig. S4
      \\
      One-way R-filter&reflects&absorbs&$\cal{R/A}$&II, III, IV, V, VII, VIII&\cite{Huang2016}
      \\
      Transparent 1-way reflector&transmits and reflects&transmits&$\cal{TR/T}$& II, III, V, VIII
      & figs. \ref{fig_reflector}, S5
      \\
      \hline\hline
    \end{tabular}}
  \end{table}
\end{landscape}


\begin{table}
  \caption{Device types allowed for a given symmetry.
  \vspace*{.3cm}\label{tab:chapter1_AllowedDevices}}
  \centering
  \scalebox{0.9}{
  \begin{tabular}{cc}
    \hline\hline
    Symmetry& Allowed devices
    \\
    \hline
    I&All types
    \\
    II&None
    \\
    III&None
    \\
    IV&$\cal{TR/R,TR/T}$
    \\
    V&$\cal{TR/R}$
    \\
    VI & $\cal{R/A, TR/T}$
    \\
    VII&$\cal{TR/T}$
    \\
    VIII&$\cal{T/A,TR/R}$
    \\
    \hline\hline
  \end{tabular}}

\end{table}


%%%%%%%%%

\begin{figure}
  \begin{center}
    (a)\includegraphics[width = 0.6\columnwidth]{Figures/fig_T_A_pot_abs.pdf}\\
    (b)\includegraphics[width = 0.6\columnwidth]{Figures/fig_T_A_pot_arg.pdf}\\
    (c)\includegraphics[width = 0.6\columnwidth]{Figures/fig_T_A_prob.pdf}
  \end{center}
  \caption{(Color online) One-way T-filter ($\cal{T/A}$, $\left|T^l\right|=1,T^r=R^l=R^r=0$) with  potential $V(x,y)=|V(x,y)|e^{i\phi(x,y)}$ set
  for $k_0 = 1/d$.
  (a) Absolute value $\left| V(x,y) \right|$;    (b) Argument $\phi(x,y)$;
  (c) Transmission and reflection coefficients:
  $\left| R^l \right|^2$ (black, solid line), $\left| T^l \right|^2$ (green, solid line),
  $\left| R^r \right|^2$ (blue, tick, dashed line), $\left| T^r \right|^2$ (red, dotted line). $V_0 = \hbar^2/(2m d^3)$.\label{fig_device_T_A}}
\end{figure}


%
%
%

\section{Designing potentials for asymmetric devices\label{examples}}

\begin{figure}
  \begin{center}
    (a) \includegraphics[width = 0.6\columnwidth]{Figures/fig_TR_T_local_pot.pdf}\\
    (b) \includegraphics[width = 0.6\columnwidth]{Figures/fig_TR_T_local_prob_1.pdf}\\
    (c) \includegraphics[width = 0.6\columnwidth]{Figures/fig_TR_T_local_prob_2.pdf}
  \end{center}
  \caption{\label{fig_reflector}(Color online) Transparent 1-way reflector with a local PT potential:
  (a) Approximation of the potential (\ref{num}), real part (green solid line), imaginary part (blue dashed line).
  (b,c) Transmission and reflection coefficients versus momentum $k d$;
  left incidence: $\left| R^l \right|^2$ (black, solid line), $\left| T^l \right|^2$ (green, solid line);
  right incidence: $\left| R^r \right|^2$ (blue, tick, dashed line), $\left| T^r \right|^2$ (red, dotted line, coincides with green, solid line).
  $\epsilon/d = 10^{-4}$.
  (b) $\alpha= 1.0 \hbar^2/(4\pi m)$ (c) $\alpha = 1.225 \hbar^2/(4\pi m)$
  (the black, solid line coincides here mostly with the red, dotted and green, solid lines).
  \label{fig_TR_T_local}}
\end{figure}


We will show  how to design non-local potentials
leading to the asymmetric devices.
For simplicity we look for  non-local potentials $V(x,y)$ with local support
that vanish  for $|x| >d$ and $|y| >d$.

To obtain the potentials for the asymmetric devices I follow an inverse scattering approach similar to \cite{Palao1998}. It starts by imposing an ansatz for the wavefunctions and the potential in the stationary Schr\"{o}dinger equation
%
\begin{eqnarray}
  \frac{\hbar^2k^2}{2m} \psi (x) = - \frac{\hbar^2}{2m} \frac{d^2}{dx^2} \psi (x)
  +\!\!\int_{-d}^d \!dy V(x, y) \psi(y).
  \label{Schroedinger}
\end{eqnarray}
%
The free parameters are fixed making use of the boundary conditions.
The form of the wavefunction incident from the left is
$\psi_l(x) = e^{i k x} + R^l e^{-i k x}$ for $x < -d$ and $\psi_l (x) = T^l e^{i k x}$ for $x > d$,
where  $k=p/\hbar$.
The wavefunction incident from the right is instead
$\psi_r(x) = e^{-ikx} T^r$ for $x < -d$ and $\psi_r (x) = e^{-i k x} + R^r e^{i k x}$ for $x > d$.

Our strategy is to assume  polynomial forms for the two wavefunctions in the interval $|x| < d$,
$\psi_l (x) = \sum_{j=0}^5 c_{l,j} x^j$ and $\psi_r (x) = \sum_{j=0}^5 c_{r,j} x^j$, and also a
polynomial ansatz of finite degree for the potential $V(x,y) = \sum_i \sum_j v_{ij} x^i y^j$.
Inserting these ansatzes in eq. (\ref{Schroedinger}) and from the conditions that $\psi_{l,r}$
and their derivatives must be continuous, all coefficients $c_{l,j}\,,c_{r,j}$ and $v_{ij}$ can be determined.
Symmetry properties of the potential can also be imposed via additional conditions on
the potential coefficients $v_{ij}$. For example we may use this method to obtain a one-way T-filter ($\cal{T/A}$) device (third device in table \ref{tab:chapter1_DevicesDescription}) with a nonlocal PT-pseudohermitian potential (symmetry VIII of table \ref{tab:chapter1_SymmetriesTable}) for a chosen wavevector $k = k_0$. The absolute value and argument of the resulting potential $V(x,y)$ are shown in figs. \ref{fig_device_T_A}(a) and \ref{fig_device_T_A}(b) together with its scattering coefficients as function of the incident wave vector, fig. \ref{fig_device_T_A}(c). As can be seen in fig. \ref{fig_device_T_A}(c) the imposed scattering coefficients are fulfilled exactly for the chosen wavevector. They are also satisfied approximately in a neighborhood of $k_0$. In the  Supplemental Material, Sec. II, we give further details about the construction of this potential and we work out other asymmetric devices of fig. \ref{cases}.
%
% ---------------------------------------- Asymmetric Reflection -----------------------------------
%
%
%
%

%
%%%%%%%%%%%%%%%%%%%%%%%%%%%%%%%%%%%%%%%%%%%%%%

%%%%%%%%%%%%%%%%%%%%%%%%%%%%%%%%%%%%%%%%%%%%%
%

\section{Extending the scattering asymmetry to a broad incident-momentum domain\label{ext}}
%
%
%
%
The inversion technique just described may be generalized
to extend the range of incident momenta for which the potential works by imposing additional
conditions and increasing correspondingly the number of parameters in the wavefunction ansatz,
for example we may impose that the derivatives of the  amplitudes,  in one or more orders,  vanish at $k_0$,
or  0/1 values for the coefficients not only at  $k_0$ but at a series of grid points $k_1$, $k_2$, ... $k_N$,
as in \cite{Brouard1994,Palao1998a,Palao1998,Muga2004}.

Here we put forward instead a method that provides a very broad working-window domain.
While we make formally use of the Born approximation, the exact numerical
computations demonstrate the robustness and accuracy of the approach to achieve that objective by
making use of an adjustable parameter in the potential. The very special role of the Born approximation in inverse problems has been
discussed and demonstrated in \cite{Snieder1990,Mostafazadeh2014,Horsley2015}.
Specifically we study a transparent one-way reflector ${\cal{TR/T}}$.
Our aim is now to find a local PT-symmetric potential such that asymmetric reflection results,
$T^l = T^r = 1, R^r = 0, |R^l|=1$ for a broad range of incident momenta. A similar goal
was pursued in \cite{Longhi2014} making use of a supersymmetric transformation,
without imposing $|R^l|=1$.

In the Born approximation and for a local potential $V(x)$, the reflection amplitudes take the simple form
%
\begin{eqnarray}
  R^l=-\frac{2\pi i m}{p}\langle -p|V|p\rangle,
  \;
  R^r=-\frac{2\pi i m}{p}\langle p|V|-p\rangle.
\end{eqnarray}
%
Defining the Fourier transform
%
\begin{eqnarray}
  \widetilde V (k) = \frac{1}{\sqrt{2\pi}} \int_{-\infty}^\infty dx \, V(x) e^{-i k x}
\end{eqnarray}
%
we get for $k=p/\hbar>0$:
%
\begin{eqnarray}
  R^l=-\frac{\sqrt{2\pi} i m}{k \hbar^2} \widetilde V (-2k),
  \;
  R^r=-\frac{\sqrt{2\pi} i m}{k\hbar^2} \widetilde V (2 k).
\end{eqnarray}
%
Assuming that the potential is local and PT-symmetrical, we calculate the transition coefficient
from them using generalized unitarity as
$|T|^2=1-{R^r}^*R^l$.

To build a ${\cal{TR/T}}$ device we demand:
$\widetilde V(k) = \sqrt{2\pi} \alpha k$ ($k < 0$) and $\widetilde V(k) = 0$ ($k \ge 0$).
By inverse Fourier transformation, this implies
%
\begin{eqnarray}
  V(x) &=&
  %-\alpha \frac{\partial}{\partial x} \left[P \frac{1}{x} + i \pi \delta(x) \right]
  %\nonumber\\
  %&=&
  -\alpha \frac{\partial}{\partial x} \lim_{\epsilon\to 0} \frac{1}{x - i \epsilon}
  = \alpha \lim_{\epsilon\to 0} \frac{1}{(x - i \epsilon)^2}
  \nonumber\\
  %&=& \alpha \lim_{\epsilon\to 0} \frac{(x + i \epsilon)^2}{(x^2 + \epsilon^2)^2}\\
  &=& \alpha \lim_{\epsilon\to 0} \left[\frac{x^2 - \epsilon^2}{(x^2 + \epsilon^2)^2} + i
  \frac{2 x\epsilon}{(x^2 + \epsilon^2)^2}\right],
  \label{num}
\end{eqnarray}
%
which is indeed a local, $PT$-symmetric potential for $\alpha$ real.
$\alpha$ is directly related to the reflection coefficient, within the Born approximation,
$R^l = 4 \pi i m \alpha/\hbar^2$. As the Born approximation may differ from exact results
we shall keep $\alpha$ as an adjustable parameter
in the following.

In a possible physical implementation, the potential in eq. (\ref{num}) will
be approximated by keeping a small finite $\epsilon>0$, see fig. \ref{fig_TR_T_local}(a).
%\begin{eqnarray}
%V(x) = \alpha \left[\frac{x^2 - \epsilon^2}{(x^2 + \epsilon^2)^2} + i
%\frac{2 x\epsilon}{(x^2 + \epsilon^2)^2}\right],
%\end{eqnarray}
%with a finite, small $\epsilon>0$.
Then, its Fourier transform is
$\widetilde V(k) = \sqrt{2\pi} \alpha k e^{\epsilon k}$ ($k < 0$) and $\widetilde V(k)=0$ ($k \ge 0$).
In figs. \ref{fig_TR_T_local}(b) and (c), the resulting coefficients for $\epsilon/d=10^{-4}$ and  two different values
of $\alpha$ are shown.  These figures have been calculated by
numerically solving the Schr\"odinger equation exactly.
% and demonstrate that
%$\alpha$ can indeed  be adjusted so that $\left| R^l \right|^2 \approx 1$.
Remarkably, the Born approximation contains all the information required to build the required potential shape
up to a global factor.  Such a prominent role of the Born approximation in inverse problems has been noted
in different applications \cite{Snieder1990,Mostafazadeh2014,Horsley2015}. For a range of $\alpha$, the potential gives $|R^r|\approx 0$, a nearly constant $|R^l|^2$, and
$|T^r|= |T^l|\approx1$  in a broad $k$-domain, see fig. \ref{fig_TR_T_local}(b). Adjusting  the value of $\alpha$, fig. \ref{fig_TR_T_local}(c),
sets $|R^l|\approx 1$ as desired.
%
%Remarkably, except for very low $k$,
%the potential is reflectionless from the right
%and provides full transmission independently of $\alpha$. As well, $|R^2|^2$ stays constant with respect to $k$ for different $\alpha$,
%so that that up to the adjustable parameter the Born approximation provides in fact the required potential shape, a surprising
%in inverse problems \cite{Snieder1990,Mostafazadeh2014,Horsley2015}.
%Figure \ref{fig_TR_T_local}(c) demonstrates that $\alpha$ can indeed be adjusted so that the  potential works
%as intended, i.e., as a transparent one-way reflector,  for a very broad range of $k$ values.

%
%
%
%
%
\section{Discussion}
%
%
Scattering asymmetries are necessary to develop technologically relevant devices such
as one-way mirrors, filters and  barriers, invisibility cloaks, diodes, or Maxwell demons.
So far much effort has been devoted to build and apply local PT-symmetric potentials but the possible scattering asymmetries with them are
quite limited. We find that six  device types with asymmetric scattering are possible
when imposing $0$ or $1$ scattering coefficients.
PT-symmetry can only realize one of them, but this symmetry  is just one among eight possible symmetries of complex non-local potentials.
The eight symmetries arise from the discovery that Klein's four-group
$\{1, \Pi, \Theta, \Theta\Pi\}$, combined with two possible relations among the Hamiltonian, its adjoint,
and the symmetry operators of the group, eqs. (1) and (2),
produce all possible  equalities among potential matrix elements after complex conjugation, coordinate inversion, the identity, and transposition.
In other words, to have all possible such equalities, the conventional definition of a symmetry $A$ in terms of its commutation with the Hamiltonian $H$ is not enough, and $A$-pseudohermiticity must be considered as well on the same footing.
Extending the concept of what a symmetry is for complex, non-local potentials is
a fundamental, far-reaching step of this work.
This group theoretical analysis and classification is not only esthetically pleasing, but also of practical importance, as it reveals
the underlying structure and span of the possibilities available in principle to manipulate the asymmetrical response of a potential
for a structureless particle.

We provide a potential for a One-way T-filter ($\mathcal{T}/\mathcal{A}$) and a PT potential for a transparent One-way reflector ($\mathcal{T}\mathcal{R}/\mathcal{T}$) that works in a broad domain of incident momenta. Although the present theory is for the scattering of quantum particles, the analogies between quantum physics and optics suggest to extend the concepts and results for optical asymmetric devices.
%
    % Asymmetric scattering by non-Hermitian potentials
%!TEX root = ../Thesis.tex
%Chapter 2

\chapter{$S$-matrix pole symmetries for non-Hermitian scattering Hamiltonians}
\label{Chapter2}
\lhead{Chapter 2. \emph{$S$-matrix pole symmetries for non-Hermitian scattering Hamiltonians}}

The complex eigenvalues of some non-Hermitian Hamiltonians, e.g. parity-time symmetric Hamiltonians, come in complex-conjugate pairs. We show that for non-Hermitian scattering Hamiltonians (of a structureless particle in one dimension) possesing one of four certain symmetries, the poles of the $S$-matrix eigenvalues in the complex  momentum plane are symmetric about the imaginary axis, i.e. they  are complex-conjugate pairs in complex-energy plane. This applies even to states which are not bounded eigenstates of the system, i.e. antibound or virtual states, resonances, and antiresonances. Example potentials with such symmetries are constructed and their pole structures and scattering properties are calculated.
%
\newpage
%
\section{Introduction}

Much work on non-hermitian (NH) physics has focused on PT-symmetric Hamiltonians, as they may have a purely real spectrum \cite{Bender1998}. More recently, other NH and non-PT Hamiltonians, have been shown to hold real eigenvalues \cite{Nixon2016,Chen2017,Yang2017}. Work on scattering by PT-symmetric potentials was at first rather scarce \cite{Muga2004,Ruschhaupt2005,Cannata2007,Znojil2015}. However, scattering has been later investigated intensely in connection with spectral singularities and reflection asymmetries for left or right incidence (i.e. unidirectional invisibility) \cite{Mostafazadeh2009,Longhi2014,Mostafazadeh2013}, in most cases restricting the analysis to local potentials. As was shown in chapter \ref{Chapter1}, different devices with asymmetrical scattering responses (i.e., with different transmission and/or reflection for right and left incidence in a 1D setting) are possible if one makes use of non-local potentials. Chapter \ref{Chapter1} provides the selection rules for the transmission and reflection coefficient asymmetries based on eight basic Hamiltonian symmetries. Four of theses symmetries are a standard conmutation between a symmetry operator $A$ and the Hamiltonian, eq. \eqref{eq:chapter1_symmetry}, and the other four are $A$-pseudohermicity, \eqref{eq:chapter1_pseudoSymmetry}. $A$ is an element of the Klein 4-group $\mathbf{K}_4 = \left\{1,\Pi,\Theta,\Theta\Pi\right\}$.

A set of works by A. Mostafazadeh \cite{Mostafazadeh2002,Mostafazadeh2002a,Mostafazadeh2002b} gives the sufficient and necesary conditions for hamiltonians with a discrete spectrum to have real or complex conjugate pairs of eigenvalues. Hamiltonians satisying eq. \eqref{eq:chapter1_pseudoSymmetry} for a hermitian linear operator $A$, \textit{i.e.} $A$-pseudohermicity have a spectrum in which the eigenvalues come in complex conjugate pairs, some of them can even be real-valued. Moreover, the results of \cite{Mostafazadeh2002b} show that if a Hamiltonian conmutes with a hermitian antilinear operator, then there will exist a hermitian linear operator $A$ for which the Hamiltonian is $A$-pseudohermitian and, therefore, it will have complex conjugate pairs of eigenvalues. However, an aspect uncovered in refs. \cite{Mostafazadeh2002,Mostafazadeh2002a,Mostafazadeh2002b} is that when the Hamiltonian is $A$-pseudohermitian, the poles of the scattering matrix can have the same properties as the discrete spectrum, they can also come in complex conjugate pairs.

This chapter aims at extending the results in chapter \ref{Chapter1} and refs. \cite{Mostafazadeh2002,Mostafazadeh2002a,Mostafazadeh2002b} in several directions:

i) I will provide an alternative characterization of the 8 symmetries formed by the elements of the Klein 4-group and the relations \eqref{eq:chapter1_symmetry}, \eqref{eq:chapter1_pseudoSymmetry} in terms of the invariance of $H$ with respect to the action of superoperators.

ii) Moreover,
four of these eight symmetries imply the same
type of pole structure of $S$-matrix eigenvalues in the complex momentum plane that was found for PT symmetry \cite{Muga2004},
namely, zero-pole correspondence at complex-conjugate points, and poles on the imaginary axis or forming symmetrical pairs with respect to the imaginary
axis. This configuration with poles located on the imaginary  axis or as symmetrical pairs has some important consequences. In particular, it provides stability of the real energy eigenvalues with respect to parameter variations of the potential. While a simple pole on the imaginary axis can move along that axis when a parameter is changed, it cannot move off this axis (since this would violate the pole-pair symmetry) or bifurcate. The formation of pole pairs occurs near special  parameter values for which two poles on the imaginary axis collide. When the poles are mapped to the energy complex plane $E = p^2/2m$, they have the same symmetry structure of complex conjugate pairs as the discrete eigenvalues of $A$-pseudohermitian Hamiltonian, which expands the results of refs. \cite{Mostafazadeh2002,Mostafazadeh2002a,Mostafazadeh2002b}.

The remainder of the chapter is organized as follows. In section \ref{sec:SymTheory} we characterize the symmetry operations defined in chapter \ref{Chapter1} as the invariance of the Hamiltonian with respect to the action of eight linear or antilinear superoperators. In section \ref{sec:SPoles} I discuss the physical consequences of the symmetries in the pole structure of the scattering matrix eigenvalues. Four symmetries are shown to lead to complex poles corresponding to real energies or conjugate (energy) pairs.  In section \ref{sep_pot_sec} I exemplify the general results with separable potentials exhibiting parity-pseudohermiticity and time-reversal symmetry. These are the two non-trivial symmetries of the four (in the sense that the other two, hermiticity and PT-symmetry, are already well discussed). In section \ref{sec:RealEigenConclusions} we discuss and summarize our results.
    % Symmetries of non-Hermitian potentials S-matrix poles
%!TEX root = ../Thesis.tex
%Chapter 3

\chapter{Physical Implementation of non-hermitian and non-local hamiltonians}
\label{Chapter3}
\lhead{Chapter 3. \emph{Physical Implementation of non-hermitian and non-local hamiltonians}}
%
Non-Hermitian, one-dimensional  potentials which are also non-local,
allow for scattering asymmetries, namely, asymmetric transmission or reflection responses to the incidence of a particle from left or right.
The  symmetries of the potential
imply selection rules for transmission and reflection. In particular, parity-time (PT)
symmetry or the symmetry  of any local potential do not allow for asymmetric transmission.
We put forward a feasible  quantum-optical implementation
of non-Hermitian, non-local, non-PT potentials to implement different scattering asymmetries, including transmission
asymmetries.
%
\newpage
%
\section{Introduction}
The asymmetric response of diodes, valves, or rectifiers to input direction is of paramount importance in many different fields and technologies, from hydrodynamics  to microelectronics,  as well as in biological systems. We  expect a wealth of applications of such response asymmetries also in the microscopic quantum realm, in particular in  circuits or operations carrying or processing quantum information with moving atoms.
So far  devices such as Maxwell demons, which  let atoms pass one way, have  been instrumental, first as ideal devices to  understand the second law \cite{Maxwell1875,Rex1990}, and also  as practical sorting devices \cite{Ruschhaupt2004,Raizen2005,Dudarev2005,Ruschhaupt2006a,Ruschhaupt2006,Ruschhaupt2006b,Ruschhaupt2007,Raizen2009,Jerkins2010}.

Asymmetric transmission and reflection
probabilities for one-dimensional (1D) particle scattering off  a potential center are not possible if the Hamiltonian is Hermitian \cite{Muga2004,Mostafazadeh2018}.
Non-Hermitian (NH) Hamiltonians representing effective interactions have a long history in nuclear, atomic, and molecular physics, and have become common in optics, where wave equations in waveguides could simulate  Schr\"odinger equations \cite{Ruschhaupt2005,Longhi2017a,Konotop2016}.
%
Non-Hermitian Hamiltonians constructed by analytically continuing Hermitian ones are useful and efficient tools to find resonances \cite{Moiseyev2011}.
They can also be set phenomenologically, e.g. to describe gain and loss
\cite{Ruschhaupt2005},
or be found as effective Hamiltonians for a subspace from a Hermitian Hamiltonian of a larger system
by projection \cite{Feshbach1958,Ruschhaupt2004,Muga2004}.
%

Much of the recent interest in Non-Hermitian Hamiltonians focuses on  parity-time (PT) symmetric Hamiltonians \cite{Bender1998,Znojil2015}  because of their spectral properties and useful applications, mostly in optics  \cite{Longhi2017a,Konotop2016,Longhi2014}, but  alternative symmetries are also being studied \cite{Nixon2016,Nixon2016a,Chen2017,Ruschhaupt2017,Simon2018,Simon2019a,Alana2020,Bernard2002,Kawabata2019}. Symmetry operations
on NH Hamiltonians can be systematized into group structures \cite{Ruschhaupt2017,Simon2019a,Alana2020}.  In particular for
1D particle scattering off a potential center, the different Hamiltonian symmetries imply
selection rules for asymmetric transmission and reflection \cite{Ruschhaupt2017,Simon2019a}.\footnote{Throughout the paper we assume a linear theory for systems whose wave equation is linear in the wavefunction.}
Whereas hermiticity does not allow for any asymmetry in transmission and reflection probabilities,   PT symmetry or
the symmetry of local potentials, technically ``pseudohermiticy with respect to time reversal'' \cite{Ruschhaupt2017},
do not allow for  asymmetric transmission \cite{Muga2004,Mostafazadeh2018}, see symmetries II (Hermiticity), VII (PT symmetry),  and VI
(time-reversal pseudohermiticity) in table \ref{condi}.
(Here a ``local potential'' is defined as one whose only non-zero elements in coordinate representation are diagonal, whereas a non-local one has  non-zero nondiagonal elements.)
%
%
Thus  non-local, non-PT, and non-Hermitian potentials are needed to implement a rich set of scattering
asymmetries, and in particular asymmetric transmission.

In this paper we put forward a physical realization of  effective NH, non-local  Hamiltonians which do not posses PT symmetry.
Non-local potentials for asymmetric scattering had been constructed as mathematical models \cite{Ruschhaupt2017}, but a physical implementation  had been so far elusive.
Using Feshbach's projection technique it is found that the
effective potentials for a ground-state atom crossing a laser beam in a region of space are generically non-local and non-Hermitian. Shaping the spatial-dependence of the, generally complex, Rabi frequency, and selecting a specific laser detuning allows us to produce different potential symmetries and asymmetric scattering effects, including asymmetric transmission.

After a lightning review of Hamiltonian symmetries and the corresponding scattering selection rules in Sec. \ref{ssh},
we shall  explain in Sec. \ref{enl} how to generate different NH symmetries in a quantum optical setting of an atom impinging on a laser illuminated region. Finally we provide specific example devices (constructed using numerical optimisation) with different asymmetric scattering responses in Sec. \ref{exa}. Realistic experimental parameters are also examined. The asymmetric behavior can be intuitively understood based on a classical approximation of the motion and the non-commutativity of rotations on the Bloch sphere, which gives good estimates for the potential parameters, see Sec. \ref{class}.

%
%
%
\section{Symmetries of Scattering Hamiltonians\label{ssh}}
%
%
We consider one-dimensional  scattering Hamiltonians  $H=H_0+V$, where $H_{0}$
is the kinetic energy for a particle of mass $m$,
and
$V$ is the  potential, which is assumed to decay fast enough on both sides so that $H$ has a continuous spectrum and scattering eigenfunctions. These eigenfunctions may be chosen so that asymptotically, i.e., far from the potential center,  they are superpositions of an incident plane wave and a reflected plane wave on one side, and a transmitted plane wave on the other side. Reflected and transmitted waves include corresponding amplitudes, whose squared-modulii
(scattering coefficients hereafter) sum to one for Hermitian potentials. Instead, NH potentials
may produce  absorption or gain.


There are eight different symmetries that $H$ could fulfill, see table \ref{condi},
with the forms
%
\begin{eqnarray}
	AH&=&HA,
	\\
	AH&=&H^{\dagger}A,
	\label{pseudohermiticityPhysicalImplementation}
\end{eqnarray}
%
%
where $A$ is a unitary or antiunitary operator in the Klein four-group \linebreak $K_{4}=\lbrace 1,\Pi,\theta,\Pi\theta \rbrace$ \cite{Ruschhaupt2017}.
Relation \eqref{pseudohermiticityPhysicalImplementation} is called here $A$-pseudohermiticity of $H$ \cite{Mostafazadeh2010,Ruschhaupt2017}.
The operators $1$, $\Pi$, $\theta$ and $\Pi\theta$ are the identity, parity, time reversal,
and the consecutive (commuting) application of both operators.
Acting on position eigenvectors $|x\rangle$,
$\Pi c|x\rangle =c|-x\rangle$, and $\theta c |x\rangle = c^* |x\rangle$, for any  complex number $c$. Note that symmetry I is a trivial symmetry and is satisfied for all Hamiltonians.

%
The eight symmetries  may be regarded as the invariance of the Hamiltonian with respect to
eight symmetry operations that form the Abelian group E8 \cite{Simon2019a}. They are all operations that can be done by inversion, transposition, complex conjugation, and their combinations.
Making use of generalized unitarity relations and  the relations implied by the symmetries on $S$-matrix elements,
the transmission and reflection amplitudes for right and left incidence, $T^r$, $R^r$ and $T^l$, $R^l$, can be  related
to each other,
as well as their modulii \cite{Ruschhaupt2017}.  ``Right and left  incidence'' are here shorthands for ``incidence {\it from} the right'',
and ``incidence { from} the left'', respectively.

The possible asymmetric responses are allowed or forbidden, according to selection rules,  by the symmetries of the Hamiltonian.
If we impose that the transmission and reflection coefficients  have only 0 or 1 values,  a convenient reference scenario for  devices intended to manage quantum-information applications, six possible scattering asymmetries may be identified \cite{Ruschhaupt2017}, see table \ref{table2PhysicalImplementation}. It is useful to label them according to the response  to incidence from the left/right. The possible responses are encoded in the  letters ${\cal{A}}$,
for ``absorption'', and ${\cal{T}}$ and ${\cal{R}}$ for ``transmission'' and ``reflection'', separated by ``$/$''. The
letters on the left of $/$ are for left incidence, and the ones on the right are for right incidence. For example ${\cal T/A}$ means transmission for left incidence and absorption for right
incidence. From the selection rules \cite{Ruschhaupt2017}, it is possible to determine which symmetries allow for a given device, see table \ref{table2PhysicalImplementation}.

The relations between the symmetries and ``reciprocity'' are surely worth spelling out, in view of many works and discussions in optics \cite{Lin2011,Feng2011,Fan2012,Peng2014}. ``Reciprocity'' is a somewhat vague term with different meanings for different authors and communities, the reviews \cite{Potton2004} and \cite{Deak2012} give  some useful  background. A primary formulation regards reciprocity as the property of detecting the  same effects when interchanging source and detector without changing the scatterer. This  concept has lead to  different formalizations that fix in more detail what is exactly meant by ``same effects'' and ``interchanging source and detector''.
In 1D scattering problems we may first  distinguish  a reciprocity for scattering amplitudes or for scattering coefficients (their modulus squared). We shall hereafter focus on  coefficients as in the rest of  the paper.
Another distinction can be made between reflection and transmission reciprocities, namely, a system with $|R^l|^2=|R^r|^2$
would be ``reflection reciprocal'' and if $|T^l|^2=|T^r|^2$ the system would be transmission reciprocal.\footnote{Incidentally, for reflection reciprocity, ``interchanging source and detector'' has to be understood in momentum space rather than spatially.}
A formal definition of reciprocity is that, for some antiunitary operator $K$ \cite{Deak2012},
%
\begin{equation}
	\label{reci}
	HK=KH^\dagger.
\end{equation}
%
It follows that the scattering transition  matrix obeys in momentum representation \cite{Deak2012}
%
\begin{equation}\label{tt}
	\la p|{\sf T}|p'\ra=\la Kp'|{\sf T}|Kp\ra.
\end{equation}
%
In our symmetry classification, symmetries VI and VIII obey by definition reciprocity conditions of the form \eqref{reci}.
Inserting the results in the exact forms of transmission and reflection amplitudes, which depend on diagonal and non-diagonal elements of the transition matrix, respectively,   see e.g.  \cite{Muga2004},
different physical consequences follow:  In symmetry VI, $K=\Theta$, $|Kp\ra=|-p\ra$, and the reciprocity condition implies transmission reciprocity.
In symmetry VIII, $K=\Pi\Theta$ and $|K p\ra=|p\ra$, so the reciprocity condition implies reflection reciprocity.
A first relevant observation is this: an arbitrary reciprocity condition of the form \eqref{reci}, does not necessarily imply symmetrical transmission. A second point is that ``scattering selection rules'',  i.e., the set of forbidden phenomena, or compulsory relations among right and left coefficients, see table 1 in \cite{Ruschhaupt2017}, depend as well on generalized unitary relations. Putting together the effect of symmetries on transition or
$S$-matrix elements and generalized unitarity relations, it turns out that
symmetries II (Hermiticity) and III (parity) are not capable of any,  reflection or transmission, asymmetry; symmetries VI (time reversal pseudohermiticity) and VII (PT symmetry) allow for reflection asymmetry
but not for transmission asymmetry;  symmetries V (time-reversal symmetry) and VIII (PT pseudohermiticity) allow for
transmission asymmetry but not for reflection asymmetry, whereas I (trivial symmetry) and IV (parity pseudohermiticity) allow for both scattering asymmetries.
Note also the importance on non-locality for asymmetric transmission: All local potentials do satisfy automatically symmetry VI, and are therefore necessarily transmission reciprocal. Let us insist once more than all these results are for linear (Schr\"odinger) dynamics.
Nonlinearity allows to break down these selection rules \cite{Lin2011,Peng2014,Xu2014}.




%---------------------------------------------------------------------------------------------
\begin{table}[t]

	\caption{Conditions leading to  specific symmetries in the potential \eqref{effpot}. A given symmetry also implies others, see the last column.\label{condi}}
	\hspace*{-0.1cm}
	\scalebox{.94}{
	\begin{tabular}{lcc}
		\hline\hline
		Symmetry& Conditions & Implies
		\\
		\hline
		(I)\;$1H=H1$ &   none & -
		\\
		(II)\;$1H=H^\dagger 1$ &  $q=-q^{*}$ (i.e. $\operatorname{Re}q=0$) & I
		\\
		(III)\;$ \Pi H=H\Pi$ &  $\Omega(x)=e^{i\phi}\Omega(-x)$ & I
		\\
		(IV) $\Pi H=H^\dagger \Pi$ &  $q=-q^{*}$ \& $\Omega(x)=e^{i\phi}\Omega(-x)$ & III,\! II,\! I
		\\
		(V) $\Theta H=H\Theta$ &  $q=-q^{*}$ \& $\Omega(x)=e^{i\phi}\Omega(x)^*$ & {\small VI,\! II,\! I}
		\\
		(VI) $\Theta H=H^\dagger\Theta$ &  $\Omega(x)=e^{i\phi}\Omega(x)^*$ & I
		\\
		(VII) $\Theta\Pi H=H\Theta \Pi$ &  $q=-q^{*}$ \& $\Omega(x)=e^{i\phi}\Omega(-x)^*$ & VIII,\! II,\! I
		\\
		(VIII)\,$\Theta\Pi H=H^\dagger \Theta \Pi$ &  $\Omega(x)=e^{i\phi}\Omega(-x)^*$  & I
		\\
		\hline\hline
	\end{tabular}
	}
\end{table}
%---------------------------------------------------------------------------------------------

%%%%%%%%%
\begin{table}[t]
	\caption{Device types for  transmission and/or reflection asymmetry in the first row (see main text for nomenclature,
	binary values ($0$ or $1$) for the transmission and reflection coefficients are considered here as an ideal case).
	The second row gives the corresponding symmetries  that allow
	each device.
	\label{devices}}
	\vspace*{.0cm}
	\label{table2PhysicalImplementation}
	\centering
	\scalebox{0.90}{
	\begin{tabular}{cccccc}
		$\cal{TR/A}$ & $\cal{T/R}$ & $\cal{T/A}$ & $\cal{TR/R}$ & $\cal{R/A}$ & $\cal{TR/T}$ \\
		I            & I           & I,VIII      & I,VIII       & I,VI        & I, IV, VI, VII
	\end{tabular}}
\end{table}

%
%
%
%
%

%
\section{Effective non-local potential for the ground state of a two-level atom\label{enl}}
%
The key task is now to physically realize some of the potential and device types described in the previous section. We start with a two-level atom with ground level $|1\ra$ and excited state $|2\ra$ impinging onto a laser illuminated region. For a full account of the model and further references see
\cite{Ruschhaupt2009}. The motion is assumed one dimensional, either because the atom is confined in a waveguide or because the direction $x$ is uncoupled to
the others.
We only account explicitly for atoms before the first spontaneous emission in the wavefunction
\cite{Hegerfeldt1996,Damborenea2002,Navarro2003}.
If the excited atom emits a spontaneous photon it disappears from the coherent wavefunction ensemble.
We assume that no resetting into the ground state occurs. The physical mechanism
may be an irreversible decay into a third level  \cite{Oberthaler1996}, or atom ejection from the waveguide or the privileged 1D direction due to  random recoil  \cite{Streed2006}.
The state ${\bf\Phi}_k=\left(\begin{smallmatrix}\phi_k^{(1)}\\\phi_k^{(2)}\end{smallmatrix}\right)$
for the atom before the first spontaneous emission impinging with wavenumber $k$
in a laser adapted  interaction picture,
obeys, after applying the rotating wave approximation,  an effective stationary Schr\"odinger equation
with a time-independent Hamiltonian \cite{Ruschhaupt2004a,Ruschhaupt2009}
%
${\cal H}{\bf\Phi}_k(x)=E{\bf\Phi}_k(x)$,
%
where
%
\begin{eqnarray}
	{\cal H}&=& H_0{\bf 1}+ {\cal V}=\frac{1}{2m}\left(
	{{p}^2\atop 0}{0\atop {p}^2}\right)+ {\cal V}(x),
	\\
	{\cal V}(x) &=&
	\frac{\hbar}{2}\left(
	{0\atop \Omega(x)^*}\;\;\;\;
	{\Omega(x)\atop -(2\Delta+i\gamma)}
	\right).
\end{eqnarray}
%
We assume perpendicular incidence of the atom on the laser sheet for simplicity, oblique incidence is treated e.g. in  \cite{Ruschhaupt2007,Ruschhaupt2009}.
Here $E=\hbar^2 k^2/2m$ is the energy, and
$\Omega(x)$ is the position-dependent, on-resonance Rabi frequency, where real and imaginary parts may be controlled independently
using two  laser field quadratures  \cite{Zhang2013};
$\gamma$ is the inverse of the life time of the excited state;
$\Delta=\omega_{L}-\omega_{12}$
is the detuning (laser angular frequency minus the atomic transition
angular frequency $\omega_{12}$);
${p}=-i\hbar\partial/\partial x$ is the momentum operator;
and ${\bf 1}=|1\ra\la 1|+|2\ra\la 2|$ is the unit operator
for the internal-state space.
Complementary projectors
%
$P=|1\ra\la 1|$ and $Q=|2\ra\la 2|$
%
are defined to select ground and excited state components.
Using the partitioning
technique \cite{Feshbach1958,Feshbach1962,Levine1969},
we find for the ground
state amplitude $\phi_k^{(1)}$ the equation
%
\begin{equation}\label{effecti}
	E\phi_k^{(1)}(x) = H_0\phi_k^{(1)}(x)+\!
	\int\! dy\, \la x,1|{\cal W}(E)|y,1\ra \phi_k^{(1)}(y),
\end{equation}
%
where
%
$
{\cal W}(E)=P{\cal V}P+P{\cal V}Q(E+i0-Q{\cal H}Q)^{-1}Q{\cal V}P,
$
%
is generically non local and energy dependent. Specifically, we have now achieved
a physical realization of an effective (in general) non-local, non-Hermitian potential whose kernel has the form
%
\begin{eqnarray}
	\hspace*{-.3cm}V (x,y) = \la x,1|{\cal W}(E)|y,1\ra = \frac{m}{4} \frac{e^{i|x-y|q}}{i q}
	\Omega(x)\Omega(y)^*,
	\label{effpot}
\end{eqnarray}
%
%
where
$
q=\frac{\sqrt{2mE}}{\hbar}(1+\mu)^{1/2},\;\;
{\rm Im}\,q\ge 0,
\label{qeq}
$ and
$
\mu=\frac{2\Delta+i\gamma}{2E/\hbar}.
$
%
Eq. \eqref{effpot} is worked out  in momentum representation to do the integral
using the residue theorem.
This is a generalized, non-local version of the effective potentials known for the ground state
\cite{Chudesnikov1991,Oberthaler1996}, which are found from Eq. \eqref{effpot}  in the large $\mu$ limit \cite{Ruschhaupt2004a}.
The reflection and transmission amplitudes $R^{r,l}$ and  $T^{r,l}$ may be calculated directly
using  the potential  \eqref{effpot} or as  corresponding amplitudes for
transitions from ground state to ground state in the full two-level theory see Appendix \ref{Appendix:NumericalCalculationOfTandR}.


%
\subsection{Possible symmetries of the non-local potential}
%
The necessary conditions for the different symmetries of the potential \eqref{effpot} are outlined in the second column of  table \ref{condi}. For example,
symmetry III (parity) requires that $V(x,y)=V(-x,-y)$ \cite{Ruschhaupt2017}. Inserting the functional form of the potential from Eq. \eqref{effpot} into this condition, it results in the requirement $\Omega(x) \Omega(y)^* = \Omega(-x) \Omega(-y)^*$. This is fulfilled if $\Omega(x)=\Omega(-x)e^{i \phi}$ with some arbitrary phase freedom $\phi$.

Since $\Omega(x)$ does not depend on $q$, symmetries IV, V and VII imply that symmetry II is obeyed as well (Hermiticity).
Moreover symmetry III (parity) should be discarded for our purpose since it does not allow for asymmetric transmission or reflection
\cite{Ruschhaupt2017}.
This leaves us with three interesting symmetries to explore:
VI, which allows for  asymmetric reflection; VIII which allows for asymmetric transmission, and  I,
which in principle allows for arbitrary asymmetric responses, except for physical limitations imposed by
the two-level model see Appendix \ref{Appendix:NumericalCalculationOfTandR}.


%
%
As seen from  table \ref{condi}, $\operatorname{Re}(q)=0$ makes the potential Hermitian so we shall avoid this condition.
If $\gamma=0$,   $\mu \in \mathbb{R}$. Hence $\mu+1<0$ gives $\operatorname{Re}(q)=0$ and $\mu+1>0$ gives
$\operatorname{Im}(q)=0$. $\mu+1>0$ amounts to a condition on the detuning compared to the incident energy, namely $\Delta>-E/\hbar$.
In the following examples we implement potentials with symmetries VIII, VI, and I, with detunings and energies satisfying the condition $\mu+1>0$.
%
%
%

%
%
%
\section{Design of asymmetric devices\label{exa}}
%
%
We will now apply this method to physically realize non-local potentials of the form \eqref{effpot}.
%
We  shall work out  explicitly a ${\cal T/A}$ device with symmetry  VIII, a ${\cal R/A}$ device with symmetry  VI, and a ``partial''-${\cal TR/A}$ device,  having 1/2 transmission and reflection coefficients from the left, with symmetry I. The  ${\cal T/A}$ and the ``partial''-${\cal TR/A}$ device have transmission asymmetry so they cannot be built with local or $PT$-symmetric  potentials.
%, whereas   to design asymmetric devices which can be shown not to be implementable with a one-dimensional, one-channel, local potential.
Let us  motivate the effort with some possible applications, relations  and analogies of these devices.
${\cal T/A}$ and ${\cal R/A}$ are, respectively, transmission and reflection filters. They are analogous to
half-wave electrical rectifiers that either let the signal from one side ``pass'' (transmitted) or change its sign (reflected)
while suppressing the other half signal.  They may play the role of half-rectifiers in atomtronic circuits.
A ${\cal T/A}$ device allows us, for example, to empty a region of selected particles, letting them go away while not letting particles in.
The ``atom diode'' devices worked out e.g. in \cite{Ruschhaupt2004,Ruschhaupt2006a,Ruschhaupt2006,Ruschhaupt2007}
where of type ${\cal R/A}$. As the mechanism behind them was adiabatic, a broad range of momenta with the desired asymmetry
could be achieved. In comparison the current approach is not necessarily adiabatic so it can be adapted to faster processes.

As for the ``partial''-${\cal RT/A}$ device,  it reflects and transmits from one side while absorbing from the
other side. In an optical analogy an observer from the left perceives it as a darkish mirror.
An observer from the right ``sees'' the other side because of the allowed transmission
but cannot be seen from the left since  nothing is transmitted from right to left. Our device is necessarily ``partial'' one as
there cannot be  net probability gain because of the underlying two-level system, and a ``full'' version with both reflection and transmission coefficients equal to one
would need net gain.

The three devices are worked out for $\gamma=0$, a valid approximation for  hyperfine transitions. %, although this is by no means a necessary assumption.
We assume  for the Rabi frequencies the forms
%
\begin{eqnarray}
	\Omega_{\rm VIII} (x) &=& a [g(x+x_0) + i  g(x-x_0)],
	\nonumber\\
	\Omega_{\rm VI} (x) &=&  b g(x+x_0) + c g(x-x_0),
	\nonumber\\
	\Omega_{\rm I} (x) &=&  - i b g(x+x_0) + c g(x-x_0),
	\label{3omegas}
\end{eqnarray}
%
in terms of smooth, realizable Gaussians $g(x) =  \exp[-{x^2}/{w^2}]$.
We fix $2 d$ as an effective finite width
of the potential area beyond which the potential is negligible and assumed to vanish. We will express in the following the different length parameters as a multiple of $d$ to keep results general.
In addition, we will use as a scaling factor for the velocity $v_{d} = {\hbar}/({m d})$, and for time $\tau={m d^2}/{\hbar}$.

In the following calculations we fix the width of the Gaussians to be $w= {\sqrt{2}}d/{10}$.
We always first set a target velocity $v_0$ to achieve the desired asymmetric scattering response.
The real parameters $a$, $b$, $c$, $x_0$ in Eq. \eqref{3omegas}, and  $\Delta$
are then numerically optimized with the GRAPE (Gradient Ascent Pulse Engineering) algorithm \cite{Khaneja2005,Wu2015}.

The Rabi frequencies will fulfill the indicated symmetries VIII, VI, and I. $\Omega_{\rm VI}(x)$ should not be even (i.e. $b \neq c$) to avoid symmetry II. In addition, $\Omega_{\rm I}(x)$ should not fulfill any other symmetry than ${\rm I}$.
The corresponding Rabi frequencies $\Omega(x)$ are  depicted in Figs. \ref{fig_t_a}, top row.
The scattering coefficients are shown in the bottom row. Fig. \ref{fig_t_a} demonstrates that the three potentials satisfy the asymmetric response conditions imposed
at the selected velocity and also in a region nearby.

The ``partial''-${\cal TR/A}$  device fullfills $\absq{T^l} = \absq{R^l} = 1/2$ and full absorption from the right.
The potential we use for that device has symmetry I only, i.e., ``no symmetry'' other than the trivial commutation with the identity. No other potential symmetry would allow this type of device.

The effective non-local potential $V(x,y)$, see Eq. \eqref{effpot}, corresponding to the $v/v_d$ ratios used for the three devices is shown in Fig. \ref{fig_poten1}. Note that the non-local potential has dimensions  energy/length, so  we
divide the absolute value by a factor $V_0=\hbar^2/(m d^3)$ to plot a dimensionless quantity.


% -------------------------------------------------------------
\begin{figure*}
	\begin{center}
		\includegraphics[width=0.28\linewidth]{Figures/asym_fig_t_a_400_pot.pdf}
		\includegraphics[width=0.28\linewidth]{Figures/asym_fig_r_a_400_pot.pdf}
		\includegraphics[width=0.28\linewidth]{Figures/asym_fig_1_2_tr_a_8_pot.pdf}\\
		\includegraphics[width=0.29\linewidth]{Figures/asym_fig_t_a_400.pdf}
		\includegraphics[width=0.29\linewidth]{Figures/asym_fig_r_a_400.pdf}
		\includegraphics[width=0.29\linewidth]{Figures/asym_fig_1_2_tr_a_8.pdf}
	\end{center}
	\caption{Left column: ${\cal T/A}$ device with symmetry VIII.
	Top: $\Omega_{\rm VIII}(x)$;
	Bottom:  transmission and reflection coefficients. $v_0/v_d=400$, $a\tau = 2618.19$,
	$x_0/d = 0.1532$, $\tau\Delta = 1413.01$.
	%
	Middle column: ${\cal R/A}$ device with symmetry VI.
	Top: $\Omega_{\rm VI} (x)$ (it is real); Bottom:  transmission and reflection coefficients. $v_0/v_d=400$,
	$b \tau =  -244516.1$,
	$c\tau = 167853.9$,
	$x_0/d = 0.1679$,
	$\tau\Delta= 193.508$.
	%
	Right column: ``Partial''-${\cal TR/A}$ device with symmetry I.
	Top:  $\Omega_{\rm I}(x)$, real (blue, solid line) and imaginary parts (orange, dashed line);
	Bottom: transmission and reflection coefficients. $v_0/v_d=8$, $b\tau =  102.6520$,
	$c \tau =  165.8355$,
	$x_0/d = 0.1648$,
	$\tau\Delta= 90.5337$. In all cases $\tau={m d^2}/{\hbar}$ and $v_{d} = {\hbar}/({m d})$.
	\label{fig_t_a}}
\end{figure*}
% -------------------------------------------------------------

%--------------------------------------------------------------------------
\begin{figure*}
	\begin{center}
		\includegraphics[width=0.28\linewidth]{Figures/asym_fig_t_a_400_eff_pot_abs.pdf}
		\includegraphics[width=0.28\linewidth]{Figures/asym_fig_r_a_400_eff_pot_abs.pdf}
		\includegraphics[width=0.28\linewidth]{Figures/asym_fig_1_2_tr_a_8_eff_pot_abs.pdf}\\
		\includegraphics[width=0.29\linewidth]{Figures/asym_fig_t_a_400_eff_pot_arg.pdf}
		\includegraphics[width=0.29\linewidth]{Figures/asym_fig_r_a_400_eff_pot_arg.pdf}
		\includegraphics[width=0.29\linewidth]{Figures/asym_fig_1_2_tr_a_8_eff_pot_arg.pdf}
	\end{center}
	\caption{Nonlocal potentials  $V(x,y)$: absolute value (top), argument (bottom).
	Left column: Potential for ${\cal T/A}$ device with symmetry VIII.
	Middle column: Potential for ${\cal R/A}$ device with symmetry VI.
	Right column: ``Partial''-${\cal TR/A}$ device with symmetry I.
	$V_0=\hbar^2/(md^3)$.}
	\label{fig_poten1}
\end{figure*}
%--------------------------------------------------------------------------

In the parameter optimization we see that increasing the velocities further does not pose a problem for the ${\cal T/A}$
device, it is more challenging for a ${\cal R/A}$ device, and it is quite difficult for the partial-${\cal RT/A}$ device.  The device ${\cal T/A}$ is feasible for an experimental implementation  as the ratio $v_0/v_d$ can be easily increased to desired values, for  reasonable values of the
Rabi frequency and laser waist \cite{Zeyen2016}.

Moreover the velocity width with the desired behavior is much broader for ${\cal T/A}$. Therefore a ${\cal T/A}$
device is the best candidate for
an experimental implementation.
As a check of feasibility, let us assume a Beryllium ion. Its hyperfine structure provides a good  two-level system
for which we can neglect decay (i.e. $\gamma\approx 0$ is indeed realistic). We have $m=1.49\times 10^{-26}$ kg
and set a length $d=10\, \mu$m compatible with the small laser waists (in this case 1.4 $\mu$m) achieved for individual ion
addressing \cite{Zeyen2016}. The scaling factors take the values
%
\begin{eqnarray}
	v_d&=&0.67\, {\rm mm/s},
	\nonumber\\
	\tau&=&1.49\times 10^{-2}\, {\rm s},
	\nonumber
\end{eqnarray}
%
which gives  $v\approx$ 27 cm/s for $v/v_d=400$, (again, we see no major obstacle to get devices for higher velocities,
in particular the classical approximations in Sec. \ref{class} can be used to  estimate the values of the parameters)
and Rabi frequencies, see Fig. \ref{fig_t_a},  in the hundreds of kHz range. The relative ion-laser beam velocity could be as well
implemented  by moving the beam in the laboratory frame.

%
%
\section{Classical approximation for ${\cal T/A}$ device \label{class}}
%
%
In a ${\cal T/A}$ device such as the one presented an incident plane wave from the left ends up as a pure transmitted wave with no reflection or absorption.
However, a wave incident from the right is fully absorbed. How can that be? Should not the velocity-reversed motion
of the transmitted wave lead to the reversed incident wave?
%Obviously that is not the case, and a formal answer to that question  is that $V$ is not time-reversal invariant.
For a more intuitive understanding we may seek help in the underlying two-level model.
In the larger space the potential is again local and Hermitian. A simple semiclassical
approximation is to assume that the particle moves with  constant speeds $\pm v$ for left ($v>0$) or right ($-v<0$) incidence,  so that at a given time it is subjected to  the $2\times2$ time-dependent potentials
${\cal V}(\pm vt)$. The incidence from the left and right give different time dependences for the potential. The scattering problem then reduces to solving the time-dependent Schr\"odinger equation for the amplitudes of a two-level atom with time-dependent potential, i.e. to solving the following time-dependent Schr\"odinger equation ($\gamma = 0$)
%
\begin{eqnarray}
	i \hbar \frac{\partial}{\partial t} \chi_\pm(t)
	= {\cal V} (\pm v t) \chi_\pm(t),
\end{eqnarray}
%
with the appropriate boundary conditions $\chi_+ (-\infty) = \chi_- (-\infty) =\left(\begin{smallmatrix} 1\\ 0\end{smallmatrix}\right)$. The  solutions for $v/v_d = 400$
%9.1
are shown in Fig. \ref{fig_t_a_approx}.
In Fig. \ref{fig_t_a_approx}(a), $\chi_+ (t)$ (left incidence) is depicted:  the particle ends  with high probability in the ground state at final time. In Fig. \ref{fig_t_a_approx}(b), $\chi_- (t)$ (right incidence) demonstrates  the ground state population is transferred to the excited state. Projected onto the ground-state level alone,
this corresponds to full absorption of the ground state population at final time.

For an  even rougher but also illustrative picture,  again in a semiclassical time-dependent framework, we  may substitute the smooth Gaussians for Re$(\Omega)$ and Im$(\Omega)$ in Fig. \ref{fig_t_a} by two simple, contiguous square functions of height
$\Omega>0$ and width $\tilde{w} > 0$. Then, the $2\times2$ potential at a given time is, in terms of Pauli matrices,
%
\begin{eqnarray}
	{\cal V} (x) = \frac{\hbar}{2}\Delta (\sigma_Z-{\mathbf 1})+ \frac{\hbar}{2} \left\{\begin{array}{cc}
	\Omega\sigma_X & -\tilde w < x < 0\\
	-\Omega\sigma_Y & 0 < x < \tilde w\\
	0 & \mbox{otherwise}
	\end{array}\right.
\end{eqnarray}
%
where $x = \pm v t$ and let ${\sf T}=2 \tilde w/v$.

% ------------------------------------------------------------------------------
\begin{figure}
	\begin{center}
		\includegraphics[width=0.48\linewidth]{Figures/asym_fig_t_a_approx_left.pdf}
		\includegraphics[width=0.48\linewidth]{Figures/asym_fig_t_a_approx_right.pdf}
	\end{center}
	\caption{Simplified model of the asymmetric ${\cal T/A}$ device with symmetry VIII: (a) $\chi_+(t)$, (b) $\chi_-(t)$; ground-state population $\absq{\chi_{\pm(t),1}}$ (blue, solid line), excited-
	$\absq{\chi_{\pm(t),2}}$ (orange, dashed line). $v/v_d = 400$, $a\tau = 2618.19$,
	$x_0/d = 0.1532$, $\tau\Delta = 1413.01$.
	\label{fig_t_a_approx}}
\end{figure}
% ------------------------------------------------------------------------------

% ------------------------------------------------------------------------------
\begin{figure}
	\fbox{
	\begin{minipage}{8cm}
		\flushleft  (a) Order of rotations:  first  $R_1({\sf T}/2)$ (left figure) and then $R_2({\sf T}/2)$ (right figure)
		\begin{center}
			\vspace*{-0.32cm}
			\includegraphics[width=0.49\linewidth]{Figures/"asym_fig_sphere_left1"}\,\includegraphics[width=0.49\linewidth]{Figures/"asym_fig_sphere_left2"}\\[0.1cm]
		\end{center}
		(b) Order of rotations: first $R_2({\sf T}/2)$ (left figure) and then $R_1({\sf T}/2)$ (right figure).
		\vspace*{-0.32cm}
		\begin{center}
			\includegraphics[width=0.49\linewidth]{Figures/"asym_fig_sphere_right1"}\,\includegraphics[width=0.49\linewidth]{Figures/"asym_fig_sphere_right2"}
		\end{center}
		\vspace*{-.8cm}
	\end{minipage}
	}
	\caption{Simplified time-dependent model of the asymmetric ${\cal T/A}$ device with symmetry VIII: Bloch sphere explaining non time-reversal invariance, see text for details. The state trajectories are depicted in two-steps on the sphere. The rotation axes are
	also depicted. (a) The process simulates incidence from the left. The state starts and ends in $|1\ra$. (b) The process simulates incidence from the right. The state starts at $|1\ra$ and ends at $|2\ra$.
	\label{fig_t_a_simple2}}
\end{figure}
% ------------------------------------------------------------------------------


The time-evolution of this process, $\chi_\pm (t)$,
up to a phase factor may be regarded as
two consecutive rotations $R_j=e^{-i{\beta}{\bf n}_j\cdot {\boldsymbol{\sigma}}/2}$ ($j=1,2$), with $\beta=\frac{{\sf T}}{2}\sqrt{\Omega^2+\Delta^2}$, of the two-level state on the Bloch sphere about the axes
%
\begin{eqnarray}
	{\bf n}_1&=&\frac{1}{\sqrt{\Omega^2+\Delta^2}}(\Omega,0,\Delta), %real
	\\
	{\bf n}_2&=&\frac{1}{\sqrt{\Omega^2+\Delta^2}}(0,-\Omega,\Delta). %imag
\end{eqnarray}
%
The initial state at time $t=-{\sf T}/2$ is again $\chi_+ (-{\sf T}/2) = \chi_- (-{\sf T}/2) =\left(\begin{smallmatrix} 1\\ 0\end{smallmatrix}\right)$.
The unitary time-evolution operator to reach the final time ${\sf T}/2$ takes the form
$e^{i\Delta {\sf T}/2}R_2 R_1$ for  incidence from the left ($\chi_+$) and
$e^{i\Delta {\sf T}/2}R_1 R_2$ for incidence from the right ($\chi_-$).
The time ${\sf T}$ and the parameters $\Omega, \Delta$ will be fixed to reproduce the results of the full calculation with the exact model, namely,
so that the system starts in the ground state to end either in the ground state
($\absq{\chi_{+} ({\sf T}/2)} = 1$)
or in the excited state by performing the rotations in one order or the reverse order
($\absq{\chi_{-} ({\sf T}/2)} = 0$). This gives $\Omega/\Delta = \sqrt{2}$ and ${\sf T}= 4\pi/(3 \sqrt{3} \Delta)$. It follows that ${\bf n}_1=\frac{1}{\sqrt{3}}(\sqrt{2},0,1)$ and ${\bf n}_2=\frac{1}{\sqrt{3}}(0,-\sqrt{2},1)$.

The different outcomes can thus be understood as the result of the \linebreak non-commutativity of rotations on the Bloch sphere, see
Fig. \ref{fig_t_a_simple2}: In Fig. \ref{fig_t_a_simple2}(a), first the rotation $R_1({\sf T}/2)$ and then the rotation $R_2({\sf T}/2)$ are performed. Starting in the ground state $\ket{1}$, the system ends up  in the excited state $\ket{2}$.
In Fig. \ref{fig_t_a_simple2}(b),  first the rotation $R_2({\sf T}/2)$ and then the rotation $R_1({\sf T}/2)$ are performed:  now the system starts and ends  in the ground state $\ket{1}$.

These results can be even used to approximate the parameters of the potential in the quantum setting.
As an approximation of the height $a$ we assume that the area $a \int_{-\infty}^\infty dx \, g(x) = a \sqrt{\pi} w$
is equal to $\tilde w \Omega = {{\sf T}} v_0 \Omega/2 =
v_0 \pi ({2}/{3})^{3/2}$. This results in an
approximation $a \approx \frac{v_0}{w} \sqrt{\pi}\, ({2}/{3})^{3/2}$. As an additional approximation, we
assume that $(a/\sqrt{2})/\Delta \approx {\Omega}/{\Delta} = \sqrt{2}$, so we get
$\Delta \approx a/2 \approx \frac{v_0}{2 w} \sqrt{\pi}\, ({2}/{3})^{3/2}$. A comparison between
these approximations and the numerically achieved parameters, see Fig. \ref{fig_t_a_param}, shows a good agreement
over a large velocity range. This allows one to find good initial values for further numerical optimization.

% ------------------------------------------------------------------------------
\begin{figure}
	\begin{center}
		\includegraphics[width=0.48\linewidth]{Figures/asym_fig_t_a_param1.pdf}
		\includegraphics[width=0.48\linewidth]{Figures/asym_fig_t_a_param2.pdf}
	\end{center}
	\caption{Asymmetric ${\cal T/A}$ device with symmetry VIII: comparison between numerically achieved parameters (red dots) and approximated parameters (blue, solid lines) versus velocity $v_0$.
	(a) Height of Rabi frequency $a$, (b) detuning $\Delta$.
	\label{fig_t_a_param}}
\end{figure}
% ------------------------------------------------------------------------------

%
%
%
%

%
%
\section{Discussion}
Non-Hermitian Hamiltonians display many interesting phenomena which are
impossible for a  Hermitian Hamiltonian  acting on the same Hilbert space. In particular, in the Hilbert space of
a single, structureless particle on a line formed by square integrable normalizable functions, Hermitian Hamiltonians do not allow, within a linear theory, for asymmetric scattering transmission and reflection coefficients.
% for right/left incidence of a particle off a potential center.
However,
non-Hermitian Hamiltonians do.   Since devices of technological interest, such as one-way filters for transmission or reflection, one-way barriers, one-way mirrors, and others, may be built based on such scattering response asymmetries, there is both fundamental
interest and applications in sight to implement Non-Hermitian scattering Hamiltonians.    This paper is a step forward in that direction, specifically we propose a quantum-optical implementation of potentials with asymmetric scattering response.
They are non-local and non-PT symmetrical, which allows for asymmetric transmission.

In general the chosen Hilbert space may  be regarded as a subspace of a larger space. For example,  the space of a ``structureless particle'' in 1D is the ground-state subspace
for a particle with internal structure, consisting of two-levels in the simplest scenario.
It is then possible to regard the Non-Hermitian physics in the reduced space
as a projection of the larger space, which may itself be driven by a  Hermitian or a Non-Hermitian Hamiltonian.
We have seen the Hermitian option in our examples, where we assumed a zero decay constant, $\gamma=0$, for the excited state.
A non-zero $\gamma$ implies  a Non-Hermitian  Hamiltonian in the larger two-level space. The description may still be
enlarged,  including  quantized field modes to account for the atom-field interaction with a Hermitian Hamiltonian.
As an outlook, depending on the application, there might be the need for a more fundamental and detailed descriptive level. Presently we discuss the desired physics (i.e., the scattering asymmetries) at the level of the smallest 1D space of the ground state, while taking refuge in the
two-level space to find a feasible physical implementation.
    % Physical Implementation of non-Hermitian and non-local Hamiltonians


% ------------ PART II: Heat Rectification ---------------------------------------------------------
\part{Heat rectification in mesoscopic systems\label{partII}}
%!TEX root = ../Thesis.tex
%Chapter 4

\chapter{Local Rectification of Heat Flux}
\label{Chapter4}
\lhead{Chapter 4. \emph{Local Rectification of Heat Flux}} % Write in your own chapter title to set the page header
%
In this chapter, a model for an atom-chain thermal rectifier is presented. The atoms in the chain are trapped in on-site harmonic potentials, and interact with their nearest neighbours by Morse potentials (or also by harmonic potentials in a simplified version). The chain is homogeneous except for a local modification of the interactions and trapping potential at one site, the ``impurity''. The rectification mechanism is due here to the localized impurity, the only asymmetrical element of the structure, apart from the externally imposed temperature bias, and does not rely on putting in contact different materials or other known mechanisms such as grading or long-range interactions.  The effect survives if all interaction forces are linear except the ones for the impurity.

The rest of the chapter is organized as follows. In section \ref{sec:homogeneous_chain}, I shall describe the homogeneous 1D chain, without the impurity.  For this system, I numerically solve the dynamical equations, to show that the usual heat conduction applies. In section \ref{sec:Impurity_rectifier}, I modify the potentials for one of the atoms and demonstrate the rectification effect. I also observe rectification when all the interaction Morse potentials are substituted by harmonic oscillators. Finally, in section \ref{sec:chapter4_Discussion}, I summarize and discuss the results of this chapter.

\section{Homogeneous one-dimensional chain\label{sec:homogeneous_chain}}

I start with a homogeneous 1D chain with $N$ atoms coupled at both extremes to heat baths, at different temperatures $T_h$ and $T_c$ for ``hot" and ``cold" respectively. The baths are modeled with a Nos\' e-Hoover method as described in \cite{Martyna1992}. Atoms $1$ and $N$ represent the first and the $N$-th atom in the chain, from left to right, that will be in contact with the baths. All the atoms are subjected to on-site potentials and to nearest-neighbor interactions, and their equilibrium positions $y_{i0}$ are assumed to be equally spaced by a distance $a$.
$x_i= y_i-y_{i0}$,
$i=1,...,N$, represent the displacements from the equilibrium positions of the corresponding atoms
with positions $y_i$.

%%%%%%%%%%%%%%%%%%%%%%%
\begin{figure}
\centering
\includegraphics[width=0.65\linewidth]{Figures/FIG1.pdf}
\caption{(a) On-site potentials: harmonic potential centered at the equilibrium position of each atom (dashed blue line) as a function of the displacement from this position $x_i=y_i-y_{i0}$ in $a-$units, and the on-site potential for the impurity, $i=N/2+1$
($N$ even, red solid line). (b) Interaction potentials as a function of the distance between nearest neighbors: Morse potential
(blue dashed line) valid for all atoms except for $i=N/2+1$, $N$ even, where the modified potential (red solid line) is used.
The harmonic approximation of the Morse potential is also depicted (eq. (\ref{Vhar}), black dots, only used for fig. \ref{fig:chapter4_figure5}, below).
Parameters: $D=0.5$, $g=1$, $\gamma = 45$, $d=100$ and $b=105$, used throughout the chapter.
}
\label{fig:chapter4_figure1}
\end{figure}
%%%%%%%%%%%%%%%%%%%%%%%%%

The classical Hamiltonian of the atom chain can be written in a general form as
%
\begin{equation}
\label{GH}
%GH=general Hamiltonian
H=\sum_{i=1}^{N} H_i,
\end{equation}
%
with
%
\begin{eqnarray}
\label{GH2}
%GH=general Hamiltonian
H_1&=&{{p^2_1} \over {2m}} +U_1(x_1)+V_L,
\nonumber\\
H_i&=&{{p^2_i} \over {2m}} +U_i(x_i)+V_i(x_{i-1},x_i)  \quad i=2,...,N-1,
 \nonumber\\
H_N&=&{{p^2_N} \over {2m}} +U_N(x_N)+V_N(x_{N-1},x_N) + V_R,
\end{eqnarray}
%
where the $p_i$ are the momenta, $U_i(x_i)$ is the on-site potential for the $i$th atom, and $V_i(x_{i-1},x_i)$ represents the atom-atom interaction potential. $V_R$ and $V_L$ are the interactions coupling the boundary atoms to the Nos\'e-Hoover thermostats, see \cite{Martyna1992}.

%%%%%%%%%%%%%%%%%%%%%%%%%%%%
\begin{figure}
\centering
\includegraphics[width=0.65\linewidth]{Figures/FIG2.pdf}
\caption{Symmetric temperature profiles for a homogeneous chain, without impurity.  For $T_{h}=T_{L}$, $T_c=T_R$ (red solid dots) the (absolute value of) the heat flux is $J_{L\rightarrow R}$, equal to $J_{R\rightarrow L}$ for the reverse configuration of the bath temperatures, $T_{h}=T_{R}$, $T_c=T_L$
(black empty squares). Parameters as in fig. \ref{fig:chapter4_figure1}.}
\label{fig:chapter4_figure2}
\end{figure}
%%%%%%%%%%%%%%%%%%%%%%%%%%%%%%

There are a large number of 1D models that obey this general Hamiltonian. Different choices of the trapping and interaction potentials would give different conductivity behaviors. I choose a simple form of the Hamiltonian in which each atom is subjected to a harmonic on-site potential and a Morse interaction potential between nearest neighbors (see fig. \ref{fig:chapter4_figure1}, dashed lines),
%
\begin{eqnarray}
\label{HO}
%HO=Harmonic oscillator
U_i(x_i)&=&{1 \over 2} m \omega^2 x^2_i,
%\end{equation}
%\begin{equation}
\\
\label{IH}
%IP=Interaction potential
V_i(x_{i-1},x_i)&=&D\left \{e^{-\alpha [x_i-x_{i-1}]}-1\right \}^2,
\end{eqnarray}
%
where $\omega$ is the trapping angular frequency, and $D$ and $\alpha$ are time-independent parameters of the Morse potential.
A ``minimalist version'' of the model where $V$ becomes the harmonic limit of eq. (\ref{IH}), dotted line in fig. 1,
 will also be considered in the final discussion,
%
\begin{equation}
\label{Vhar}
{V}_i(x_{i-1},x_i)=k(x_i-x_{i-1})^2/2,\;k=2D\alpha^2.
\end{equation}
%
For convenience, dimensionless units are used and the mass of all particles is set to unity.

I start by studying the homogeneous configuration with no impurity and potentials (\ref{HO}) and (\ref{IH}), solving numerically the dynamical equations for  the Hamiltonian (\ref{GH}) with a Runge-Kutta-Fehlberg algorithm. I have chosen a low number of atoms, $N=20$,  with thermal baths at $T_h=0.20$ and $T_c=0.15$ at both ends of the chain with 16 thermostats each. The real temperature is related to the dimensionless one through $T_{real}=T m a^2 \omega^2/k_B$ so, for typical values  $m\approx10^{-26}$ kg, $\omega \approx 10^{13}$ s$^{-1}$, $a\approx 10^{-10}$ m, and using $k_B =1.38 \times 10^{-23}$ JK$^{-1}$,
the dimensionless temperatures $0.15,\, 0.20$, translate into $100,\, 150$ K. It is advisable to use temperatures around these values in order to ensure that the displacements of the particles are realistic \cite{Casati1984}.

%%%%%%%%%%%%%%%%%%%%%%%%%%%%%%%%%%
\begin{figure}
\centering
\includegraphics[width=0.65\linewidth]{Figures/FIG3.pdf}
\caption{Temperature profile along the homogeneous chain for different number of atoms: 100 (dotted black line), 125 (dashed blue line) and 150 (solid red line). The atom sites have been rescaled with the total number of atoms.
%, showing the convergence of the spatial profile of the local temperature $T_i$.
The time averages have been carried over a time interval of $\approx 2 \times 10^6$ after a transient of $\approx 1\times 10^5$. In the inset (a), the product $JN$ vs. $N$ demonstrates that for long chains $JN$ is independent of $N$. In (b) the linear dependence of $J$ with $\Delta T$ for a fixed number of atoms, $N=100$, is shown. Parameters as in fig. \ref{fig:chapter4_figure1}.}
\label{fig:chapter4_figure3}
\end{figure}
%%%%%%%%%%%%%%%%%%%%%%%%%%%%%%%%%%%%%%%%

First I demonstrate the conductivity behavior of the model.
%that our system satisfies Fourier's heat law for the heat flux, $J=\kappa \nabla T$.
%, so it shows normal thermal conductivity. ESTE CONCEPTO TIENE QUE VER CON EL TAMA�O?
To this end, I calculate the local heat flux $J_i$ and temperature $T_i$, performing the numerical integration
%of eq. (\ref{GH2})
for long enough times to reach the stationary state.
The local temperature is found as the time average $T_i= \langle p_i^2 / m \rangle$, whereas
%After a transient, the local temperature is given by the time average $T_i=\langle p_i^2\rangle$.
$J_i$,  from the continuity equation
%, $\dot H(x,t)+divJ(x,t)=0,$
\cite{Hu1998}, is given by
%
%Fourier law: temperature gradient vanishes with N
\begin{equation}
\label{heatflux}
J_i=-\dot x_i {{\partial V(x_{i-1},x_{i})} \over {\partial x_i}}.
\end{equation}
%
From now on I only consider the time average $\langle J_i (t)\rangle$, which converges to a constant value for all sites once the system is in the stationary nonequilibrium state. I depict the temperature profiles, for $N=20$, first with $T_L=T_h$ and $T_R=T_c$
($L$ and $R$ stand for left and right) and after switching the positions of the thermal baths in fig. \ref{fig:chapter4_figure2}. The profiles are symmetric, as expected, and the heat flux does not have a preferred direction  \cite{Hu1998,Terraneo2002}. Denoting the absolute values of the fluxes from the left (when $T_L=T_h$) as
$J_{L\rightarrow R}$, and from the right (when $T_R=T_h$) as
$J_{R\rightarrow L}$, I find that $J_{L\rightarrow R}=J_{R\rightarrow L}=J=1.6\times 10^{-2}$, in the dimensionless units, consistent with the values found in other models \cite{Terraneo2002,Hu1998}.

%%%%%%%%%%%%%%%%%%%%%%%%%%%%%%%%%%%
\begin{figure}
\centering
\includegraphics[width=0.65\linewidth]{Figures/FIG4b.pdf}
\caption{Temperature profile for the chain of $N=20$ atoms, with an impurity in the $N/2+1$ position, with $T_L=T_h$ and $T_R=T_c$ (circles) and with the thermostat baths switched (squares).
Parameters as in fig. \ref{fig:chapter4_figure1}.
(a) $T_c=0.15$, $T_h=0.2$. $J_{L\rightarrow R}=0.00769$ vs $J_{R\rightarrow L}=0.00581$, with gives a rectification $R=31 \% $; (b) $T_c=0.025$, $T_h=0.325$. $J_{L\rightarrow R}=0.0499$ vs  $J_{R\rightarrow L}=0.0140$, with $R=256 \%$.}
\label{fig:chapter4_figure4}
\end{figure}
%%%%%%%%%%%%%%%%%%%%%%%%%%%%%%%%%%%%%%%%%

The profile of the temperature is linear with boundary non-linearities at the edges, close to the thermal baths,  due to the boundary conditions \cite{Lepri1997}. In fig. \ref{fig:chapter4_figure3}, I depict $T_i$ vs $i/N$ for $N=100, 125$ and $150$ with the same boundary conditions. For these
larger atom numbers  I have connected the first 3 and the last 3 atoms to the Nos\'e-Hoover baths.
%The temperature gradient scales as $N^{-1}$, which is also true for many other different models \cite{Hu1998}.
In the inset (a) of fig. \ref{fig:chapter4_figure3}  the product $JN$ vs. $N$ is plotted, showing that for a low $N$ limit there is a well defined conductivity per unit length whereas for longer chains, $JN$ tends to be constant  which indicates a normal thermal conductivity independent of the length. Fixing the number of atoms to 100, as in the inset (b) of fig. \ref{fig:chapter4_figure3},  I observe a linear dependence between the flux and $\Delta T$.
%Fourier law, $J=\kappa \nabla T$, is fulfilled.

\section{Impurity-based thermal rectifier \label{sec:Impurity_rectifier}}

To rectify the heat flux I modify the potentials for site $j=N/2+1$ with $N$ even, as
%
%\begin{equation}
%\label{IMP1}
%IMP1=impurity in absolute position
%U_j(y_j,t)=d e^{-b [y_j(t)-y_{d}]} +ge^{-\gamma [y_j(t)-y_{j-1}(t)-\epsilon]}
%\end{equation}
%with $y_{d}=y_{d,0}-a/3$.  Written in terms of the displacements, $x_j$,
%
\begin{eqnarray}
\label{IMP}
%IMP=impurity
U_j(x_j,t)&=&d e^{-b [x_j(t)+a/3]},
\\
V_j(x_{j-1},x_j,t)&=&ge^{-\gamma [x_j(t)-x_{j-1}(t)+a/2]}.
\end{eqnarray}
%
All the parameters involved, $d, b$, and $g,\gamma$ are time-independent. In fig. \ref{fig:chapter4_figure1} the modifications introduced with respect to the ordinary sites are shown (solid lines).  The different on-site and interaction terms introduce soft-wall potentials
(instead of hard-walls to aid in integrating the dynamical equations) that make it difficult for the impurity to transmit its excitation to the left whereas left-to-right transmission is still possible.
This effect is facilitated by the relative size of the coefficients, $a/3<a/2$, that determine the position of the walls.
% that I fixed after some experimentation.
These positions imply that an impurity excited by a hot right bath cannot affect its left cold neighbour near its equilibrium position at the $j-1$ site.
However, if the left $j-1$ atom is excited from a hot bath on the left,
it can get close enough to the impurity to kick it and transfer kinetic energy.

\begin{figure}
\centering
\includegraphics[width=0.65\linewidth]{Figures/FIG5new.pdf}
\caption{Rectification factor $R$ as a function of the temperature difference between ends of the chain of atoms, $\Delta T$.
%The rectification factor shows a very strong dependency on $\Delta T$.
I have changed both $T_h$ and $T_c$ according to $T_c=0.15-(\Delta T-0.05)/2$ and $T_h=0.2+(\Delta T-0.05)/2$, with $N=20$,  keeping the rest of parameters as in fig. \ref{fig:chapter4_figure1}.
Interatomic potentials: Morse potential, eq. (\ref{IH}) (black line with circles, see the temperature profiles of extreme points in fig. \ref{fig:chapter4_figure4}); harmonic potential, eq. (\ref{Vhar}) (red line with squares).}
\label{fig:chapter4_figure5}
\end{figure}

After extensive numerical simulations, I have chosen the values of these parameters as in fig. \ref{fig:chapter4_figure1}, such that the conductivity in the forward direction, $J_{L\rightarrow R}$, and the rectification factor, defined as $R=(J_{L\rightarrow R}-J_{R\rightarrow L}) / J_{R\rightarrow L}\times 100$,
are both large for $T_h=0.2$, $T_c=0.15$. A large $R$ without a large $J_{L\rightarrow R}$ could in fact be useless \cite{Roberts2011}.
%($R=0$ would represent a perfectly symmetric heat conduction.).
Note that the parameters are not necessarily the optimal combination, which in any case would depend on the exact definition of ``optimal'' (technically on how $J_{L\rightarrow R}/J$ and $R$ are weighted and combined in a cost function and on the limits imposed on the
parameter values). This definition is an interesting question but it goes beyond the scope of this chapter, which is to demonstrate and discuss the effect of the localized impurity.

I have used again $N=20$ atoms connected to baths of 16 thermostats each, with the same temperatures as for the homogeneous chain, and numerically solved the dynamical equations
to calculate the local temperature and the heat flux for both configurations of the baths. The interatomic potential for the regular atoms is the Morse potential (\ref{IH}).
In fig. \ref{fig:chapter4_figure4}(a), the temperature profiles show a clear asymmetry between ${L\rightarrow R}$ and ${R\rightarrow L}$. Specifically, I find $J_{L\rightarrow R}=7.6 \times 10^{-3}$ and $J_{R\rightarrow L}=5.8 \times 10^{-3}$ which gives
$R=31\%$. The effect decays with longer chains,  with, for example, $R=19\%$ for $N=100$, and R=17.8\% for $N=150$.

\begin{figure}
\centering
\includegraphics[width=0.65\linewidth]{Figures/FIG6.pdf}
\caption{Temperature profile for a harmonic interacting chain of $N=20$ atoms, with an impurity in the $N/2+1$ position, with $T_L=T_h$ and $T_R=T_c$ (circles) and with the thermostat baths switched (squares), for (a) $\Delta T = 0.05$ and (b)  $\Delta T = 0.3$. The corresponding rectification factors are (a) $R=18\%$ and (b) $R=85\%$. Parameters regarding the impurity are the same as in fig. \ref{fig:chapter4_figure1}.
}
\label{fig:chapter4_figure6}
\end{figure}

These temperature profiles depend on the difference between the bath temperatures, see e.g. fig. \ref{fig:chapter4_figure4}(b). Increasing the temperature gap, but  keeping $T_h$ low enough so that the displacement of the atoms from their equilibrium positions is realistic, I find higher values of $R$. Figure \ref {fig:chapter4_figure5} shows the strong dependence of $R$ with $\Delta T$ (black circles). I have changed both $T_h$ and $T_c$ so that the mean temperature $(T_c+T_h)/2$ remains constant.

\section{Discussion\label{sec:chapter4_Discussion}}

I have presented  a scheme for thermal rectification using a one-dimensional chain of atoms which is homogeneous except
for the special interactions of one of them, the impurity, and the couplings with the baths at the boundaries. These proof-of-principle results for an impurity-based rectification mechanism may encourage further exploration of the impurity-based rectification, in particular of the effect of different forms for the impurity on-site potential and its interactions with neighboring atoms.
In contrast to the majority of chain models, the structural asymmetry in the present model is only in the impurity. The idea of a localized effect was already implicit in early works on a two-segment Frenkel-Kontorova
model \cite{Li2004,Hu2006}, where rectification depended crucially on the interaction constant coupling between the two segments.
However, the coupling interaction was symmetrical and the asymmetry was provided by the different nature
(parameters) of the segments put in contact.
Also different from common chain models are the potentials chosen here. Instead of using the Morse potential as an on-site model, see e.g.  \cite{Terraneo2002},
I have considered a natural setting where this potential characterizes the interatomic interactions,
and the on-site potential is symmetrical with respect to the equilibrium position, and actually harmonic.
The numerical results indicate that this model is consistent with normal conduction,
and also helps to isolate and identify the local-impurity mechanism for rectification.
In this regard it is useful to consider a further simplification, in the spirit of the minimalists models
proposed by Pereira \cite{Pereira2017}, so as to distill further the essence of the local rectification mechanism.
If the Morse interatomic interaction is substituted by the corresponding harmonic interaction, see the black dotted line in fig. \ref{fig:chapter4_figure1}(b), the rectification effect remains, albeit slightly reduced, see fig. \ref{fig:chapter4_figure5}. The chain is then perfectly linear with the only non-linear exception  localized
at the impurity.
The temperature dependent feature mentioned in \cite{Pereira2017} as the second necessary condition for rectification besides asymmetry, is here localized in the impurity too, and consists of a different
capability to transfer kinetic energy depending on the temperatures on both sides of the impurity.
Figure \ref{fig:chapter4_figure6} shows temperature profiles for the purely harmonic chain to be compared with the Morse-interaction
chain in fig. \ref{fig:chapter4_figure4}. Flatter profiles are found on both sides of the impurity, as corresponds to the abnormal transport expected for harmonic chains \cite{Lepri2003}. It would be interesting to combine the impurity effect with other rectification mechanisms (such as grading, long-range interactions, or use of different segments), or with more impurities in series to enhance further the rectification effect.

Even though the motivation was to mimic the effect of a localized atom diode that lets atoms pass only one way,
unlike the atom diode \cite{Ruschhaupt2004}, all interactions in the present model
are elastic. The model may be extended by adding an irreversible,  dissipative element so as to induce not only rectification but a truly Maxwell demon for heat transfer \cite{Skordos1992,Ruschhaupt2006}.
On the experimental side, one dimensional chains of neutral atoms in optical lattices can be implemented with cold atoms \cite{Bloch2005}.
An impurity with different internal structure could be subjected to a different on-site potential imprinted by a holographic mask \cite{Bakr2009}, and asymmetrical interatomic interactions could be implemented by trapping a controllable polar molecule or mediated by atoms in parallel lattices \cite{Gollub2014}.
    % Heat Rectification with local impurities
%!TEX root = ../Thesis.tex
%Chapter 5

\chapter{Engineering fast and stable splitting of matter waves}
\label{Chapter5}
\lhead{Chapter 5. \emph{Engineering fast and stable splitting of matter waves}} % Write in your own chapter title to set the page header
%
Some lines to start the chapter, acting as kind of abstract.
%
\newpage
%
\section{Introduction}

Now, you can write the content of the chapter and organize it into sections.
    % Heat Rectification in graded chains of trapped ions
%!TEX root = ../Thesis.tex
%Chapter 6

\chapter{Rectification in a minimal model}
\label{Chapter6}
\lhead{Chapter 6. \emph{Rectification in a minimal model}} % Write in your own chapter title to set the page header
%
We study heat rectification in a minimalistic model composed of two unequal atoms subjected to linear forces and in contact with effective Langevin baths
induced by Doppler lasers. Analytic expressions of the heat currents in the steady state are spelled out. Asymmetric heat transport is found in this linear system if both the bath temperatures and the temperature dependent bath-system couplings are exchanged. The model can be realized with two ions  in  either common or individual traps. This physical setting allows for a natural temperature
dependence of the coupling to the baths.
We also explore the parameter space of the model to optimize asymmetric heat current and find
conditions for maximal rectification. High rectification corresponds to a good match of the power spectra of the ions for forward temperature bias and
mismatch  for reverse bias, which may be understood by the behavior of dissipative normal modes.
%
\newpage
%
\section{Introduction \label{sec:Introduction}}
%
Heat rectification is the physical phenomenon, analogous to electrical current rectification in diodes, in which heat current through a device or medium (the thermal diode or rectifier) is not symmetric with respect to the exchange of the bath temperatures at the boundaries. It was  first observed in 1936 by Starr in a junction between copper and
cuprous oxide \cite{Starr1936}. The theoretical work started much later  using as rectifiers simple anharmonic chain models
with different segments \cite{Terraneo2002,Li2004}.
These papers sparked much research that
continues to this day. The field remains very active driven by potential applications in fundamental science and technology for thermal management and signal processing, and also because
none of the proposals so far appears to be efficient enough for
practical purposes, i.e., highly conducting in one direction and insulating in the other one. The studies have also branched into different subfields
and systems (e.g. quantum or  classical \cite{Pereira2019}, and for macroscopic, mesoscopic,  or microscopic devices),
that need  specific treatments. A full account of the
developments and results is out of the scope of this introduction, but  several good reviews  are available for a broad perspective  \cite{Roberts2011,Li2012,Pereira2019,Ma2019}. We merely mention in passing,  important  progress on
nanostructures \cite{Li2012,Ma2019},
macroscopic devices \cite{Roberts2011},  or time dependent drivings \cite{Li2012,Riera-Campeny2019}.
Instead we shall focus  on some aspects more closely connected to the present work that help to  motivate it and put it in context.
The rectification in the first  models was explained by the different temperature dependences of the phonon bands (power spectra) of the  segments  \cite{Terraneo2002,Li2004}. A match or mismatch of the spectra of neighboring parts implies corresponding good or bad conduction so the
sign of the temperature bias may affect the conduction and lead to rectification when the spectra of the parts are affected differently
by the bias reversal. Interaction potential anharmonicities (i.e. non-linear forces) imply dependences of the spectra on temperature and thus have been regarded recurrently as an essential element for rectification   \cite{Li2012,Li2008,Hu2006,Zeng2008,Segal2005,Segal2005b,Katz2016,Benenti2016}.
However Pereira pointed out \cite{Pereira2017} that anharmonicity is not a necessary condition for rectification.
Rectification also occurs in simple (minimalistic) harmonic models
that incorporate some structural asymmetry and temperature-dependence of the model parameters. This dependence may indeed result from
an underlying, more intricate  anharmonic  system by linearization of the stochastic dynamics \cite{Pereira2017,Pereira2019},
or it may have a different origin \cite{Simon2019}.
In general minimalistic models, harmonic or otherwise,  provide insight and serve to guide further work towards effective rectifiers.

To look for higher rectification factors, the use of graded  materials \cite{Yang2007}
and long range interactions (LRI) was  put forward \cite{Pereira2013,Chen2015}.
It was noted recently that LRI naturally occurs as a result of the Coulomb interaction in
a  chain of cold ions in Paul traps \cite{Simon2019}, and that this system may serve to bridge the gap between
simple models and experimental realizations. In Ref. \cite{Simon2019} the gradation was incorporated
by ramping the frequencies of individual traps. Moreover the linear regime (when the potentials are well approximated harmonically)
is realistic for trapped ions, and  shows rectification because of the temperature dependence of  the coupling to the effective bath
induced by Doppler cooling lasers.  Trapped ions constitute a well-developed and tested  architecture for fundamental research,
quantum information processing and
quantum technologies such as detectors or metrology. This architecture is  in principle scalable in driven ion circuits, see e.g.  \cite{Bruzewicz2019}.
Controllable heat rectification in this context
would be a useful asset for energy management.

%Several of the trends and ideas mentioned so far motivate this article. In particular
In this article we put forward, along this line of trapped ion physics, an even simpler, minimalistic model
for rectification
implemented by two neighboring atoms of different mass interacting harmonically
and in contact with thermal baths with temperature dependent couplings (LRI does not play any role in the two-ion configuration but it
would affect the dynamics of longer chains).
The model admits a  natural realization in terms of two trapped ions subjected to  Doppler cooling lasers, which provide the
necessary temperature dependence of the coupling parameters. Apart from the possibility of a physical realization, another interesting
feature is the analytical treatment, which facilitates greatly the exploration in parameter  space
to  identify regimes of maximal rectification.
The explicit solution of the stationary regime also provides tools for a better understanding of the physics and enhanced control.
For example the match or mismatch of the spectra of the two masses for forward and reverse bias configurations, which will be made evident
for the parameters with maximal
rectification, may be analyzed in terms of dissipative normal modes characterized by complex eigenvalues.


The model may be compared and related to other simple models. The localization of the
structural asymmetry in a  small spatial region, by a ``defect'', ``impurity'', or asymmetrical molecule has been proposed e.g. in
several anharmonic models \cite{Segal2005b,Pons2017,Alexander2020}.
Segal and Nitzan proposed models with some similarities to ours \cite{Segal2005,Segal2005b}, specifically an anharmonic chain
with different couplings to both baths. They also worked out  quantum models \cite{Segal2005,Segal2005b} in terms of an N-level
system asymmetrically coupled to the baths. Both types of models have ``harmonic limits'', which in the chain is reached by making the potentials
harmonic, and in the quantum one by taking $N$ to infinity assuming equispaced levels.
The asymmetrical couplings however, were not interchanged when reversing the temperature bias
(in these models that interchange would have suppressed the asymmetry because the forward bias configuration becomes a mirror image of the reverse bias one), so that
the harmonic limit did not give any rectification.










%In 2002 a paper by Terraneo \textit{et al.} \cite{Terraneo2002} demonstrated heat rectification numerically for a chain of nonlinear oscillators in contact with two thermal baths at different temperatures. Since then, there has been a growing interest in heat rectification  \cite{Roberts2011,Ye2017,Wang2008,Wang2007,Casati2006,Joulain2016,Chang2006,Kobayashi2009,Leitner2013,Elzouka2017,Pons2017,Alexander2020}, and
%Much effort has been devoted to  understand the underlying physical mechanism responsible for  rectification \cite{Pereira2019}.
%
%However, a work by Pereira \textit{et al.} \cite{Pereira2017} showed that rectification can also be found in effective harmonic systems if two requirements are met: some kind of structural asymmetry, and features that depend on the temperature so they change as the baths are inverted. Indeed,  in this article we demonstrate rectification in a minimalistic model of two harmonic oscillators where the coupling to the baths depends on the temperature.
%
%This will be justified with a particular physical set up with trapped ions and lasers.

%

The article is organized as follows: In Section \ref{sec:Physical_Model}
we describe the physical model and its dynamical equations. In Section \ref{sec:covMatrix} we introduce the  covariance matrix and  derive the equation  that it satisfies in the steady state. In Section \ref{sec:solutions} we solve the covariance matrix equation and find analytical expressions for the steady-state temperatures of the masses and heat currents. In Section \ref{sec:TrappedIonSetUp} we relate the parameters of the model to those of Doppler cooled trapped ions. In Section \ref{sec:lookingForR} we look for configurations with high rectification. We also study the power spectra of the oscillators, which confirm the match or mismatch pattern for rectification. Finally, in Section \ref{sec:Conclusions} we summarize the results and present the conclusions.

\begin{figure}
  \center
  \includegraphics[width=1.1\linewidth]{Figures/model_diagram.pdf}
  \caption{Diagram of the model described in Section \ref{sec:Physical_Model}. Two masses are coupled to each other through a spring constant $k$. Each mass is harmonically trapped and connected to a bath characterized by its temperature $T_i$ and its friction coefficient $\gamma_i$. }
  \label{fig:model_diagram}
\end{figure}
%
%
%
%
\section{Physical Model \label{sec:Physical_Model}}
%
%
%
%
The physical model consists of two masses $m_1$ and $m_2$ coupled to each other by a harmonic interaction with spring constant $k$ and natural length $x_e$. The masses $m_1$ and $m_2$ are confined by  harmonic potentials centered at $x_L$, $x_R$ with spring constants $k_L$, $k_R$  respectively (see Fig. \ref{fig:model_diagram}). The Hamiltonian describing this model is
%
\begin{equation}
  H = \frac{p_1^2}{2m_1} + \frac{p_2^2}{2m_2} + V(x_1,x_2),
  \label{eq:HamiltonianOriginalCordinates}
\end{equation}
%
with $V(x_1,x_2)=\frac{k}{2}\left( x_1 - x_2 - x_e \right)^2 + \frac{k_L}{2}\left( x_1 - x_L \right)^2 + \frac{k_R}{2}\left( x_2 - x_R \right)^2$,  where $\{x_i,p_i\}_{i=1,2}$ are the position and momentum of each mass. Switching from the original coordinates $x_i$ to displacements with respect to the equilibrium positions of the system $q_i = x_i - x_i^{eq}$, where $x_i^{eq}$ are the solutions to $\partial_{x_i}V(x_1,x_2)=0$, the Hamiltonian can be written as
%
\begin{align}
  H &= \frac{p_1^2}{2m_1} + \frac{p_2^2}{2m_2} + \frac{k+k_L}{2}q_1^2\nonumber\\ &+ \frac{k+k_R}{2}q_2^2 - k q_1 q_2 + V(x_1^{eq},x_2^{eq}).
  \label{eq:Hamiltonian}
\end{align}
%
Dropping the constant term, this has the form of  the Hamiltonian of a system around a stable equilibrium point,
%
\begin{equation}
  H = \frac{1}{2} \overrightarrow{p}^\mathsf{T}\mathbb{M}^{-1}\overrightarrow{p} + \frac{1}{2} \overrightarrow{q}^\mathsf{T}\mathbb{K}\overrightarrow{q},
\label{generic}
\end{equation}
%
where $\overrightarrow{q} = \left(q_1,q_2\right)^\mathsf{T}$, $\overrightarrow{p} = \left(p_1,p_2\right)^\mathsf{T}$, $\mathbb{M} = diag(m_1,m_2)$ is the mass matrix of the system and $\mathbb{K}$ is the Hessian matrix of the potential at the equilibrium point, i.e., $\mathbb{K}_{ij} = \partial^2_{x_i,x_j}V(\overrightarrow{x})\Big|_{\overrightarrow{x} = \overrightarrow{x}^{eq}}$. In this model  $\mathbb{K}_{11} = k + k_L$, $\mathbb{K}_{22} = k + k_R$ and $\mathbb{K}_{12} = \mathbb{K}_{21} = -k$.
We shall see later that
the generic form (\ref{generic}) can be adapted to different physical settings, in particular to
two ions in individual traps, or to two ions in a common trap.

The  masses are in contact with Langevin baths, which will be denoted as $L$ (for left) and $R$ (for right), at temperatures $T_{L}$ and $T_R$ for  the mass $m_1$ and $m_2$ respectively (see Fig. \ref{fig:model_diagram}). The equations of motion of the system, taking into account the Hamiltonian and the Langevin baths are
%
\begin{align}
  \dot{q}_1 &= \frac{p_1}{m_1},\;\;\;\;
  \dot{q}_2 = \frac{p_2}{m_2},\nonumber
  \\
  \dot{p}_1 &= -(k+k_L)q_1 + k q_2 -\frac{\gamma_L}{m_1} p_1 + \xi_L(t),\nonumber
  \\
  \dot{p}_2 &= -(k+k_R)q_2 + k q_1 -\frac{\gamma_R}{m_2} p_2 + \xi_R(t),
\end{align}
%
where $\gamma_L$, $\gamma_R$ are the friction coefficients of the baths and $\xi_L(t)$, $\xi_R(t)$ are Gaussian white-noise-like forces. The Gaussian forces have zero mean over noise realizations ($\expval{ \xi_L(t) } = \expval{ \xi_R(t) } = 0 $) and satisfy the correlations $\expval{ \xi_L(t)\xi_R(t') } = 0$, $\expval{ \xi_L(t)\xi_L(t') } = 2D_L\delta(t-t')$, $\expval{ \xi_R(t)\xi_R(t') } = 2D_R\delta(t-t')$. $D_L$ and $D_R$ are the diffusion coefficients, which satisfy the fluctuation-dissipation theorem, $D_L = \gamma_L k_B T_L$, $D_R =\gamma_R k_B T_R$, where  $k_B$ is the Boltzmann constant.

It is useful to define the phase-space vector $\overrightarrow{r}(t) = \left( \overrightarrow{q}, \mathbb{M}^{-1}\overrightarrow{p} \right)^\mathsf{T}$ (note that $\overrightarrow{v} = \mathbb{M}^{-1}\overrightarrow{p}$ is just the velocity vector).  The equations of motion are
%
\begin{equation}
  \dot{\overrightarrow{r}}(t) = \mathbb{A} \, \overrightarrow{r}(t) + \mathbb{L}\overrightarrow{\xi}(t),
  \label{eq:vectorEqOfMotion}
\end{equation}
%
with
%
\begin{align}
  \mathbb{A} &=
  \left(
  \begin{array}{cc}
    \mathbb{0}_{2 \times 2} & \mathbb{1}_{2 \times 2}
    \\
    -\mathbb{M}^{-1}\mathbb{K} & -\mathbb{M}^{-1}\bbGamma
  \end{array}
  \right),
  \nonumber
  \\
  \mathbb{L} &=
  \left(
  \begin{array}{c}
    \mathbb{0}_{2\times 2} \\ \mathbb{M}^{-1}
  \end{array}
  \right),
  \label{eq:Dynamical_Matrix}
\end{align}
%
and $\overrightarrow{\xi}(t) = \left( \xi_L(t),\xi_R(t) \right)^\mathsf{T}$, $\bbGamma = diag(\gamma_L,\gamma_R)$. $\mathbb{0}_{n\times n}$ and $\mathbb{1}_{n\times n}$ are the $n$-dimensional squared 0 matrix and identity matrix respectively. With the vector notation the correlation of the white-noise forces can be written as
%
\begin{equation}
  \expval{\overrightarrow{\xi}(t)\overrightarrow{\xi}(t')^\mathsf{T}} = 2 \mathbb{D}\delta(t-t'),
\end{equation}
%
where $\mathbb{D} = diag(D_L,D_R)$.
%
%
%
%
%
%
\section{Covariance matrix in the steady state\label{sec:covMatrix}}
%
%
%
%
%
%
We define the covariance matrix of the system as
%
\begin{equation}
\mathbb{C}(t) = \expval{\overrightarrow{r}(t)\overrightarrow{r}(t)^\mathsf{T}}.
\end{equation}
%
This matrix is important because the heat transport properties can be extracted from it. In particular, the kinetic temperatures of the masses, $T_1(t)$ and  $T_2(t)$, are
%
\begin{align}
  T_1(t) &= \frac{\expval{ p_1^2(t)}}{m_1 k_B} = \frac{m_1 C_{3,3}(t)}{k_B},
  \nonumber\\
   T_2(t) &= \frac{\expval{ p_2^2(t)}}{m_2 k_B} = \frac{m_2 C_{4,4}(t)}{k_B}.
  \label{eq:Temperature_definition}
\end{align}
%
One approach to find the covariance matrix is to solve Eq. \eqref{eq:vectorEqOfMotion}. However, this requires solving the equations explicitly or simulate them numerically many times to find the covariance matrix for the ensemble of simulated stochastic trajectories. Instead, we proceed by looking for an ordinary differential equation that gives the evolution of the covariance matrix as described in \cite{Sarkka2019,Rieder1967,Casher1971}. Differentiating $\mathbb{C}(t)$ with respect to time and using Eq. \eqref{eq:vectorEqOfMotion} we get
%
\begin{align}
  \frac{d}{dt}\mathbb{C}(t) &=
  \mathbb{A}\mathbb{C}(t) +
  \mathbb{C}(t) \mathbb{A}^\mathsf{T}
  \nonumber\\
  &+
  \mathbb{L}\expval{ \overrightarrow{\xi}(t)\overrightarrow{r}(t)^\mathsf{T}}
  %\nonumber\\
 % &+
+
  \expval{ \overrightarrow{r}(t)\overrightarrow{\xi}(t)^\mathsf{T}}\mathbb{L}^\mathsf{T}.
  \label{eq:evolutionOfCovariances}
\end{align}
%
The solution of Eq. \eqref{eq:evolutionOfCovariances} allows us to find the local temperatures of the masses as a function of the bath temperatures (Eq. \eqref{eq:Temperature_definition}) at all times. In particular, we are interested in the covariance matrix in the steady state, i.e., for $t\to \infty$. %According to the Novikov Theorem \cite{Novikov1965} we can write down the covariance matrix in the steady state without having to integrate the differential equation. We now show how to get the steady-state covariance matrix.

In the steady state, the covariance matrix is constant ($\frac{d}{dt}\mathbb{C}(t)=0$), therefore it satisfies
%
\begin{align}
  &\mathbb{A}\mathbb{C}^{s.s.} +
  \mathbb{C}^{s.s.} \mathbb{A}^\mathsf{T}=
  \nonumber\\
  &- \mathbb{L}\expval{ \overrightarrow{\xi}\overrightarrow{r}^\mathsf{T}}^{s.s.}
  - \expval{ \overrightarrow{r}\overrightarrow{\xi}^\mathsf{T}}^{s.s.}\mathbb{L}^\mathsf{T},
  \label{eq:SteadyStateEquationToyModel_raw}
\end{align}
%
with $\small\{\cdot\small\}^{s.s.}\equiv \lim\limits_{t \to \infty} \small\{\cdot\small\}(t)$. Equation \eqref{eq:SteadyStateEquationToyModel_raw} is an algebraic equation whose solution is the steady-state covariance matrix $\mathbb{C}^{s.s.}$. However, the two terms $\expval{ \overrightarrow{\xi}\overrightarrow{r}^\mathsf{T}}^{s.s.}$ and  $\expval{\overrightarrow{r}\overrightarrow{\xi}^\mathsf{T}}^{s.s.}$ need to be calculated before working out the solution.
%One approach to calculate $\expval{\overrightarrow{\xi}\overrightarrow{r}^\mathsf{T}}^{s.s.}$ would be to solve Eq. , but this is exactly what we are trying to avoid. It is here when
%The Novikov theorem comes useful, since it lets us compute $\expval{ \overrightarrow{\xi}\overrightarrow{r}^\mathsf{T}}^{s.s.}$ without having to integrate the equations of motion.
Using Novikov's theorem and the $\delta$-correlation of the noises, we find the $ij$-th component of $\expval{ \overrightarrow{\xi}(t)\overrightarrow{r}(t)^\mathsf{T}}$ without solving Eq. \eqref{eq:vectorEqOfMotion},
%
\begin{align}
  \expval{ \xi_i(t) r_j(t) } &= \sum_{k=1}^2 \int_0^t d\tau\,\expval{ \xi_i(t) \xi_k(\tau)}
  \,
  \expval{ \frac{\delta r_j(t)}{\delta \xi_k(\tau)} }\nonumber
  \\
  &= \sum_{k=1}^2 \mathbb{D}_{ik}
  \,
  \lim_{\tau \to t^{-}}
  \,
  \expval{ \frac{\delta r_j(t)}{\delta \xi_k(\tau)} },
\end{align}
%
where $\lim_{\tau \to t^{-}}$ is the limit when $\tau$ goes to $t$ from below. Evaluation of the functional derivative ${\delta r_j(t)}/{\delta \xi_k(\tau)}$ for the $\tau \to t^{-}$ limit gives
%
\begin{equation}
  \expval{ \overrightarrow{\xi}(t)\overrightarrow{r}(t)^\mathsf{T}} = \mathbb{D}\mathbb{L}^\mathsf{T}.
\end{equation}
%
Now, the algebraic equation that gives the steady-state covariance matrix becomes
%
\begin{equation}
  \mathbb{A}\mathbb{C}^{s.s.} +
  \mathbb{C}^{s.s.}\mathbb{A}^\mathsf{T}
  =
  -\mathbb{B},
  \label{eq:SteadyStateEquationToyModel}
\end{equation}
%
with $\mathbb{B} = 2 \mathbb{L}\mathbb{D}\mathbb{L}^\mathsf{T}$. By definition, the covariance matrix is  symmetric, but there are also  additional restrictions imposed by the equations of motion and the steady-state condition, which reduce the dimensionality of the problem of solving Eq. \eqref{eq:SteadyStateEquationToyModel} \cite{Simon2019}. Since ${d \expval{ q_i q_j }}/{dt} = 0$ in the steady state, we have
%
\begin{align}
  \expval{ p_1 q_1}^{s.s.} &= \expval{ p_2 q_2}^{s.s.} = 0,\nonumber\\
  \frac{\expval{ p_1 q_2}^{s.s.}}{m_1}&=-\frac{\expval{ q_1 p_2}^{s.s.}}{m_2}.
  \label{eq:ExtraConditionSteadyState}
\end{align}
%
Taking \eqref{eq:ExtraConditionSteadyState} into account, the steady-state covariance matrix takes the form
%
\begin{equation}
  \begin{split}
    \mathbb{C}^{s.s.} =
    \left(
    \begin{array}{cccc}
      \expval{ q_1^2}^{s.s.}  & \expval{ q_1 q_2}^{s.s.}  & 0 & \frac{\expval{ p_2 q_1}^{s.s.} }{m_2} \\
      \expval{ q_1 q_2}^{s.s.}  & \expval{ q_2^2}^{s.s.}  & -\frac{\expval{ p_2 q_1}^{s.s.} }{m_2} & 0 \\
      0 & -\frac{\expval{ p_2 q_1}^{s.s.} }{m_2} & \frac{\expval{ p_1^2}^{s.s.} }{m_1^2} & \frac{\expval{ p_1 p_2}^{s.s.} }{m_1 m_2} \\
      \frac{\expval{ p_2 q_1}^{s.s.} }{m_2} & 0 & \frac{\expval{ p_1 p_2}^{s.s.} }{m_1 m_2} & \frac{\expval{ p_2^2}^{s.s.} }{m_2^2} \\
      \end{array}
      \right)
    \end{split}
    \label{eq:steadyStateCovarianceMatrix}\,.
\end{equation}
%
The explicit set of equations for the components of $\mathbb{C}^{s.s}$ can be found in Appendix \ref{Appendix:SteadyStateEquations}.
%
%
%
%
%
\section{Solutions\label{sec:solutions}}
%
%
%
%

%
In this section we use the solution to Eq. \eqref{eq:SteadyStateEquationToyModel} to write down the temperatures and currents in the steady state. We use {\it Mathematica} to find analytic expressions for the temperatures,
%
\begin{align}
  T_1 &= \frac{T_L \mathcal{P}_{1,L}(k) + T_R \mathcal{P}_{1,R}(k)}{\mathcal{D}(k)},\nonumber
  %
  \\
  %
  T_2 &= \frac{T_L \mathcal{P}_{2,L}(k) + T_R \mathcal{P}_{2,R}(k)}{\mathcal{D}(k)},
  %
  \label{eq:ModelBTemperatures}
\end{align}
%
where $\mathcal{D}(k) =  \sum\limits_{n=0}^2 \mathcal{D}_n k^n$ and $\mathcal{P}_{i,(L/R)}(k) = \sum\limits_{n=0}^2 a_{i,n,(L/R)} k^n$ are polynomials in the coupling constant $k$ with coefficients
%

%\begin{widetext}
  \begin{align}
    \mathcal{D}_0 &= a_{1,0,L} = a_{2,0,R}
    \nonumber\\
    & = \gamma _L \gamma _R\! \left[h^{(1)}\! \left(\gamma_L k_R +\gamma_R k_L \right)+\left(m_1 k_R-m_2 k_L\right)^2\right]\!,
    \nonumber\\
    %
    \mathcal{D}_1 &= a_{1,1,L} = a_{2,1,R}
    \nonumber\\
    &= \gamma _L \gamma _R\! \left[h^{(0)} h^{(1)}\!+2 \left(m_1-m_2\right) \left(m_1 k_R-m_2 k_L\right)\right]\!,
    %
    \nonumber\\
    %
    \mathcal{D}_2 &= h^{(0)} h^{(2)},\nonumber
    %
    \\
    %
    a_{1,2,L} &= \gamma _L \left(m_2 h^{(1)} + \gamma_R (m_1 - m_2)^2 \right),\nonumber
    %
    \\
    %
    a_{1,2,R} &= h^{(1)} m_1 \gamma_R,\nonumber
    %
    \\
    %
    a_{2,2,L} &= h^{(1)} m_2 \gamma_L,\nonumber
    %
    \\
    %
    a_{2,2,R} &= \gamma _R \left( m_1 h^{(1)} + \gamma_L (m_1-m_2)^2 \right),\nonumber
    %
    \\
    %
    a_{1,0,R} &= a_{1,1,R} = a_{2,0,L} = a_{2,1,L} = 0,
    %
    \label{eq:SolutionPolynomialCoefficients}
  \end{align}
%\end{widetext}
%
where
%
\begin{equation}
h^{(n)}\equiv \gamma_R m_1^n + \gamma_L m_2^n.
\end{equation}
%
The currents from the baths to the masses \cite{Simon2019} are
%
\begin{equation}
%  \begin{split}
    J_L = k_B \frac{\gamma_L}{m_1} \left( T_L - T_1 \right),\;\;\;
    J_R = k_B \frac{\gamma_R}{m_2} \left( T_R - T_2 \right),
    \label{eq:currents_definition}
%  \end{split}
\end{equation}
\\
%
with $T_i$ given by Eq. \eqref{eq:ModelBTemperatures}. Since, in the steady state, $J_L = -J_R$ we will use the shorthand notation $J \equiv J_L$. Substituting Eq. \eqref{eq:ModelBTemperatures} into Eq.  \eqref{eq:currents_definition} we get for the heat current
%
% \begin{equation}
%   J = k_B \frac{k^2\gamma_L \gamma_R h^{(1)}}{\mathcal{D}(k)}(T_L - T_R).
%   \label{eq:CurrentsInModelB}
% \end{equation}
%
\begin{equation}
  J = \kappa\;(T_L - T_R),
  \label{eq:CurrentsInModelB}
\end{equation}
%
where $\kappa = k_B {k^2\gamma_L \gamma_R h^{(1)}}/{\mathcal{D}(k)}$ acts as an effective thermal conductance.
%, which depends on the parameters of the system, i.e., the masses and spring constants, and also on the friction coefficients of the baths.
%From Eq. \eqref{eq:CurrentsInModelB} it could be thought that inverting the temperatures of the baths would only lead to an exchange of heat currents. However, since the thermal conductance $\kappa$ depends on the friction coefficients, the exchange of the baths implies a change in its value. Moreover, it is possible to have temperature-dependent friction coefficients, as it happens in the physical set-up of laser-cooled trapped ions described in Section \ref{sec:TrappedIonSetUp}.
%
%
%
%
\section{Relation of the Model to a trapped ion setup \label{sec:TrappedIonSetUp}}
%
%
%
%
%As we mentioned, the parameters  can be related to the elements of the Hessian matrix of a system in a stable equilibrium position.
In this section we discuss the realization of the model with  a pair of trapped ions. We consider two different setups: two ions in a collective trap, and two ions in individual traps. Later in Section \ref{sec:lookingForR} we shall focus on two ions in individual traps to illustrate the analysis of rectification.

In both setups we assume strong confinement in the radial direction, making the effective dynamics one-dimensional. We will also assume that the confinement in the axial direction is purely electrostatic, which makes the effective spring constant independent of the mass of the ions \cite{Leibfried2003}. Additionally, we will relate the temperatures and friction coefficients of the Langevin baths to those corresponding to Doppler cooling.
%
%
\subsection{Collective trap}
%
%
Consider two ions of unit charge with masses $m_1$ and $m_2$ trapped in a collective trap. Assuming strong radial confinement and purely electrostatic axial confinement, both ions feel the same harmonic oscillator potential with trapping constant $k_{trap}$ \cite{Leibfried2003}. The potential describing the system is
%
\begin{equation}
  V_{collective} = \frac{1}{2}k_{trap} \left( x_1^2 + x_2^2\right) + \frac{\mathcal{C}}{x_2-x_1},
\end{equation}
%
with $\mathcal{C}={Q^2}/({4\pi\varepsilon_0})$. The equilibrium positions for this potential are
%
\begin{equation}
  x_2^{eq} = -x_1^{eq} =
  \label{eq:equilibriumPositionsCollectiveTrap}\left(\frac{1}{2}\right)^{2/3} \left(\frac{Q^2}{4\pi\varepsilon_0 k_{trap}}\right)^{1/3}.
\end{equation}
%
Assuming small oscillations of the ions around the equilibrium positions, the Hessian matrix of the system is
%
\begin{align}
  \mathbb{K}_{1,2} &= -\frac{Q^2}{2\pi\varepsilon_0}\frac{1}{(x_2^{eq}-x_1^{eq})^3} = -k_{trap},\nonumber
  \\
  \mathbb{K}_{1,1} &= k_{trap} + \frac{Q^2}{2\pi\varepsilon_0}\frac{1}{(x_2^{eq}-x_1^{eq})^3} = 2 k_{trap},\nonumber
  \\
  \mathbb{K}_{2,2} &= k_{trap} + \frac{Q^2}{2\pi\varepsilon_0}\frac{1}{(x_2^{eq}-x_1^{eq})^3} = 2 k_{trap}.
  \label{eq:HessianOffDiagonalCollective}
\end{align}
%
Using Eq. \eqref{eq:HessianOffDiagonalCollective} we can relate the parameters of this physical setup to those of the model described in Section \ref{sec:Physical_Model},
%
\begin{equation}
  k_L = k_R = k = k_{trap}.
\end{equation}
%
%
%
\subsection{Individual on-site traps}
%
%
%
We can make the same assumptions for the axial confinement as in the previous subsection but now each of the ions is in an individual trap with spring constants $k_{trap,L}$ and $k_{trap,R}$ respectively. The potential of the system is
%
\begin{align}
    V_{individual} &= \frac{1}{2}k_{trap,L}\left(x_1 -x_L\right)^2 +\frac{1}{2}k_{trap, R}\left(x_2 -x_R\right)^2 \nonumber \\&+ \frac{\mathcal{C}}{x_2-x_1},
\end{align}
%
where $x_L$ and $x_R$ are the center positions of the on-site traps. The elements of the Hessian matrix in the equilibrium position are
%
\begin{align}
  \mathbb{K}_{1,2} &= -\frac{Q^2}{2\pi\varepsilon_0}\frac{1}{(x_2^{eq}-x_1^{eq})^3},\nonumber
  \\
  \mathbb{K}_{1,1} &= k_{trap,L} + \frac{Q^2}{2\pi\varepsilon_0}\frac{1}{(x_2^{eq}-x_1^{eq})^3},\nonumber
  \\
  \mathbb{K}_{2,2} &= k_{trap,R} + \frac{Q^2}{2\pi\varepsilon_0}\frac{1}{(x_2^{eq}-x_1^{eq})^3}.
  \label{eq:HessianOffDiagonalOnSite}
\end{align}
%
Comparing the parameters in Eq. \eqref{eq:HessianOffDiagonalOnSite} with those in the model described in Section \ref{sec:Physical_Model} we identify
\begin{align}
  k_L &= k_{trap,L},\nonumber\\
  k_R &= k_{trap,R},\nonumber\\
  k &= \frac{Q^2}{2\pi\varepsilon_0}\frac{1}{(x_2^{eq}-x_1^{eq})^3}\,.
\end{align}
%
In this case, the analytic expressions for the equilibrium positions are more complicated. We get for the distance between the equilibrium positions of the ions
%
\begin{align}
  &(x_2 - x_1)^{(eq)} = \frac{1}{3} \Delta x_{LR}\nonumber\\
  &- \frac{1}{6}\Big[ \frac{2^{2/3}\zeta}{k_{trap,L} k_{trap,R} (k_{trap,L} + k_{trap,R})}\nonumber\\
  &+ \frac{2^{4/3} k_{trap,L} k_{trap,R} (k_{trap,L} + k_{trap,R}) (x_R-x_L)^2}{\zeta} \Big]\,,
\end{align}
%
where $\Delta x_{LR} = (x_R-x_L)$ and $\zeta = \left( Y - \eta \right)^{(1/3)}$, with
%
\begin{eqnarray}
&&Y = 3 \sqrt{3} \bigg\{\mathcal{C} k_{trap,L}^4 k_{trap,R}^4 \left(k_{trap,L}+k_{trap,R}\right)^{7}
\nonumber\\
&&\times\left[4 k_{trap,L} k_{trap,R} \Delta x_{LR}^3+27 \mathcal{C} \left(k_{trap,L}+k_{trap,R}\!\right)\!\right]\!\!\bigg\}^{\!1/2}\!,
\nonumber
%
\\
  %%
&&\eta =  k_{trap,L}^2 k_{trap,R}^2 \left(k_{trap,L}+k_{trap,R}\right)^{3}
\nonumber\\
&&\times\left[2 k_{trap,L} k_{trap,R} \Delta x_{LR}^3+27 \mathcal{C} \left(k_{trap,L}+k_{trap,R}\right)\right]\!.
\end{eqnarray}
%
In this setup, the coupling between the ions $k$ can be controlled by changing the distance between the on-site traps.
%
%
%
\subsection{Optical molasses and Langevin baths}
%
%
%
Trapped ions may be cooled down by counterpropagating lasers which are red-detuned with respect to an internal atomic transition of the ions. This technique is known as Doppler cooling or optical molasses \cite{Chu1985,Cohen1992,Metcalf1999,Metcalf2003}. The off-resonant absorption of laser photons by the ions exerts a damping-like force that slows them down. The spontaneous emission of the ions produces heating due to the random recoil generated by the emitted photons. The friction and recoil force balance, so eventually the ion thermalizes to a finite temperature.
Thus the effect of the lasers on the ion is equivalent to a Langevin bath with temperature $T_{molass}$ and friction coefficient $\gamma_{molass}$. The temperature and friction coefficients are controlled with the laser intensity $I$ and frequency detuning $\delta$ with respect to the selected internal atomic transition by the expressions \cite{Cohen1992,Metcalf2003,Ruiz2014},
%
\begin{align}
  \gamma_{molass}(I,\delta) &= -4 \hbar \left(\frac{\delta + \omega_0}{c}\right)^2 \left(\frac{I}{I_0}\right)\frac{2\delta/\Gamma}{\left[1 + (2\delta/\Gamma)^2\right]^2},\nonumber\\
  %
  T_{molass}(\delta) &= -\frac{\hbar \Gamma}{4 k_B} \frac{1+(2\delta/\Gamma)^2}{(2\delta/\Gamma)},
  \label{eq:DopplerCoolingToyModel}
\end{align}
%
where $\omega_0$ is the frequency of the internal atomic transition, $\Gamma$ is the natural width (decay rate) of the excited state, and $I_0$ is the saturation intensity. For fixed $\Gamma$ and $I$, $\gamma_{molass}$ depends on $\delta$, and thus, indirectly, on the temperature $T_{molass}$.
%
%
%
\section{Looking for rectification\label{sec:lookingForR}}
%
%
%
%First let us define what we exactly mean by \textit{rectification}.
There is rectification if the flux $J$  for the forward temperature bias is different from the flux $\tilde{J}$ for reverse bias
with the baths exchanged.  To measure rectification, we will use the rectification coefficient $0\le R\le 1$ defined as
%
\begin{equation}
  R = \frac{\abs{|J|-|\tilde{J}|}}{\max(|J|,|\tilde{J}|)}.
  \label{eq:Rectification}
\end{equation}
%
The important point here is to define what is  meant by \textit{exchanging the baths}. We consider that a bath is characterized, not only by its temperature $T$ but also by its coupling  to the system by means of the friction coefficient $\gamma$, so, exchanging the baths is achieved by exchanging both the temperatures and the friction coefficients, as summarized in Table \ref{tab:reversed_bath}. For generic models this
choice is a matter of definition, but for trapped ions it is a natural way to proceed.

When implementing temperatures and friction coefficients by lasers according to
%, this exchange operation is performed by changing the values of the intensities and detunings acting on each ion (
Eq. \eqref{eq:DopplerCoolingToyModel}, the exchange operation is straightforward when the two ions are either of the same species or isotopes of each other, since the only required action to exchange temperatures is to exchange the detunings without modifying the intensities. The detuning exchange in turn automatically exchanges the friction coefficients. However, for two different species, which involve two different atomic transitions, the laser wavelengths and the decay rates $\Gamma$ depend on the species. Then, exchanging the temperatures by modifying the detunings, keeping the laser intensities constant, does not necessarily imply an exchange of the friction coefficients. Nevertheless it is possible to adjust the laser intensities so that the friction coefficients get exchanged and that is the assumption hereafter. In terms of the analysis of rectification in Ref. \cite{Pereira2017}, we are adding a temperature dependent feature to the system, namely,  the friction coefficients depend on the bath temperature
and are exchanged as the baths are reversed.

\begin{figure}
  \center
  \includegraphics[width=\linewidth]{Figures/RwMPlota.pdf}
  \caption{Rectification, $R$, in the $k_L k_R$ plane for $k = 1.17$ fN/m, $\gamma_L = 6.75\times 10^{-22}$ kg/s, and $\gamma_R = 4.64\gamma_L$, $m_1 = 24.305$ a.u., $m_2 = 40.078$ a.u. The dashed  line represents Eq.  (\ref{eq:MaxRLines}).}
  \label{fig:Fig_rectification_K_plane}
\end{figure}


%that is, the ratio between the difference of heat currents and the largest one. As defined, $R=0$ for no asymmetry of the heat currents and $R=1$ when they are maximally asymmetric.

\begin{table}[]
\center
\caption{Definition of forward and reversed (exchanged) bath configurations.}
\begin{tabular}{lcc}
\hline
                 & forward                & reversed                                                       \\ \hline
Bath Friction    & $\gamma_L$, $\gamma_R$ & $\tilde{\gamma}_L =\gamma_R $,  $\tilde{\gamma}_R =\gamma_L $   \\
Bath Temperature & $T_L$, $T_R$           & $\tilde{T}_L =T_R $,  $\tilde{T}_R =T_L $                     \\
\hline
\end{tabular}
\label{tab:reversed_bath}
\end{table}
%
%
\subsection{Parametric exploration}
%
%
%
We have explored thoroughly the space formed by the parameters of the model $m_1,m_2,k,k_L,k_R,\gamma_L,\gamma_R$, to find
and maximize asymmetric heat transport. We have fixed the values of some of the parameters to realistic ones while varying the rest. Unless stated otherwise the masses are
$m_1 = 24.305$ a.u. and $m_2 = 40.078$ a.u., which correspond to Mg and Ca, whose ions are broadly used in trapped-ion physics. According to Eq. \eqref{eq:CurrentsInModelB} and the corresponding expression for $\tilde{J}$ with the substitutions in Table \ref{tab:reversed_bath},
rectification does not formally depend on the bath temperatures in this model for given friction coefficients.
Of course the friction coefficients depend on the temperature indirectly, but also on laser intensities, see Eq. \eqref{eq:DopplerCoolingToyModel}, so in the parametric space $m_1,m_2,k,k_L,k_R,\gamma_L,\gamma_R$ there is no need to specify the bath temperatures to analyze the rectification in the following. The bath temperatures will be needed though
to calculate the power spectra, and play an implicit role in the central assumption that their exchange implies an exchange of
friction coefficients.

Figure \ref{fig:Fig_rectification_K_plane} depicts the values of the rectification after sweeping the $k_L k_R$ plane for fixed values of $k$, $\gamma_L$, and $\gamma_R$. There is a ridge in the $k_L,k_R$ plane for which the rectification is maximal. $\partial_{k_L}R = 0$ may be
solved explicitly but the solution is too long to be displayed here. In a weak dissipation regime
(${\gamma_L}/{m_1}<<\sqrt{{k}/{m_1}}$, ${\gamma_R}/{m_2}<<\sqrt{{k}/{m_2}}$), a Taylor series around $(\gamma_L,\gamma_R) = (0,0)$ gives in zeroth order
%.i.e., ${\gamma_L}/{m_1}<<\sqrt{{k}/{m_1}}$, ${\gamma_R}/{m_2}<<\sqrt{{k}/{m_2}}$, the ridge is given approximately by
a straight line for the ridge,
%
\begin{equation}
  \frac{k+k_R}{m_2} = \frac{k+k_L}{m_1}.
  \label{eq:MaxRLines}
\end{equation}
%
Eq. \eqref{eq:MaxRLines} implies the resonance condition $\omega_L = \omega_R$
for the effective oscillation frequencies $\omega_L = \sqrt{{(k+k_L)}/{m_1}}$ and $\omega_R = \sqrt{{(k+k_R)}/{m_2}}$,
see Eq. (\ref{eq:Hamiltonian}). The lowest order correction to  Eq. \eqref{eq:MaxRLines} implies a small shift of the line,
keeping the same slope,
%
\begin{equation}
  \frac{k+k_R}{m_2} = \frac{k+k_L}{m_1} + \frac{(m_2\gamma_L+m_1\gamma_R)(m_1\gamma_L+m_2\gamma_R)}{2m_1m_2(m_2^2-m_1^2)}.
  \label{eq:MaxRLines_correction}
\end{equation}
%
In a trapped-ion context the condition \eqref{eq:MaxRLines} may be imposed by adjusting the distance of the traps for fixed $k_L$ and $k_R$. Besides the line of maximum rectification, Fig. \ref{fig:Fig_rectification_K_plane} also shows two lines where rectification is zero.
At these lines forward and backward fluxes cross.
%, The lines correspond to the boundaries in which the heat conductance for the reversed configuration surpasses the forward configuration. Inside of these boundaries heat propagates easily for a forward bias ($T_L>T_R$), whereas the opposite happens outside. In Fig. \ref{fig:Fig_rectification_K_plane}, the lines of zero rectification look parallel to the one of maximum rectification. This is however only a limiting behavior for weak dissipation.
Solving $R=0$ with a Taylor series around $(\gamma_L,\gamma_R) = (0,0)$ gives, up to second order in friction coefficients,  the two approximate solutions
%
\begin{align}
k_R &= k_L\left[\frac{m_2}{m_1}\pm\frac{1}{2k}\sqrt{\frac{m_2\gamma_L\gamma_R^3}{m_1^3}}\right]
\nonumber\\
&+k\left[\frac{m_2}{m_1}\left(1\pm \frac{ 2 m_1 m_2 \gamma_R + (m_1^2 + m_2^2)\gamma_L }{2\sqrt{m_1 m_2^3 \gamma_L \gamma_R}} \right)-1\right]
\nonumber\\
&\pm\frac{1}{2}\sqrt{\frac{m_2\gamma_L\gamma_R^3}{m_1^3}} + \gamma_R\frac{(m_1^2+m_2^2)\gamma_L + m_1m_2\gamma_R}{2m_1^2(m_2-m_1)}.
\label{eq:zeroRlines}
\end{align}
%
The term $\pm\frac{1}{2k}\sqrt{{m_2\gamma_L\gamma_R^3}/{m_1^3}}$ in Eq. \eqref{eq:zeroRlines} makes the slopes of the
two zero-rectification lines different from each other and also from the maximum-rectification line. This difference is however
hardly noticeable for weak dissipation as in  Fig. \ref{fig:Fig_rectification_K_plane}.

Interestingly, along the maximum line  \eqref{eq:MaxRLines} the rectification no longer depends on the spring constants of the model,
see  Eqs. \eqref{eq:CurrentsInModelB}  and \eqref{eq:Rectification},
%
\begin{equation}
    R=
    \begin{cases}
      1-\frac{a+g}{1+ag} &\text{ if }a>1,g>1\text{ or }a<1,g<1\\
      1-\frac{1+ag}{a+g} &\text{ if }a>1,g<1\text{ or }a<1,g>1,
    \end{cases}
  \label{eq:maxRExpression}
\end{equation}
%
it only depends on the mass and friction coefficient ratios $a$ and $g$
%
\begin{align}
  a = m_2/m_1,\;\;\;\;
  g = \gamma_R/\gamma_L.
\end{align}
%
%The maximal rectification found does not scale with the magnitude of the masses or the friction coefficients, just with their ratios.
Besides a high value of $R$, it is desirable to have a significant current $J_{max}$
%when a forward temperature bias is applied to the rectifier
%, as in some cases an increase of $R$ is accompanied by an overall decrease of the heat currents
\cite{Simon2019}. Using again  Eq. \eqref{eq:MaxRLines} in the expression for the currents \eqref{eq:CurrentsInModelB}, the maximum current $J_{\max} = \max(\big|{J}\big|,\big|\tilde{J}\big|)$ is
%
\begin{align}
    &J_{\max}=\begin{cases}
   \frac{k_B g\gamma_L k^2 \abs{T_L-T_R}}{(a+g)(g\gamma_L^2(k_L+k)+k^2m_1)} &
   \text{ if }\begin{cases}a>1,g>1\\
   \text{ or }a<1,g<1\end{cases}
    \\
    \frac{k_B g\gamma_L k^2 \abs{T_L-T_R}}{(1+ag)(g\gamma_L^2(k_L+k)+k^2m_1)}&\text{ if }
    \begin{cases}a>1,g<1
    \\\text{ or }a<1,g>1\end{cases}
    \end{cases}
    \label{eq:maxJExpression}
\end{align}
%
Now we analyze how the parameters $a$ and $g$ affect the maximum current $J_{max}$ in \eqref{eq:maxJExpression}. To do this, we can divide the $ag$ plane in four quadrants by the axes $a = 1$ and $g = 1$ (in those axes $R = 0$). In Eq. \eqref{eq:maxJExpression} the parameter $a$ appears only in the denominator, thus for a higher $a$, a smaller current is found. The quadrants with $a < 1$ will be better for achieving large currents. $g$ appears both in the numerator and denominator so there is no obvious advantageous quadrant for this parameter.

Equation \eqref{eq:maxRExpression} is symmetric upon the transformations $a \leftrightarrow 1/a$ and $g \leftrightarrow 1/g$. Using a logarithmic scale for $a$ and $g$, the resulting $R$ map is symmetric with respect to the $a=1$ and $g=1$ axes. We can thus limit ourselves to analyze the quadrant $a > 1$, $g > 1$.
%, as the results in other quadrants will be equivalent upon transformations $a \leftrightarrow 1/a$ and $g \leftrightarrow 1/g$.


\begin{figure}
  \center
  \includegraphics[width=\linewidth]{Figures/Rade.pdf}
  \caption{Rectification factor, $R$, given by Eq. \eqref{eq:maxRExpression}.}
  \label{fig:R_g_a_plane}
\end{figure}

Figure \ref{fig:R_g_a_plane} shows the rectification given by Eq. \eqref{eq:maxRExpression} in terms of $a$ and $g$. Along any diagonal line (parallel to the solid cyan or the dashed green lines), the maximum value is at the center, that is, when $a = g$. For constant $a$, a larger $g$ always increases $R$, but making the  ratio between friction coefficients $g$ arbitrarily large is not  realistic in a trapped-ion setup.
%Since $g$ is defined as the ratio between the friction coefficients, increasing it means making either $\gamma_L$ go to 0 or $\gamma_R$ to infinity.
Making $\gamma_L$ go to 0 decouples one of the ions from the bath, so the heat current tends to vanish in any direction. Also, increasing $\gamma_R$ arbitrarily is impossible since it is a function of the laser detuning (Eq. \eqref{eq:DopplerCoolingToyModel}) which is physically bounded
by the existence of other levels. Although Eq. \eqref{eq:DopplerCoolingToyModel} suggests that boosting the laser intensity can also increase the friction coefficient, this is not an option since Eq. \eqref{eq:DopplerCoolingToyModel} is just an approximation for low laser intensities. When going to higher intensities, the emission/absorption of photons by the ion is saturated and the friction coefficient reaches a finite value proportional to the width $\Gamma$ of the excited state \cite{Metcalf2003}. As a compromise between feasibility and high $R$, let as assume that the ratio between the friction coefficients $g$ to be equal to the mass ratio $a$. As shown  in Fig. \ref{fig:R_g_a_plane}, along the solid-cyan and dashed-green diagonal lines the maximum $R$ is achieved for $a = g$. The effect of varying  the common value $c$ of $a$ and $g$, $c=a = g$, may be seen in
Fig. \ref{fig:Fig_PerfectRectification}, which  shows the rectification in Eq. \eqref{eq:maxRExpression}. $R$ tends to one for large $c$.
%
%
\subsection{Spectral match/mismatch approach to rectification}
%
%
%
\begin{figure}
  \center
  \includegraphics[width=\linewidth]{Figures/CC.pdf}
  \caption{Rectification for different values of $c=m_2/m_1=\gamma_R/\gamma_L$ when the maximum condition in the $k_L k_R$ plane is satisfied (Eq. \eqref{eq:MaxRLines}).}
  \label{fig:Fig_PerfectRectification}
\end{figure}

%The match/mismatch between the power spectra of the particles controls the heat currents in the system . A good match between the power spectra of the two ions in a large range of frequencies yields a higher heat current through the system while the mismatch  reduces the heat current.
%Therefore, we can understand rectification through the match/mismatch of the phonon bands of the ions \cite{Terraneo2002}.
If there is a good match between the phonon spectra of the ions (i.e., their peaks overlap in a broad range of frequencies) for a certain baths configuration, and mismatch when the baths exchange, the system will present heat rectification \cite{Terraneo2002,Li2004}.
We have studied the spectra of the ions in our model for several sets of parameters exhibiting no rectification or strong rectification. The spectra are calculated  through the spectral density matrix. For a real-valued stochastic process $\overrightarrow{x}(t)$, its spectral density matrix is defined as \cite{Sarkka2019}
%
\begin{equation}
  \mathbb{S}_{\overrightarrow{x}}(\omega) \equiv \expval{ \overrightarrow{X}(\omega) \overrightarrow{X}^\mathsf{T}(-\omega) },
  \label{eq:SpectralDensityDefinition}
\end{equation}
%
with $\overrightarrow{X}(\omega)$ being the Fourier transform of $\overrightarrow{x}(t)$ (we use the convention of factors of $1$ and ${1}/{(2\pi)}$ for the transform and the inverse transform). A justification of the use of the spectral density matrix to understand heat transport arises from the Wiener-Khinchin theorem \cite{Sarkka2019}, which says that the correlation matrix of a stationary stochastic process in the steady state is the inverse Fourier transform of its spectral density matrix $\expval{\overrightarrow{r}(t)\overrightarrow{r}^\mathsf{T}(t+\tau)} = \mathcal{F}^{-1}[\mathbb{S}_{\overrightarrow{r}}(\omega)](\tau)$. Thus  the covariance matrix in the steady state is
%
\begin{equation}
  \mathbb{C}^{s.s.} = \frac{1}{2\pi} \int_{-\infty}^{\infty}d\omega\;\mathbb{S}_{\overrightarrow{r}}(\omega).
  \label{eq:Wiener-Khinchin}
\end{equation}
%
Eq. \eqref{eq:Wiener-Khinchin} directly connects the spectral density matrix to the steady-state temperature
since  $T_1^{s.s.} = {m_1 C_{3,3}^{s.s.}}/{k_B}$ and $T_2^{s.s.} = {m_2 C_{4,4}^{s.s.}}/{k_B}$, and, therefore, to the heat currents,
see  Eqs.(\ref{eq:Temperature_definition}) and (\ref{eq:currents_definition}).


\begin{figure*}[t]
  \center
  \includegraphics[width=.8\linewidth]{Figures/SpectrumComparative.pdf}
  \caption{Spectral densities of the velocities of the ions ($r_3$ and $r_4$) corresponding to $T_L=\tilde{T}_R=2$ mK, $T_R=\tilde{T}_L=1$ mK, and  two values of $c$ in Fig. \ref{fig:Fig_PerfectRectification}: (a), (b) for $c=1$ and (c), (d) for $c=10$. Solid, black lines are for the left ion spectral density ${\cal{S}}_1(\omega)$ and dashed, blue lines for the right ion spectral density
 ${\cal{S}}_2(\omega)$. Dot-dashed, vertical lines mark the frequencies of the normal modes of the system. The spectra are multiplied by their corresponding masses so that  the areas are proportional to the steady-state temperatures, see  Eq. \eqref{eq:Wiener-Khinchin}. (a) and (b) correspond to $R = 0$:  the overlap between the phonon bands is the same in forward and reversed configurations. (c) and (d) correspond to $R\approx 0.8$:  in the forward configuration (c)  the phonons match better than in the reversed configuration (d).}
  \label{fig:Figure_Spectra}
\end{figure*}

The Fourier transform of the vector process $\overrightarrow{r}(t)$ describing the evolution of our system, see Eq. (\ref{eq:vectorEqOfMotion}),
is $\overrightarrow{R}(\omega) = \left( i \omega - \mathbb{A} \right)^{-1}\mathbb{L}\overrightarrow{\Xi}(\omega)$ with $\overrightarrow{\Xi}(\omega)$ being the Fourier transform of the white noise $\overrightarrow{\xi}(t)$. Note that $\overrightarrow{\Xi}(\omega)$ is not square-integrable, however its spectral density is $\mathbb{S}_{\overrightarrow{\xi}}(\omega) = 2 \mathbb{D}$ \cite{Sarkka2019}, which is flat as expected for a white noise. Therefore, the spectral density matrix of the system is
%
\begin{equation}
  \mathbb{S}_{\overrightarrow{r}} (\omega)= 2 \left(  \mathbb{A} - i\omega\right)^{-1}\mathbb{L}\mathbb{D}\mathbb{L}^\mathsf{T}\left(  \mathbb{A} + i\omega\right)^{-\mathsf{T}}.
  \label{eq:SpectralDensityToyModelB}
\end{equation}
%
%As we can see in Eq. \eqref{eq:SpectralDensityToyModelB},
The imaginary part of the eigenvalues of the dynamical matrix $\mathbb{A}$ correspond to the peaks in the spectrum whereas the real part dictates their width. Eq. \eqref{eq:SpectralDensityToyModelB} gives after direct computation
%
\begin{equation}
  \mathbb{S}_{\overrightarrow{r}}(\omega) = 2 k_B \frac{\gamma_L T_L\mathbb{S}_L(i\omega)+\gamma_L T_R\mathbb{S}_R(i\omega)}{(m_1 m_2)^2 P_\mathbb{A}(i\omega)P_\mathbb{A}(-i\omega)},
\end{equation}
%
where $P_\mathbb{A}(\lambda)$ is the characteristic polynomial of the dynamical matrix $\mathbb{A}$ and $\mathbb{S}_L(\omega)$, $\mathbb{S}_R(\omega)$ are the matrix polynomials in the angular frequency $\omega$ whose coefficients are defined in Appendix \ref{Appendix:SpectralDensity}. We give
%Equation \eqref{eq:SpectralDensitiesVelocities} gives the full expressions of
the spectral densities for the velocities, ${\cal{S}}_1\equiv\mathbb{S}_{3,3}(\omega) = \expval{R_3(\omega)R_3(-\omega)}$ for the left ion, and ${\cal S}_2\equiv\mathbb{S}_{4,4}(\omega) = \expval{R_4(\omega)R_4(-\omega)}$ for the right ion, since they are the elements related to the calculation of the heat current using Eq. \eqref{eq:Wiener-Khinchin},
%
  \begin{align}
    {\cal S}_1(\omega) &= 2 k_B \frac{\gamma_R k^2 T_R \omega ^2+\gamma_L T_L \left[\omega ^4 \left(\gamma_R^2-2 k m_2-2 k_R m_2\right)+\omega ^2 (k+k_R)^2+m_2^2 \omega ^6\right]}{(m_1 m_2)^2 P_\mathbb{A}(i\omega)P_\mathbb{A}(-i\omega)},\nonumber\\
    %
%    \nonumber\\
    %
%    \mathbb{S}_{4,4}(\omega)
{\cal S}_2(\omega)
&= 2 k_B \frac{\gamma_L k^2 T_L \omega ^2+\gamma_R T_R \left[\omega ^4 \left(\gamma_L^2-2 k m_1-2 k_L m_1\right)+\omega ^2 (k+k_L)^2+m_1^2 \omega ^6\right]}{(m_1 m_2)^2 P_\mathbb{A}(i\omega)P_\mathbb{A}(-i\omega)}.
    \label{eq:SpectralDensitiesVelocities}
  \end{align}
%
%The spectral densities of the masses depend explicitly on the temperatures on the baths, as well as implicitly through the dependence of the friction coefficients if the laser cooling baths are used.
Figure \ref{fig:Figure_Spectra} depicts a series of plots of the spectra given by Eq. \eqref{eq:SpectralDensitiesVelocities}, corresponding to two points in Fig. \ref{fig:Fig_PerfectRectification}. (The calculation for the reverse bias is done with the substitutions in Table \ref{tab:reversed_bath}.)
For $c=1$ (Fig. \ref{fig:Figure_Spectra}(a) and (b)) there is no rectification, since the spectra match in the forward (a) and reversed (b) configurations. However, for $c=10$ (Fig. \ref{fig:Figure_Spectra}(c) and (d), $R\approx 0.8$) the picture is very different: there is a good match between the spectra in the forward configuration but not for the reversed configuration. It is interesting to analyze how the system changes from $c=1$ to $c=10$
% ( Fig. \ref{fig:Figure_Spectra} (a),(b) to Fig. \ref{fig:Figure_Spectra} (c),(d)
using the dissipative normal modes of the system, which may be found by diagonalizing the dynamical matrix $\mathbb{A}$, Eq. \eqref{eq:Dynamical_Matrix}. The frequencies of the peaks in Fig. \ref{fig:Figure_Spectra} are given by the imaginary part of the eigenvalues of $\mathbb{A}$. Likewise, the width depends on  the real part of the eigenvalues. For the forward configuration, the normal frequencies (position of the peaks) come closer to each other as $c$ is increased, while the widths remain practically constant. To understand why the real part remains practically constant, we recall that we have chosen to work with spring constants that satisfy Eq. \eqref{eq:MaxRLines} and making the mass and friction coefficient ratios equal to $c$, \textit{i.e.}, $ c\equiv m_2/m_1 = \gamma_R/\gamma_L$. The (dissipative) terms in $\mathbb{A}$ responsible for the real parts in the eigenvalues are,
for the forward configuration,  $\gamma_L/m_1$ and $\gamma_R/m_2 = (c \gamma_L)/(c m_1) = \gamma_L/m_1$, which are constant for every value of $c$. On the contrary, in the reverse bias configuration  the dissipative terms in the dynamical matrix  are $\tilde{\gamma}_L/m_1 = \gamma_R/m_1 = c\gamma_L/m_1$ and $\tilde{\gamma}_R/m_2 = \gamma_L/ (c m_1)$, with opposite behavior with respect to $c$. The real parts of the eigenvalues
also behave quite differently for reverse bias, one of them gets closer to the imaginary axis for $c=10$,
see Fig. \ref{fig:Figure_Spectra} (d), where this mode  concerns mostly the right ion,
the only one excited at the peak frequency, while the other eigenvalue  moves far from the imaginary axis so a peak is not noticeable
at the imaginary value (left dotted-dashed line) any more.
%The different behavior of the spectra in the reversed bias with respect to the forward bias explains why heat currents are lower for the reversed bias.
%In fact, in the limit for high values of $c$, which leads to perfect rectification, two purely dissipative eigenvalues appear (the imaginary part is 0) which could indicate low heat current.


%Figure \ref{fig:Figure_Spectra} only shows the elements (3,3) and (4,4) in the diagonal of $\mathbb{S}$ but the remaining elements, including off-diagonal ones, exhibit a similar behavior.
%
\section{Conclusions \label{sec:Conclusions}}
%
We have studied heat rectification in a model composed of two coupled harmonic oscillators connected to Langevin baths, which could be realized with trapped ions and optical molasses. This simple model allows analytical treatment but still has enough complexity to examine different ingredients that can produce rectification. %We have also derived analytical expressions for the heat currents and local temperatures.
Our results demonstrate in a simple but realistic model that harmonic systems can rectificate heat current if they have features which depend on the temperature  \cite{Pereira2017}. We implement this notion of temperature-dependent features by defining the baths exchange operation as an exchange of both temperatures and coupling parameters of the baths to the system. The temperature dependence of the bath-system coupling  occurs naturally in laser-cooled trapped ion setups.

We have also studied the phonon spectra of the system, aided by a normal mode analysis,
comparing the match/mismatch of the phonon bands, to reach the conclusion that the band match/mismatch description for heat rectification is also valid for systems which are purely harmonic, as long as there are temperature-dependent features.
We hope this article sheds more light into the topic of heat rectification and that encourages more research regarding its physical implementation on chains of trapped ions.
    % Rectification in a minimal model

\addtocontents{toc}{\vspace{2.0em}}
% ------------ CONCLUSIONS -------------------------------------------------------------------------
%!TEX root = ../Thesis.tex

\chapter*{Conclusions} % Write in your own chapter title
\label{Conclusions}
\lhead{\emph{Conclusions}} % Write in your own chapter title to set the page header

My pleasure.
 %Conclusions


% ------------ APPENDICES --------------------------------------------------------------------------
\addtocontents{toc}{\vspace{2em}} % Add a gap in the Contents, for aestheticsy

\part*{Appendices}
\addcontentsline{toc}{part}{Appendices}

\appendix % Cue to tell LaTeX that the following 'chapters' are Appendices

%Properties of separable potentials
%%!TEX root = ../Thesis.tex

\chapter{Properties of separable potentials}
\label{Appendix:SeparablePotentials}
\lhead{Appendix C. \emph{Properties of separable potentials}}

\section{Transition operator}
\label{Appendix:SeparablePotentials_TransitionOperator}
%%%%%%%%%%%%%%%%%
 For a separable potential $V=V_0 \ketbra{\phi}{\chi}$, the transition operator becomes
\begin{eqnarray}
T_{op}=\alpha \ketbra{\phi}{\chi}
\end{eqnarray}
where $\alpha=V_{0}+V_{0}^{2}\bra{\chi}G(E)\ket{\phi}$. Then using the Lippmann-Schwinger equation we get that
%
\begin{eqnarray}
T_{op}(E)&=&V+VG_{0}(E)T_{op}(E)
\nonumber \\
&=& \left[V_{0}+\alpha V_{0} \bra{\chi}G_{0}(E)\ket{\phi}\right]\ketbra{\phi}{\chi}
\end{eqnarray}
%
where $G_{0}(E)=(E-H_{0})^{-1}$ is the Green's operator for free motion. Solving for $\alpha$ now gives
%
\begin{eqnarray}
\alpha =\frac{V_{0}}{1-V_{0} \bra{\chi}G_{0}(E)\ket{\phi}}
		   =\frac{V_{0}}{1-V_{0} Q_{0}(E)}.
\end{eqnarray}
%
%%%%%%%%%%%%%%%%%
%
%
%
%
\section{$S$-matrix eigenvalues}
\label{Appendix:SeparablePotentials_AmplitudesAndEigenvaluesofS}
%
The eigenvalues for the S-matrix are given by Eq. \eqref{eq:chapter2_SEigenvalues} in terms of the reflection and transmission amplitudes. For a separable potential, using Eq. \eqref{eq:chapter1_amplitudesFromTOperator}, we can simplify the transmission and reflection coefficients as
\begin{eqnarray}
T^{l}&=&1-\frac{2 \pi i m}{p} \alpha \phi(p) \chi^{*}(p),
\nonumber \\
T^{r}&=&1-\frac{2 \pi i m}{p} \alpha \phi(-p) \chi^{*}(-p),
\nonumber \\
R^{l}&=&-\frac{2 \pi i m}{p} \alpha \phi(-p) \chi^{*}(p),
\nonumber \\
R^{r}&=&-\frac{2 \pi i m}{p} \alpha \phi(p) \chi^{*}(-p).
\nonumber \\
\end{eqnarray}
%
If we now define
%
\begin{equation}
\Gamma=\frac{2 \pi i m}{p} \alpha \left[\phi(p) \chi^{*}(p)+\phi(-p) \chi^{*}(-p)\right],
\end{equation}
%
we can write the eigenvalues as simply
%
\begin{eqnarray}
S_{j}&=&1-\frac{\Gamma-(-1)^{j}\Gamma}{2}.
\end{eqnarray}
%
Note that $S_2=1$ for all $p$. Clearly the following relation must also always hold for the reflection and transmission amplitudes,
%
\begin{equation}
T^l + T^r - T^l T^r + R^l R^r = 1.
\end{equation}
%
%
%
% \section{Uniqueness of bound state}
% %
% A separable potential  can only have at most one bound state $\ket{\psi_{E}}$.
% In momentum representation,
% %
% \begin{eqnarray}
% \braket{p}{\psi_{E}}&=&\bra{p}\frac{V_{0}}{E-H_{0}}\ket{\phi}\braket{\chi}{\psi_{E}}
% \nonumber \\
% &=&\frac{M}{p^{2}-q_{B}^{2}}\braket{p}{\phi},
% \end{eqnarray}
% %
% where $M=-2 m V_{0} \braket{\chi}{\psi_{E}}$ and $q_{B}^2=2 m E<0$. Suppose there is a second bound state $\ket{\psi_{E'}}$,
% with corresponding quantities $M'$ and $q_{B'}^2$. Then,
% %
% \begin{eqnarray}
% \braket{\psi_{E'}}{\psi_{E}}&=&M M' \int_{-\infty}^\infty dp \left|\braket{p}{\phi}\right| \frac{1}{p^{2}-q_{B}^{2}}\frac{1}{p^{2}-q_{B'}^{2}}. \nonumber \\		\end{eqnarray}
% %
% Since $MM'\ne 0$ and the integral is positive the overlap cannot be zero
% so there cannot be two bound states.


%Numerical calculation of transmission and reflection coefficients
%%!TEX root = ../Thesis.tex

\chapter{Numerical calculation of transmission and reflection coefficients}
\label{Appendix:NumericalCalculationOfTandR}
\lhead{Appendix B. \emph{Numerical calculation of transmission and reflection...}}

%

Here we will discuss how to numerically solve  the
stationary Schr\"odinger equation for the two-level system
by the invariant imbedding method \cite{Singer1982,Band1994}.
%

%
Let the potential ${\cal V}(x)$ be non-zero in the region $-d < x < d$.
We  introduce the following dimensionless variables: $\bar k = (2mE)^{1/2}2d/\hbar$, $\bar x = x/(2d) + 1/2$,
$\bar\Omega (\bar x)= (4md^2/\hbar) \Omega(x)$ and $\bar\Gamma = (4md^2/\hbar) (\gamma-2i\Delta)$.
%
%
The non-Hermitian dimensionless Hamiltonian for the system takes the form
%
\begin{eqnarray}
	\bar {\cal H}&=& \bar {\cal H}_{0}+\bar {\cal V}(\bar x), \\
	\bar {\cal H}_{0}&=&- \frac{\partial^2 }{\partial {\bar x}^2}+\left(\begin{array}{cc}
	0 & 0 \\
	0 & -i\bar\Gamma
	\end{array}\right), \\
	\bar {\cal V}(\bar x)&=& \left(\begin{array}{cc}
	0 & \bar\Omega (\bar x)\\
	\bar\Omega (\bar x)^{*} & 0
	\end{array}\right).
\end{eqnarray}
%
To set the matrices we use as in the main text the convention for internal states $\ket{1} = \left(\begin{smallmatrix}1\\0\end{smallmatrix}\right)$ and $\ket{2} = \left(\begin{smallmatrix}0\\1\end{smallmatrix}\right)$.
%
To simplify the notation, we will from now on drop the bars above variables and operators for the remaining part of this section A.
The corresponding stationary Schr\"odinger equation is now
%
\begin{eqnarray}
	k^{2} \psi^{(1)}(x)&=&-\frac{\partial^2}{\partial x^2}\psi^{(1)}(x)+\Omega(x) \psi^{(2)}(x),
	\nonumber \\
	k^{2} \psi^{(2)}(x)&=&-\frac{\partial^2}{\partial x^2}\psi^{(2)}(x)+\Omega(x)^* \psi_{1}(x)-i \Gamma \psi^{(2)}(x).
\end{eqnarray}
%
Let us denote as  $|{\Psi}_\alpha(x)\ra$  the wave vector for the atom impinging in internal level $\alpha$, $\alpha=1,2$.
This vector has ground and excited state components, generically $\braket{\beta}{\psi_{\alpha}(x)}$, $\beta=1,2$, which are still functions of $x$.
We can define the matrices $F(x)$ and $\widetilde{F}(x)$ as
%
\begin{eqnarray}
	F_{\beta,\alpha} (x) = \braket{\beta}{\psi_{\alpha}(x)},
	\quad
	\widetilde{F}_{\beta,\alpha} (x) = \braket{\beta}{\widetilde{\psi}_{\alpha}(x)},
\end{eqnarray}
%
so the stationary Schr\"odinger equation can be rewritten as
%
\begin{eqnarray}
	\left[k^2-{\cal H}_{0}-{\cal V}(x)\right]F(x)&=&0,
	\nonumber\\
	\left[k^2-{\cal H}_{0}-{\cal V}(x)\right]\widetilde{F}(x)&=&0.
\end{eqnarray}
%

\section{Free motion, ${\cal V}=0$}
%
%
When ${\cal V}(x)=0$ we get
%
\begin{eqnarray}
	\left[k^2-{\cal H}_{0}\right]\ket{\psi_{\alpha}(x)}&=&0,
	\nonumber\\
	\left[k^2-{\cal H}_{0}\right]\ket{\widetilde{\psi}_{\alpha}(x)}&=&0,
\end{eqnarray}
%
for $\alpha=1,2$.
We can write down the solutions for particles ``coming'' from the left $\ket{\psi_{\alpha}(x)}$ in  internal state $\ket{\alpha}$ as
%
\begin{equation}
	\ket{\psi_{1}(x)} = \left(\begin{array}{c}
	\frac{1}{\sqrt{k}} e^{i k x}\\
	0
	\end{array}\right)\!,
	\,
	\ket{\psi_{2}(x)} = \left(\!\!\begin{array}{c}
	0\\
	\frac{1}{\sqrt[4]{k^2+i \Gamma}} e^{i \sqrt{k^2+i \Gamma}x}
	\end{array}\!\!\right)\!,
	% \nonumber
\end{equation}
%
where we assume the branch $\operatorname{Im} \sqrt{k^2+i \Gamma}\ge 0$.
$\ket{\psi_2(x)}$ is a regular traveling wave only for real $\sqrt{k^2+i\Gamma})$. If the square root has an imaginary part,  $\ket{\psi_2(x)}$  decays  from left to right.
%
The solutions for incidence from the right  $\ket{\widetilde{\psi}_{\alpha}(x)}$ in internal state $\ket{\alpha}$ are similarly
%
\begin{equation}
	\ket{\widetilde{\psi}_{1}(x)} = \left(\!\begin{array}{c}
	\frac{1}{\sqrt{k}} e^{-i k x}\\
	0
	\end{array}\!\right)\!,
	\ket{\widetilde{\psi}_{2}(x)} = \left(\!\!\begin{array}{c}
	0\\
	\frac{1}{\sqrt[4]{k^2+i \Gamma}} e^{-i \sqrt{k^2+i \Gamma}x}
	\end{array}\!\!\right)\!.
	% \nonumber
\end{equation}
%
The normalization is chosen in such a way that the dimensionless probability current
is constant (and equal) for all solutions with real $\sqrt{k^2+i\Gamma}$.


The solutions are given by $F(x) = h_+ (x)$ and $\widetilde F (x) = h_- (x)$, where
%
\begin{equation}
	h_{\pm}(x)=\left(\begin{array}{cc}
	\frac{1}{\sqrt{k}}e^{\pm i k x} & 0\\
	0 & \frac{1}{\sqrt[4]{k^2+i \Gamma}} e^{\pm i \sqrt{k^2+i \Gamma} x} \\
	\end{array}\right).
\end{equation}
%
The Wronskian is $W(h_{+},h_{-})(x)=2i$ so that these are linearly independent solutions.


\section{General case}
To solve the general case, we construct the Green's function defined by
%
\begin{equation}
	(k^2-{\cal H}_{0})G_{0}(x,x')=\delta(x-x')\mathbf{1}.
\end{equation}
%
It is given by
%
\begin{align}
	G_{0}(x,x')&=W^{-1} \begin{cases}
	h_{+}(x)h_{-}(x')  & x>x' , \\ h_{+}(x')h_{-}(x) & x'>x ,
	\end{cases}\nonumber \\
	&= -\frac{i}{2} \left(\!\begin{array}{cc}
	\frac{1}{k}e^{ i k \abs{x-x'}} & 0\\
	0 & \frac{e^{ i \sqrt{k^2+i \Gamma} \abs{x-x'}}}{\sqrt{k^2+i \Gamma}} \\
	\end{array}\!\right).
\end{align}
%
The Green's function allows us to solve for   $F(x)$ and $\widetilde{F}(x)$ in integral form,
%
\begin{eqnarray}
	F(x)=h_{+}(x)+\int_{-\infty}^{\infty} dx' G_{0}(x,x') {\cal V}(x') F(x'),
	\nonumber\\
	\widetilde{F}(x)=h_{-}(x)+\int_{-\infty}^{\infty} dx' G_{0}(x,x') {\cal V}(x') \widetilde{F}(x').
	\label{eqf}
\end{eqnarray}


\section{Asymptotic form of the solutions}
%
From eq. \eqref{eqf} we find the following asymptotic forms of  $F(x)$ and $\widetilde{F}(x)$:
%
\begin{eqnarray}
	F_{\eta}(x)&=& \begin{cases}
	h_{+}(x)+h_{-}(x)R  & x<0  \\ h_{+}(x)T & x>1
	\end{cases},
	\nonumber\\
	\widetilde{F}_{\eta}(x)&=&\begin{cases}
	h_{-}(x)\widetilde{T}  &  x<0  \\ h_{-}(x)+h_{+}(x)\widetilde{R} & x>1
	\end{cases},
\end{eqnarray}
%
where the $R$ and $T$ matrices for incidence from the left are given by
%
\begin{eqnarray}
	R &=& W^{-1}\int_{0}^{1} dx' h_{+}(x'){\cal V}(x')F (x'),
	\nonumber\\
	T &=& \mathbf{1}+W^{-1}\int_{0}^{1} dx' h_{-}(x'){\cal V}(x')F (x'),
\end{eqnarray}
%
whereas, for right incidence,
%
\begin{eqnarray}
	\widetilde{R} &=& W^{-1}\int_{0}^{\eta} dx' h_{-}(x'){\cal V}(x')\widetilde{F}_{\eta}(x'),
	\nonumber\\
	\widetilde{T} &=& \mathbf{1}+W^{-1}\int_{0}^{\eta} dx' h_{+}(x'){\cal V}(x')\widetilde{F}_{\eta}(x').
\end{eqnarray}
%
In particular, for left incidence  in the ground-state, we get  if $x < 0$,
%
\begin{eqnarray}
	\hspace*{-.5cm}|\psi_1 (x)\ra = \left(\!\!\begin{array}{c} \frac{1}{\sqrt{k}}e^{i k x} \\ 0\end{array}\!\!\right)
	+ \left(\!\!\begin{array}{c}
	R_{1,1} \frac{1}{\sqrt{k}}e^{-i k x} \\
	R_{2,1} \frac{1}{\sqrt[4]{k^2+i \Gamma}} e^{- i \sqrt{k^2+i \Gamma} x}
	\end{array}\!\!\right)\!,
	\label{as-}
\end{eqnarray}
%
and, if $x>1$,
%
\begin{eqnarray}
	|\psi_1 (x)\ra = \left(\begin{array}{c}
	T_{1,1} \frac{1}{\sqrt{k}}e^{i k x} \\
	T_{2,1} \frac{1}{\sqrt[4]{k^2+i \Gamma}} e^{i \sqrt{k^2+i \Gamma} x}
	\end{array}\right).
	\label{as+}
\end{eqnarray}
%
When $\sqrt{k^2+i \Gamma}$ is real, the elements of $T$ and $R$ in Eqs. \eqref{as-} and \eqref{as+}
are transmission and reflection amplitudes for waves traveling away from the interaction region.
However when  ${\rm Im}\sqrt{k^2+i \Gamma}>0$ the waves for the excited state $2$ are evanescent.
In scattering theory parlance the channel is ``closed'', so  the $T_{2,1}$ and $R_{2,1}$ are just proportionality factors
rather than proper transmission
and reflection amplitudes for travelling waves. By continuity however, it is customary to keep the same notation
and even terminology for closed or open channels.


In a similar way, for right incidence in the ground state and
$x > 1$,
%
\begin{eqnarray}
	\hspace*{-.5cm}|\widetilde\psi_1 (x)\ra = \left(\!\!\begin{array}{c} \frac{1}{\sqrt{k}}e^{-i k x} \\ 0\end{array}\!\!\right)
	\!+\! \left(\!\!\begin{array}{c}
	\widetilde R_{1,1} \frac{1}{\sqrt{k}}e^{i k x}
	\\
	\widetilde R_{2,1} \frac{1}{\sqrt[4]{k^2+i \Gamma}} e^{i \sqrt{k^2+i \Gamma} x}
	\end{array}\!\!\right)\!\!,
\end{eqnarray}
%
whereas, for $x<0$,
%
\begin{eqnarray}
	|\widetilde \psi_1 (x)\ra = \left(\begin{array}{c}
	\widetilde T_{1,1} \frac{1}{\sqrt{k}}e^{-i k x} \\
	\widetilde T_{2,1} \frac{1}{\sqrt[4]{k^2+i \Gamma}} e^{-i \sqrt{k^2+i \Gamma} x}
	\end{array}\right).
\end{eqnarray}
%
Note that alternative definitions of the amplitudes may be found in many works,
without momentum prefactors.

The amplitudes relevant for the main text are $T^l=T_{1,1}$,
$T^r=\widetilde{T}_{1,1}$, $R^l=R_{1,1}$, and $R^r=\widetilde{R}_{1,1}$.  The following
subsection explains how to compute them.


\section{Differential equations for $R$ and $T$ matrices}
To solve for $R$ and $T$ we will use cut-off versions of the  potential,
%

\begin{equation}
	{\cal V}_{\eta} (x) =\begin{cases}
	{\cal V}(x)  & 0\leq x \leq \eta , \\ 0& \text{Otherwise}
	\end{cases},
\end{equation}
%
where $0 \le \eta \le 1$, and corresponding matrices
%
%
\begin{eqnarray}
	R_{\eta}&=& W^{-1}\int_{0}^{\eta} dx' h_{+}(x'){\cal V}(x')F_{\eta}(x'),
	\nonumber\\
	T_{\eta}&=& \mathbf{1}+W^{-1}\int_{0}^{\eta} dx' h_{-}(x'){\cal V}(x')F_{\eta}(x'),
	\nonumber\\
	%
	\widetilde{R}_{\eta} &=& W^{-1}\int_{0}^{\eta} dx' h_{-}(x'){\cal V}(x')\widetilde{F}_{\eta}(x'),
	\nonumber\\
	\widetilde{T}_{\eta} &=& \mathbf{1}+W^{-1}\int_{0}^{\eta} dx' h_{+}(x'){\cal V}(x')\widetilde{F}_{\eta}(x').
\end{eqnarray}
%
Taking the derivative of these matrices with respect to $\eta$,
we find a set
of four coupled differential equations,
%
\begin{eqnarray}
	\frac{d R_{\eta}}{d \eta}&=& W^{-1} \widetilde{T}_{\eta} h_{+}(\eta){\cal V}(\eta)h_{+}(\eta)T_{\eta}, \\
	\frac{d T_{\eta}}{d \eta}&=& W^{-1} \left[h_{-}(\eta)+\widetilde{R}_{\eta}h_{+}(\eta)\right]{\cal V}(\eta)h_{+}(\eta)T_{\eta},\\
	\frac{d \widetilde{R}_{\eta}}{d \eta}&=& W^{-1}\!\!\left[h_{-}(\eta)\!+\!\widetilde{R}_{\eta}h_{+}(\eta)\right]\!{\cal V}(\eta)\!\left[h_{-}(\eta)\!+\!h_{+}(\eta)\widetilde{R}_{\eta}\right]\!, \nonumber\\
	\label{a20a}\\
	\frac{d \widetilde{T}_{\eta}}{d \eta}&=& W^{-1} \widetilde{T}_{\eta} h_{+}(\eta){\cal V}(\eta)\left[h_{-}(\eta)+h_{+}(\eta)\widetilde{R}_{\eta}\right].
	\label{a20b}
\end{eqnarray}
%
The initial conditions are $R_{0} = \widetilde{R}_{0}=0$ and $T_{0} = \widetilde{T}_{0}=\mathbf{1}$.


\section{Improving numerical efficiency}
%
{The equations \eqref{a20a} and \eqref{a20b}} involve only  matrices for incidence from the right, they  do not couple to any left-incidence matrix, whereas the equations for left incidence
amplitudes involve couplings with amplitudes for right incidence. This asymmetry is due to the way we do the potential slicing. The asymmetry  is not ``fundamental''
but we can use it for our advantage to simplify calculations. We can solve equations \eqref{a20a} and \eqref{a20b} to get amplitudes for right incidence.
To get amplitudes for left incidence we use a mirror image of the potential and solve also these two equations.
Thus it is enough to find an efficient numerical method to solve equations \eqref{a20a} and \eqref{a20b}.
In principle, one can now solve these differential equations from $\eta=0$ to $1$ to get all reflection and transmission amplitudes using the boundary conditions $\widetilde{R}_{0}=0$ and $\widetilde{T}_{0}=\mathbf{1}$. However due to the exponential nature of the
free-space solutions $h_{\pm}(x)$ especially if ${\rm Im}\sqrt{k^2+i \Gamma}>0$, this is not very efficient numerically.

To avoid this problem we make new definitions,
%
\begin{eqnarray}
	\hat{S}_{\eta}&=&\mathbf{1}+h_{+}(\eta)\widetilde{R}_{\eta}h_{-}^{-1}(\eta),
	\nonumber\\
	\hat{T}_{\eta}&=&h_{+}(0)\widetilde{T}_{\eta}h_{-}^{-1}(\eta),
	\nonumber\\
	\hat{\cal V}(\eta)&=&W^{-1}h_{+}^{2}(0){\cal V}(\eta),
	\nonumber\\
	\hat{Q}&=&i h_{+}^{-2}(0).
\end{eqnarray}
%
Rewriting {the equations \eqref{a20a} and \eqref{a20b}} in terms of these new variables we get
%
\begin{eqnarray}
	\frac{d \hat{S}_{\eta}}{d \eta}&=&-2 \hat{Q}+\hat{Q}\hat{S}_{\eta}+\hat{S}_{\eta}\left[\hat{Q}+\hat{\cal V}(\eta)\hat{S}_{\eta}\right],
	\nonumber\\
	\frac{d \hat{T}_{\eta}}{d \eta}&=& \hat{T}_{\eta}\left[\hat{Q}+\hat{\cal V}(\eta)\hat{S}_{\eta}\right],
\end{eqnarray}
%
with initial conditions $\hat{T}_{0} =\hat{S}_{0}=\mathbf{1}$.

Let us consider solely incidence in the ground state. For right incidence in the ground state,
the reflection coefficients and transmission coefficient are
%
\begin{eqnarray}
	\widetilde R_{1,1} &=& e^{-2ik} \left[(\hat S_{\eta=1})_{1,1} - 1 \right],
	\nonumber\\
	\widetilde R_{2,1} &=&
	\frac{\sqrt[4]{k^2+i \Gamma}}{\sqrt{k}} e^{-ik-i\sqrt{k^2+i \Gamma}} (\hat S_{\eta=1})_{2,1},
	\nonumber\\
	\widetilde T_{1,1} &=& e^{-i k} (\hat T_{\eta=1})_{1,1},
	\nonumber\\
	\widetilde T_{2,1} &=& \frac{\sqrt[4]{k^2+i \Gamma}}{\sqrt{k}} e^{-i k} (\hat T_{\eta=1})_{2,1}.
\end{eqnarray}
\vspace*{1cm}


\section{Bounds from unitarity}
%
The $S$-matrix
%
\begin{equation}
	S=\left(\begin{array}{cccc}
	T_{11}&T_{12}&\widetilde R_{11}&\widetilde R_{12}
	\\
	T_{21}&T_{22}&\widetilde R_{21}&\widetilde R_{22}
	\\
	R_{11}&R_{12}&\widetilde T_{11}&\widetilde T_{12}
	\\
	R_{21}&R_{22}&\widetilde T_{21}&\widetilde T_{22}
	\end{array}\right)
\end{equation}
%
is unitary for Hermitian Hamiltonians,  in particular when $\gamma=0$.
Unitarity implies relations among the matrix elements and in particular
%
\begin{eqnarray}
	1&\ge& |R_{11} |^2+|T_{11} |^2,\label{a28}
	\\
	1&\ge& |\widetilde R_{11}|^2+|\widetilde T_{11}|^2, \label{a29}
	\\
	1&\ge& |\widetilde R_{11} |^2+|T_{11} |^2,
	\label{a30}\\
	1&\ge& |R_{11} |^2+|\widetilde T_{11}|^2.\label{a31}
\end{eqnarray}
%
While the first two equations \eqref{a28} and \eqref{a29} are rather obvious because of  probability conservation, the last two equations
\eqref{a30} and \eqref{a31} are less so, and
set physical  limits to the possible asymmetric devices that can be constructed in the ground state subspace.


%Full set of steady-state equations for the components of $\mathbb{C}^{s.s}$
%%!TEX root = ../Thesis.tex

\chapter{Full set of steady-state equations for the components of $\mathbb{C}^{s.s}$}
\label{Appendix:SteadyStateEquations}
\lhead{Appendix E. \emph{Full set of steady-state equations for the components of $\mathbb{C}^{s.s}$}}

Here we present the full set of equations for the covariance matrix elements in the steady state,
%
\begin{equation}
  \begin{split}
    \frac{2 k \expval{ p_2 q_1}^{s.s.} }{m_1 m_2}+\frac{2 \gamma _L \expval{ p_1^2}^{s.s.} }{m_1^3}&=\frac{2 D_L}{m_1^2},
    %
    \\
    -\frac{2 k \expval{ p_2 q_1}^{s.s.} }{m_2^2}+\frac{2 \gamma _R \expval{ p_2^2}^{s.s.} }{m_2^3}&=\frac{2 D_R}{m_2^2},
    %
    \\
    %
    -\frac{\left(k_L+k\right) \expval{ q_1 q_2}^{s.s.} }{m_1}+\frac{k \expval{ q_2^2}^{s.s.} }{m_1}+\frac{\gamma _L \expval{ p_2 q_1}^{s.s.} }{m_1 m_2}+\frac{\expval{ p_1 p_2}^{s.s.} }{m_1 m_2}&=0,
    %
    \\
    %
    \frac{\left(k_L+k\right) \expval{ p_2 q_1}^{s.s.} }{m_1 m_2}-\frac{\left(k_R+k\right) \expval{ p_2 q_1}^{s.s.} }{m_2^2}+\frac{\gamma _L \expval{ p_1 p_2}^{s.s.} }{m_1^2 m_2}+\frac{\gamma _R \expval{ p_1 p_2}^{s.s.} }{m_1 m_2^2}&=0,
    %
    \\
    %
    -\frac{\left(k_L+k\right) \expval{ q_1^2}^{s.s.} }{m_1}+\frac{k \expval{ q_1 q_2}^{s.s.} }{m_1}+\frac{\expval{ p_1^2}^{s.s.} }{m_1^2}&=0,
    %
    \\
    %
    -\frac{\left(k_R+k\right) \expval{ q_2^2}^{s.s.} }{m_2}+\frac{k \expval{ q_1 q_2}^{s.s.} }{m_2}+\frac{\expval{ p_2^2}^{s.s.} }{m_2^2}&=0,
    %
    \\
    %
    -\frac{\left(k_R+k\right) \expval{ q_1 q_2}^{s.s.} }{m_2}+\frac{k \expval{ q_1^2}^{s.s.} }{m_2}-\frac{\gamma _R \expval{ p_2 q_1}^{s.s.} }{m_2^2}+\frac{\expval{ p_1 p_2}^{s.s.} }{m_1 m_2}&=0
  \end{split}
  \label{eq:SteadyStateEquationsModelB_Explicite}
\end{equation}


%Complete expressions for the Spectral Density Matrix
%%!TEX root = ../Thesis.tex

\chapter{Complete expressions for the Spectral Density Matrix}
\label{Appendix:SpectralDensity}
\lhead{Appendix F. \emph{Complete expressions for the Spectral Density Matrix}}

%
In section \ref{sec:lookingForR} we used the characteristic polynomial $P_{\mathbb{A}}(\lambda)$ of the dynamical matrix $\mathbb{A}$ for the calculation of the spectral density matrix. $P_{\mathbb{A}}(\lambda)$ is defined as
\begin{equation}
  \begin{split}
    P_{\mathbb{A}}(\lambda) &\equiv\det(\mathbb{A}-\lambda)\\
    &= \lambda ^4 \\&+ \lambda ^3 \left(\frac{\gamma_L}{m_1}+\frac{\gamma_R}{m_2}\right) \\ &+ \lambda^2\frac{ (\gamma_L \gamma_R+m_2 (k+k_L)+m_1 (k+k_R))}{m_1 m_2}\\ &+ \lambda \frac{  (\gamma_R (k+k_L)+\gamma_L (k+k_R))}{m_1 m_2}\\ &+\frac{k (k_L+k_R)+k_L k_R}{m_1 m_2}.
  \end{split}
\end{equation}
%
We also used the polynomials  $\mathbb{S}_L(\lambda)=\sum\limits_{n=0}^6 \lambda^n \mathbb{s}_{L,n}$ and $\mathbb{S}_R(\lambda)=\sum\limits_{n=0}^6 \lambda^n \mathbb{s}_{R,n}$. There are 14 different polynomial coefficients, which are $4\times 4$ matrices. This is the full list of coefficients,
%
\begingroup
\allowdisplaybreaks
\begin{align}
    \mathbb{s}_{L,0} &=
    \left(
    \begin{array}{cccc}
      (k+k_R)^2 & k (k+k_R) & 0 & 0 \\
      k (k+k_R) & k^2 & 0 & 0 \\
      0 & 0 & 0 & 0 \\
      0 & 0 & 0 & 0
    \end{array}
    \right),
    %
    \nonumber\\
    %
    \mathbb{s}_{R,0} & =
    \left(
    \begin{array}{cccc}
      k^2 & k (k+k_L) & 0 & 0 \\
      k (k+k_L) & (k+k_L)^2 & 0 & 0 \\
      0 & 0 & 0 & 0 \\
      0 & 0 & 0 & 0
    \end{array}
    \right),
    %
    \nonumber\\
    %
    \mathbb{s}_{L,1} &= \left(
    \begin{array}{cccc}
      0 & k \gamma_R & -(k+k_R)^2 & -k (k+k_R) \\
      -k \gamma_R & 0 & -k (k+k_R) & -k^2 \\
      (k+k_R)^2 & k (k+k_R) & 0 & 0 \\
      k (k+k_R) & k^2 & 0 & 0
    \end{array}
    \right),
    %
    \nonumber\\
    %
    \mathbb{s}_{R,1} & = \left(
    \begin{array}{cccc}
      0 & -k \gamma_L & -k^2 & -k (k+k_L) \\
      k \gamma_L & 0 & -k (k+k_L) & -(k+k_L)^2 \\
      k^2 & k (k+k_L) & 0 & 0 \\
      k (k+k_L) & (k+k_L)^2 & 0 & 0
    \end{array}
    \right),
    %
    \nonumber\\
    %
    \mathbb{s}_{L,2} &= \left(
    \begin{array}{cccc}
      2 (k+k_R) m_2-\gamma_R^2 & k m_2 & 0 & -k \gamma_R \\
      k m_2 & 0 & k \gamma_R & 0 \\
      0 & k \gamma_R & -(k+k_R)^2 & -k (k+k_R) \\
      -k \gamma_R & 0 & -k (k+k_R) & -k^2
    \end{array}
    \right),
    %
    \nonumber\\
    %
    \mathbb{s}_{R,2} & = \left(
    \begin{array}{cccc}
      0 & k m_1 & 0 & k \gamma_L \\
      k m_1 & 2 (k+k_L) m_1-\gamma_L^2 & -k \gamma_L & 0 \\
      0 & -k \gamma_L & -k^2 & -k (k+k_L) \\
      k \gamma_L & 0 & -k (k+k_L) & -(k+k_L)^2
    \end{array}
    \right),
    %
    \nonumber\\
    %
    \mathbb{s}_{L,3} &= \left(
    \begin{array}{cccc}
      0 & 0 & \gamma_R^2-2 (k+k_R) m_2 & -k m_2 \\
      0 & 0 & -k m_2 & 0 \\
      2 (k+k_R) m_2-\gamma_R^2 & k m_2 & 0 & -k \gamma_R \\
      k m_2 & 0 & k \gamma_R & 0
    \end{array}
    \right),
    %
    \nonumber\\
    %
    \mathbb{s}_{R,3} & =\left(
    \begin{array}{cccc}
      0 & 0 & 0 & -k m_1 \\
      0 & 0 & -k m_1 & \gamma_L^2-2 (k+k_L) m_1 \\
      0 & k m_1 & 0 & k \gamma_L \\
      k m_1 & 2 (k+k_L) m_1-\gamma_L^2 & -k \gamma_L & 0
    \end{array}
    \right),
    %
    \nonumber\\
    %
    \mathbb{s}_{L,4} &= \left(
    \begin{array}{cccc}
      m_2^2 & 0 & 0 & 0 \\
      0 & 0 & 0 & 0 \\
      0 & 0 & \gamma_R^2-2 (k+k_R) m_2 & -k m_2 \\
      0 & 0 & -k m_2 & 0
    \end{array}
    \right),
    %
    \nonumber\\
    %
    \mathbb{s}_{R,4} & = \left(
    \begin{array}{cccc}
      0 & 0 & 0 & 0 \\
      0 & m_1^2 & 0 & 0 \\
      0 & 0 & 0 & -k m_1 \\
      0 & 0 & -k m_1 & \gamma_L^2-2 (k+k_L) m_1
    \end{array}
    \right),
    %
    \nonumber\\
    %
    \mathbb{s}_{L,5} &= \left(
    \begin{array}{cccc}
      0 & 0 & -m_2^2 & 0 \\
      0 & 0 & 0 & 0 \\
      m_2^2 & 0 & 0 & 0 \\
      0 & 0 & 0 & 0
    \end{array}
    \right),
    %
    \nonumber\\
    %
    \mathbb{s}_{R,5} & = \left(
    \begin{array}{cccc}
      0 & 0 & 0 & 0 \\
      0 & 0 & 0 & -m_1^2 \\
      0 & 0 & 0 & 0 \\
      0 & m_1^2 & 0 & 0
    \end{array}
    \right),
    %
    \nonumber\\
    %
    \mathbb{s}_{L,6} &= \left(
    \begin{array}{cccc}
      0 & 0 & 0 & 0 \\
      0 & 0 & 0 & 0 \\
      0 & 0 & -m_2^2 & 0 \\
      0 & 0 & 0 & 0
    \end{array}
    \right),
    %
    \nonumber\\
    %
    \mathbb{s}_{R,6} & = \left(
    \begin{array}{cccc}
      0 & 0 & 0 & 0 \\
      0 & 0 & 0 & 0 \\
      0 & 0 & 0 & 0 \\
      0 & 0 & 0 & -m_1^2
    \end{array}
    \right)\,.
\end{align}%
\endgroup



\addtocontents{toc}{\vspace{2.0em}}  % Add a gap in the Contents, for aesthetics
\backmatter
% \pagestyle{empty}  % Page style needs to be empty for this page

% ------------------ BIBLIOGRAPHY ----------------------------------------------------------------
\label{Bibliography}
\lhead{\emph{Bibliography}}             % Change the left side page header to "Bibliography"
\bibliographystyle{sofia}               % modificado por mi a partir de utphys
\bibliography{Bibliography_Thesis}      % The references (bibliography) information are stored in the file named "Bibliography_Thesis.bib"

\end{document}  % The End
%% ----------------------------------------------------------------
