%% ----------------------------------------------------------------
%% Thesis.tex -- MAIN FILE (the one that you compile with LaTeX)
%% ----------------------------------------------------------------

% Set up the document
%\documentclass[a4paper, 11pt, oneside]{Thesis}  % Use the "Thesis" style, based on the ECS Thesis style by Steve Gunn
%\documentclass[a4paper, 11pt, twoside, openright]{Thesis} %en a4
\documentclass[a4paper, 12pt, twoside, openright]{Thesis} %para reducir a b5
\graphicspath{{Figures/}}  % Location of the graphics files (set up for graphics to be in PDF format)

% Include any extra LaTeX packages required
\usepackage[square, numbers, comma, sort&compress]{natbib}  % Use the "Natbib" style for the references in the Bibliography
\usepackage{verbatim}  % Needed for the "comment" environment to make LaTeX comments
\usepackage{vector}  % Allows "\bvec{}" and "\buvec{}" for "blackboard" style bold vectors in maths
\usepackage{mathtools}

\usepackage{parskip} %para que haga el indent al principio de cada parrafo
\setlength{\parindent}{15pt}
\usepackage{indentfirst} %para que ponga el indent en el primer parrafo tambien, sino no lo hace

\hypersetup{urlcolor=blue, colorlinks=true}  % Colours hyperlinks in blue, but this can be distracting if there are many links.

%% ----------------------------------------------------------------
\begin{document}

%Abreviaciones de comandos y nuevos comandos.
\def\la{\langle}
\def\ra{\rangle}
\def\beq{\begin{equation}}
\def\eeq{\end{equation}}
\def\beqa{\begin{eqnarray}}
\def\eeqa{\end{eqnarray}}
\def\blankpage{
\newpage
\begin{equation}
\nonumber
\end{equation}
\newpage}
\def\ssst{\scriptscriptstyle}   %%para hacer más pequeñas algunas ecuaciones

\newcommand{\beqas}{\begin{eqnarray*}}
\newcommand{\eeqas}{\end{eqnarray*}}
\newcommand {\fexp} [1] {\exp \left( #1 \right)}
\newcommand {\fabs}[1] {\left| #1 \right|}
\newcommand{\cR}{{\cal{R}}}
\newcommand{\new}[1]{{\it #1}}

\frontmatter	  % Begin Roman style (i, ii, iii, iv...) page numbering

% Set up the Title Page
%----------------------------------------------------------------------------------------------------------------------------------------------------------
%----------------------------------------------------------------------------------------------------------------------------------------------------------
%----------------------------------------------------------------------------------------------------------------------------------------------------------
\begin{titlepage}
\thispagestyle{empty} %\vspace*{0.1cm}
% \hspace{-0.3cm}\includegraphics[scale=0.25]{ehu}\hspace{0.3cm}
{\centering
\includegraphics[scale=0.25]{ehu}
}
\bigskip
{\centering \large
\par \vspace{1.5cm}

\hrule\vspace*{0.3cm}

{\LARGE \bf {Asymmetric Quantum Devices\\and Heat Transport}}

\vspace{0.3cm}\hrule \vspace{2cm}
{\LARGE \bf{Miguel \'{A}ngel Sim\'{o}n Mart\'{i}nez}}\\
\vspace{1.25cm}
{\it{Supervisors:}} \\
\vspace{0.1cm}
{\large \bf {Prof. Juan Gonzalo Muga Francisco}}\\
{\large \bf {Prof. Maria Luisa Pons Barba}}\\
% \vspace{2.2cm}
\vfill
% \begin{figure}[h]
% {\centering {
%
% }\par}
% \end{figure}
% \vspace{1.0cm}
Departamento de Qu\'{\i}mica-F\'{\i}sica\\
Facultad de Ciencia y Tecnolog\'ia\\
Universidad del Pa\'is Vasco/Euskal Herriko Unibertsitatea\\ (UPV/EHU)\\
\vspace{1.0cm}
Leioa, Abril 2021\\
} \pagebreak
\end{titlepage}
%we use the title page document
%\newpage
%\mbox{}
%\thispagestyle{plain} % para que no se numere esta p�gina

%This is another way of generating a title page, (in fact below it is written how to use )
%\title  {Shortcuts to adiabaticity in the double well}
%\authors  {\texorpdfstring
%           {\href{sofia.martinez@ehu.eus}{Sof\'ia Mart\'inez Garaot}}
%            {}
%            }

%\addresses  {\groupname\\\deptname\\\univname}  % Do not change this here, instead these must be set in the "Thesis.cls" file, please look through it instead
%\date       {\today}
%\subject    {}
%\keywords   {}

%\maketitle
%----------------------------------------------------------------------------------------------------------------------------------------------------------
%----------------------------------------------------------------------------------------------------------------------------------------------------------
%----------------------------------------------------------------------------------------------------------------------------------------------------------

\setstretch{1.3}  % It is better to have smaller font and larger line spacing than the other way round

% Define the page headers using the FancyHdr package and set up for one-sided printing
\fancyhead{}  % Clears all page headers and footers
\rhead{\thepage}  % Sets the right side header to show the page number
\lhead{}  % Clears the left side page header

\pagestyle{fancy}  % Finally, use the "fancy" page style to implement the FancyHdr headers

\clearpage  % Declaration ended, now start a new page

%Meto una pagina en blanco
\pagestyle{empty}  % No headers or footers for the following pages
\null\vfill

\vfill\vfill\vfill\vfill\vfill\vfill\null
\clearpage  % Empty page ended, start a new page


%% ----------------------------------------------------------------
% The "Funny Quote Page"
\pagestyle{empty}  % No headers or footers for the following pages

\null\vfill
% Now comes the "Funny Quote", written in italics
\textit{``Insert deep/meaningful/clever quote by someone who you admire or hate''}

\begin{flushright}
Someone who you admire or hate
\end{flushright}

\vfill\vfill\vfill\vfill\vfill\vfill\null
\clearpage  % Funny Quote page ended, start a new page
%% ----------------------------------------------------------------

%% The Abstract Page
%\addtotoc{Abstract}  % Add the "Abstract" page entry to the Contents
%\abstract{
%\addtocontents{toc}{\vspace{1em}}  % Add a gap in the Contents, for aesthetics
%
%The Thesis Abstract is written here (and usually kept to just this page). The page is kept centered vertically so can expand into the blank space above the title too\ldots
%
%}
%
%\clearpage  % Abstract ended, start a new page
%%% ----------------------------------------------------------------

%Meto una pagina en blanco
\pagestyle{empty}  % No headers or footers for the following pages

\null\vfill

\vfill\vfill\vfill\vfill\vfill\vfill\null
\clearpage  % Empty page ended, start a new page

\setstretch{1.3}  % Reset the line-spacing to 1.3 for body text (if it has changed)

% The Acknowledgements page, for thanking everyone
\acknowledgements{
\addtocontents{toc}{\vspace{1em}}  % Add a gap in the Contents, for aesthetics
Do not forget about the people that has been by your side and supported you.
\it{Esta Tesis ha sido financiada a trav\'es de una beca predoctoral concedida por el Gobierno Vasco/Eusko Jaurlaritza}

}

\clearpage  % End of the Acknowledgements
%% ----------------------------------------------------------------

\pagestyle{fancy}  %The page style headers have been "empty" all this time, now use the "fancy" headers as defined before to bring them back


%% ----------------------------------------------------------------
\lhead{\emph{Contents}}  % Set the left side page header to "Contents"
\tableofcontents  % Write out the Table of Contents

%% ----------------------------------------------------------------
\lhead{\emph{List of Figures}}  % Set the left side page header to "List if Figures"
\listoffigures  % Write out the List of Figures

%%% ----------------------------------------------------------------
%\lhead{\emph{List of Tables}}  % Set the left side page header to "List of Tables"
%\listoftables  % Write out the List of Tables

%% ----------------------------------------------------------------
\setstretch{1.5}  % Set the line spacing to 1.5, this makes the following tables easier to read
\clearpage  % Start a new page

%% Begin the Dedication page
\setstretch{1.3}  % Return the line spacing back to 1.3
\pagestyle{empty}  % Page style needs to be empty for this page
\dedicatory{dedicatory here}

\addtocontents{toc}{\vspace{1em}}  % Add a gap in the Contents, for aesthetics
\chapter{List of publications} % Write in your own chapter title
\label{Publications}
\lhead{\emph{List of publications}} % Write in your own chapter title to set the page header
\hspace{-0.5 cm}{{\large\bf \rm I)}} {\bf The results of this Thesis are based on the following articles}
\section*{Published Articles}

\begin{enumerate}

  \item M. Pons, Y. Y. Cui, A. Ruschhaupt, {\bf M. A. Sim\'{o}n} and J. G. Muga\\
  {\it Local rectification of heat flux}\\
  \href{https://doi.org/10.1209/0295-5075/119/64001}{EPL {\bf 119}, 64001 (2017).}

  \item A. Ruschhaupt, T. Dowdall, {\bf M. A. Sim\'{o}n} and J. G. Muga\\
  {\it Asymmetric scattering by non-Hermitian potentials}\\
  \href{https://doi.org/10.1209/0295-5075/120/20001}{EPL {\bf 120}, 20001 (2017).}

  \item {\bf M. A. Sim\'{o}n}, A. Buend\'{i}a and J. G. Muga\\
  {\it Symmetries and Invariants for Non-Hermitian Hamiltonians}\\
  \href{https://doi.org/10.3390/math6070111}{Mathematics {\bf 6}, 111 (2018).}

  \item {\bf M. A. Sim\'{o}n}, A. Buend\'{i}a, A. Kiely, A. Mostafazadeh and J. G. Muga\\
  {\it $S$-matrix pole symmetries for non-Hermitian scattering Hamiltonians}\\
  \href{https://doi.org/10.1103/PhysRevA.99.052110}{Phys. Rev. A {\bf 99}, 052110 (2019).}

  \item {\bf M. A. Sim\'{o}n}, S. Mart\'{i}nez-Garaot, M. Pons and J. G. Muga\\
  {\it Asymmetric heat transport in ion crystals}\\
  \href{https://doi.org/10.1103/PhysRevE.100.032109}{Phys. Rev. E {\bf 100}, 032109 (2019).}

  \item A. Alaña, S. Mart\'{i}nez-Garaot, {\bf M. A. Sim\'{o}n} and J. G. Muga\\
  {\it Symmetries of (${N \times N}$ ) non-Hermitian Hamiltonian matrices}\\
  \href{https://doi.org/10.1088/1751-8121/ab7781}{J. Phys. A: Math. Theor. {\bf 53}, 135304 (2020).}

\end{enumerate}

\section*{Preprints}

\begin{enumerate}

  \item A. Ruschhaupt, A. Kiely, {\bf M. A. Sim\'{o}n} and J. G. Muga\\
  {\it Quantum-optical implementation of non-Hermitian potentials for asymmetric scattering}\\
  \href{https://arxiv.org/abs/2008.01702}{arXiv:2008.01702 [quant-ph] (2020)}\\
  (\href{https://journals.aps.org/pra/accepted/2f07eYb5Ibe1ea6603ef59232d2f3864c12b1a62d}{Accepted for publication in Phys. Rev. A})

  \item {\bf M. A. Sim\'{o}n}, A. Alaña, M. Pons, A. Ruiz-Garc\'{i}a and J. G. Muga\\
  {\it Heat rectification with a minimal model of two harmonic oscillators}\\
  \href{https://arxiv.org/abs/2010.10432}{arXiv:2010.10432 [cond-mat.stat-mech] (2020)}

\end{enumerate}

\vspace{1.25 cm}

\hspace{-0.65  cm}{{\large\bf \rm II)}} {\bf Other articles produced during the Thesis period}
\section*{Published  Articles not included in this Thesis}

\begin{enumerate}

  \item M. Palmero, {\bf M. A. Sim\'{o}n} and D. Poletti\\
  {\it Towards Generation of Cat States in Trapped Ions Set-Ups via FAQUAD Protocols and Dynamical Decoupling}\\
  \href{https://doi.org/10.3390/e21121207}{Entropy {\bf 21}, 1207 (2019)}

  \item {\bf M. A. Sim\'{o}n}, M. Palmero, S. Mart\'{i}nez-Garaot and J. G. Muga\\
  {\it Trapped-ion Fock-state preparation by potential deformation}\\
  \href{https://doi.org/10.1103/PhysRevResearch.2.023372}{Phys. Rev. Research {\bf 2}, 023372 (2020)}

\end{enumerate}


%% ----------------------------------------------------------------
\mainmatter	  % Begin normal, numeric (1,2,3...) page numbering
\pagestyle{fancy}  % Return the page headers back to the "fancy" style

% Include the chapters of the thesis, as separate files

\addtocontents{toc}{\vspace{0.6em}}

\addcontentsline{toc}{chapter}{Introduction}
%!TEX root = ../Thesis.tex

% \chapter*{Introduction} % Write in your own chapter title
\label{Introduction}
\lhead{\emph{Introduction}} % Write in your own chapter title to set the page header

Devices that control the flow of energy or matter play a prominent role in technology. A key device is the rectifier, which allows currents only one way. A rectifier behaves like a corridor with a trap door that can be opened from left to right but is closed otherwise. The most notable of such devices is the electric diode, which is a vital part of computers, digital devices, and AC/DC current conversion systems. Without the diode most of the technology that we have today would not exist.

The electric diode is an electrical component that allows electrical current to flow asymmetrically with respect to the sign of the potential difference that is applied to it. Typically, a diode is composed by the union of a $p$-semiconductor with an $n$-semiconductor. When a forward-bias potential $\Delta V$ is applied to the $p$-$n$ junction (by connecting the positive pole of a battery to the $p$-semiconductor), electrical current will flow through the diode. However, the $p$-$n$ junction acts as an electrical insulator if a reversed-bias potential $-\Delta V$ is applied.

Motivated by the technological impact of the diode, analogous devices have been developed in other physical scenarios, like optics. An optical equivalent to the diode is the optical isolator, which is used to allow one-way light propagation \cite{Saleh1991}. This device is based on the non-reciprocal rotation of the polarization direction of polarized light in materials that are in a magnetic field, known as Faraday Rotation (see ref. \cite{Yariv1984}). The optical isolator is a critical component in optical devices to protect delicate light sources from back-propagating light.

At this point we can see that a common ingredient between devices which show a diode-like behaviour is some kind of internal structural asymmetry. In the electric diode this asymmetry comes from the asymmetric distribution of charge carriers: electrons in the $n$-side, and holes in the $p$-side. In the optical isolator the orientation of the magnetic field breaks the symmetry of the system.

This Thesis is devoted to explore the physics and possible designs of devices that implement a \textit{diodic} or rectifying mechanism for an asymmetric transport of matter or energy. The Thesis
is divided into two parts: In part \ref{partI}, I look for asymmetric particle scattering of 1-dimensional quantum potentials and in part \ref{partII}, I will study thermal rectification in chains of oscillators. There follows an introduction to these two parts.


\section*{Introduction to part I: non-Hermitian systems and asymmetric scattering}

The current interest to develop new quantum technologies is boosting applied
and fundamental research on quantum phenomena and systems with potential
applications in logic circuits, metrology, communications or sensors. Robust basic devices performing elementary operations are needed to perform complex tasks when combined in a circuit. With the development of new quantum technologies in mind, the objective of this part is to design 1-dimensional scattering potentials for a quantum, spinless particle of mass $m$ that lead to transmission and reflection coefficients (squared modulus of the amplitudes) which differ for wave packets coming from the left or the right.

To find an asymmetric scattering behavior, I will use non-Hermitian and non-local potentials \cite{Muga2004,Mostafazadeh2018}. Although non-local and non-Hermitian potentials might seem uncommom and extraordinary in quantum physics to some, they appear naturally when applying partitioning techniques to describe the effective interactions in a subspace of a larger system with a Hermitian Hamiltonian by projection \cite{Feshbach1958,Ruschhaupt2004,Muga2004}. Non-Hermitian Hamiltonians representing effective interactions have a long history in nuclear, atomic, and molecular physics, and have become common in optics, where wave equations in waveguides could simulate the Schr\"odinger equation \cite{Ruschhaupt2005,Longhi2017a,Konotop2016}. Non-Hermitian Hamiltonians can also be set phenomenologically, e.g. to describe dissipation \cite{Ruschhaupt2005}. Recently there has been a lot of interest in non-Hermitian Hamiltonians \cite{Nixon2016,Nixon2016a,Chen2017,Ruschhaupt2017,Simon2018,Simon2019a,Alana2020,Bernard2002,Kawabata2019}, in particular, the ones having parity-time (PT) symmetry \cite{Bender1998,Znojil2015} because of their spectral properties and useful applications, mostly in optics  \cite{Longhi2017a,Konotop2016,Longhi2014}. However, I shall emphasize that symmetries different from PT exist and are necessary to produce certain forms of asymmetric scattering.

The contents of this part of the Thesis will be organized as follows. In chapter \ref{Chapter1}, I will use non-Hermitian and non-local potentials to design potentials with asymmetric scattering coefficients for left/right incidence. Symmetries for non-Hermitian Hamiltonians will be generalized using the concept of pseudohermiticity \cite{Mostafazadeh2002} and used to derive useful selection rules for the transmission and reflection coefficients. In chapter \ref{Chapter2}, I will derive a set of properties of the eigenvalues of scattering potentials that extend previous results for discrete non-Hermitian Hamiltonians by using the generalized symmetries. In chapter \ref{Chapter3}, I will present a possible physical realization for asymmetric scattering Hamiltonians in a quantum optics setup.


\section*{Introduction to part II: Heat rectification in mesoscopic systems}

Radiation, heat and electricity are prominent mechanisms of energy transport. In particular, the two last mechanisms play a dominant role in technology. Modern information processing rests on electronic devices like the diode and the transistor. However, there is not an analogous technology to control heat currents driven by phonons. An explanation could be that phonons are more difficult to control than electrons since (contrary to them) they do not have mass or electrical charge \cite{Li2012}. However, it would be interesting to explore the design of \textit{phononic} devices due to the richness of different physical mechanisms that mediate heat transport. The thermal rectifier, or thermal diode, would be a primary building block to develop \textit{phononics} \cite{Li2012}. In this part I study thermal rectification in chains of oscillators with the design of a thermal diode in mind.

Thermal rectification is the physical phenomenon, analogous to electrical current rectification in diodes, in which heat current through a device or medium (the thermal diode or rectifier) is not symmetric with respect to the exchange of the bath temperatures at the boundaries. It was  first observed in 1936 by Starr in a junction between copper and cuprous oxide \cite{Starr1936}. The theoretical work started much later using as rectifiers simple anharmonic chain models
with different segments \cite{Terraneo2002,Li2004}. These papers sparked much research that continues to this day. Research on thermal rectification has gained a lot of attention in recent years as a key ingredient to build prospective devices to control heat flows similarly to electrical currents \cite{Roberts2011,Li2012}. There are  proposals to engineer thermal logic circuits \cite{Ye2017} in which information, stored in thermal memories \cite{Wang2008}, would be processed in thermal gates \cite{Wang2007}. Such thermal gates, as their electronic counterparts,  would require thermal diodes and thermal transistors to operate \cite{Li2006,Joulain2016}.
Heat rectifying devices would also be quite useful in nanoelectronic circuits, letting delicate components dissipate heat while being protected from external heat sources \cite{Roberts2011}.

Most work on thermal diodes has been theoretical with only a few experiments (see refs. \cite{Chang2006,Kobayashi2009,Leitner2013,Elzouka2017}).
A relevant attempt to build a thermal rectifier was based on a graded structure made of carbon and boron nitride nanotubes that transports heat between a pair of heating/sensing circuits \cite{Chang2006}. One of the ends of the nanotube is covered with a deposition of another material, which makes the heat flow better from the covered end to the uncovered end. However, rectifications were small, with rectification factors around $7\%$.

Much of the theoretical effort in thermal rectification research has been aimed at improving the rectification factors and the features of the rectifiers. The first approach to designing thermal diodes consisted in using chains of oscillators segmented into two or more regions with different properties \cite{Terraneo2002,Li2004,Li2008,Hu2006}, which is reminiscent of the idea of the $p$-$n$ junction in electric diodes. However, it was soon noticed that the performance of segmented rectifiers was very sensitive to the size of the device, \textit{i.e.}, rectification decreases when increasing the length of the rectifier \cite{Hu2006}. To overcome this limitation two ideas were proposed. The first one consisted in using graded rather than segmented chains, \textit{i.e.}, chains where some physical property varies continuously along the site position such as the mass of particles in the chain \cite{Wang2012,Chen2015,Romero-Bastida2017,Yang2007,Romero-Bastida2013,Dettori2016,Pereira2010,Pereira2011,Avila2013}. The second one consisted in using chains with long-range interactions (LRI), such that all the elements in the chain interact with all the rest \cite{Chen2015,Bagchi2017,Pereira2013}. The rationale behind these proposals was that in a graded system, new asymmetric rectifying channels are created, while the long-range interactions create
also new transport channels, avoiding the usual decay of heat flow with size \cite{Chen2015}. Besides a stronger rectification power, LRI graded chains are expected to have better heat conductivity than segmented ones. This is an important point for technological applications, because devices with high rectification factors are not useful if the currents that flow through them are very small.

Another main focus of the theoretical research in thermal rectification is the search for the fundamental factors that contribute to the emergence of rectification. Historically, the crucial elements for having rectification have been the presence of some structural asymmetry in the system and of non-linear (anharmonic) forces \cite{Zeng2008,Katz2016,Li2008,Hu2006,Benenti2016,Li2012,Segal2005,Segal2005b}, which lead to a temperature dependence of the phonon bands or power spectral densities. A match or mismatch of the phonon bands of neighboring parts of the chain implies corresponding good or bad conduction so the
sign of the temperature bias may affect the conduction and lead to rectification when the spectra of the parts are affected differently by the bias reversal. However, more recent research pointed out that anharmonicity is not a necessary condition for an asymmetric match/mismatch and therefore for rectification \cite{Pereira2017}. Rectification also occurs in simple (minimalistic) harmonic models that incorporate some structural asymmetry and temperature-dependence of the model parameters \cite{Pereira2017}. This dependence may indeed result from an underlying, more intricate  anharmonic system by linearization of the stochastic dynamics \cite{Pereira2017,Pereira2019}, or it may have a different origin \cite{Simon2019}.


The contents of this part of the Thesis will be organized as follows. In chapter \ref{Chapter4}, I present a model of a thermal rectifier that relies on a localized impurity in the middle of a chain of atoms. In chapter \ref{Chapter5}, a proposal for a thermal rectifer in a chain of trapped ions with a graded frequency distribution is presented. Finally, in chapter \ref{Chapter6}, I study heat transport in a solvable model of two connected oscillators to explore the origin an optimization of thermal rectification.
 %Introduction

%!TEX root = ../Thesis.tex
%Chapter 1

\chapter{Local rectification of heat flux}
\label{ChapterLocalRectification}
\lhead{Chapter X \emph{Local rectification of heat flux}} % Write in your own chapter title to set the page header
%
We present a chain-of-atoms model where heat is rectified, with different fluxes from the hot to the cold baths
located at the chain boundaries when the temperature bias is reversed. The chain is homogeneous except for boundary effects
and a local modification of the interactions
at one site, the ``impurity''. The rectification mechanism is due here to the localized impurity, the only asymmetrical element of the structure, apart from the externally imposed temperature bias,  and does not rely on putting in contact different materials or other known mechanisms such as grading or long-range interactions.  The effect survives if all interaction forces are linear except the ones for the impurity.
%
\newpage
%

\section{Introduction}

In spite of much work on thermal rectification
after the first model proposal in 2002 \cite{Terraneo2002} (for a broad perspective on heat rectification
see \cite{Roberts2011,Li2012}), the manipulation of phononic
heat fluxes is still far from being completely controlled as no efficient and feasible thermal diodes have been found \cite{Chen2015,Pereira2017}.
The thermal rectifier, a device where the heat
current changes when the temperature bias of the thermal baths at the boundaries is reversed,
is one of the key tools needed to manipulate heat currents and build thermal circuits.
A wealth of research is underway to meet the challenge posed by  a ``near standstill'' of the field \cite{Chen2015,Pereira2017},
combined with the prospects of widespread and impactful practical applications.
Together with experimental progress, at this stage work exploring new models is important to test possibilities that may become feasible as control capabilities improve \cite{Roberts2011}.
In this paper we propose,
motivated by previous work on ``atom diodes"  \cite{Ruschhaupt2004,Raizen2005}, a rectifying scheme
based on the effect of a local defect, or impurity,
in an otherwise homogeneous system.

To date, there have been several proposals of systems that could be used to rectify heat flows at the nanoscale.
A common scheme is based on coupling two or more different homogeneous segments, modelled with
chains of atoms with nonlinear interactions (which in this context means non-linear forces, i.e., anharmonic potentials) \cite{Terraneo2002,Li2004,Hu2006,Peyrard2006,Benenti2016}
or with a temperature and position dependent
conductivity assuming expressions for the heat current \cite{Peyrard2006,Hu2006a}.
%Other proposals \cite{Hu2006} make use of a two-segment Frenkel Kontorova model \cite {Li2004} finding size effects such us the reversal of the rectification as the size of the system increases, depending on the strength of the coupling between both segments.
Basic ingredients for thermal rectification have been considered to be  the asymmetry in the system and non-linear interactions
\cite{Zeng2008,Li2012,Benenti2016}, which lead to a temperature dependence of the phonon bands, or power spectral densities,  of the
weakly coupled \cite{Hu2006} segments. These bands match or mismatch at the interfaces, depending of the sign of the temperature bias, leading, respectively, to heat flow or insulating conditions. In fact, alternative mechanisms due to band mixing appear for stronger coupling
or long chains \cite{Hu2006}, and Pereira \cite{Pereira2017},  based on minimalist models,
has recently reformulated the conditions leading to rectification  as the combination of asymmetry plus the existence of some feature of the system that depends on the temperature (nonlinearity is certainly a possible cause of such dependence).
Other systems proposed are graded materials, such as a chain with an uneven distribution of mass \cite{Chang2006,Zeng2008,Chen2015}, and long-range interactions  have been shown to be able to amplify the rectification and avoid or mitigate the decay of the effect with system size \cite{Pereira2013,Chen2015}.
%Mass graded materials can be built, but up to now the rectification experimentally observed is small \cite{Chang2006}.
Also, recent models and experiments use asymmetrical homogeneous or inhomogeneous nanostructures and, in particular, graphene  \cite{Wang2014,Wang2017}.
Finally, we mention for completeness theoretical works
%farther from our model
that consider the use of a quantum system \cite{Roberts2011}, such as a three-level system, with each level coupled to a thermal bath \cite{Joulain2016}, or a double well with different frequencies to implement the asymmetry \cite{Katz2016}.

The model proposed in this paper is a one-dimensional chain of atoms where all, except one of them, are trapped in on-site harmonic potentials and interact with their nearest neighbours by Morse potentials (or also by harmonic
potentials in a simplified version). Unlike most chain models, the structural asymmetry is, except for the boundaries,
only in the impurity, which is subjected to a different on-site potential and interaction with one of its neighbors. The chain is connected to thermal baths at different temperatures at the boundaries.
Let us clarify our terminology:
The structural asymmetry we have in mind amounts,
to different interaction potentials of one or several atoms in the
chain (other than the boundary atoms) with neighbors on the right and on the left, and/or asymmetric onsite potentials. In this regard  chains with Morse interatomic potentials per se are not considered structurally asymmetric here as long as their parameters are the same on both sides of a given atom, even if a single Morse potential is obviously asymmetric with respect to its minimum \cite{Wang2015}.
%each end, and we have chosen the well known Nos\'e-Hoover model to simulate the thermal baths \cite{Martyna1992}.
%The manipulation of the potentials that affect one of the atoms makes the excitations easier to be transferred from one side, providing a mechanism for the inhomogeneity needed.

First, we shall describe the
%initial system used, that consists of a
homogeneous 1D chain, without the impurity,  %For this system, we numerically solve the dynamical equations, to show that the usual heat conduction applies. Once the system is connected to the baths at different temperature, after a transient time, the steady regime with constant value of the heat flux, $J$, is reached, and
and show its heat conduction behavior.
%Our goal is to obtain a system that presents a direction dependent heat flux when the positions of the baths are interchanged. With this aim,
Then  we modify the potentials for one of the atoms and demonstrate the rectification effect.


\section{Homogeneous one-dimensional chain}

We start with a homogeneous $1D$ chain with $N$ atoms coupled at both extremes to heat baths, at different temperatures $T_h$ and $T_c$ for ``hot" and ``cold" respectively. The baths are modeled with a Nos\' e-Hoover method as described in \cite{Martyna1992}.
Atoms $1$ and $N$ represent the first and the $N$-th atom in the chain, from left to right,
that will be in contact with the baths. All the atoms are subjected to on-site potentials and to nearest-neighbor interactions, and their equilibrium positions $y_{i0}$ are assumed to be equally spaced by a distance $a$.
$x_i= y_i-y_{i0}$,
$i=1,...,N$, represent the displacements from the equilibrium positions of the corresponding atoms
with positions $y_i$.

The classical Hamiltonian of the atom chain can be written in a general form as
%
\begin{equation}
\label{GH}
%GH=general Hamiltonian
H=\sum_{i=1}^{N} H_i,
\end{equation}
%
with
%
\begin{eqnarray}
\label{GH2}
%GH=general Hamiltonian
H_1&=&{{p^2_1} \over {2m}} +U_1(x_1)+V_L,
\nonumber\\
H_i&=&{{p^2_i} \over {2m}} +U_i(x_i)+V_i(x_{i-1},x_i)  \quad i=2,...,N-1,
 \nonumber\\
H_N&=&{{p^2_N} \over {2m}} +U_N(x_N)+V_N(x_{N-1},x_N) + V_R,
\end{eqnarray}
%
where the $p_i$ are the momenta, $U_i(x_i)$ is the on-site potential for the $i$th atom, and $V_i(x_{i-1},x_i)$ represents the atom-atom interaction potential. $V_R$ and $V_L$ are the interactions coupling the boundary atoms to the Nos\'e-Hoover thermostats, see \cite{Martyna1992}.

There are a large number of $1D$ models that obey this general Hamiltonian. Using different potentials for the trapping and interactions we would get different conductivity behaviors.
%%%%%%%%%%%%%%%%%%%%%%%
\begin{figure}
\centering
\includegraphics[width=8.8cm]{Figures/FIG1.pdf}
\caption{(Color online) (a) On-site potentials: harmonic potential centered at the equilibrium position of each atom (dashed blue line) as a function of the displacement from this position $x_i=y_i-y_{i0}$ in $a-$units, and the on-site potential for the impurity, $i=N/2+1$
($N$ even, red solid line). (b) Interaction potentials as a function of the distance between nearest neighbors: Morse potential
(blue dashed line) valid for all atoms except for $i=N/2+1$, $N$ even, where the modified potential (red solid line) is used.
The harmonic approximation of the Morse potential is also depicted (eq. (\ref{Vhar}), black dots, only used for fig. 5, below).
Parameters: $D=0.5$, $g=1$, $\gamma = 45$, $d=100$ and $b=105$, used throughout the paper.
}
\label{figure1}
\end{figure}
%%%%%%%%%%%%%%%%%%%%%%%%%
%
We choose a simple form of the Hamiltonian in
%(\ref{IH})
which each atom is subjected to a harmonic on-site potential and a Morse interaction potential between nearest neighbors (see fig. \ref{figure1}, dashed lines),
%
\begin{eqnarray}
\label{HO}
%HO=Harmonic oscillator
U_i(x_i)&=&{1 \over 2} m \omega^2 x^2_i,
%\end{equation}
%\begin{equation}
\\
\label{IH}
%IP=Interaction potential
V_i(x_{i-1},x_i)&=&D\left \{e^{-\alpha [x_i-x_{i-1}]}-1\right \}^2,
\end{eqnarray}
%
where $\omega$ is the trapping angular frequency, and $D$ and $\alpha$ are time independent parameters of the Morse potential.
A ``minimalist version'' of the model where $V$ becomes the harmonic limit of eq. (\ref{IH}), dotted line in fig. 1,
 will also be considered in the final discussion,
%
\begin{equation}
\label{Vhar}
{V}_i(x_{i-1},x_i)=k(x_i-x_{i-1})^2/2,\;k=2D\alpha^2.
\end{equation}
%
For convenience, dimensionless units are used and the mass of all particles is set to unity.
%%%%%%%%%%%%%%%%%%%%%%%%%%%%
\begin{figure}
\centering
\includegraphics[width=8.8cm]{Figures/{FIG2.pdf}}
\caption{(Color online) Symmetric temperature profiles for a homogeneous chain, without impurity.  For $T_{h}=T_{L}$, $T_c=T_R$ (red solid dots) the (absolute value of) the heat flux is $J_{L\rightarrow R}$, equal to $J_{R\rightarrow L}$ for the reverse configuration of the bath temperatures, $T_{h}=T_{R}$, $T_c=T_L$
(black empty squares). Parameters as in fig. 1.}
\label{figure2}
\end{figure}
%%%%%%%%%%%%%%%%%%%%%%%%%%%%%%

We start by studying the homogeneous configuration with no impurity and potentials (\ref{HO}) and (\ref{IH}), solving numerically the dynamical equations for  the Hamiltonian (\ref{GH}) with a Runge-Kutta-Fehlberg algorithm. We have chosen a low number of atoms, $N=20$,  with thermal baths at $T_h=0.20$ and $T_c=0.15$ at both ends of the chain with 16 thermostats each. The real temperature is related to the dimensionless one through $T_{real}=T m a^2 \omega^2/k_B$ so, for typical values  $m\approx10^{-26}$ kg, $\omega \approx 10^{13}$ s$^{-1}$, $a\approx 10^{-10}$ m, and using $k_B =1.38 \times 10^{-23}$ JK$^{-1}$,
the dimensionless temperatures $0.15,\, 0.20$, translate into $100,\, 150$ K. It is advisable to use temperatures around these values so that we ensure that the displacements of the particles are realistic \cite{Casati1984}.


First we demonstrate the conductivity behavior of the model.
%that our system satisfies Fourier's heat law for the heat flux, $J=\kappa \nabla T$.
%, so it shows normal thermal conductivity. ESTE CONCEPTO TIENE QUE VER CON EL TAMA�O?
To this end, we calculate the local heat flux $J_i$ and temperature $T_i$, performing the numerical integration
%of Eq. (\ref{GH2})
for long enough times to reach the stationary state.
The local temperature is found as the time average $T_i= \langle p_i^2 / m \rangle$, whereas
%After a transient, the local temperature is given by the time average $T_i=\langle p_i^2\rangle$.
$J_i$,  from the continuity equation
%, $\dot H(x,t)+divJ(x,t)=0,$
\cite{Hu1998}, is given by
%
%Fourier law: temperature gradient vanishes with N
\begin{equation}
\label{heatflux}
J_i=-\dot x_i {{\partial V(x_{i-1},x_{i})} \over {\partial x_i}}.
\end{equation}
%
From now on we only consider the time average $\langle J_i (t)\rangle$, which converges to a constant value for all sites once the system is in the stationary nonequilibrium state. We depict the temperature profiles, for $N=20$, first with $T_L=T_h$ and $T_R=T_c$
($L$ and $R$ stand for left and right) and after switching the positions of the thermal baths in fig. \ref{figure2}. The profiles are symmetric, as expected, and the heat flux does not have a preferred direction  \cite{Hu1998,Terraneo2002}. Denoting the absolute values of the fluxes from the left (when $T_L=T_h$) as
$J_{L\rightarrow R}$, and from the right (when $T_R=T_h$) as
$J_{R\rightarrow L}$, we find that $J_{L\rightarrow R}=J_{R\rightarrow L}=J=1.6\times 10^{-2}$, in the dimensionless units, consistent with the values found in other models \cite{Terraneo2002,Hu1998}.
%%%%%%%%%%%%%%%%%%%%%%%%%%%%%%%%%%
\begin{figure}
\centering
\includegraphics[width=8.8cm]{Figures/{FIG3.pdf}}
\caption{(Color online) Temperature profile along the homogeneous chain for different number of atoms: 100 (dotted black line), 125 (dashed blue line) and 150 (solid red line). The atom sites have been rescaled with the total number of atoms.
%, showing the convergence of the spatial profile of the local temperature $T_i$.
The time averages have been carried over a time interval of $\approx 2 \times 10^6$ after a transient of $\approx 1\times 10^5$. In the inset (a), the product $JN$ vs. $N$ demonstrates that for long chains $JN$ is independent of $N$. In (b) the linear dependence of $J$ with $\Delta T$ for a fixed number of atoms, $N=100$, is shown. Parameters as in fig. 1.}
\label{figure3}
\end{figure}
%%%%%%%%%%%%%%%%%%%%%%%%%%%%%%%%%%%%%%%%

The profile of the temperature is linear with boundary nonlinearities at the edges, close to the thermal baths,  due to the boundary conditions \cite{Lepri1997}. In fig. \ref{figure3} we depict $T_i$ vs $i/N$ for $N=100, 125$ and $150$ with the same boundary conditions. For these
larger atom numbers  we have connected the first 3 and the last 3 atoms to the Nos\'e-Hoover baths.
%The temperature gradient scales as $N^{-1}$, which is also true for many other different models \cite{Hu1998}.
In the inset (a) of fig. \ref{figure3}  the product $JN$ is plotted vs. $N$ showing that in a low $N$ limit there is a well defined conductivity per unit length whereas for longer chains, $JN$ tends to be constant  which indicates a normal thermal conductivity independent of the length. Fixing the number of atoms to 100, as in the inset (b) of fig. \ref{figure3},  we observe a linear dependence between the flux and $\Delta T$.
%Fourier law, $J=\kappa \nabla T$, is fulfilled.

\section{Impurity-based thermal rectifier}

To rectify the heat flux
we modify the potentials for site $j=N/2+1$ with $N$ even, as
%
%\begin{equation}
%\label{IMP1}
%IMP1=impurity in absolute position
%U_j(y_j,t)=d e^{-b [y_j(t)-y_{d}]} +ge^{-\gamma [y_j(t)-y_{j-1}(t)-\epsilon]}
%\end{equation}
%with $y_{d}=y_{d,0}-a/3$.  Written in terms of the displacements, $x_j$,
%
\begin{eqnarray}
\label{IMP}
%IMP=impurity
U_j(x_j,t)&=&d e^{-b [x_j(t)+a/3]},
\\
V_j(x_{j-1},x_j,t)&=&ge^{-\gamma [x_j(t)-x_{j-1}(t)+a/2]}.
\end{eqnarray}
%
All the parameters involved, $d, b$, and $g,\gamma$ are time independent. In fig. \ref{figure1} the modifications introduced with respect to the ordinary sites are shown (solid lines).  The different on-site and interaction terms introduce soft-wall potentials
(instead of hard-walls to aid integrating the dynamical equations) that make it difficult for the impurity to transmit its excitation to the left whereas left-to-right transmission is still possible.
This effect is facilitated by the relative size of the coefficients, $a/3<a/2$, that determine the position of the walls.
% that we fixed after some experimentation.
These positions imply that an impurity excited by a hot right bath cannot affect its left cold neighbour near its equilibrium position at the $j-1$ site.
However, if the left $j-1$ atom is excited from a hot bath on the left,
it can get close enough to the impurity to kick it and transfer kinetic energy.
The asymmetrical behavior relies on the asymmetry of the potentials and the temperatures of
neighboring atoms; it does not require
breaking time-reversal invariance. All collisions are elastic and time reversible.

After extensive numerical simulations, we have chosen the values of these parameters as in fig. 1, such that the conductivity in the forward direction, $J_{L\rightarrow R}$, and the rectification factor, defined as $R=(J_{L\rightarrow R}-J_{R\rightarrow L}) / J_{R\rightarrow L}\times 100$,
are both large for $T_h=0.2$, $T_c=0.15$. A large $R$ without a large $J_{L\rightarrow R}$ could in fact be useless \cite{Roberts2011}.
%($R=0$ would represent a perfectly symmetric heat conduction.).
Note that the parameters are not necessarily the optimal combination, which in any case would depend on the exact definition of ``optimal'' (technically on how $J_{L\rightarrow R}/J$ and $R$ are weighted and combined in a cost function and on the limits imposed on the
parameter values). This definition is an interesting question but it goes beyond the focus of our paper,
which is to demonstrate and discuss the effect of the localized inpurity.

We have used again $N=20$ connected to baths of 16 thermostats each, with the same temperatures as for the homogeneous chain, and numerically solved the dynamical equations
to calculate the local temperature and the heat flux for both configurations of the baths. The interatomic potential for the regular atoms is the Morse potential (\ref{IH}).
In fig. \ref{figure4}(a), the temperature profiles show a clear asymmetry between ${L\rightarrow R}$ and ${R\rightarrow L}$. Specifically, we find $J_{L\rightarrow R}=7.6 \times 10^{-3}$ and $J_{R\rightarrow L}=5.8 \times 10^{-3}$ which gives
$R=31\%$. The effect decays with longer chains,  with, for example, $R=19\%$ for $N=100$, and R=17.8\% for $N=150$.

These temperature profiles depend on the difference between the bath temperatures, see e.g. fig. \ref{figure4}(b). Increasing the temperature gap, but  keeping $T_h$ low enough so that the displacement of the atoms from their equilibrium positions is realistic, we find higher values of $R$. Figure \ref {figure5} shows the strong dependence of $R$ with $\Delta T$ (black circles).
%, for the same values of the potential parameters as in fig. \ref{figure1}.
We have changed both $T_h$ and $T_c$ so that the mean temperature $(T_c+T_h)/2$ remains constant.




%Future... 100 atoms, R=0.82, worse.
%Possibility of including long range interactions? more atoms manipulated?

%%%%%%%%%%%%%%%%%%%%%%%%%%%%%%%%%%%
\begin{figure}
\centering
\includegraphics[width=8.8cm]{Figures/{FIG4b.pdf}}
\caption{(Color online) Temperature profile for the chain of $N=20$ atoms, with an impurity in the $N/2+1$ position, with $T_L=T_h$ and $T_R=T_c$ (circles) and with the thermostat baths switched (squares).
%, for $g=1, \gamma = 45, d=100$ and $b=105.$ The rest of parameters are the same as in fig. 3 where the profile of the temperature with no impurity was shown. The difference in the temperature profiles can be clearly noticed, also confirmed by the heat fluxes
Parameters as in fig. 1.
(a) $T_c=0.15$, $T_h=0.2$. $J_{L\rightarrow R}=0.00769$ vs $J_{R\rightarrow L}=0.00581$, with gives a rectification $R=31 \% $; (b) $T_c=0.025$, $T_h=0.325$. $J_{L\rightarrow R}=0.0499$ vs  $J_{R\rightarrow L}=0.0140$, with $R=256 \%$.}
\label{figure4}
\end{figure}
%%%%%%%%%%%%%%%%%%%%%%%%%%%%%%%%%%%%%%%%%



\begin{figure}
\centering
\includegraphics[width=8.8cm]{Figures/{FIG5new.pdf}}
\caption{(Color online) Rectification factor $R$ as a function of the temperature difference between ends of the chain of atoms, $\Delta T$.
%The rectification factor shows a very strong dependency on $\Delta T$.
We have changed both $T_h$ and $T_c$ according to $T_c=0.15-(\Delta T-0.05)/2$ and $T_h=0.2+(\Delta T-0.05)/2$, with $N=20$,  keeping the rest of parameters as in fig. \ref{figure1}.
Interatomic potentials: Morse potential, eq. (\ref{IH}) (black line with circles, see the temperature profiles of extreme points in fig. 4); harmonic potential, eq. (\ref{Vhar}) (red line with squares).}
\label{figure5}
\end{figure}


\section{Discussion}

We have presented  a scheme for thermal rectification using a one-dimensional chain of atoms which is homogeneous except
for the special interactions of one of them, the impurity, and the couplings with the baths at the boundaries.
% in such a way
%that the heat flux is different when the temperature bias of the baths at chain boundaries is reversed.
Our  proof-of-principle results for an impurity-based rectification mechanism encourage further exploration of the
impurity-based rectification, in particular
of the effect of different forms for the impurity on-site potential and its interactions with neighboring atoms.
In contrast to the majority of chain models, the structural asymmetry in our model
is only in the impurity. The idea of a localized effect was already implicit in early works on a two-segment Frenkel-Kontorova
model \cite{Li2004,Hu2006}, where rectification depended crucially on the interaction constant coupling between the two segments.
However, the coupling interaction was symmetrical and the asymmetry was provided by the different nature
(parameters) of the segments put in contact.
Also different from common chain models are the potentials chosen here. Instead of using the Morse
potential as an on-site model, see e.g.  \cite{Terraneo2002},
we have considered a natural setting where this potential characterizes the interatomic interactions,
and the on-site potential is symmetrical with respect to the equilibrium position, and actually harmonic.
The numerical results indicate that this model is consistent with normal conduction,
and also helps to isolate and identify the local-impurity mechanism for rectification.
In this regard it is useful to consider a further simplification, in the spirit of the minimalists models
proposed by Pereira \cite{Pereira2017}, so as to distill further the essence of the local rectification mechanism.
If the Morse interatomic interaction is substituted by the corresponding harmonic interaction, see the black dotted line in fig. 1b, the rectification effect remains, albeit slightly reduced, see fig. \ref{figure5}. The chain is then perfectly linear with the only nonlinear exception  localized
at the impurity.
The temperature dependent feature mentioned in \cite{Pereira2017} as the second necessary condition for rectification besides asymmetry, is here localized in the impurity too, and consists of a different
capability to transfer kinetic energy depending on the temperatures on both sides of the impurity.
Figure 6 shows temperature profiles for the purely harmonic chain to be compared with the Morse-interaction
chain in fig. 4. Flatter profiles are found on both sides of the impurity,
as corresponds to the abnormal transport expected for harmonic chains \cite{Lepri2003}.
It would be interesting to combine the impurity effect with other rectification mechanisms (such as grading, long-range interactions, or use of different segments), or with more impurities in series to enhance further the rectification effect.


\begin{figure}
\centering
\includegraphics[width=8.8cm]{Figures/FIG6.pdf}
\caption{(Color online) Temperature profile for a harmonic interacting chain of $N=20$ atoms, with an impurity in the $N/2+1$ position, with $T_L=T_h$ and $T_R=T_c$ (circles) and with the thermostat baths switched (squares), for (a) $\Delta T = 0.05$ and (b)  $\Delta T = 0.3$. The corresponding rectification factors are (a) $R=18\%$ and (b) $R=85\%$. Parameters regarding the impurity are the same as in fig. 1.
}
\label{figure6}
\end{figure}

Even though our motivation was to mimic the effect of a localized atom diode that lets atoms pass only one way,
unlike the atom diode \cite{Ruschhaupt2004}, all interactions in the present model
are elastic. The model may be extended by adding an irreversible,  dissipative element so as to induce not only rectification but a truly Maxwell demon for heat transfer \cite{Skordos1992,Ruschhaupt2006}.
On the experimental side, one dimensional chains of neutral atoms in optical lattices can be implemented with cold atoms \cite{Bloch2005}.
An impurity with different internal structure could be subjected to a different on-site potential imprinted by a holographic mask \cite{Bakr2009}, and asymmetrical interatomic interactions
could be implemented by trapping a controllable polar molecule or mediated by atoms in parallel lattices \cite{Gollub2014}.


We are  indebted to G. Casati for raising our attention to thermal rectification and for providing information on his work. We acknowledge financial support by the
Basque Government (Grant No.  IT986-16) and MINECO/FEDER,UE (Grant No. FIS2015-67161-P).
 %Rectification chain of Ions
\input{./Chapters/RectificationChainOfIons} %Rectification chain of Ions
%!TEX root = ../Thesis.tex
%Chapter 1

\chapter{Rectification in a toy model}
\label{ChapterToyModel}
\lhead{Chapter Toy Model. \emph{blah blah}} % Write in your own chapter title to set the page header
%
We study heat rectification in a minimalistic model composed of two masses subjected to on-site and coupling
linear forces in contact with effective Langevin baths induced by laser interactions.  Analytic expressions of the heat currents in the steady state are spelled out.  Asymmetric heat transport is found in this linear system if both the bath temperatures and the temperature dependent bath-system couplings
are also exchanged.
%
\newpage
%
\section{Introduction \label{sec:Introduction}}
%
Heat rectification, firstly observed in 1936 by Starr \cite{Starr1936}, is the physical phenomenon, analogous to electrical current rectification in diodes, in which heat current through a device or medium is not symmetric with respect to the exchange of the baths at the boundaries. In the limiting case the device allows heat to propagate in one direction from the hot to the cold bath while it behaves as a thermal insulator in the opposite direction when the baths are exchanged.  In 2002 a paper by Terraneo \textit{et al.} \cite{Terraneo2002} demonstrated heat rectification numerically for a chain of nonlinear oscillators in contact with two thermal baths at different temperatures. Since then, there has been a growing interest in heat rectification  \cite{Pereira2019,Roberts2011,Li2012,Ye2017,Wang2008,Wang2007,Casati2006,Joulain2016,Chang2006,Kobayashi2009,Leitner2013,Elzouka2017,Pons2017,Alexander2020}, and the field remains very active because of the potential applications in fundamental science and technology, and the
fact that none of the proposals so far appears to be efficient and robust for
practical purposes.

Much effort has been devoted to  understand the underlying physical mechanism responsible for  rectification \cite{Pereira2019}.
%
In early times some kind of anharmonicity,  i.e. non-linear forces, in the substrate potential or in the particle-particle interactions, was identified as a fundamental requisite for rectification \cite{Li2012,Li2008,Hu2006,Zeng2008,Katz2016,Benenti2016}. This non-harmonic behavior leads to a temperature dependence of the phonon bands. The match/mismatch of the phonon bands (power spectra) governs the heat transport in the chain, allowing it when the bands match or obstructing it if they mismatch \cite{Terraneo2002,Li2004}. However, a work by Pereira \textit{et al.} \cite{Pereira2017} showed that rectification can also be found in effective harmonic systems if two requirements are met: some kind of structural asymmetry, and features that depend on the temperature so they change as the baths are inverted. Indeed,  in this article we demonstrate rectification in a minimalistic model of two harmonic oscillators where the coupling to the baths depends on the temperature.
%
This will be justified with a particular physical set up with trapped ions and lasers.

%

The article is organized as follows. In Section \ref{sec:Physical_Model}
we describe the physical model and its dynamical equations. In Section \ref{sec:covMatrix} we describe the dynamics of the system in terms of a covariance matrix. We also derive a set of algebraic equations that gives as solution the covariance matrix in the steady state. In Section \ref{sec:solutions} we solve the covariance matrix equations and find analytical expressions for the steady-state temperatures of the masses and heat currents. In Section \ref{sec:TrappedIonSetUp} we relate the parameters of our model to those in a physical set-up of Doppler cooled trapped ions. In Section \ref{sec:lookingForR} we make a parameter sweep looking for configurations which yield high rectification. We also study the power spectra of the oscillators, which confirm the match/mismatch patterns in cases where there is rectification. In Section \ref{sec:Conclusions} we summarize our results and present our conclusions.

\begin{figure}
  \includegraphics[width=1.1\linewidth]{Figures/model_diagram.pdf}
  \caption{Diagram of the model described in Section \ref{sec:Physical_Model}. Two ions coupled to each other through a spring constant $k$. Each ion is harmonically trapped and connected to a bath characterized by its temperature $T_i$ and its friction coefficient $\gamma_i$. }
  \label{fig:model_diagram}
\end{figure}

\section{Physical Model \label{sec:Physical_Model}}

The physical model consists of two masses $m_1$ and $m_2$ coupled to each other by a harmonic interaction with spring constant $k$ and natural length $x_e$. Each of the masses $m_1$ and $m_2$ are confined by a harmonic potential with spring constants $k_L$, $k_R$ and equilibrium positions $x_L$, $x_R$ respectively (see Fig. \ref{fig:model_diagram}). The Hamiltonian describing this model is
%
\begin{equation}
  H = \frac{p_1^2}{2m_1} + \frac{p_2^2}{2m_2} + V(x_1,x_2),
  \label{eq:HamiltonianOriginalCordinates}
\end{equation}
%
with $V(x_1,x_2)=\frac{k}{2}\left( x_1 - x_2 - x_e \right)^2 + \frac{k_L}{2}\left( x_1 - x_L \right)^2 + \frac{k_R}{2}\left( x_2 - x_R \right)^2$,  where $\{x_i,p_i\}_{i=1,2}$ are the position and momentum of each mass. Switching from the original coordinates $x_i$ to displacements with respect to the equilibrium positions of the system $q_i = x_i - x_i^{eq}$, where $x_i^{eq}$ are the solutions to $\partial_{x_i}V(x_1,x_2)=0$, the Hamiltonian can be written as
%
\begin{align}
  H &= \frac{p_1^2}{2m_1} + \frac{p_2^2}{2m_2} + \frac{k+k_L}{2}q_1^2\nonumber\\ &+ \frac{k+k_R}{2}q_2^2 - k q_1 q_2 + V(x_1^{eq},x_2^{eq}).
  \label{eq:Hamiltonian}
\end{align}
%
This has the form of  the Hamiltonian of a system around a stable equilibrium point
%
\begin{equation}
  H = \frac{1}{2} \overrightarrow{p}^\mathsf{T}\mathbb{M}^{-1}\overrightarrow{p} + \frac{1}{2} \overrightarrow{q}^\mathsf{T}\mathbb{K}\overrightarrow{q},
\label{generic}
\end{equation}
%
where $\overrightarrow{q} = \left(q_1,q_2\right)^\mathsf{T}$, $\overrightarrow{p} = \left(p_1,p_2\right)^\mathsf{T}$, $\mathbb{M} = diag(m_1,m_2)$ is the mass matrix of the system and $\mathbb{K}$ is the Hessian matrix of the potential at the equilibrium point, i.e., $\mathbb{K}_{ij} = \partial^2_{x_i,x_j}V(\overrightarrow{x})\Big|_{\overrightarrow{x} = \overrightarrow{x}^{eq}}$. In this model  $\mathbb{K}_{11} = k + k_L$, $\mathbb{K}_{22} = k + k_R$ and $\mathbb{K}_{12} = \mathbb{K}_{21} = -k$.
We shall see later that
the generic form (\ref{generic}) can be adapted to different physical settings, in particular to
two ions in individual traps, or to two ions in a common trap.

The  masses are in contact with Langevin baths, which will be denoted as $L$ (for left) and $R$ (for right), at temperatures $T_{L}$ and $T_R$ for  the mass $m_1$ and $m_2$ respectively (see Fig. \ref{fig:model_diagram}). The equations of motion of the system, taking into account the Hamiltonian and the Langevin baths are
%
\begin{align}
  \dot{q}_1 &= \frac{p_1}{m_1},\nonumber
  \\
  \dot{q}_2 &= \frac{p_2}{m_2},\nonumber
  \\
  \dot{p}_1 &= -(k+k_L)q_1 + k q_2 -\frac{\gamma_L}{m_1} p_1 + \xi_L(t),\nonumber
  \\
  \dot{p}_2 &= -(k+k_R)q_2 + k q_1 -\frac{\gamma_R}{m_2} p_2 + \xi_R(t),
\end{align}
%
where $\gamma_L$, $\gamma_R$ are the friction coefficients of the baths and $\xi_L(t)$, $\xi_R(t)$ are Gaussian white-noise-like forces. The Gaussian forces have zero mean ($\expval{ \xi_L(t) } = \expval{ \xi_R(t) } = 0 $) and satisfy the correlations $\expval{ \xi_L(t)\xi_R(t') } = 0$, $\expval{ \xi_L(t)\xi_L(t') } = 2D_L\delta(t-t')$, $\expval{ \xi_R(t)\xi_R(t') } = 2D_R\delta(t-t')$. $D_L$ and $D_R$ are the diffusion coefficients, which satisfy the fluctuation-dissipation theorem: $D_L = \gamma_L k_B T_L$, $D_R =\gamma_R k_B T_R$ ($k_B$ is the Boltzmann constant).

It is useful to define the phase-space vector $\overrightarrow{r}(t) = \left( \overrightarrow{q}, \mathbb{M}^{-1}\overrightarrow{p} \right)^\mathsf{T}$ (note that $\overrightarrow{v} = \mathbb{M}^{-1}\overrightarrow{p}$ is just the velocity vector) so the equations of motion for this vector are
%
\begin{equation}
  \dot{\overrightarrow{r}}(t) = \mathbb{A} \, \overrightarrow{r}(t) + \mathbb{L}\overrightarrow{\xi}(t),
  \label{eq:vectorEqOfMotion}
\end{equation}
%
with
%
\begin{align}
  \mathbb{A} &=
  \left(
  \begin{array}{cc}
    \mathbb{0}_{2 \times 2} & \mathbb{1}_{2 \times 2}
    \\
    -\mathbb{M}^{-1}\mathbb{K} & -\mathbb{M}^{-1}\Gamma
  \end{array}
  \right),
  \nonumber
  \\
  \mathbb{L} &=
  \left(
  \begin{array}{c}
    \mathbb{0}_{2\times 2} \\ \mathbb{M}^{-1}
  \end{array}
  \right),
\end{align}
%
and $\overrightarrow{\xi}(t) = \left( \xi_L(t),\xi_R(t) \right)^\mathsf{T}$, $\Gamma = diag(\gamma_L,\gamma_R)$. $\mathbb{0}_{n\times n}$ and $\mathbb{1}_{n\times n}$ are the $n$-th dimensional squared 0 matrix and identity matrix respectively. With the vector notation the correlation of the white-noise forces can be written as
%
\begin{equation}
  \expval{\overrightarrow{\xi}(t)\overrightarrow{\xi}(t')^\mathsf{T}} = 2 \mathbb{D}\delta(t-t'),
\end{equation}
%
with $\mathbb{D} = diag(D_L,D_R)$.
%
%
%
%
%
%
\section{Covariance matrix in the steady state\label{sec:covMatrix}}
%
%
%
%
%
%
We define the covariance matrix of the system as $\mathbb{C}(t) = \expval{\overrightarrow{r}(t)\overrightarrow{r}(t)^\mathsf{T}}$. This matrix is important because the heat transport properties can be extracted from it. In particular, the kinetic temperatures of the masses, $T_1(t)$ and  $T_2(t)$, are
%
\begin{align}
  T_1(t) &= \frac{\expval{ p_1^2(t)}}{m_1 k_B} = \frac{m_1 C_{3,3}(t)}{k_B},
  \nonumber\\
   T_2(t) &= \frac{\expval{ p_2^2(t)}}{m_2 k_B} = \frac{m_2 C_{4,4}(t)}{k_B}.
  \label{eq:Temperature_definition}
\end{align}
%
One approach to find the covariance matrix is to solve Eq. \eqref{eq:vectorEqOfMotion}. However, this requires solving the equations explicitly or simulate them numerically many times to find the covariance matrix for the ensemble of simulated stochastic trajectories. Instead, we proceed by looking for an ordinary differential equation that gives the evolution of the covariance matrix as described in \cite{Sarkka2019,Rieder1967,Casher1971}. Differentiating $\mathbb{C}(t)$ with respect to time and using Eq. \eqref{eq:vectorEqOfMotion} we get
%
\begin{align}
  \frac{d}{dt}\mathbb{C}(t) &=
  \mathbb{A}\mathbb{C}(t) +
  \mathbb{C}(t) \mathbb{A}^\mathsf{T}
  \nonumber\\
  &+
  \mathbb{L}\expval{ \overrightarrow{\xi}(t)\overrightarrow{r}(t)^\mathsf{T}}
  \nonumber\\
  &+
  \expval{ \overrightarrow{r}(t)\overrightarrow{\xi}(t)^\mathsf{T}}\mathbb{L}^\mathsf{T}.
  \label{eq:evolutionOfCovariances}
\end{align}
%
The solution of Eq. \eqref{eq:evolutionOfCovariances} allows us to find the local temperatures of the masses as a function of the bath temperatures (Eq. \eqref{eq:Temperature_definition}) at all times. In particular, we are interested in the covariance matrix in the steady state, i.e., for $t\to \infty$. According to the Novikov Theorem \cite{Novikov1965} we can write down the covariance matrix in the steady state without having to integrate the differential equation. We now show how to get the steady-state covariance matrix.

In the steady state, the covariance matrix is constant ($\frac{d}{dt}\mathbb{C}(t)=0$), therefore it satisfies
%
\begin{align}
  &\mathbb{A}\mathbb{C}^{s.s.} +
  \mathbb{C}^{s.s.} \mathbb{A}^\mathsf{T}=
  \nonumber\\
  &- \mathbb{L}\expval{ \overrightarrow{\xi}\overrightarrow{r}^\mathsf{T}}^{s.s.}
  - \expval{ \overrightarrow{r}\overrightarrow{\xi}^\mathsf{T}}^{s.s.}\mathbb{L}^\mathsf{T},
  \label{eq:SteadyStateEquation_raw}
\end{align}
%
with $\{\cdot\}^{s.s.}\equiv \lim\limits_{t \to \infty} \{\cdot\}(t)$. Equation \eqref{eq:SteadyStateEquation_raw} is an algebraic equation whose solution is the steady-state covariance matrix $\mathbb{C}^{s.s.}$. However, the two terms $\expval{ \overrightarrow{\xi}\overrightarrow{r}^\mathsf{T}}^{s.s.}$ and  $\expval{\overrightarrow{r}\overrightarrow{\xi}^\mathsf{T}}^{s.s.}$ need to be calculated before working out the solution. One approach to calculate $\expval{\overrightarrow{\xi}\overrightarrow{r}^\mathsf{T}}^{s.s.}$ would be to solve Eq. \eqref{eq:vectorEqOfMotion}, but this is exactly what we are trying to avoid. It is here when the Novikov theorem comes useful, since it lets us compute $\expval{ \overrightarrow{\xi}\overrightarrow{r}^\mathsf{T}}^{s.s.}$ without having to integrate the equations of motion. Using this theorem and the $\delta$-correlation of the noises, we find the $ij$-th component of $\expval{ \overrightarrow{\xi}(t)\overrightarrow{r}(t)^\mathsf{T}}$,
%
\begin{align}
  \expval{ \xi_i(t) r_j(t) } &= \sum_{k=1}^2 \int_0^t d\tau\,\expval{ \xi_i(t) \xi_k(\tau)}
  \,
  \expval{ \frac{\delta r_j(t)}{\delta \xi_k(\tau)} }\nonumber
  \\
  &= \sum_{k=1}^2 \mathbb{D}_{ik}
  \,
  \lim_{\tau \to t^{-}}
  \,
  \expval{ \frac{\delta r_j(t)}{\delta \xi_k(\tau)} },
\end{align}
%
where $\lim\limits_{\tau \to t^{-}}$ is the limit when $\tau$ goes to $t$ from below. Evaluation of the functional derivative ${\delta r_j(t)}/{\delta \xi_k(\tau)}$ for the $\tau \to t^{-}$ limit gives
%
\begin{equation}
  \expval{ \overrightarrow{\xi}(t)\overrightarrow{r}(t)^\mathsf{T}} = \mathbb{D}\mathbb{L}^\mathsf{T}.
\end{equation}
%
Now, the algebraic equation that gives the steady-state covariance matrix becomes
%
\begin{equation}
  \mathbb{A}\mathbb{C}^{s.s.} +
  \mathbb{C}^{s.s.}\mathbb{A}^\mathsf{T}
  =
  -\mathbb{B},
  \label{eq:SteadyStateEquationToyModel}
\end{equation}
%
with $\mathbb{B} = 2 \mathbb{L}\mathbb{D}\mathbb{L}^\mathsf{T}$. By definition, the covariance matrix is  symmetric, but there are also  additional restrictions imposed by the equations of motion and the steady-state condition, which reduce the dimensionality of the problem of solving Eq. \eqref{eq:SteadyStateEquationToyModel} \cite{Simon2019}. Since ${d \expval{ q_i q_j }}/{dt} = 0$ in the steady state, we have
%
\begin{align}
  \expval{ p_1 q_1}^{s.s.} &= \expval{ p_2 q_2}^{s.s.} = 0,\nonumber\\
  \frac{\expval{ p_1 q_2}^{s.s.}}{m_1}&=-\frac{\expval{ q_1 p_2}^{s.s.}}{m_2}.
  \label{eq:ExtraConditionSteadyState}
\end{align}
%
Taking \eqref{eq:ExtraConditionSteadyState} into account, the steady-state covariance matrix takes the form
%
\begin{equation}
  \begin{split}
    \mathbb{C}^{s.s.} =
    \left(
    \begin{array}{cccc}
      \expval{ q_1^2}^{s.s.}  & \expval{ q_1 q_2}^{s.s.}  & 0 & \frac{\expval{ p_2 q_1}^{s.s.} }{m_2} \\
      \expval{ q_1 q_2}^{s.s.}  & \expval{ q_2^2}^{s.s.}  & -\frac{\expval{ p_2 q_1}^{s.s.} }{m_2} & 0 \\
      0 & -\frac{\expval{ p_2 q_1}^{s.s.} }{m_2} & \frac{\expval{ p_1^2}^{s.s.} }{m_1^2} & \frac{\expval{ p_1 p_2}^{s.s.} }{m_1 m_2} \\
      \frac{\expval{ p_2 q_1}^{s.s.} }{m_2} & 0 & \frac{\expval{ p_1 p_2}^{s.s.} }{m_1 m_2} & \frac{\expval{ p_2^2}^{s.s.} }{m_2^2} \\
      \end{array}
      \right)
    \end{split}
    \label{eq:steadyStateCovarianceMatrix}\,.
\end{equation}
%
The explicit set of equations for the components of $\mathbb{C}^{s.s}$ can be found in Appendix \ref{AppStationaryStateEquations}.
%
%
%
%
%
\section{Solutions\label{sec:solutions}}
%
%
%
%

%
In this section we use the solution to Eq. \eqref{eq:SteadyStateEquationToyModel} to write down the temperatures and currents in the steady state. We use Mathematica to obtain analytic expressions for the temperatures,
%
\begin{align}
  T_1 &= \frac{T_L \mathcal{P}_{1,L}(k) + T_R \mathcal{P}_{1,R}(k)}{\mathcal{D}(k)},\nonumber
  %
  \\
  %
  T_2 &= \frac{T_L \mathcal{P}_{2,L}(k) + T_R \mathcal{P}_{2,R}(k)}{\mathcal{D}(k)},
  %
  \label{eq:ModelBTemperatures}
\end{align}
%
where $\mathcal{D}(k) =  \sum\limits_{n=0}^2 \mathcal{D}_n k^n$ and $\mathcal{P}_{i,(L/R)}(k) = \sum\limits_{n=0}^2 a_{i,n,(L/R)} k^n$ are polynomials in the coupling constant $k$ with coefficients
%
\begin{align}
  \mathcal{D}_0 &= a_{1,0,L} = a_{2,0,R} = \gamma _L \gamma _R \left[h^{(1)} \left(\gamma_L k_R +\gamma_R k_L \right)+\left(m_1 k_R-m_2 k_L\right)^2\right],\nonumber
  %
  \\
  %
  \mathcal{D}_1 &= a_{1,1,L} = a_{2,1,R} = \gamma _L \gamma _R \left[h^{(0)} h^{(1)}+2 \left(m_1-m_2\right) \left(m_1 k_R-m_2 k_L\right)\right],\nonumber
  %
  \\
  %
  \mathcal{D}_2 &= h^{(0)} h^{(2)},\nonumber
  %
  \\
  %
  a_{1,2,L} &= \gamma _L \left(m_2 h^{(1)} + \gamma_R (m_1 - m_2)^2 \right),\nonumber
  %
  \\
  %
  a_{1,2,R} &= h^{(1)} m_1 \gamma_R,\nonumber
  %
  \\
  %
  a_{2,2,L} &= h^{(1)} m_2 \gamma_L,\nonumber
  %
  \\
  %
  a_{2,2,R} &= \gamma _R \left( m_1 h^{(1)} + \gamma_L (m_1-m_2)^2 \right),\nonumber
  %
  \\
  %
  a_{1,0,R} &= a_{1,1,R} = a_{2,0,L} = a_{2,1,L} = 0,
  %
  \label{eq:SolutionPolynomialCoefficients}
\end{align}
%
where $h^{(n)}\equiv \gamma_R m_1^n + \gamma_L m_2^n$. The currents from the baths to the masses \cite{Simon2019} are
%
\begin{equation}
  \begin{split}
    J_L &= k_B \frac{\gamma_L}{m_1} \left( T_L - T_1 \right),\\
    J_R &= k_B \frac{\gamma_R}{m_2} \left( T_R - T_2 \right),
    \label{eq:currents_definition}
  \end{split}
\end{equation}
\\
%
with $T_i$ given by Eq. \eqref{eq:ModelBTemperatures}. Since, in the steady state, $J_L = -J_R$ we will use the shorthand notation $J \equiv J_L$. Substituting Eq. \eqref{eq:ModelBTemperatures} into Eq.  \eqref{eq:currents_definition} we get for the heat current
%
% \begin{equation}
%   J = k_B \frac{k^2\gamma_L \gamma_R h^{(1)}}{\mathcal{D}(k)}(T_L - T_R).
%   \label{eq:CurrentsInModelB}
% \end{equation}
%
\begin{equation}
  J = \kappa\;(T_L - T_R),
  \label{eq:CurrentsInModelB}
\end{equation}
%
where $\kappa = k_B {k^2\gamma_L \gamma_R h^{(1)}}/{\mathcal{D}(k)}$ acts as an effective thermal conductance, which depends on the parameters of the system, i.e., the masses and spring constants, and also on the friction coefficients of the baths. From Eq. \eqref{eq:CurrentsInModelB} it could be thought that inverting the temperatures of the baths would only lead to an exchange of heat currents. However, since the thermal conductance $\kappa$ depends on the friction coefficients, the exchange of the baths implies a change in its value. Moreover, it is possible to have temperature-dependent friction coefficients, as it happens in the physical set-up of laser-cooled trapped ions described in Section \ref{sec:TrappedIonSetUp}.
%
%


\section{Relation of the Model to a trapped ion set-up \label{sec:TrappedIonSetUp}}

As we mentioned, the parameters $k$, $k_L$ and $k_R$ can be related to the elements of the Hessian matrix of a system in a stable equilibrium position. In this section we will identify these parameters with the Hessian matrix of a pair of trapped ions. Here we consider two different set-ups: two ions in a collective trap, and two ions in individual traps. In Section \ref{sec:lookingForR} we focus on two ions in individual traps to illustrate the analysis of rectification.

In both set-ups we assume strong confinement in the radial direction, making the effective dynamics one-dimensional. We will also assume that the confinement in the axial direction is purely electrostatic, which makes the effective spring constant independent of the mass of the ions \cite{Leibfried2003}. Additionally, we will relate the temperatures and friction coefficients of the Langevin baths to those corresponding to Doppler cooling.

\subsection{Collective trap}

Consider two ions of unit charge with masses $m_1$ and $m_2$ trapped in a collective trap. Assuming strong radial confinement and purely electrostatic axial confinement, both ions feel the same harmonic oscillator potential with trapping constant $k_{trap}$ \cite{Leibfried2003}. The potential describing the system is
%
\begin{equation}
  V_{collective} = \frac{1}{2}k_{trap} \left( x_1^2 + x_2^2\right) + \frac{\mathcal{C}}{x_2-x_1},
\end{equation}
%
with $\mathcal{C}=\frac{Q^2}{4\pi\varepsilon_0}$. The equilibrium positions for this potential are
%
\begin{equation}
  x_2^{eq} = -x_1^{eq} =
  \label{eq:equilibriumPositionsCollectiveTrap}\left(\frac{1}{2}\right)^{2/3} \left(\frac{Q^2}{4\pi\varepsilon_0 k_{trap}}\right)^{1/3}.
\end{equation}
%
Assuming small oscillations of the ions around the equilibrium positions, the Hessian matrix of the system is
%
\begin{align}
  \mathbb{K}_{1,2} &= -\frac{Q^2}{2\pi\varepsilon_0}\frac{1}{(x_2^{eq}-x_1^{eq})^3} = -k_{trap},\nonumber
  \\
  \mathbb{K}_{1,1} &= k_{trap} + \frac{Q^2}{2\pi\varepsilon_0}\frac{1}{(x_2^{eq}-x_1^{eq})^3} = 2 k_{trap},\nonumber
  \\
  \mathbb{K}_{2,2} &= k_{trap} + \frac{Q^2}{2\pi\varepsilon_0}\frac{1}{(x_2^{eq}-x_1^{eq})^3} = 2 k_{trap}.
  \label{eq:HessianOffDiagonalCollective}
\end{align}
%
Using Eq. \eqref{eq:HessianOffDiagonalCollective} we can relate the parameters of this physical set-up to those of the model described in Section \ref{sec:Physical_Model} to find
%
\begin{equation}
  k_L = k_R = k = k_{trap}.
\end{equation}
%

\subsection{Individual on-site traps}

We can make the same assumptions for the axial confinement as in the previous subsection but now each of the ions is in an individual trap with spring constants $k_{trap,L}$ and $k_{trap,R}$ respectively. The potential of the system is
%
\begin{align}
    V_{individual} &= \frac{1}{2}k_{trap,L}\left(x_1 -x_L\right)^2 +\frac{1}{2}k_{trap, R}\left(x_2 -x_R\right)^2 \nonumber \\&+ \frac{\mathcal{C}}{x_2-x_1},
\end{align}
%
where $x_L$ and $x_R$ are the center positions of the on-site traps. The elements of the Hessian matrix in the equilibrium position are
%
\begin{align}
  \mathbb{K}_{1,2} &= -\frac{Q^2}{2\pi\varepsilon_0}\frac{1}{(x_2^{eq}-x_1^{eq})^3},\nonumber
  \\
  \mathbb{K}_{1,1} &= k_{trap,L} + \frac{Q^2}{2\pi\varepsilon_0}\frac{1}{(x_2^{eq}-x_1^{eq})^3},\nonumber
  \\
  \mathbb{K}_{2,2} &= k_{trap,R} + \frac{Q^2}{2\pi\varepsilon_0}\frac{1}{(x_2^{eq}-x_1^{eq})^3}.
  \label{eq:HessianOffDiagonalOnSite}
\end{align}
%
Comparing the parameters in Eq. \eqref{eq:HessianOffDiagonalOnSite} with those in the model described in Section \ref{sec:Physical_Model} we identify
\begin{align}
  k_L &= k_{trap,L},\nonumber\\
  k_R &= k_{trap,R},\nonumber\\
  k &= \frac{Q^2}{2\pi\varepsilon_0}\frac{1}{(x_2^{eq}-x_1^{eq})^3}\,.
\end{align}
%
In this case, the analytic expressions for the equilibrium positions are more complicated. We get for the distance between the equilibrium positions of the ions
%
\begin{align}
  &(x_2 - x_1)^{(eq)} = \frac{1}{3} \Delta x_{LR}\nonumber\\
  &- \frac{1}{6}\Big[ \frac{2^{2/3}\zeta}{k_{trap,L} k_{trap,R} (k_{trap,L} + k_{trap,R})}\nonumber\\
  &+ \frac{2^{4/3} k_{trap,L} k_{trap,R} (k_{trap,L} + k_{trap,R}) (x_R-x_L)^2}{\zeta} \Big]\,,
\end{align}
%
where $\Delta x_{LR} = (x_R-x_L)$ and $\zeta = \left( Y - \eta \right)^{(1/3)}$, with
%
\begin{align}
  &Y = 3 \sqrt{3} \bigg\{\mathcal{C} k_{trap,L}^4 k_{trap,R}^4 \left(k_{trap,L}+k_{trap,R}\right)^{7}\times\nonumber\\& \quad\quad \left[4 k_{trap,L} k_{trap,R} \Delta x_{LR}^3+27 \mathcal{C} \left(k_{trap,L}+k_{trap,R}\right)\right]\bigg\}^{(1/2)},\nonumber
  %
  \\
  %%
  &\eta =  k_{trap,L}^2 k_{trap,R}^2 \left(k_{trap,L}+k_{trap,R}\right)^{3}\times\nonumber\\ &\quad\quad\left[2 k_{trap,L} k_{trap,R} \Delta x_{LR}^3+27 \mathcal{C} \left(k_{trap,L}+k_{trap,R}\right)\right]\,.
\end{align}
%
In this set-up, the coupling between the ions $k$ can be controlled by changing the distance between the on-site traps.



\subsection{Optical molasses and Langevin baths}

Trapped ions may be cooled down by a pair of counterpropagating lasers which are red-detuned with respect to an internal atomic transition of the ions. This technique is known as Doppler cooling or optical molasses \cite{Chu1985,Cohen1992,Metcalf1999,Metcalf2003}. The off-resonant absorption of laser photons by the ions exerts a damping-like force that slows them down. The spontaneous emission of the ions produces heating due to the random recoil generated by the emitted photons. Both, the friction and recoil force are in balance, and eventually the ion thermalizes to a finite temperature.
Thus the effect of the lasers on the ion is equivalent to a Langevin bath with temperature $T_{molass}$ and friction coefficient $\gamma_{molass}$. The temperature and friction coefficients are controlled with the laser intensity $I$ and frequency detuning $\delta$ with respect to the selected internal transition by the expressions \cite{Cohen1992,Metcalf2003,Ruiz2014},
%
\begin{align}
  \gamma_{molass}(I,\delta) &= -4 \hbar \left(\frac{\delta + \omega_0}{c}\right)^2 \left(\frac{I}{I_0}\right)\frac{2\delta/\Gamma}{\left[1 + (2\delta/\Gamma)^2\right]^2},\nonumber\\
  %
  T_{molass}(\delta) &= -\frac{\hbar \Gamma}{4 k_B} \frac{1+(2\delta/\Gamma)^2}{(2\delta/\Gamma)},
  \label{eq:DopplerCoolingToyModel}
\end{align}
%
where $\omega_0$ is the frequency of the selected internal atomic transition, $\Gamma$ is the natural width of the excited state, and $I_0$ is the saturation intensity.
%
%
%
\section{Looking for rectification\label{sec:lookingForR}}
%
%
%
%First let us define what we exactly mean by \textit{rectification}.
We will say that we observe rectification whenever the heat current $J$ for a configuration of the baths changes when we exchange the baths to $\tilde{J}$. The important point here is to define what is  meant by \textit{exchanging the baths}. We consider that a bath is characterized, not only by its temperature $T$ but also by its coupling  to the system by means of the friction coefficient $\gamma$, so, exchanging the baths is achieved by exchanging both the temperatures and the friction coefficients, as summarized in Table \ref{tab:reversed_bath}.

When implementing temperatures and friction coefficients by lasers, this exchange operation is performed by changing the values of the intensities and detunings acting on each ion (Eq. \eqref{eq:DopplerCoolingToyModel}). The exchange operation is straightforward when the two ions are either of the same species or isotopes of each other, since the only required action is to exchange the values of the detunings of the lasers without modifying the intensities. However, if we deal with two different species, i.e., with two different atomic transitions, the laser wavelengths and the decay rates  depend on the species. Then, exchanging the temperatures by modifying the detunings, keeping the laser intensities constant, does not necessarily imply an exchange of the friction coefficients. Nevertheless it is possible to adjust the laser intensities so that the friction coefficients get exchanged and that is the assumption hereafter. The idea of implementing a bath exchange like this follows the same line of thought as \cite{Pereira2017}, since we are adding a temperature dependent feature to the system -the friction coefficients- that changes as the baths are inverted.

\begin{figure}
  \includegraphics[width=\linewidth]{Figures/RwMPlota.pdf}
  \caption{Rectification, $R$, in the $k_L k_R$ plane for $k = 1.17 \times$ fN/m, $\gamma_L = 6.75\times 10^{-22}$ kg/s, and $\gamma_R = 4.64\gamma_L$.}
  \label{fig:Fig_rectification_K_plane}
\end{figure}

To measure rectification, we will use the rectification coefficient $R$ defined as
%
\begin{equation}
  R = \frac{\abs{J-\tilde{J}}}{\max(J,\tilde{J})},
  \label{eq:Rectification}
\end{equation}
%
that is, the ratio between the difference of heat currents and the largest one. As defined, $R=0$ for no asymmetry of the heat currents and $R=1$ when they are maximally asymmetric.

\begin{table}[]
\caption{Definition of forward and reversed (exchanged) bath configurations.}
\begin{tabular}{lcc}
\hline
                 & forward                & reversed                                                       \\ \hline
Bath Friction    & $\gamma_L$, $\gamma_R$ & $\tilde{\gamma}_L =\gamma_R $,  $\tilde{\gamma}_R =\gamma_L $   \\
Bath Temperature & $T_L$, $T_R$           & $\tilde{T}_L =T_R $,  $\tilde{T}_R =T_L $                     \\
\hline
\end{tabular}
\label{tab:reversed_bath}
\end{table}

\subsection{Parametric exploration}

We have explored thoroughly the space formed by the parameters of the model to find asymmetric heat transport, namely, $m_1,m_2,k,k_L,k_R,\gamma_L,\gamma_R$. We have fixed the values of some of the parameters to realistic ones while we have varied the rest. We have set the masses to $m_1 = 24.305$ a.u. and $m_2 = 40.078$ a.u., which correspond to Mg and Ca, whose ions are broadly used in trapped-ion physics. The temperatures are also fixed and, as Eq. \eqref{eq:CurrentsInModelB} shows, rectification does not formally depend on the temperature in this model, unless we set the friction coefficients as a function of temperature using Eq. \eqref{eq:DopplerCoolingToyModel} explicitly.

Figure \ref{fig:Fig_rectification_K_plane} depicts the values of the rectification after sweeping the $k_L k_R$ plane for fixed values of $k$, $\gamma_L$, and $\gamma_R$. A remarkable result from this figure is that parallel lines appear alternating minima and maxima of $R$. With a numerical fitting, we find that the line corresponding to the highest maximum value of $R$ is determined by
%
\begin{equation}
  \frac{k+k_L}{m_1} = \frac{k+k_R}{m_2}.
  \label{eq:MaxRLines}
\end{equation}
%
In a trapped-ion context the condition \eqref{eq:MaxRLines} may be imposed by adjusting the distance of the traps for fixed $k_L$ and $k_R$. It is also  remarkable that when Eq. \eqref{eq:MaxRLines} is satisfied, the rectification no longer depends on the spring constants of the model. This last result can be  found assuming  Eq. \eqref{eq:MaxRLines} when calculating the currents with Eq. \eqref{eq:CurrentsInModelB} and $R$ with Eq. \eqref{eq:Rectification},
%
\begin{equation}
    R=
    \begin{cases}
      1-\frac{a+g}{1+ag} &\text{ if }(a+g)<(1+ag)\\
      1-\frac{1+ag}{a+g} &\text{ if }(a+g)>(1+ag)\\
      0 &\text{ if } (a+g)=(1+ag)\,,
    \end{cases}
  \label{eq:maxRExpression}
\end{equation}
%
where $a$ and $g$ are the mass and friction coefficients ratios
%
\begin{align}
  a &= m_2/m_1,\nonumber\\
  g &= \gamma_R/\gamma_L.
\end{align}
%
The maximal rectification found does not scale with the magnitude of the masses or the friction coefficients, just with their ratios. Besides a high $R$, it is important to have non-vanishing heat currents
%, as in some cases an increase of $R$ is accompanied by an overall decrease of the heat currents
\cite{Simon2019}. Using again  Eq. \eqref{eq:MaxRLines} in the expression for the currents \eqref{eq:CurrentsInModelB}, the maximum current $J_{\max} = \max(\big|{J}\big|,\big|\tilde{J}\big|)$ is
%
\begin{align}
    &J_{\max}=\begin{cases}
   \frac{k_B g\gamma_L k^2 \abs{T_L-T_R}}{(a+g)(g\gamma_L^2(k_L+k)+k^2m_1)} & \text{ if }(a+g)<(1+ag)
    \\
    \frac{k_B g\gamma_L k^2 \abs{T_L-T_R}}{(1+ag)(g\gamma_L^2(k_L+k)+k^2m_1)}& \text{ if }(a+g)>(1+ag)\,.
    \end{cases}
    \label{eq:maxJExpression}
\end{align}
%
Now we analyze how the parameters $a$ and $g$ affect the maximum current $J_{max}$ in \eqref{eq:maxJExpression}. To do this, we can divide the $ag$ plane in four quadrants by the axes $a = 1$ and $g = 1$ (in those axes $R = 0$). In Eq. \eqref{eq:maxJExpression} the parameter $a$ appears only in the denominator, thus for a higher $a$, a smaller current is found. The quadrants with $a < 1$ will be better for achieving large currents. However, $g$ appears both in the numerator and denominator so there is no obvious advantageous quadrant for this parameter.

Equation \eqref{eq:maxRExpression} is symmetric upon the transformations $a \leftrightarrow 1/a$ and $g \leftrightarrow 1/g$. Using a logarithmic scale for $a$ and $g$, the resulting $R$ map will be symmetric with respect to the $a=1$ and $g=1$ axes. We can limit ourselves to analyze the quadrant $a > 1$, $g > 1$, as the results in other quadrants will be equivalent upon transformations $a \leftrightarrow 1/a$ and $g \leftrightarrow 1/g$.


\begin{figure}
  \includegraphics[width=\linewidth]{Figures/Rade.png}
  \caption{Rectification factor, $R$, given by Eq. \eqref{eq:maxRExpression}.}
  \label{fig:R_g_a_plane}
\end{figure}

Fig. \ref{fig:R_g_a_plane} shows the rectification given by Eq. \eqref{eq:maxRExpression} in terms of $a$ and $g$. Along any diagonal line (parallel to the solid cyan or the dashed green lines), the maximum value is at the center, that is, when $a = g$. However, if we fix $a$, increasing $g$ always increases $R$. Although we could increase $g$ arbitrarily to get more rectification this is not a realistic option in a trapped-ion set-up. Since $g$ is defined as the ratio between the friction coefficients, increasing it means making either $\gamma_L$ go to 0 or $\gamma_R$ to infinity. Making $\gamma_L$ go to 0 decouples one of the ions from the bath, so the heat current tends to vanish in any direction. Also, increasing $\gamma_R$ arbitrarily is impossible since the Doppler cooling friction coefficient as a function of the laser detuning (Eq. \eqref{eq:DopplerCoolingToyModel}) is bounded. Although Eq. \eqref{eq:DopplerCoolingToyModel} suggests that boosting the laser intensity can also increase the friction coefficient, this is not an option since Eq. \eqref{eq:DopplerCoolingToyModel} is just an approximation for low laser intensities. When going to higher intensities, the emission/absorption of photons by the ion is saturated and the friction coefficient reaches a finite value proportional to the width $\Gamma$ of the excited state \cite{Metcalf2003}. As a compromise between feasibility and high $R$, we set the ratio between the friction coefficients $g$ to be equal to the mass ratio $a$. As shown  in Fig. \ref{fig:R_g_a_plane}, along the solid-cyan and dashed-green diagonal lines the maximum $R$ is achieved for $a = g$. Fig. \ref{fig:Fig_PerfectRectification} shows the rectification in Eq. \eqref{eq:maxRExpression} for the line $a = g$. When both parameters are large enough, the rectification goes to 1.
%
%
\subsection{Spectral match/mismatch approach to rectification}
%
%
%
\begin{figure}
  \includegraphics[width=\linewidth]{Figures/CC-eps-converted-to.pdf}
  \caption{Rectification for different values of $c=m_2/m_1=\gamma_R/\gamma_L$ when the maximum condition in the $k_L k_R$ plane is satisfied (Eq. \eqref{eq:MaxRLines}).}
  \label{fig:Fig_PerfectRectification}
\end{figure}

The match/mismatch between the power spectra of the particles controls the heat currents in the system \cite{Terraneo2002,Li2004}. A good match between the power spectra of the two ions in a large range of frequencies yields a higher heat current through the system while the mismatch  reduces the heat current.
%Therefore, we can understand rectification through the match/mismatch of the phonon bands of the ions \cite{Terraneo2002}.
If there is a good match between the spectra of the ions (i.e., their peaks overlap in a broad range of frequencies) for a certain baths configuration, and mismatch when the baths exchange, the system will present heat rectification.

We have studied the phonon spectra of our model for several sets of parameters exhibiting no rectification or strong rectification. The phonon spectra of the ions is calculated through the spectral density matrix. For a real-valued stochastic process $\overrightarrow{x}(t)$, its spectral density matrix is defined as \cite{Sarkka2019}
%
\begin{equation}
  \mathbb{S}_{\overrightarrow{x}}(\omega) \equiv \expval{ \overrightarrow{X}(\omega) \overrightarrow{X}^\mathsf{T}(-\omega) },
  \label{eq:SpectralDensityDefinition}
\end{equation}
%
with $\overrightarrow{X}(\omega)$ being the Fourier transform of $\overrightarrow{x}(t)$ (we are using the convention of multiplying by a factor of $1$ and $\frac{1}{2\pi}$ for the transform and its inverse operation). A justification of the use of the spectral density matrix to understand heat transport arises from the Wiener-Khinchin theorem \cite{Sarkka2019}, which says that the correlation matrix of a stationary stochastic process in the steady state is the inverse Fourier transform of its spectral density matrix $\expval{\overrightarrow{r}(t)\overrightarrow{r}^\mathsf{T}(t+\tau)} = \mathcal{F}^{-1}[\mathbb{S}_{\overrightarrow{r}}(\omega)](\tau)$. This result allows us to write down the covariance matrix in the steady state through the spectral density as
%
\begin{equation}
  \mathbb{C}^{s.s.} = \frac{1}{2\pi} \int_{-\infty}^{\infty}d\omega\;\mathbb{S}_{\overrightarrow{r}}(\omega).
  \label{eq:Wiener-Khinchin}
\end{equation}
%
Eq. \eqref{eq:Wiener-Khinchin} directly connects the spectral density matrix to the steady-state temperature and, therefore, to the heat currents (in Section \ref{sec:covMatrix} we saw that  $T_1^{s.s.} = {m_1 C_{3,3}^{s.s.}}/{k_B}$ and $T_2^{s.s.} = {m_2 C_{4,4}^{s.s.}}/{k_B}$).


\begin{figure}[t]
  \includegraphics[width=\linewidth]{Figures/SpectrumComparative.pdf}
  \caption{Spectral densities of the velocities of the ions ($r_3$ and $r_4$) corresponding to different values of $c$ in Fig. \ref{fig:Fig_PerfectRectification}: (a), (b) for $c=1$ and (c), (d) for $c=10$. Solid, black lines correspond to the left ion velocity spectral density $\mathbb{S}_{3,3}(\omega)$ and dashed, blue lines correspond to the right ion velocity spectral density $\mathbb{S}_{4,4}(\omega)$. (a) and (b) correspond to $R = 0$:  the overlap between the phonon bands is the same in the forward and reversed configurations. (c) and (d) correspond to $R\approx 0.8$:  in the forward configuration (c)  the phonons match better than in the reversed configuration (d).}
  \label{fig:Figure_Spectra}
\end{figure}


For the vector process $\overrightarrow{r}(t)$ describing the evolution of our system we have $\overrightarrow{R}(\omega) = \left( i \omega - \mathbb{A} \right)^{-1}\mathbb{L}\overrightarrow{\Xi}(\omega)$ with $\overrightarrow{\Xi}(\omega)$ being the Fourier transform of the white noise $\overrightarrow{\xi}(t)$. Note that $\overrightarrow{\Xi}(\omega)$ does not strictly exist, because it is not square-integrable, however its spectral density is $\mathbb{S}_{\overrightarrow{\xi}}(\omega) = 2 \mathbb{D}$ \cite{Sarkka2019}, which is flat as expected for a white noise. Therefore, the spectral density matrix of the system is
%
\begin{equation}
  \mathbb{S}_{\overrightarrow{r}} = 2 \left(  \mathbb{A} - i\omega\right)^{-1}\mathbb{L}\mathbb{D}\mathbb{L}^\mathsf{T}\left(  \mathbb{A} + i\omega\right)^{-\mathsf{T}}.
  \label{eq:SpectralDensityToyModelB}
\end{equation}
%
As we can see in Eq. \eqref{eq:SpectralDensityToyModelB}, the imaginary part of the eigenvalues of the dynamical matrix $\mathbb{A}$ correspond to the peaks in the spectrum whereas the real part dictates their width. The spectral density matrix of our model is
%
\begin{equation}
  \mathbb{S}_{\overrightarrow{r}}(\omega) = 2 k_B \frac{\gamma_L T_L\mathbb{S}_L(i\omega)+\gamma_L T_R\mathbb{S}_R(i\omega)}{(m_1 m_2)^2 P_\mathbb{A}(i\omega)P_\mathbb{A}(-i\omega)},
\end{equation}
%
where $P_\mathbb{A}(\lambda)$ is the characteristic polynomial of the dynamical matrix $\mathbb{A}$ and $\mathbb{S}_L(\omega)$, $\mathbb{S}_R(\omega)$ are the matrix polynomials in the angular frequency $\omega$ whose coefficients are defined in Appendix \ref{AppSpecDenMat}. Equation \eqref{eq:SpectralDensitiesVelocities} gives the full expressions of the spectral densities for the velocities, $\mathbb{S}_{3,3}(\omega) = \expval{R_3(\omega)R_3(-\omega)}$ for the left ion, and $\mathbb{S}_{4,4}(\omega) = \expval{R_4(\omega)R_4(-\omega)}$ for the right ion, since they are the elements related to the calculation of the heat current using Eq. \eqref{eq:Wiener-Khinchin},
%
\begin{align}
  \mathbb{S}_{3,3}(\omega) &= 2 k_B \frac{\gamma_R k^2 T_R \omega ^2+\gamma_L T_L \left[\omega ^4 \left(\gamma_R^2-2 k m_2-2 k_R m_2\right)+\omega ^2 (k+k_R)^2+m_2^2 \omega ^6\right]}{(m_1 m_2)^2 P_\mathbb{A}(i\omega)P_\mathbb{A}(-i\omega)},\nonumber\\
  %
%    \nonumber\\
  %
  \mathbb{S}_{4,4}(\omega) &= 2 k_B \frac{\gamma_L k^2 T_L \omega ^2+\gamma_R T_R \left[\omega ^4 \left(\gamma_L^2-2 k m_1-2 k_L m_1\right)+\omega ^2 (k+k_L)^2+m_1^2 \omega ^6\right]}{(m_1 m_2)^2 P_\mathbb{A}(i\omega)P_\mathbb{A}(-i\omega)}.
  \label{eq:SpectralDensitiesVelocities}
\end{align}
%
Figure \ref{fig:Figure_Spectra} depicts a series of plots of the spectra given by Eq. \eqref{eq:SpectralDensitiesVelocities} that correspond to two points in Fig. \ref{fig:Fig_PerfectRectification}. For $c=1$ (Fig. \ref{fig:Figure_Spectra}(a) and (b)) there is no rectification, since the spectra match in the forward (a) and reversed (b) configurations. However, for $c=10$ ((Fig. \ref{fig:Figure_Spectra}(c) and (d))) the picture is very different: there is a good match between the spectra in the forward configuration whereas in the reversed configuration the spectra are less correlated, giving as a result higher rectification ($R \approx 0.8$). Figure \ref{fig:Figure_Spectra} only shows the elements (3,3) and (4,4) in the diagonal of $\mathbb{S}$ but the remaining elements, including off-diagonal ones, exhibit a similar behavior.
% with respect to rectification.
%
\section{Conclusions \label{sec:Conclusions}}
%
We have studied heat rectification in a model composed of two coupled harmonic oscillators connected to baths. This simple model allows analytical treatment but still has enough complexity to examine different ingredients that can produce rectification. %We have also derived analytical expressions for the heat currents and local temperatures.
Our results demonstrate in a simple but realistic system that harmonic systems can rectificate heat current if they have features which depend on the temperature  \cite{Pereira2017}. We implement this notion of temperature-dependent features by defining the baths exchange operation as an exchange of both temperatures and coupling parameters of the baths to the system. This kind of temperature-dependent features happens naturally in laser-cooled trapped ion set-ups.

We have also studied the phonon spectra of the system, comparing the match/mismatch of the phonon bands, to reach the conclusion that the band match/mismatch description for heat rectification is also valid for systems which are harmonic, as long as there are temperature-dependent features.
We hope this article sheds more light into the topic of heat rectification and that encourages more research regarding its physical implementation on chains of trapped ions.

\section{acknowledgements}
We thank Daniel Alonso for fruitful discussions and comments. This work was supported by the Basque Country Government (Grant No. IT986-16), by Grants PGC2018-101355-B-I00 (MCIU/AEI/FEDER,UE) and FIS2016-80681P, and by the Spanish MICINN and European Union (FEDER) (Grant No. FIS2017-82855-P). M.A.S. acknowledges support by the Basque Government predoctoral program (Grant No. PRE-2019-2-0234).

\appendix




\section{Full set of steady-state equations for the components of $\mathbb{C}^{s.s}$  \label{AppStationaryStateEquations}}
Here we present the full set of equations for the covariance matrix elements in the steady state,
%
\begin{equation}
  \begin{split}
    \frac{2 k \expval{ p_2 q_1}^{s.s.} }{m_1 m_2}+\frac{2 \gamma _L \expval{ p_1^2}^{s.s.} }{m_1^3}&=\frac{2 D_L}{m_1^2},
    %
    \\
    -\frac{2 k \expval{ p_2 q_1}^{s.s.} }{m_2^2}+\frac{2 \gamma _R \expval{ p_2^2}^{s.s.} }{m_2^3}&=\frac{2 D_R}{m_2^2},
    %
    \\
    %
    -\frac{\left(k_L+k\right) \expval{ q_1 q_2}^{s.s.} }{m_1}+\frac{k \expval{ q_2^2}^{s.s.} }{m_1}+\frac{\gamma _L \expval{ p_2 q_1}^{s.s.} }{m_1 m_2}+\frac{\expval{ p_1 p_2}^{s.s.} }{m_1 m_2}&=0,
    %
    \\
    %
    \frac{\left(k_L+k\right) \expval{ p_2 q_1}^{s.s.} }{m_1 m_2}-\frac{\left(k_R+k\right) \expval{ p_2 q_1}^{s.s.} }{m_2^2}+\frac{\gamma _L \expval{ p_1 p_2}^{s.s.} }{m_1^2 m_2}+\frac{\gamma _R \expval{ p_1 p_2}^{s.s.} }{m_1 m_2^2}&=0,
    %
    \\
    %
    -\frac{\left(k_L+k\right) \expval{ q_1^2}^{s.s.} }{m_1}+\frac{k \expval{ q_1 q_2}^{s.s.} }{m_1}+\frac{\expval{ p_1^2}^{s.s.} }{m_1^2}&=0,
    %
    \\
    %
    -\frac{\left(k_R+k\right) \expval{ q_2^2}^{s.s.} }{m_2}+\frac{k \expval{ q_1 q_2}^{s.s.} }{m_2}+\frac{\expval{ p_2^2}^{s.s.} }{m_2^2}&=0,
    %
    \\
    %
    -\frac{\left(k_R+k\right) \expval{ q_1 q_2}^{s.s.} }{m_2}+\frac{k \expval{ q_1^2}^{s.s.} }{m_2}-\frac{\gamma _R \expval{ p_2 q_1}^{s.s.} }{m_2^2}+\frac{\expval{ p_1 p_2}^{s.s.} }{m_1 m_2}&=0
  \end{split}
  \label{eq:SteadyStateEquationsModelB_Explicite}
\end{equation}

%
%
\section{Complete expressions for the Spectral Density Matrix\label{AppSpecDenMat}}
%
%
In Section \ref{sec:lookingForR} we used the characteristic polynomial $P_{\mathbb{A}}(\lambda)$ of the dynamical matrix $\mathbb{A}$ for the calculation of the spectral density matrix. $P_{\mathbb{A}}(\lambda)$ is defined as
\begin{equation}
  \begin{split}
    \det(\mathbb{A}-\lambda) &= \lambda ^4 \\&+ \lambda ^3 \left(\frac{\gamma_L}{m_1}+\frac{\gamma_R}{m_2}\right) \\ &+ \lambda^2\frac{ (\gamma_L \gamma_R+m_2 (k+k_L)+m_1 (k+k_R))}{m_1 m_2}\\ &+ \lambda \frac{  (\gamma_R (k+k_L)+\gamma_L (k+k_R))}{m_1 m_2}\\ &+\frac{k (k_L+k_R)+k_L k_R}{m_1 m_2}.
  \end{split}
\end{equation}
%
We also used the polynomials $\mathbb{S}_L(\lambda)$ and $\mathbb{S}_R(\lambda)$, which are defined as $\mathbb{S}_L(\lambda)=\sum\limits_{n=0}^6 \lambda^n \mathbb{s}_{L,n}$ and $\mathbb{S}_R(\lambda)=\sum\limits_{n=0}^6 \lambda^n \mathbb{s}_{R,n}$. There are 14 different polynomial coefficients, which are $4\times 4$ matrices, which makes very cumbersome to include them in the main text. This is the full list of coefficients,
%
\begin{equation}
  \begin{aligned}
    \mathbb{s}_{L,0} &=
    \left(
    \begin{array}{cccc}
      (k+k_R)^2 & k (k+k_R) & 0 & 0 \\
      k (k+k_R) & k^2 & 0 & 0 \\
      0 & 0 & 0 & 0 \\
      0 & 0 & 0 & 0
    \end{array}
    \right),
    &
    \mathbb{s}_{R,0} & =
    \left(
    \begin{array}{cccc}
      k^2 & k (k+k_L) & 0 & 0 \\
      k (k+k_L) & (k+k_L)^2 & 0 & 0 \\
      0 & 0 & 0 & 0 \\
      0 & 0 & 0 & 0
    \end{array}
    \right),
    %
    \\
    %
    \mathbb{s}_{L,1} &= \left(
    \begin{array}{cccc}
      0 & k \gamma_R & -(k+k_R)^2 & -k (k+k_R) \\
      -k \gamma_R & 0 & -k (k+k_R) & -k^2 \\
      (k+k_R)^2 & k (k+k_R) & 0 & 0 \\
      k (k+k_R) & k^2 & 0 & 0
    \end{array}
    \right),
    &
    \mathbb{s}_{R,1} & = \left(
    \begin{array}{cccc}
      0 & -k \gamma_L & -k^2 & -k (k+k_L) \\
      k \gamma_L & 0 & -k (k+k_L) & -(k+k_L)^2 \\
      k^2 & k (k+k_L) & 0 & 0 \\
      k (k+k_L) & (k+k_L)^2 & 0 & 0
    \end{array}
    \right),
    %
    \\
    %
    \mathbb{s}_{L,2} &= \left(
    \begin{array}{cccc}
      2 (k+k_R) m_2-\gamma_R^2 & k m_2 & 0 & -k \gamma_R \\
      k m_2 & 0 & k \gamma_R & 0 \\
      0 & k \gamma_R & -(k+k_R)^2 & -k (k+k_R) \\
      -k \gamma_R & 0 & -k (k+k_R) & -k^2
    \end{array}
    \right),
    &
    \mathbb{s}_{R,2} & = \left(
    \begin{array}{cccc}
      0 & k m_1 & 0 & k \gamma_L \\
      k m_1 & 2 (k+k_L) m_1-\gamma_L^2 & -k \gamma_L & 0 \\
      0 & -k \gamma_L & -k^2 & -k (k+k_L) \\
      k \gamma_L & 0 & -k (k+k_L) & -(k+k_L)^2
    \end{array}
    \right),
    %
    \\
    %
    \mathbb{s}_{L,3} &= \left(
    \begin{array}{cccc}
      0 & 0 & \gamma_R^2-2 (k+k_R) m_2 & -k m_2 \\
      0 & 0 & -k m_2 & 0 \\
      2 (k+k_R) m_2-\gamma_R^2 & k m_2 & 0 & -k \gamma_R \\
      k m_2 & 0 & k \gamma_R & 0
    \end{array}
    \right),
    &
    \mathbb{s}_{R,3} & =\left(
    \begin{array}{cccc}
      0 & 0 & 0 & -k m_1 \\
      0 & 0 & -k m_1 & \gamma_L^2-2 (k+k_L) m_1 \\
      0 & k m_1 & 0 & k \gamma_L \\
      k m_1 & 2 (k+k_L) m_1-\gamma_L^2 & -k \gamma_L & 0
    \end{array}
    \right),
    %
    \\
    %
    \mathbb{s}_{L,4} &= \left(
    \begin{array}{cccc}
      m_2^2 & 0 & 0 & 0 \\
      0 & 0 & 0 & 0 \\
      0 & 0 & \gamma_R^2-2 (k+k_R) m_2 & -k m_2 \\
      0 & 0 & -k m_2 & 0
    \end{array}
    \right),
      &
      \mathbb{s}_{R,4} & = \left(
    \begin{array}{cccc}
      0 & 0 & 0 & 0 \\
      0 & m_1^2 & 0 & 0 \\
      0 & 0 & 0 & -k m_1 \\
      0 & 0 & -k m_1 & \gamma_L^2-2 (k+k_L) m_1
    \end{array}
    \right),
    %
    \\
    %
    \mathbb{s}_{L,5} &= \left(
    \begin{array}{cccc}
      0 & 0 & -m_2^2 & 0 \\
      0 & 0 & 0 & 0 \\
      m_2^2 & 0 & 0 & 0 \\
      0 & 0 & 0 & 0
    \end{array}
    \right),
    &
    \mathbb{s}_{R,5} & = \left(
    \begin{array}{cccc}
      0 & 0 & 0 & 0 \\
      0 & 0 & 0 & -m_1^2 \\
      0 & 0 & 0 & 0 \\
      0 & m_1^2 & 0 & 0
    \end{array}
    \right),
    %
    \\
    %
    \mathbb{s}_{L,6} &= \left(
    \begin{array}{cccc}
      0 & 0 & 0 & 0 \\
      0 & 0 & 0 & 0 \\
      0 & 0 & -m_2^2 & 0 \\
      0 & 0 & 0 & 0
    \end{array}
    \right),
    &
    \mathbb{s}_{R,6} & = \left(
    \begin{array}{cccc}
      0 & 0 & 0 & 0 \\
      0 & 0 & 0 & 0 \\
      0 & 0 & 0 & 0 \\
      0 & 0 & 0 & -m_1^2
    \end{array}
    \right)\,.
  \end{aligned}
\end{equation}
%
 %Rectification Toy Model

%!TEX root = ../Thesis.tex

\chapter*{Conclusions} % Write in your own chapter title
\label{Conclusions}
\lhead{\emph{Conclusions}} % Write in your own chapter title to set the page header

My pleasure.
 %Conclusions
\addcontentsline{toc}{chapter}{Conclusions}

%% ----------------------------------------------------------------
% Now begin the Appendices, including them as separate files

\addtocontents{toc}{\vspace{2em}} % Add a gap in the Contents, for aestheticsy

\appendix % Cue to tell LaTeX that the following 'chapters' are Appendices
\part*{Appendix}
%\addcontentsline{toc}{part}{Appendix}
\addtocontents{toc}{\vspace{0.6em}}

\input{./Appendices/AppendixA}	% Interaction versus asymmetry for adiabatic following

\addtocontents{toc}{\vspace{0.6em}}  % Add a gap in the Contents, for aesthetics
\backmatter
\pagestyle{empty}  % Page style needs to be empty for this page

%% ----------------------------------------------------------------
\label{Bibliography}
\lhead{\emph{Bibliography}}  % Change the left side page header to "Bibliography"
%\bibliographystyle{apsrev}  % Use the "unsrtnat" BibTeX style for formatting the Bibliography
%\bibliographystyle{h-physrev3} % formato PRA sin url ni issn ni hyperlinks
\bibliographystyle{sofia} %modificado por mi a partir de utphys
%\bibliographystyle{utphys} %formato arXiv con hyperlinks
\bibliography{Bibliography_Thesis}  % The references (bibliography) information are stored in the file named "Bibliography_Thesis.bib"

\end{document}  % The End
%% ----------------------------------------------------------------
