%%!TEX root = ../Thesis.tex

\chapter{Numerical calculation of transmission and reflection coefficients}
\label{Appendix:NumericalCalculationOfTandR}
\lhead{Appendix D. \emph{Numerical calculation of transmission and reflection coefficients}}

%

Here we will discuss how to numerically solve  the
stationary Schr\"odinger equation for the two-level system
by the invariant imbedding method \cite{Singer1982,Band1994}.
%

%
Let the potential ${\cal V}(x)$ be non-zero in the region $-d < x < d$.
We  introduce the following dimensionless variables: $\bar k = (2mE)^{1/2}2d/\hbar$, $\bar x = x/(2d) + 1/2$,
$\bar\Omega (\bar x)= (4md^2/\hbar) \Omega(x)$ and $\bar\Gamma = (4md^2/\hbar) (\gamma-2i\Delta)$.
%
%
The non-Hermitian dimensionless Hamiltonian for the system takes the form
%
\begin{eqnarray}
	\bar {\cal H}&=& \bar {\cal H}_{0}+\bar {\cal V}(\bar x), \\
	\bar {\cal H}_{0}&=&- \frac{\partial^2 }{\partial {\bar x}^2}+\left(\begin{array}{cc}
	0 & 0 \\
	0 & -i\bar\Gamma
	\end{array}\right), \\
	\bar {\cal V}(\bar x)&=& \left(\begin{array}{cc}
	0 & \bar\Omega (\bar x)\\
	\bar\Omega (\bar x)^{*} & 0
	\end{array}\right).
\end{eqnarray}
%
To set the matrices we use as in the main text the convention for internal states $\ket{1} = \left(\begin{smallmatrix}1\\0\end{smallmatrix}\right)$ and $\ket{2} = \left(\begin{smallmatrix}0\\1\end{smallmatrix}\right)$.
%
To simplify the notation, we will from now on drop the bars above variables and operators for the remaining part of this section A.
The corresponding stationary Schr\"odinger equation is now
%
\begin{eqnarray}
	k^{2} \psi^{(1)}(x)&=&-\frac{\partial^2}{\partial x^2}\psi^{(1)}(x)+\Omega(x) \psi^{(2)}(x),
	\nonumber \\
	k^{2} \psi^{(2)}(x)&=&-\frac{\partial^2}{\partial x^2}\psi^{(2)}(x)+\Omega(x)^* \psi_{1}(x)-i \Gamma \psi^{(2)}(x).
	\nonumber
\end{eqnarray}
%
Let us denote as  $|{\Psi}_\alpha(x)\ra$  the wave vector for the atom impinging in internal level $\alpha$, $\alpha=1,2$.
This vector has ground and excited state components, generically $\braket{\beta}{\psi_{\alpha}(x)}$, $\beta=1,2$, which are still functions of $x$.
We can define the matrices $F(x)$ and $\widetilde{F}(x)$ as
%
\begin{eqnarray}
	F_{\beta,\alpha} (x) = \braket{\beta}{\psi_{\alpha}(x)},
	\quad
	\widetilde{F}_{\beta,\alpha} (x) = \braket{\beta}{\widetilde{\psi}_{\alpha}(x)},
\end{eqnarray}
%
so the stationary Schr\"odinger equation can be rewritten as
%
\begin{eqnarray}
	\left[k^2-{\cal H}_{0}-{\cal V}(x)\right]F(x)&=&0,
	\nonumber\\
	\left[k^2-{\cal H}_{0}-{\cal V}(x)\right]\widetilde{F}(x)&=&0.
\end{eqnarray}
%

\section{Free motion, ${\cal V}=0$}
%
%
When ${\cal V}(x)=0$ we get
%
\begin{eqnarray}
	\left[k^2-{\cal H}_{0}\right]\ket{\psi_{\alpha}(x)}&=&0,
	\nonumber\\
	\left[k^2-{\cal H}_{0}\right]\ket{\widetilde{\psi}_{\alpha}(x)}&=&0,
\end{eqnarray}
%
for $\alpha=1,2$.
We can write down the solutions for particles ``coming'' from the left $\ket{\psi_{\alpha}(x)}$ in  internal state $\ket{\alpha}$ as
%
\begin{eqnarray}
	\ket{\psi_{1}(x)} = \left(\begin{array}{c}
	\frac{1}{\sqrt{k}} e^{i k x}\\
	0
	\end{array}\right)\!,
	\,
	\ket{\psi_{2}(x)} = \left(\!\!\begin{array}{c}
	0\\
	\frac{1}{\sqrt[4]{k^2+i \Gamma}} e^{i \sqrt{k^2+i \Gamma}x}
	\end{array}\!\!\right)\!,
	\nonumber
\end{eqnarray}
%
where we assume the branch $\operatorname{Im} \sqrt{k^2+i \Gamma}\ge 0$.
$\ket{\psi_2(x)}$ is a regular traveling wave only for real $\sqrt{k^2+i\Gamma})$. If the square root has an imaginary part,  $\ket{\psi_2(x)}$  decays  from left to right.
%
The solutions for incidence from the right  $\ket{\widetilde{\psi}_{\alpha}(x)}$ in internal state $\ket{\alpha}$ are similarly
%
\begin{eqnarray}
	\ket{\widetilde{\psi}_{1}(x)} = \left(\!\begin{array}{c}
	\frac{1}{\sqrt{k}} e^{-i k x}\\
	0
	\end{array}\!\right)\!,
	\ket{\widetilde{\psi}_{2}(x)} = \left(\!\!\begin{array}{c}
	0\\
	\frac{1}{\sqrt[4]{k^2+i \Gamma}} e^{-i \sqrt{k^2+i \Gamma}x}
	\end{array}\!\!\right)\!.
	\nonumber
\end{eqnarray}
%
The normalization is chosen in such a way that the dimensionless probability current
is constant (and equal) for all solutions with real $\sqrt{k^2+i\Gamma}$.


The solutions are given by $F(x) = h_+ (x)$ and $\widetilde F (x) = h_- (x)$, where
%
\begin{equation}
	h_{\pm}(x)=\left(\begin{array}{cc}
	\frac{1}{\sqrt{k}}e^{\pm i k x} & 0\\
	0 & \frac{1}{\sqrt[4]{k^2+i \Gamma}} e^{\pm i \sqrt{k^2+i \Gamma} x} \\
	\end{array}\right).
\end{equation}
%
The Wronskian is $W(h_{+},h_{-})(x)=2i$ so that these are linearly independent solutions.


\section{General case}
To solve the general case, we construct the Green's function defined by
%
\begin{equation}
	(k^2-{\cal H}_{0})G_{0}(x,x')=\delta(x-x')\mathbf{1}.
\end{equation}
%
It is given by
%
\begin{eqnarray}
	G_{0}(x,x')&=&W^{-1} \begin{cases}
	h_{+}(x)h_{-}(x')  & x>x' , \\ h_{+}(x')h_{-}(x) & x'>x ,
	\end{cases} \\
	&=& -\frac{i}{2} \left(\!\begin{array}{cc}
	\frac{1}{k}e^{ i k \abs{x-x'}} & 0\\
	0 & \frac{e^{ i \sqrt{k^2+i \Gamma} \abs{x-x'}}}{\sqrt{k^2+i \Gamma}} \\
	\end{array}\!\right). \nonumber
\end{eqnarray}
%
The Green's function allows us to solve for   $F(x)$ and $\widetilde{F}(x)$ in integral form,
%
\begin{eqnarray}
	F(x)=h_{+}(x)+\int_{-\infty}^{\infty} dx' G_{0}(x,x') {\cal V}(x') F(x'),
	\nonumber\\
	\widetilde{F}(x)=h_{-}(x)+\int_{-\infty}^{\infty} dx' G_{0}(x,x') {\cal V}(x') \widetilde{F}(x').
	\label{eqf}
\end{eqnarray}


\section{Asymptotic form of the solutions}
%
From eq. \eqref{eqf} we find the following asymptotic forms of  $F(x)$ and $\widetilde{F}(x)$:
%
\begin{eqnarray}
	F_{\eta}(x)&=& \begin{cases}
	h_{+}(x)+h_{-}(x)R  & x<0  \\ h_{+}(x)T & x>1
	\end{cases},
	\nonumber\\
	\widetilde{F}_{\eta}(x)&=&\begin{cases}
	h_{-}(x)\widetilde{T}  &  x<0  \\ h_{-}(x)+h_{+}(x)\widetilde{R} & x>1
	\end{cases},
\end{eqnarray}
%
where the $R$ and $T$ matrices for incidence from the left are given by
%
\begin{eqnarray}
	R &=& W^{-1}\int_{0}^{1} dx' h_{+}(x'){\cal V}(x')F (x'),
	\nonumber\\
	T &=& \mathbf{1}+W^{-1}\int_{0}^{1} dx' h_{-}(x'){\cal V}(x')F (x'),
\end{eqnarray}
%
whereas, for right incidence,
%
\begin{eqnarray}
	\widetilde{R} &=& W^{-1}\int_{0}^{\eta} dx' h_{-}(x'){\cal V}(x')\widetilde{F}_{\eta}(x'),
	\nonumber\\
	\widetilde{T} &=& \mathbf{1}+W^{-1}\int_{0}^{\eta} dx' h_{+}(x'){\cal V}(x')\widetilde{F}_{\eta}(x').
\end{eqnarray}
%
In particular, for left incidence  in the ground-state, we get  if $x < 0$,
%
\begin{eqnarray}
	\hspace*{-.5cm}|\psi_1 (x)\ra = \left(\!\!\begin{array}{c} \frac{1}{\sqrt{k}}e^{i k x} \\ 0\end{array}\!\!\right)
	+ \left(\!\!\begin{array}{c}
	R_{1,1} \frac{1}{\sqrt{k}}e^{-i k x} \\
	R_{2,1} \frac{1}{\sqrt[4]{k^2+i \Gamma}} e^{- i \sqrt{k^2+i \Gamma} x}
	\end{array}\!\!\right)\!,
	\label{as-}
\end{eqnarray}
%
and, if $x>1$,
%
\begin{eqnarray}
	|\psi_1 (x)\ra = \left(\begin{array}{c}
	T_{1,1} \frac{1}{\sqrt{k}}e^{i k x} \\
	T_{2,1} \frac{1}{\sqrt[4]{k^2+i \Gamma}} e^{i \sqrt{k^2+i \Gamma} x}
	\end{array}\right).
	\label{as+}
\end{eqnarray}
%
When $\sqrt{k^2+i \Gamma}$ is real, the elements of $T$ and $R$ in Eqs. \eqref{as-} and \eqref{as+}
are transmission and reflection amplitudes for waves traveling away from the interaction region.
However when  ${\rm Im}\sqrt{k^2+i \Gamma}>0$ the waves for the excited state $2$ are evanescent.
In scattering theory parlance the channel is ``closed'', so  the $T_{2,1}$ and $R_{2,1}$ are just proportionality factors
rather than proper transmission
and reflection amplitudes for travelling waves. By continuity however, it is customary to keep the same notation
and even terminology for closed or open channels.


In a similar way, for right incidence in the ground state and
$x > 1$,
%
\begin{eqnarray}
	\hspace*{-.5cm}|\widetilde\psi_1 (x)\ra = \left(\!\!\begin{array}{c} \frac{1}{\sqrt{k}}e^{-i k x} \\ 0\end{array}\!\!\right)
	\!+\! \left(\!\!\begin{array}{c}
	\widetilde R_{1,1} \frac{1}{\sqrt{k}}e^{i k x}
	\\
	\widetilde R_{2,1} \frac{1}{\sqrt[4]{k^2+i \Gamma}} e^{i \sqrt{k^2+i \Gamma} x}
	\end{array}\!\!\right)\!\!,
\end{eqnarray}
%
whereas, for $x<0$,
%
\begin{eqnarray}
	|\widetilde \psi_1 (x)\ra = \left(\begin{array}{c}
	\widetilde T_{1,1} \frac{1}{\sqrt{k}}e^{-i k x} \\
	\widetilde T_{2,1} \frac{1}{\sqrt[4]{k^2+i \Gamma}} e^{-i \sqrt{k^2+i \Gamma} x}
	\end{array}\right).
\end{eqnarray}
%
Note that alternative definitions of the amplitudes may be found in many works,
without momentum prefactors.

The amplitudes relevant for the main text are $T^l=T_{1,1}$,
$T^r=\widetilde{T}_{1,1}$, $R^l=R_{1,1}$, and $R^r=\widetilde{R}_{1,1}$.  The following
subsection explains how to compute them.


\section{Differential equations for $R$ and $T$ matrices}
To solve for $R$ and $T$ we will use cut-off versions of the  potential,
%

\begin{equation}
	{\cal V}_{\eta} (x) =\begin{cases}
	{\cal V}(x)  & 0\leq x \leq \eta , \\ 0& \text{Otherwise}
	\end{cases},
\end{equation}
%
where $0 \le \eta \le 1$, and corresponding matrices
%
%
\begin{eqnarray}
	R_{\eta}&=& W^{-1}\int_{0}^{\eta} dx' h_{+}(x'){\cal V}(x')F_{\eta}(x'),
	\nonumber\\
	T_{\eta}&=& \mathbf{1}+W^{-1}\int_{0}^{\eta} dx' h_{-}(x'){\cal V}(x')F_{\eta}(x'),
	\nonumber\\
	%
	\widetilde{R}_{\eta} &=& W^{-1}\int_{0}^{\eta} dx' h_{-}(x'){\cal V}(x')\widetilde{F}_{\eta}(x'),
	\nonumber\\
	\widetilde{T}_{\eta} &=& \mathbf{1}+W^{-1}\int_{0}^{\eta} dx' h_{+}(x'){\cal V}(x')\widetilde{F}_{\eta}(x').
\end{eqnarray}
%
Taking the derivative of these matrices with respect to $\eta$,
we find a set
of four coupled differential equations,
%
\begin{eqnarray}
	\frac{d R_{\eta}}{d \eta}&=& W^{-1} \widetilde{T}_{\eta} h_{+}(\eta){\cal V}(\eta)h_{+}(\eta)T_{\eta}, \\
	\frac{d T_{\eta}}{d \eta}&=& W^{-1} \left[h_{-}(\eta)+\widetilde{R}_{\eta}h_{+}(\eta)\right]{\cal V}(\eta)h_{+}(\eta)T_{\eta},\\
	\frac{d \widetilde{R}_{\eta}}{d \eta}&=& W^{-1}\!\!\left[h_{-}(\eta)\!+\!\widetilde{R}_{\eta}h_{+}(\eta)\right]\!{\cal V}(\eta)\!\left[h_{-}(\eta)\!+\!h_{+}(\eta)\widetilde{R}_{\eta}\right]\!, \nonumber\\
	\label{a20a}\\
	\frac{d \widetilde{T}_{\eta}}{d \eta}&=& W^{-1} \widetilde{T}_{\eta} h_{+}(\eta){\cal V}(\eta)\left[h_{-}(\eta)+h_{+}(\eta)\widetilde{R}_{\eta}\right].
	\label{a20b}
\end{eqnarray}
%
The initial conditions are $R_{0} = \widetilde{R}_{0}=0$ and $T_{0} = \widetilde{T}_{0}=\mathbf{1}$.


\section{Improving numerical efficiency}
%
{The equations \eqref{a20a} and \eqref{a20b}} involve only  matrices for incidence from the right, they  do not couple to any left-incidence matrix, whereas the equations for left incidence
amplitudes involve couplings with amplitudes for right incidence. This asymmetry is due to the way we do the potential slicing. The asymmetry  is not ``fundamental''
but we can use it for our advantage to simplify calculations. We can solve equations \eqref{a20a} and \eqref{a20b} to get amplitudes for right incidence.
To get amplitudes for left incidence we use a mirror image of the potential and solve also these two equations.
Thus it is enough to find an efficient numerical method to solve equations \eqref{a20a} and \eqref{a20b}.
In principle, one can now solve these differential equations from $\eta=0$ to $1$ to get all reflection and transmission amplitudes using the boundary conditions $\widetilde{R}_{0}=0$ and $\widetilde{T}_{0}=\mathbf{1}$. However due to the exponential nature of the
free-space solutions $h_{\pm}(x)$ especially if ${\rm Im}\sqrt{k^2+i \Gamma}>0$, this is not very efficient numerically.

To avoid this problem we make new definitions,
%
\begin{eqnarray}
	\hat{S}_{\eta}&=&\mathbf{1}+h_{+}(\eta)\widetilde{R}_{\eta}h_{-}^{-1}(\eta),
	\nonumber\\
	\hat{T}_{\eta}&=&h_{+}(0)\widetilde{T}_{\eta}h_{-}^{-1}(\eta),
	\nonumber\\
	\hat{\cal V}(\eta)&=&W^{-1}h_{+}^{2}(0){\cal V}(\eta),
	\nonumber\\
	\hat{Q}&=&i h_{+}^{-2}(0).
\end{eqnarray}
%
Rewriting {the equations \eqref{a20a} and \eqref{a20b}} in terms of these new variables we get
%
\begin{eqnarray}
	\frac{d \hat{S}_{\eta}}{d \eta}&=&-2 \hat{Q}+\hat{Q}\hat{S}_{\eta}+\hat{S}_{\eta}\left[\hat{Q}+\hat{\cal V}(\eta)\hat{S}_{\eta}\right],
	\nonumber\\
	\frac{d \hat{T}_{\eta}}{d \eta}&=& \hat{T}_{\eta}\left[\hat{Q}+\hat{\cal V}(\eta)\hat{S}_{\eta}\right],
\end{eqnarray}
%
with initial conditions $\hat{T}_{0} =\hat{S}_{0}=\mathbf{1}$.

Let us consider solely incidence in the ground state. For right incidence in the ground state,
the reflection coefficients and transmission coefficient are
%
\begin{eqnarray}
	\widetilde R_{1,1} &=& e^{-2ik} \left[(\hat S_{\eta=1})_{1,1} - 1 \right],
	\nonumber\\
	\widetilde R_{2,1} &=&
	\frac{\sqrt[4]{k^2+i \Gamma}}{\sqrt{k}} e^{-ik-i\sqrt{k^2+i \Gamma}} (\hat S_{\eta=1})_{2,1},
	\nonumber\\
	\widetilde T_{1,1} &=& e^{-i k} (\hat T_{\eta=1})_{1,1},
	\nonumber\\
	\widetilde T_{2,1} &=& \frac{\sqrt[4]{k^2+i \Gamma}}{\sqrt{k}} e^{-i k} (\hat T_{\eta=1})_{2,1}.
\end{eqnarray}
\vspace*{1cm}


\section{Bounds from unitarity}
%
The $S$-matrix
%
\begin{equation}
	S=\left(\begin{array}{cccc}
	T_{11}&T_{12}&\widetilde R_{11}&\widetilde R_{12}
	\\
	T_{21}&T_{22}&\widetilde R_{21}&\widetilde R_{22}
	\\
	R_{11}&R_{12}&\widetilde T_{11}&\widetilde T_{12}
	\\
	R_{21}&R_{22}&\widetilde T_{21}&\widetilde T_{22}
	\end{array}\right)
\end{equation}
%
is unitary for Hermitian Hamiltonians,  in particular when $\gamma=0$.
Unitarity implies relations among the matrix elements and in particular
%
\begin{eqnarray}
	1&\ge& |R_{11} |^2+|T_{11} |^2,\label{a28}
	\\
	1&\ge& |\widetilde R_{11}|^2+|\widetilde T_{11}|^2, \label{a29}
	\\
	1&\ge& |\widetilde R_{11} |^2+|T_{11} |^2,
	\label{a30}\\
	1&\ge& |R_{11} |^2+|\widetilde T_{11}|^2.\label{a31}
\end{eqnarray}
%
While the first two equations \eqref{a28} and \eqref{a29} are rather obvious because of  probability conservation, the last two equations
\eqref{a30} and \eqref{a31} are less so, and
set physical  limits to the possible asymmetric devices that can be constructed in the ground state subspace.
