%%!TEX root = ../Thesis.tex

\chapter{Properties of separable potentials}
\label{Appendix:SeparablePotentials}
\lhead{Appendix C. \emph{Properties of separable potentials}}

\section{Transition operator}
\label{Appendix:SeparablePotentials_TransitionOperator}
%%%%%%%%%%%%%%%%%
 For a separable potential $V=V_0 \ketbra{\phi}{\chi}$, the transition operator becomes
\begin{eqnarray}
T_{op}=\alpha \ketbra{\phi}{\chi}
\end{eqnarray}
where $\alpha=V_{0}+V_{0}^{2}\bra{\chi}G(E)\ket{\phi}$. Then using the Lippmann-Schwinger equation we get that
%
\begin{eqnarray}
T_{op}(E)&=&V+VG_{0}(E)T_{op}(E)
\nonumber \\
&=& \left[V_{0}+\alpha V_{0} \bra{\chi}G_{0}(E)\ket{\phi}\right]\ketbra{\phi}{\chi}
\end{eqnarray}
%
where $G_{0}(E)=(E-H_{0})^{-1}$ is the Green's operator for free motion. Solving for $\alpha$ now gives
%
\begin{eqnarray}
\alpha =\frac{V_{0}}{1-V_{0} \bra{\chi}G_{0}(E)\ket{\phi}}
		   =\frac{V_{0}}{1-V_{0} Q_{0}(E)}.
\end{eqnarray}
%
%%%%%%%%%%%%%%%%%
%
%
%
%
\section{$S$-matrix eigenvalues}
%
%
%
%
The eigenvalues for the S-matrix are given by Eq. \eqref{Sform} in terms of the reflection and transmission amplitudes. For a separable potential, using Eq. \eqref{art}, we can simplify the transmission and reflection coefficients as
\begin{eqnarray}
T^{l}&=&1-\frac{2 \pi i m}{p} \alpha \phi(p) \chi^{*}(p),
\nonumber \\
T^{r}&=&1-\frac{2 \pi i m}{p} \alpha \phi(-p) \chi^{*}(-p),
\nonumber \\
R^{l}&=&-\frac{2 \pi i m}{p} \alpha \phi(-p) \chi^{*}(p),
\nonumber \\
R^{r}&=&-\frac{2 \pi i m}{p} \alpha \phi(p) \chi^{*}(-p).
\nonumber \\
\end{eqnarray}
%
If we now define
%
\begin{equation}
\Gamma=\frac{2 \pi i m}{p} \alpha \left[\phi(p) \chi^{*}(p)+\phi(-p) \chi^{*}(-p)\right],
\end{equation}
%
we can write the eigenvalues as simply
%
\begin{eqnarray}
S_{j}&=&1-\frac{\Gamma-(-1)^{j}\Gamma}{2}.
\end{eqnarray}
%
Note that $S_2=1$ for all $p$. Clearly the following relation must also always hold for the reflection and transmission amplitudes,
%
\begin{equation}
T^l + T^r - T^l T^r + R^l R^r = 1.
\end{equation}
%
%
%
%
%
%
%
%
\section{Uniqueness of bound state}
%
A separable potential  can only have at most one bound state $\ket{\psi_{E}}$.
In momentum representation,
%
\begin{eqnarray}
\braket{p}{\psi_{E}}&=&\bra{p}\frac{V_{0}}{E-H_{0}}\ket{\phi}\braket{\chi}{\psi_{E}}
\nonumber \\
&=&\frac{M}{p^{2}-q_{B}^{2}}\braket{p}{\phi},
\end{eqnarray}
%
where $M=-2 m V_{0} \braket{\chi}{\psi_{E}}$ and $q_{B}^2=2 m E<0$. Suppose there is a second bound state $\ket{\psi_{E'}}$,
with corresponding quantities $M'$ and $q_{B'}^2$. Then,
%
\begin{eqnarray}
\braket{\psi_{E'}}{\psi_{E}}&=&M M' \int_{-\infty}^\infty dp \left|\braket{p}{\phi}\right| \frac{1}{p^{2}-q_{B}^{2}}\frac{1}{p^{2}-q_{B'}^{2}}. \nonumber \\		\end{eqnarray}
%
Since $MM'\ne 0$ and the integral is positive the overlap cannot be zero
so there cannot be two bound states.
