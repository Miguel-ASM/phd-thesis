%%!TEX root = ../Thesis.tex

\chapter{Complete expressions for the Spectral Density Matrix}
\label{Appendix:SpectralDensity}
\lhead{Appendix F. \emph{Complete expressions for the Spectral Density Matrix}}

%
In section \ref{sec:lookingForR} we used the characteristic polynomial $P_{\mathbb{A}}(\lambda)$ of the dynamical matrix $\mathbb{A}$ for the calculation of the spectral density matrix. $P_{\mathbb{A}}(\lambda)$ is defined as
\begin{equation}
  \begin{split}
    P_{\mathbb{A}}(\lambda) &\equiv\det(\mathbb{A}-\lambda)\\
    &= \lambda ^4 \\&+ \lambda ^3 \left(\frac{\gamma_L}{m_1}+\frac{\gamma_R}{m_2}\right) \\ &+ \lambda^2\frac{ (\gamma_L \gamma_R+m_2 (k+k_L)+m_1 (k+k_R))}{m_1 m_2}\\ &+ \lambda \frac{  (\gamma_R (k+k_L)+\gamma_L (k+k_R))}{m_1 m_2}\\ &+\frac{k (k_L+k_R)+k_L k_R}{m_1 m_2}.
  \end{split}
\end{equation}
%
We also used the polynomials  $\mathbb{S}_L(\lambda)=\sum\limits_{n=0}^6 \lambda^n \mathbb{s}_{L,n}$ and $\mathbb{S}_R(\lambda)=\sum\limits_{n=0}^6 \lambda^n \mathbb{s}_{R,n}$. There are 14 different polynomial coefficients, which are $4\times 4$ matrices. This is the full list of coefficients,
%
\begingroup
\allowdisplaybreaks
\begin{align}
    \mathbb{s}_{L,0} &=
    \left(
    \begin{array}{cccc}
      (k+k_R)^2 & k (k+k_R) & 0 & 0 \\
      k (k+k_R) & k^2 & 0 & 0 \\
      0 & 0 & 0 & 0 \\
      0 & 0 & 0 & 0
    \end{array}
    \right),
    %
    \nonumber\\
    %
    \mathbb{s}_{R,0} & =
    \left(
    \begin{array}{cccc}
      k^2 & k (k+k_L) & 0 & 0 \\
      k (k+k_L) & (k+k_L)^2 & 0 & 0 \\
      0 & 0 & 0 & 0 \\
      0 & 0 & 0 & 0
    \end{array}
    \right),
    %
    \nonumber\\
    %
    \mathbb{s}_{L,1} &= \left(
    \begin{array}{cccc}
      0 & k \gamma_R & -(k+k_R)^2 & -k (k+k_R) \\
      -k \gamma_R & 0 & -k (k+k_R) & -k^2 \\
      (k+k_R)^2 & k (k+k_R) & 0 & 0 \\
      k (k+k_R) & k^2 & 0 & 0
    \end{array}
    \right),
    %
    \nonumber\\
    %
    \mathbb{s}_{R,1} & = \left(
    \begin{array}{cccc}
      0 & -k \gamma_L & -k^2 & -k (k+k_L) \\
      k \gamma_L & 0 & -k (k+k_L) & -(k+k_L)^2 \\
      k^2 & k (k+k_L) & 0 & 0 \\
      k (k+k_L) & (k+k_L)^2 & 0 & 0
    \end{array}
    \right),
    %
    \nonumber\\
    %
    \mathbb{s}_{L,2} &= \left(
    \begin{array}{cccc}
      2 (k+k_R) m_2-\gamma_R^2 & k m_2 & 0 & -k \gamma_R \\
      k m_2 & 0 & k \gamma_R & 0 \\
      0 & k \gamma_R & -(k+k_R)^2 & -k (k+k_R) \\
      -k \gamma_R & 0 & -k (k+k_R) & -k^2
    \end{array}
    \right),
    %
    \nonumber\\
    %
    \mathbb{s}_{R,2} & = \left(
    \begin{array}{cccc}
      0 & k m_1 & 0 & k \gamma_L \\
      k m_1 & 2 (k+k_L) m_1-\gamma_L^2 & -k \gamma_L & 0 \\
      0 & -k \gamma_L & -k^2 & -k (k+k_L) \\
      k \gamma_L & 0 & -k (k+k_L) & -(k+k_L)^2
    \end{array}
    \right),
    %
    \nonumber\\
    %
    \mathbb{s}_{L,3} &= \left(
    \begin{array}{cccc}
      0 & 0 & \gamma_R^2-2 (k+k_R) m_2 & -k m_2 \\
      0 & 0 & -k m_2 & 0 \\
      2 (k+k_R) m_2-\gamma_R^2 & k m_2 & 0 & -k \gamma_R \\
      k m_2 & 0 & k \gamma_R & 0
    \end{array}
    \right),
    %
    \nonumber\\
    %
    \mathbb{s}_{R,3} & =\left(
    \begin{array}{cccc}
      0 & 0 & 0 & -k m_1 \\
      0 & 0 & -k m_1 & \gamma_L^2-2 (k+k_L) m_1 \\
      0 & k m_1 & 0 & k \gamma_L \\
      k m_1 & 2 (k+k_L) m_1-\gamma_L^2 & -k \gamma_L & 0
    \end{array}
    \right),
    %
    \nonumber\\
    %
    \mathbb{s}_{L,4} &= \left(
    \begin{array}{cccc}
      m_2^2 & 0 & 0 & 0 \\
      0 & 0 & 0 & 0 \\
      0 & 0 & \gamma_R^2-2 (k+k_R) m_2 & -k m_2 \\
      0 & 0 & -k m_2 & 0
    \end{array}
    \right),
    %
    \nonumber\\
    %
    \mathbb{s}_{R,4} & = \left(
    \begin{array}{cccc}
      0 & 0 & 0 & 0 \\
      0 & m_1^2 & 0 & 0 \\
      0 & 0 & 0 & -k m_1 \\
      0 & 0 & -k m_1 & \gamma_L^2-2 (k+k_L) m_1
    \end{array}
    \right),
    %
    \nonumber\\
    %
    \mathbb{s}_{L,5} &= \left(
    \begin{array}{cccc}
      0 & 0 & -m_2^2 & 0 \\
      0 & 0 & 0 & 0 \\
      m_2^2 & 0 & 0 & 0 \\
      0 & 0 & 0 & 0
    \end{array}
    \right),
    %
    \nonumber\\
    %
    \mathbb{s}_{R,5} & = \left(
    \begin{array}{cccc}
      0 & 0 & 0 & 0 \\
      0 & 0 & 0 & -m_1^2 \\
      0 & 0 & 0 & 0 \\
      0 & m_1^2 & 0 & 0
    \end{array}
    \right),
    %
    \nonumber\\
    %
    \mathbb{s}_{L,6} &= \left(
    \begin{array}{cccc}
      0 & 0 & 0 & 0 \\
      0 & 0 & 0 & 0 \\
      0 & 0 & -m_2^2 & 0 \\
      0 & 0 & 0 & 0
    \end{array}
    \right),
    %
    \nonumber\\
    %
    \mathbb{s}_{R,6} & = \left(
    \begin{array}{cccc}
      0 & 0 & 0 & 0 \\
      0 & 0 & 0 & 0 \\
      0 & 0 & 0 & 0 \\
      0 & 0 & 0 & -m_1^2
    \end{array}
    \right)\,.
\end{align}%
\endgroup
